% !TeX root = ./main.tex
\documentclass[main]{subfiles}
\begin{document}
\chapter{Приведение уравнения II порядка к каноническому виду}
\begin{gather*}
    a_{11} x^2 + 2 a_{12} xy + a_{22} y^2 + 2b_1 +2b_2y +b_3 = 0\\
    a_{11}^2 + a_{12}^2 + a_{22}^2 \neq 0
\end{gather*}

\textbf{I шаг. } Поворот на угол $\alpha$, чтобы избавиться от $a_{12}$

\begin{theorem}
    $(x', y')$ получено поворотом $(x,y)$ на $\alpha$:
    \begin{gather*}
        \begin{cases}
            x' = x \cos \alpha + y \sin \alpha \\
            y' = - x \sin \alpha + y \cos \alpha
        \end{cases}\qquad
        \begin{cases}
            x = x' \cos \alpha - y' \sin \alpha \\
            y = x' \sin \alpha + y' \cos \alpha
        \end{cases}
    \end{gather*}
\end{theorem}
\begin{proof}
    Для доказательства используем полярную систему координат $(r, \phi) \to (r', \phi')$

    \begin{gather*}
        r' = r \qquad \phi' = \phi - \alpha\\
        x = r \cos \phi \qquad y = r \sin \phi \\
        \begin{cases}
            x' = r' \cos \phi' = r \cos (\phi - \alpha) = r \cos \phi \cos \alpha + r \sin \phi \sin \alpha  \\
            y' = r' \sin \phi' = r \sin (\phi - \alpha) = -r \cos \phi \sin \alpha + r \sin \phi \cos \alpha \\
        \end{cases}\\
        \begin{cases}
            x' = x \cos \alpha + y \sin \alpha \\
            y' = -x \sin \alpha + y \cos \alpha
        \end{cases}\\
        x = x' \cos \alpha - y' \sin \alpha\\
        y = x' \sin \alpha + y' \cos \alpha
    \end{gather*}
\end{proof}

Получили такое выражение, выясним при каком $\alpha a_{12}$ станет нулем
\begin{multline*}
    a_{11} (x' \cos \alpha - y' \sin \alpha)^2 + 2a_{12} (x' \cos \alpha - y' \sin \alpha)(x' \sin \alpha + y' \cos \alpha) + \\
    + a_{22} (x' \sin \alpha + y' \cos \alpha)^2 + ... = 0
\end{multline*}

Коэффициент при $x'y'$:
\begin{gather*}
    a_{11}(-2\cos\alpha\sin\alpha) + 2 a_{12} (\cos^2 \alpha - \sin^2 \alpha)
    + a_{22}(2\sin\alpha \cos\alpha) = 0\\
    -a_{11} \sin 2\alpha + 2a_{12} \cos 2\alpha + a_{22} \sin 2 \alpha=0 |:\cos2\alpha\\
    -a_{11} \tg 2 \alpha + a_{22} \tg 2 \alpha = -2 a_{12}\\
    \tg 2 \alpha = \frac{2 a_{12}}{a_{11} - a_{22}}\\
    \ctg 2 \alpha = \frac{a_{11} - a_{22}}{2a_{12}}
\end{gather*}
Если $a_{12} \neq 0$, то $\ctg 2 \alpha$ найдется, то найдем $\alpha \in \left[0; \frac{\pi}{2}\right]$.
Если $a_{12} = 0$, то $\alpha = 0$

\textbf{II шаг. } Теперь рассмотрим уравнение
\[a_{11} x^2 + a_{22} y^2 + 2b_1 x + 2 b_2y +b_3 = 0\]
(вообще-то везде штрихи)

\begin{lemma}
    Если $a_{11} \neq 0$, то считаем $b_1 = 0$ (иначе сдвинем переменные: $x' = x - x_0$)
    \begin{gather*}
        a_{11} x^2 + 2 b_1 x =
        a_{11}\left(x^2+2\frac{b_1}{a_{11}}x + \frac{b_1^2}{a_{11}^2} -  \frac{b_1^2}{a_{11}^2}\right) =
        a_{11} x'^2 - \frac{b_1^2}{a_{11}}\\
        x' = x + \frac{b_1}{a_{11}}
    \end{gather*}

    Аналогично если $a_{22} \neq 0 $, то считаем $b_2=0$
\end{lemma}

\section{Виды кривых}
\subsection{Эллиптический тип}
$a_{11}>0, a_{22} > 0$ (иначе умножим на $-1$)
\[a_{11} x^2 + a_{22} y^2 + b^3 = 0\]
\begin{enumerate}
    \item $b_3 < 0$ $ \frac{x^2}{a^2} + \frac{y^2}{b^2} = 1$ -- эллипс
          \[a = \sqrt{\frac{-b_3}{a_{11}}}; b = \sqrt{\frac{-b_3}{a_{22}}}\]
    \item $b_3 = 0$ $ \frac{x^2}{a^2} + \frac{y^2}{b^2} = 0$ -- точка
    \item $b_3> 0$ $ \frac{x^2}{a^2} + \frac{y^2}{b^2} = -1$ -- пустое множество или мнимый эллипс
\end{enumerate}

\subsection{Гиперболический тип}
$a_{11}>0, a_{22} < 0$ (или наоборот)
\begin{enumerate}
    \setcounter{enumi}{3}
    \item $b_3 \neq 0$ $\frac{x^2}{a^2} - \frac{y^2}{b_2} = 1$ --  гипербола
    \item $b_3 = 0$ $\frac{x^2}{a^2} - \frac{y^2}{b_2} = 0$ -- пара пересекающихся прямых
          \begin{gather*}
              \left(\frac{x}{a}+\frac{y}{b}\right)\left(\frac{x}{a} - \frac{y}{b}\right)=0\\
              \frac{x}{a} = \pm \frac{y}{b}
          \end{gather*}
\end{enumerate}

\subsection{Параболический тип}
$a_{11} = 0, a_{22} \neq 0$, считаем, что $b_2 = 0$
\[a_{22}y^2 + 2b_1x + b_3 = 0\]
\begin{enumerate}
    \setcounter{enumi}{5}
    \item Если $b_1 \neq 0$, то считаем $b_3 = 0$
          \[2b_1x + b_3 = 2b_1\left(x + \frac{b_3}{2b_1}\right) = 2b_1 x'\]
          $y^2 = 2px$ -- парабола
    \item Если $b_1 = 0, a_{22} > 0 \quad a_{22}y^2 + b_3 = 0$ \\
          $b_3 < 0$ $\frac{y^2}{b^2} = 1$ -- пара параллельных прямых
          \[\frac{y}{b} = \pm 1\]
    \item $b_3=0$ $\frac{y^2}{b^2} =0$ -- одна прямая
    \item $b_3>0$ $\frac{y^2}{b^2} =-1$ -- пустое множество или пара мнимых прямых
\end{enumerate}

\end{document}