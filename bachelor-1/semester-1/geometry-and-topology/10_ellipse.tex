% !TeX root = ./main.tex
\documentclass[main]{subfiles}
\begin{document}
\part{Кривые II порядка}
\chapter{Эллипс}
\begin{definition}[Стандартный вид прямой II порядка]
    \[a_{11}x^2 + 2a_{12}xy + a_{22}y^2 + b_1x +b_2y +b_3=0\]
    \[a_{11}^2 + a_{12}^2 + a_{22}^2 \neq 0\]
\end{definition}
\begin{definition}\label{def:ellipse1}
    Эллипс --- кривая, которая в подходящих координатах задается уравнением:
    \[\frac{x^2}{a^2} + \frac{y^2}{b^2}=1\]
\end{definition}
\begin{definition}\label{def:ellipse2}
    Пусть $F_1, F_2$ -- точки (фокусы), если $F_1 F_2 = 2c < 2a$, тогда
    ГМТ $M:$ \[F_1M + F_2M=2a\] называется эллипсом.
\end{definition}
\begin{definition}\label{def:ellipse3}
    $F_1$ -- фокус, $l_1$ --  прямая (директриса). ГМТ $M$:
    \[\frac{\dist(F_1,M)}{\dist(l_1,M)} = e < 1\]
    называется эллипсом.
\end{definition}
\begin{minipage}{0.45\textwidth}
    \begin{tikzpicture}
        \draw[axis] (-2.5, 0) -- (2.5, 0) node [right, color=blue!50] {$x$};
        \draw[axis] (0, -2.5) -- (0, 2.5) node [above, color=blue!50] {$y$};
        \draw (0, 0) ellipse [x radius = 1.7, y radius = 0.8];
        \draw (-1,0) node [below] {$F_1$};
        \fill [black] (-1,0) circle (0.05);
        \draw (1, 0) node [below] {$F_2$};
        \fill [black] (1,0) circle (0.05);
        \draw (-2, -2) -- (-2, 2) node [left] {$l_1$};
        \draw (2, -2) -- (2, 2) node [right] {$l_2$};
        \fill [black] (1.7, 0) circle (0.05);
        \draw (1.7, 0.3) node [anchor=south] {$(a,0)$};
        \fill [black] (-1.7, 0) circle (0.05);
        \draw (-1.7, 0.3) node [anchor=south] {$(-a,0)$};
        \fill [black] (0, 0.8) circle (0.05);
        \draw (0, 0.8) node [anchor=south west] {$(0,b)$};
        \fill [black] (0, -0.8) circle (0.05);
        \draw (0, -0.8) node [anchor=north east] {$(0,-b)$};
    \end{tikzpicture}
\end{minipage}
\begin{minipage}{0.45\textwidth}
    \begin{gather*}
        F_2(c,0) \qquad F_1(-c,0)\\
        l_{1,2}: x = \pm \frac{a}{e}\\
    \end{gather*}
\end{minipage}

\section*{Параметры эллипса}
\begin{itemize}
    \item $a$ -- большая полуось
    \item $b$ -- малая полуось (по умолчанию $a \ge b$)
    \item $c$ -- фокальный параметр \[a^2 = b^2 + c^2\]
    \item $e = \frac{c}{a} \in \left[0, 1\right)$ -- эксцентриситет
\end{itemize}

\section*{Доказательство}
\begin{itemize}
    \item В определении \ref{def:ellipse1} задано $a,b \implies c = \sqrt{a^2 -b^2}, e = \frac{c}{a}$
    \item В определении \ref{def:ellipse2} задано $a,c \implies b = \sqrt{a^2 -c^2}, e = \frac{c}{a}$
    \item В определении \ref{def:ellipse3} задано $d$ расстояние от фокуса до директрисы. Хотим $F(c,0);l:x = \frac{a}{e}$
          \begin{gather*}
              d = \frac{a}{e} -c = \frac{a}{e} - ae = a \left(\frac{1}{e}-e\right)\\
              a = \frac{d}{\frac{1}{e}-e}\\
              c=ae; b = \sqrt{a^2 -c^2}
          \end{gather*}
\end{itemize}

\begin{theorem}
    Определения \ref{def:ellipse1}, \ref{def:ellipse2} и \ref{def:ellipse3} равносильны.
\end{theorem}
\begin{proof}
    Докажем, что \ref{def:ellipse1} и \ref{def:ellipse2} равносильны:

    \noindent\begin{minipage}{0.45\textwidth}
        \begin{tikzpicture}
            \draw[axis] (-2.5, 0) -- (2.5, 0) node [right, color=blue!50] {$x$};
            \draw[axis] (0, -1.5) -- (0, 1.5) node [above, color=blue!50] {$y$};
            \draw (-1,0) node [below] {$F_1 (-c, 0)$};
            \fill [black] (-1,0) circle (0.05);
            \draw (1, 0) node [below] {$F_2 (c,0)$};
            \fill [black] (1,0) circle (0.05);
            \draw (0.8, 0.7) node [above] {$M$};
            \fill [black] (0.8, 0.7) circle (0.05);
        \end{tikzpicture}
    \end{minipage}
    \begin{minipage}{0.45\textwidth}
        \begin{gather*}
            F_1M+F_2M = 2a\\
            F_1M = \sqrt{(x+c)^2 + y^2}\\
            F_2M = \sqrt{(x-c)^2 + y^2}
        \end{gather*}
    \end{minipage}
    \begin{gather*}
        \sqrt{(x+c)^2 + y^2} + \sqrt{(x-c)^2 + y^2} = 2a\\
        \sqrt{(x+c)^2 + y^2} = 2a - \sqrt{(x-c)^2 + y^2}\\
        x^2 + 2cx +c^2 + y^2 = 4a^2 + x^2 - 2cx + c^2 + y^2 - 4a \sqrt{(x-c)^2 + y^2}\\
        4a \sqrt{(x-c)^2 + y^2} = 4a^2 - 4cx\quad |:4a\\
        \intertext{Расстояние от точки на эллипсе до фокуса}
        \boxed{\sqrt{(x-c)^2 + y^2} = a - ex} \qquad \boxed{\sqrt{(x+c)^2 + y^2} = a + ex}\\
        (x-c)^2 + y^2 = a^2 - 2aex +e^2x^2\\
        x^2 -2cx +c^2 +y^2 = a^2 -2aex + e^2x^2\\
        x^2(1-e^2) + y^2 = a^2 -c^2 =b^2\\
        x^2 \frac{1-e^2}{b^2} + \frac{y^2}{b^2} = 1\\
        \intertext{Нужно доказать, что:}
        \frac{b^2}{1-e^2} = a^2\\
        b^2 = a^2 -a^2e^2 \qquad a^2 e^2 = c^2\\
        \frac{x^2}{a^2} + \frac{y^2}{b^2}=1 \qedhere
    \end{gather*}
\end{proof}
\begin{proof}
    Докажем, что \ref{def:ellipse1} и \ref{def:ellipse3} равносильны:

    \noindent\begin{minipage}{0.3\textwidth}
        \begin{tikzpicture}
            \draw[axis] (0, 0) -- (2.5, 0) node [right, color=blue!50] {$x$};
            \draw (1, 0) node [below] {$F(c,0)$};
            \fill [black] (1,0) circle (0.05);
            \draw (0.8, 0.7) node [above] {$M(x,y)$};
            \fill [black] (0.8, 0.7) circle (0.05);
            \draw (2, -0.5) -- (2, 0.9) node [right] {$l$};
        \end{tikzpicture}
    \end{minipage}
    \begin{minipage}{0.60\textwidth}
        \begin{gather*}
            l: x = \frac{a}{e} \qquad \frac{\sqrt{(x-c)^2 + y^2}}{\frac{a}{e}-x} = e\\
            \sqrt{(x-c)^2 + y^2} = e\left(\frac{a}{e}-x\right) = a-ex
        \end{gather*}
    \end{minipage}

    Далее смотри равносильность \ref{def:ellipse1} и \ref{def:ellipse2}.
\end{proof}


\begin{theorem}
    Прямая $Ax + By + C = 0$ касается эллипса $\frac{x^2}{a^2} + \frac{y^2}{b^2} =1$
    \[\Leftrightarrow A^2 a^2 + B^2 b^2 = C^2\]
\end{theorem}
\begin{proof}
    Касательная имеет 1 точку пересечения с эллипсом
    \begin{gather*}
        B \neq 0 \qquad
        y = \frac{-C-Ax}{B} \qquad
        \frac{x^2}{a^2} + \frac{\left(\frac{C+Ax}{B}\right)^2}{b^2} = 1\\
        x^2b^2B^2 + a^2C^2 + a^2A^2x^2 + 2a^2ACx = a^2 b^2 B^2\\
        x^2(a^2A^2 + b^2B^2) + 2a^2ACx + \left(a^2C^2 - a^2b^2B^2\right)=0\\
        \intertext{Это уравнение имеет ровно 1 корень}
        \frac{D}{4} = 0 \qquad
        a^4 A^2 C^2 - (a^2C^2 - a^2 b^2 B^2)(a^2A^2 + b^2B^2)=0\\
        a^2 A^2 C^2 - (C^2 - b^2 B^2)(a^2A^2 + b^2B^2)=0\\
        a^2 A^2 C^2 - a^2 A^2 C^2 - b^2B^2 C^2 + a^2b^2A^2B^2 + b^4B^4=0\\
        a^2b^2A^2B^2 + b^4B^4 = b^2B^2 C^2 \\
        a^2A^2 + b^2B^2 =  C^2
    \end{gather*}
\end{proof}

\begin{theorem}
    Если $(x_0, y_0)$  -- точка на эллипсе, тогда касательная
    \[\frac{xx_0}{a^2} + \frac{yy_0}{b^2}=1\]
\end{theorem}
\begin{proof}
    \begin{gather*}
        A = \frac{x_0}{a^2} \qquad B = \frac{y_0}{b^2} \qquad C = -1\\
        a^2 A^2 + b^2B^2 = C^2\\
        a^2 \frac{x_0^2}{a^4} + b^2 \frac{y_0^2}{b^4} = 1\\
        \frac{x_0^2}{a^2} + \frac{y_0^2}{b^2} = 1
    \end{gather*}
    Отсюда $(x_0, y_0)$ -- точка на эллипсе.
\end{proof}

\begin{theorem}[Оптическое свойство эллипса]
    $l$ --касательная к эллипсу в точке $M \implies \angle(l, F_1M) = \angle(l, F_2M)$
\end{theorem}
\begin{proof}
    \

    \noindent\begin{minipage}{0.45\textwidth}
        \begin{tikzpicture}
            \pgfmathsetmacro{\a}{2}
            \pgfmathsetmacro{\b}{1.5}
            \pgfmathsetmacro{\angleO}{60}
            \pgfmathsetmacro{\angleL}{90}
            \pgfmathsetmacro{\angleP}{120}

            \draw[axis] (-2.5, 0) -- (2.5, 0) node [right, color=blue!50] {$x$};
            \draw[axis] (0, -2.5) -- (0, 2.5) node [above, color=blue!50] {$y$};

            \draw (0, 0) ellipse [x radius = \a, y radius = \b];
            \node[dot, label={below:$F_1$}] (F1) at ({-sqrt(\a*\a-\b*\b)},0) {};
            \node[dot, label={below:$F_2$}] (F2) at ({sqrt(\a*\a-\b*\b)},0) {};
            \node[dot, label={\angleO:$O$}] (O) at (\angleO:{\a} and {\b}) {};
            \draw (F1) -- (O) -- (F2);
            \node[dot, label={\angleL:$L$}] (L) at (\angleL:{\a} and {\b}) {};
            \draw (F1) -- (L) -- (F2);
            \node[dot, label={\angleP:$M$}] (P) at (\angleP:{\a} and {\b}) {};
            \draw (F1) -- (P) -- (F2);
            \draw (-2.3, 0.74) -- (1.5, 2.4);
        \end{tikzpicture}
    \end{minipage}
    \begin{minipage}{0.45\textwidth}
        \begin{gather*}
            l: \frac{xx_0}{a^2} + \frac{yy_0}{b^2} = 1 \\
            \vn = \left(\frac{x_0}{a^2}; \frac{y_0}{b^2}\right)
            \intertext{Надо доказать:}
            \cos\angle(\vn; \overrightarrow{F_1M}) = \cos\angle(\vn; \overrightarrow{F_2M})\\
            \Leftrightarrow \frac{\vn \overrightarrow{F_1M}}{|\vn||\overrightarrow{F_1M}|} = \frac{\vn \overrightarrow{F_2M}}{|\vn||\overrightarrow{F_2M}|}
        \end{gather*}
    \end{minipage}
    \begin{gather*}
        \overrightarrow{F_1M} (x_0+c;y_0) \qquad \overrightarrow{F_2M}(x_0-c;y_0)
        \intertext{Вспомним:}
        |\overrightarrow{F_1M}|=a+ex \qquad |\overrightarrow{F_2M}|=a-ex\\
        \frac{\frac{x_0}{a^2}(x_0+c) + \frac{y_0}{b^2}y_0}{a+ex}=
        \frac{\frac{x_0}{a^2}(x_0-c) + \frac{y_0}{b^2}y_0}{a-ex}\\
        \left(\frac{x_0^2}{a^2} + \frac{x_0c}{a^2} + \frac{y_0^2}{b^2}\right)(a-ex)=
        \left(\frac{x_0^2}{a^2} - \frac{x_0c}{a^2} + \frac{y_0^2}{b^2}\right)(a+ex)\\
        \frac{x_0c}{a^2}= \frac{x_0e}{a}\\
        \left(1 + \frac{x_0e}{a}\right)(a-ex)=\left(1 - \frac{x_0e}{a}\right)(a+ex)\\
        \frac{1}{a}(a+x_0e)(a-x_0e)=\frac{1}{a}(a-x_0e)(a+x_0e)
    \end{gather*}
\end{proof}
\begin{lemma}
    \

    \noindent\begin{minipage}{0.45\textwidth}
        \import{figures}{ellipse_lemma.pdf_tex}
    \end{minipage}
    \begin{minipage}{0.45\textwidth}
        \begin{gather*}
            AM+MB \to \min\\
            M_0 \text{ точка, реализующая } \min\\
            A' \text{ -- отражение } A \text{ относительно } l\\
            \min(AM+MB) = \min(A'M+MB)\\
            A'M+MB \ge A'B
        \end{gather*}
    \end{minipage}
\end{lemma}

\begin{proof}
    \

    \begin{center}
        \import{figures}{ellipse_tangent_proof.pdf_tex}
    \end{center}
    \begin{gather*}
        \underbrace{F_1M_0 + F_2M_0}_{=2a} < \underbrace{F_1M + F_2M}_{>2a}\\
        M_0 \text{ -- искомая точка из леммы}
    \end{gather*}
\end{proof}
\end{document}