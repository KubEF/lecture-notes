% !TeX root = ./main.tex
\documentclass[main]{subfiles}
\begin{document}
\chapter{Векторное произведение}
\begin{remark}
    Векторное произведение существует, только если $\dim V = 3$~(т.е. пространство
    трехмерное).
\end{remark}

\begin{definition}[Формальное]
    Пусть $\va, \vb \in V$, $\va \times \vb = \vv$ -- вектор со свойствами:
    \begin{enumerate}
        \item $\vv \perp \va, \vv \perp \vb$
        \item $|\vv| = |\va| |\vb| \sin \alpha$
        \item $(\va, \vb, \va \times \vb)$ -- правая тройка векторов
    \end{enumerate}
\end{definition}

Вопрос: что такое <<правая тройка?>> --- Ответ: нет <<правой>> или <<левой>> троек,
но про любые две тройки мы можем сказать одинаково ли они ориентированы.

\begin{definition}
    Пусть $(\vi, \vj, \vk)$ -- фиксированный ортонормированный базис,
    будем называть его правой тройкой векторов.

    Введем определения:
    \begin{center}
        $\begin{array}{c|rrr}
                    & \vi  & \vj  & \vk  \\ \hline
                \vi & \zv  & \vk  & -\vj \\
                \vj & -\vk & \zv  & \vi  \\
                \vk & \vj  & -\vi & \zv
            \end{array}$
        -- таблица умножения базисных векторов
    \end{center}

    Пусть
    \begin{gather*}
        \va = a_1 \vi + a_2 \vj + a_3 \vk = (a_1, a_2, a_3)\\
        \vb = b_1 \vi + b_2 \vj + b_3 \vk = (b_1, b_2, b_3)
    \end{gather*}
    Тогда векторное произведение $\va$ и $\vb$:
    \begin{multline*}
        \va \times \vb =
        (a_1 \vi + a_2 \vj + a_3 \vk) \times (b_1 \vi + b_2 \vj + b_3 \vk)=\\
        a_1 b_1 \vi \times \vi + a_1 b_2 \vi \times \vj + a_1 b_3 \vi \times \vj+\\
        +a_2 b_1 \vi \times \vi + a_2 b_2 \vi \times \vj + a_2 b_3 \vi \times \vj+\\
        +a_3 b_1 \vi \times \vi + a_3 b_2 \vi \times \vj + a_3 b_3 \vi \times \vj=\\
        = \vi (a_2 b_3 - a_3 b_2)+ \vj (a_3 b_1 - a_1 b_3) + \vk (a_1 b_2 - a_2 b_1)
    \end{multline*}
    \[\va \times \vb = \vi (a_2 b_3 - a_3 b_2)+ \vj (a_3 b_1 - a_1 b_3) + \vk (a_1 b_2 - a_2 b_1)\]
\end{definition}

\begin{theorem}
    Векторное произведение обладает свойствами:
    \begin{enumerate}
        \item $\va \times (\vb + \vc) = \va \times \vb + \va \times \vc$
        \item $\va \times \vb = -\vb \times \va$
        \item $\va \times \vb \perp \va, \va \times \vb \perp \vb$
        \item $|\va \times \vb| = |\va||\vb|\sin\alpha$
    \end{enumerate}
\end{theorem}
\begin{proof}
    \begin{gather*}
        \va = a_1 \vi + a_2 \vj + a_3 \vk\\
        \vb = b_1 \vi + b_2 \vj + b_3 \vk\\
        \vc = c_1 \vi + c_2 \vj + c_3 \vk
    \end{gather*}
    \begin{gather*}
        1. \qquad \vb+\vc = (b_1 + c_1) \vi + (b_2+c_2) \vj + (b_3+c_3)\vk\\
        \va \times (\vb+\vc) = \vi (a_2(b_3+c_3) - a_3 (b_2+c_2))+\vj(...)+\vk(...)\\
        \begin{multlined}
            \va \times \vb + \va \times \vc = \vi (a_2 b_3 - a_3 b_2)+\vj(...)+\vk(...)+\\
            +\vi (a_2 c_3 - a_3 c_2)+\vj(...)+\vk(...)
        \end{multlined}
    \end{gather*}
    После преобразований получим то же самое.

    2. Аналогично
    \begin{multline*}
        3. \qquad (\va\times\vb; \va) = \\
        = (\vi (a_2 b_3 - a_3 b_2)+ \vj (a_3 b_1 - a_1 b_3) + \vk (a_1 b_2 - a_2 b_1);
        a_1 \vi + a_2 \vj + a_3 \vk) = \\
        =a_1 (a_2 b_3 - a_3 b_2)+ a_2 (a_3 b_1 - a_1 b_3) + a_3 (a_1 b_2 - a_2 b_1) = 0
    \end{multline*}

    4. Будем доказывать
    $|\va \times \vb|^2 = |\va|^2 |\vb|^2 \sin^2 \alpha = |\va|^2 |\vb|^2 (1 - \cos^2 \alpha)$
    \begin{multline*}
        (a_2 b_3 - a_3 b_2)^2+(a_3 b_1 - a_1 b_3)^2+(a_1 b_2 - a_2 b_1)^2 =\\
        (a_1^2 + a_2^2 + a_3^2)(b_1^2 + b_2^2 + b_3^2)\left(1 -
        \frac{(a_1 b_1 + a_2 b_2 + a_3 b_3)^2}{(a_1^2 + a_2^2 + a_3^2)(b_1^2 + b_2^2 + b_3^2)}\right)=\\
        (a_1^2 + a_2^2 + a_3^2)(b_1^2 + b_2^2 + b_3^2) - (a_1 b_1 + a_2 b_2 + a_3 b_3)^2
    \end{multline*}
    Чтобы не расписывать слагаемые перепишем в другом виде:
    \[\sum_{i\neq j} a_i^2 b_j^2 - 2 \sum_{i<j} a_i b_i a_j b_j =
        \sum_i a_i^2 b_i^2 + \sum_{i \neq j} a_i^2 b_j^2
        - \sum_{i=1}^3 a_i^2 b_i^2 - 2 \sum_{i<j} a_i b_i a_j b_j\]
\end{proof}

\begin{remark}
    \[\va \times \vb =
        \begin{vmatrix}
            \vi & \vj & \vk \\
            a_1 & a_2 & a_3 \\
            b_1 & b_2 & b_3 \\
        \end{vmatrix}\]
\end{remark}

\begin{definition}[Ориентация]
    Пусть%
    \footnote{Здесь возможно стоит обратиться к \autoref{introduction:deteminants}}
    $\vi, \vj, \vk$ -- ОНБ (<<правая тройка>>), $\va, \vb, \vc$ -- векторы.
    \begin{gather*}
        \va = a_1 \vi + a_2 \vj + a_3 \vk\\
        \vb = b_1 \vi + b_2 \vj + b_3 \vk\\
        \vc = c_1 \vi + c_2 \vj + c_3 \vk
    \end{gather*}

    Если $\begin{vmatrix}
            a_1 & a_2 & a_3 \\
            b_1 & b_2 & b_3 \\
            c_1 & c_2 & c_3
        \end{vmatrix} > 0$, то $(\va, \vb, \vc)$ называется правой тройкой векторов.

    Если $\det < 0$, то $(\va, \vb, \vc)$ называется левой тройкой векторов.

    Если $\det = 0$, то $(\va, \vb, \vc)$ -- ЛЗ.
\end{definition}

Выводы:
\begin{enumerate}
    \item Ориентация бывает только у ЛНЗ троек -- у базисов.
    \item Ориентаций бывает ровно 2.
    \item Одинаковость ориентаций является эквивалентностью.
\end{enumerate}

\begin{remark}
    После этого можно определить $\va \times \vb$ как вектор $\perp \va \perp \vb$
    с длиной $|\va||\vb|\sin\alpha$ и с нужной ориентацией.
\end{remark}
\begin{theorem}
    $(\va, \vb, \va \times\vb)$ -- правая тройка
\end{theorem}
\begin{proof}
    \begin{gather*}
        \va = (a_1, a_2, a_3) \qquad \vb = (b_1, b_2,b_3)\\
        \va \times \vb =(a_2 b_3 - a_3 b_2) + (a_3 b_1 - a_1 b_3) + (a_1b_2 - a_2 b_1) =\\
        =\left(
        \begin{vmatrix}
            a_2 & a_3 \\
            b_2 & b_3
        \end{vmatrix};
        \begin{vmatrix}
            a_3 & a_1 \\
            b_3 & b_1
        \end{vmatrix};
        \begin{vmatrix}
            a_1 & b_1 \\
            a_2 & b_2
        \end{vmatrix}
        \right)
        = \begin{vmatrix}
            a_1             & a_2 & a_3 \\
            b_1             & b_2 & b_3 \\
            \begin{vmatrix}
                a_2 & a_3 \\
                b_2 & b_3
            \end{vmatrix} &
            \begin{vmatrix}
                a_3 & a_1 \\
                b_3 & b_1
            \end{vmatrix} &
            \begin{vmatrix}
                a_1 & a_2 \\
                b_1 & b_2
            \end{vmatrix}
        \end{vmatrix} \\
        = \begin{vmatrix}
            a_2 & a_3 \\
            b_2 & b_3
        \end{vmatrix}
        \left(a_2b_3-a_3b_2\right) -
        \begin{vmatrix}
            a_3 & a_1 \\
            b_3 & b_1
        \end{vmatrix}
        \left(a_1b_3-a_3b_1\right) +
        \begin{vmatrix}
            a_1 & a_2 \\
            b_1 & b_2
        \end{vmatrix}
        \left(a_1b_2-a_2b_1\right)\\
        =(a_2 b_3 - a_3 b_1)^2 + (a_3 b_1 - a_1 b_3)^2 + (a_1b_2 - a_2 b_1)^2 >0
    \end{gather*}
\end{proof}

\section{Геометрический смысл векторного произведения}


\begin{enumerate}
    \item Если нужен вектор $\perp \va \perp \vb$, то $\va \times \vb$ подойдет.
    \item $|\va \times \vb| = S_{\text{параллелограмма}} = |\va||\vb|\sin\alpha$
\end{enumerate}
\end{document}