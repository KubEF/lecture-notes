% !TeX root = ./main.tex
\documentclass[main]{subfiles}
\begin{document}
\chapter{Смешанное произведение}

\begin{definition}
    $\va, \vb, \vc$ -- векторы в $\R^3$
    \[(\va, \vb,\vc) = (\va\times\vb; \vc)\text{ -- смешанное произведение}\]
    Геометрический смысл: $\pm V_{\text{параллелепипеда}}$
\end{definition}
\begin{proof}
    \begin{gather*}
        (\va, \vb, \vc ) = |\va \times \vb| |\vc|\cos\alpha = S_{\va,\vb}|\vc|\cos\alpha=\pm V_{\va, \vb, \vc}
    \end{gather*}
\end{proof}

В координатах:
\begin{multline*}
    (\va \times \vb; \vc) = (a_2 b_3 - a_3 b_2; a_3 b_1 - a_1 b_3; a_1b_2 - a_2 b_1)(c_1, c_2, c_3)=\\
    a_2 b_3 c_1 - a_3 b_2 c_1  + a_3 b_1 c_2 - a_1 b_3 c_2 + a_1 b_2 c_3 - a_2 b_1 c_3 =
    \begin{vmatrix}
        a_1 & a_2 & a_3 \\
        b_1 & b_2 & b_3 \\
        c_1 & c_2 & c_3
    \end{vmatrix}
\end{multline*}

\section{Свойства}
\begin{enumerate}
    \item $(\ve+\vf, \vb, \vc) = (\ve, \vb, \vc) + (\vf, \vb, \vc)$ для каждого аргумента
    \item $(\alpha \va, \vb, \vc) = (\va, \alpha \vb, \vc) =(\va, \vb, \alpha\vc) = \alpha (\va, \vb, \vc)$
    \item $(\va, \vb, \vc) = 0 \Leftrightarrow \va, \vb, \vc$ -- ЛЗ
    \item $(\va, \vb, \vc) = (\vb,\vc, \va) = (\vc, \va, \vb) = - (\vb, \va, \vc)=
              -(\va, \vc, \vb) = - (\vc, \vb, \va)$
    \item Знак смешанного произведения -- ориентация тройки.
\end{enumerate}
\end{document}