% !TeX root = ./main.tex
\documentclass[main]{subfiles}
\begin{document}
\chapter{Векторные пространства}
\section{Понятие векторного пространства}
\begin{definition}
    Множество $V$ с двумя операциями:
    \marginpar{\raggedright Элементы $a,b,c$ -- векторы (иногда $\overrightarrow{a}$ или $\overline{a}$). \\ Элементы $\alpha, \beta$ -- скаляры.}
    $+:V\times V \to V; \quad (a,b) \mapsto a+b$ и
    $\cdot : \R \times V \to V$ называется векторным пространством (над $\R$),
    если при условии $\forall a,b,c \in V; \forall \alpha, \beta \in \R$, выполнены следующие свойства:

    \begin{enumerate}
        \item $a+(b+c) = (a+b)+c$ --- ассоциативность
        \item $a+b=b+a$ --- коммутативность
        \item $\exists \zv:\forall a \quad \zv+a=a+\zv=a$
        \item $\forall a \exists (-a): a+(-a)=\zv$
              \marginpar{\raggedright Если выполнены свойства 1--4, то $V$ называется коммутативной (абелевой) группой.}
        \item $\alpha(a+b)=\alpha a + \alpha b$ ---дистрибутивность \begin{proof}
                  \begin{gather*}
                      \alpha (a+b) = \alpha a + \alpha b \\
                      a = (x_1, y_1) \quad b = (x_2, y_2)
                  \end{gather*}
                  \begin{multline*}
                      \alpha(a+b)=\alpha((x_1, y_1)+(x_2, y_2)) = \alpha (x_1+x_2, y_1+y_2)=\\
                      (\alpha(x_1+x_2), \alpha(y_1+y_2))=(\alpha x_1 + \alpha x_2, \alpha y_1 + \alpha y_2)=\\
                      (\alpha x_1, \alpha y_1)+(\alpha x_2, \alpha y_2) = \alpha (x_1, y_1) + \alpha (x_2, y_2)= \\
                      \alpha a + \alpha b
                  \end{multline*}
              \end{proof}
        \item $(\alpha+\beta)a = \alpha a + \beta a$ --- дистрибутивность
        \item $(\alpha \cdot \beta)a = \alpha(\beta a)$ --- ассоциативность
        \item $1\cdot a = a, \forall a \in \R$
    \end{enumerate}
\end{definition}

\subsection{Свойства векторного пространства}

\begin{enumerate}
    \marginpar{$\zv$ --- векторный ноль}
    \item $\zv$ --- единственный \begin{proof}
              $\zv_1 = \zv_1 + \zv_2 = \zv_2$
          \end{proof}
    \item $-a$ --- единственный \begin{proof}
              Пусть $b_1, b_2$ -- противоположные к $a$

              $b_1 + a = \zv \quad b_2 + a = \zv$

              $b_1 = b_1 + \zv = b_1 + (a + b_2) = (b_1+a)+b_2 = \zv +b_2 = b_2$
          \end{proof}
    \item $\zv\cdot a = \zv$ \begin{proof}
              !!!
          \end{proof}
    \item $-1\cdot a = -a$ \begin{proof}
              !!!
          \end{proof}
\end{enumerate}

\subsection{Примеры векторных пространств}
\begin{enumerate}
    \item Координатная плоскость $\{(x,y)|x,y\in \R\}$
    \item Координатное трехмерное пространство $\{(x,y,z)|x,y,z\in \R\}$
    \item Строки длины $n$ из вещественных чисел \\
          $V = \{(x_1, x_2, ..., x_n)|x_i \in \R\}$ или матрицы (2d массивы)
\end{enumerate}

\section{Операции над векторами}
\begin{equation*}
    a = (x_1, y_1) \quad b = (x_2, y_2)
\end{equation*}
\subsection{Сложение}
\begin{equation*}
    a + b = (x_1+x_2, y_1+y_2)
\end{equation*}

\subsection{Умножение вектора на число}
\marginpar{\raggedright Свойства 1--8, очевидно выполняются}
\begin{equation*}
    \alpha a = (\alpha\cdot x_1, \alpha\cdot y_1)
\end{equation*}

\section{Линейные комбинация и (не)зависимость}
\begin{definition}
    $V$ - векторное пространство и векторы \\ $v_1,v_2,v_3,..., v_n \in V$.
    Система $v_1,...,v_n$ называется линейно независимой (ЛНЗ), если из
    $\alpha_1 v_1 + \alpha_2 v_2 + ... + \alpha_n v_n = 0 \implies \alpha_1=\alpha_2=...=\alpha_n =0$.
\end{definition}

\begin{definition}
    Если $\alpha_1,..., \alpha_n \in \R$, $v_1,...,v_n \in V$.
    То $\alpha_1 v_1 + \alpha_2 v_2 + ... + \alpha_n v_n$ -- линейная комбинация (ЛК)
    векторов $v_1,...,v_n$.
\end{definition}

\begin{definition}
    Если $\exists \alpha_1,..., \alpha_n$, не все $=0$, но $\alpha_1 v_1 + \alpha_2 v_2 + ... + \alpha_n v_n = 0$,
    то система $v_1,...,v_n$ называется линейно зависимой (ЛЗ).
\end{definition}

\begin{assertion}
    \marginpar{$\Leftrightarrow$ -- тогда и только тогда}
    $v_1,...,v_n$ -- ЛЗ $\Leftrightarrow$ один из этих векторов можно представить как ЛК остальных.
    $\exists i: v_i= \alpha_1 v_1 +\alpha_2 v_2 + ... + \alpha_{i-1} v_{i-1} + \alpha_{i+1} v_{i+1} + ... +\alpha_n v_n$
    \begin{proof}
        $\Rightarrow : \exists \alpha_1,...,\alpha_n (\exists i: \alpha_i \neq 0)$
        \begin{gather*}
            \alpha_1 v_1 + \alpha_2 v_2 + ... + \alpha_n v_n = 0\\
            \alpha_i v_i = - \alpha_1 v_1 - \alpha_2 v_2 - ... - \alpha_{i-1} v_{i-1} - \alpha_{i+1} v_{i+1} - ... -\alpha_n v_n\\
            \alpha_i \neq 0 \quad v_i = -\frac{\alpha_1}{\alpha_i}v_1 -... - \frac{\alpha_n}{\alpha_i}v_n
        \end{gather*}

        \begin{gather*}
            \Leftarrow: v_i = \alpha_1 v_1 + ... + \alpha_n v_n \text{ без } i\text{-ого слагаемого}\\
            \alpha_1 v_1 + \alpha_2 v_2 + ... + \mathbf{(-1)}v_i + ... + \alpha_n v_n =0\\
            \text{ЛК } = 0 \text{ не все коэффициенты} = 0
        \end{gather*}
    \end{proof}
\end{assertion}

\begin{prop}
    $v_1,...,v_n $ -- ЛНЗ, то любой его поднабор тоже ЛНЗ.\\
    $v_1,...,v_n $ -- ЛЗ, то при добавлении векторов, набор останется ЛЗ.
\end{prop}


\begin{assertion}
    $v_1,..., v_n$ -- ЛНЗ $\Leftrightarrow$ если
    \begin{gather*}
        \alpha_1 v_1 + ... + \alpha_n v_n = \beta_1 v_1 + ... + \beta_n v_n\\
        \Rightarrow \alpha_1 = \beta_1; \alpha_2 = \beta_2; ... ; \alpha_n = \beta_n
    \end{gather*}
\end{assertion}
\begin{proof}
    \begin{gather*}
        (\alpha_1 - \beta_1)v_1 + (\alpha_2 - \beta_2)v_2 + ...
        + (\alpha_n - \beta_n)v_n = \zv\\
        \alpha_i - \beta_i = 0 \Leftrightarrow v_1, ..., v_n\text{-- ЛНЗ}
    \end{gather*}
\end{proof}
\end{document}