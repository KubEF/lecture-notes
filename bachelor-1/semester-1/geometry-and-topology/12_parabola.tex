% !TeX root = ./main.tex
\documentclass[main]{subfiles}
\begin{document}
\chapter{Парабола}
\begin{definition}
    Парабола -- кривая, которая в подходящих координатах имеет уравнение:
    \[y^2=2px\]
    где $p$ -- параметр параболы
\end{definition}
\begin{definition}
    Пусть $F$ -- точка, $l$ -- прямая, тогда парабола -- ГМТ $M$:
    \[\frac{FM}{\dist(M;l)} = e = 1\]
\end{definition}
\begin{theorem}
    Определения равносильны
\end{theorem}
\begin{proof}
    \begin{gather*}
        \left(x + \frac{p}{2}\right) = \sqrt{\left(x-\frac{p}{2}\right)^2 + y^2}\\
        \left(x + \frac{p}{2}\right)^2 = \left(x-\frac{p}{2}\right)^2 + y^2\\
        y^2 = 2px
    \end{gather*}
\end{proof}

Характеристики параболы
\begin{itemize}
    \item $F$ -- фокус
    \item $l$ -- директриса
    \item $e=1$ -- эксцентриситет
\end{itemize}

\begin{theorem}
    $(x_0, y_0)$ -- точка на параболе $y^2=2px$, тогда
    \[yy_0 = p (x+x_0)\]
    -- уравнение касательной в $(x_0, y_0)$
\end{theorem}
\begin{proof}
    \begin{gather*}
        px = yy_0 - px_0\\
        y^2 = 2px = 2yy_0-2px_0\\
        y^2 - 2yy_0 + 2px_0 = 0\\
        \frac{D}{4} = y_0^2 - 2px_0 = 0
    \end{gather*}
    1 решение
\end{proof}
\begin{theorem}[Оптическое свойство параболы]

\end{theorem}

\end{document}