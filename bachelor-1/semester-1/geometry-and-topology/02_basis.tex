% !TeX root = ./main.tex
\documentclass[main]{subfiles}
\begin{document}
\chapter[Базис \texorpdfstring{$V$}{V}]{Базис векторного пространства}
\begin{definition}
    Набор $v_1, v_2, ..., v_n$ называется порождающим для $V$,
    если $\forall w \in V \exists \alpha_1,..., \alpha_n : w =
        \alpha_1 v_1 + \alpha_2 v_2 + ... + \alpha_n v_n$
\end{definition}
\begin{prop}
    Если к порождающему набору прибавить вектор, то он останется порождающим.
    Если убрать векторы из непорождающего набора векторы, то набор останется непорождающим.
\end{prop}

\begin{definition}
    $v_1, v_2, ..., v_n$ называется базисом $V$, если этот набор ЛНЗ и порождающий.
\end{definition}
\begin{theorem}[О базисе]
    Следующие определения базиса равносильны:
    \begin{enumerate}
        \item ЛНЗ и  порождающий набор
        \item Минимальный порождающий набор (минимальный по включениям)
        \item Максимальный ЛНЗ набор (максимальный по включениям)
        \item Набор $\forall w \in V \exists! \alpha_1, ..., \alpha_2:
                  w = \alpha_1 v_1 + \alpha_2 v_2 + ... + \alpha_n v_n$
    \end{enumerate}
\end{theorem}
\begin{proof}
    Цепочка доказательств: $1 \to 2 \to 4 \to 3 \to 1$ (цикл)

    $1 \to 2$. Дан $v_1, ..., v_n$ -- ЛНЗ и порождающий набор.
    Доказать, что он минимальный порождающий.

    Допустим, что $v_i$ выкинули, оставшийся набор остался порождающим
    $\Rightarrow v_i$ -- ЛК остальных $\Rightarrow$ ЛЗ $\qquad \bot$.

    $2 \to 4$. Дан $v_1, ..., v_n$ -- минимальный порождающий набор.
    Доказать $v_1, ..., v_n$ -- порождающий с единственностью коэффициентов.

    Допустим противное: $\alpha_1 v_1 + \alpha_2 v_2 + ... + \alpha_n v_n =
        \beta_1 v_1 + ... + \beta_n v_n$
    \begin{gather*}
        \alpha_1 \neq \beta_1\\
        (\alpha_i - \beta_i)v_i = (\beta_1 - \alpha_1) v_1 + ...\text{ (без $i$-ого)}
        + (\beta_n - \alpha_n)v_n\\
        v_i = \frac{\beta_1 - \alpha_1}{\alpha_i - \beta_i} + ...\text{ (без $i$-ого)}
        + \frac{\beta_n - \alpha_n}{\alpha_i - \beta_i}
    \end{gather*}
    $v_i$ -- выкинем. В любой ЛК с $v_i$ заменим $v_i$ на выражение выше $\implies$
    набор порождающий

    $4 \to 3$. Дан $v_1, ..., v_n$ -- порождающий набор с единственностью коэффициентов.
    Доказать: $v_1, ..., v_n$ -- минимальный ЛНЗ (ЛНЗ уже доказана)

    Допустим противное: $v_1, v_2, ..., v_n; u$ --  ЛНЗ набор
    \begin{gather*}
        u= \alpha_1 v_1 + ... + \alpha_n v_n (\alpha_1, ... \alpha_n \exists!)\\
        \implies v_1, ..., v_n, u \text{ -- ЛЗ } \qquad \bot
    \end{gather*}

    $3 \to 1$. Дан $v_1, ..., v_n$ -- минимальный ЛНЗ.
    Доказать $v_1, ..., v_n$ -- ЛНЗ и порождающий набор.

    \begin{align*}
        \forall w \in V    &                 &  & v_1, v_2, ..., v_n, w \text{ -- ЛЗ набор}         \\
                           &                 &  & \alpha_1 v_1 + \alpha_2 v_2 + ...
        + \alpha_n v_n +\beta w = \zv                                                               \\
        \text{Если } \beta & = 0 \implies    &  & \alpha_1 v_1 + ... + \alpha_n v_n =\zv            \\
                           &                 &  & \text{не все коэффициенты } = 0 (\alpha_i \neq 0) \\
                           &                 &  & \implies v_1,..., v_n \text{ -- ЛЗ}               \\
        \beta              & \neq 0 \implies &  & w = - \frac{\alpha_1}{\beta} v_1
        - \frac{\alpha_2}{\beta} v_2 - ...  - \frac{\alpha_n}{\beta} v_n
    \end{align*}
\end{proof}

\begin{remark}
    Любую конечную порождающую систему можно сузить до базиса.
\end{remark}
\begin{remark}
    Если есть конечный порождающий набор, то любую ЛНЗ систему можно расширить
    до базиса.
\end{remark}

\begin{definition}
    Размерность пространства равна количеству элементов в базисе.
    (пока нет доказательств корректности)
\end{definition}

\begin{lemma}
    Система линейных уравнений: $(a_{ij} \in \R; x_i \in \R; 0 \in \R)$
    \begin{equation*}
        \begin{cases}
            a_{11} x_1 + a_{12}x_2 + ... + a_{1n} x_n = 0 \\
            a_{21} x_1 + a_{22}x_2 + ... + a_{2n} x_n = 0 \\
            ...                                           \\
            a_{k1} x_1 + a_{k2}x_2 + ... + a_{kn} x_n = 0 \\
        \end{cases}
    \end{equation*}
    Имеет ненулевые решения, если $n>k$.
\end{lemma}
\begin{proof}
    Индукция по $k$.

    База $k=1$:
    \begin{gather*}
        a_{11} x_1 + a_{12}x_2 + ... + a_{1n} x_n = 0\\
        \text{Пусть } a_{11} \neq 0 \implies x_1 = -\frac{a_{12}}{a_{11}}x_2
        -\frac{a_{13}}{a_{11}}x_3 - ... - -\frac{a_{1n}}{a_{11}}x_n\\
        \forall x_2,...,x_n: x_1 \text{ выражается через них}\\
        a_{11} = 0 \implies x_1 = 1; x_2=x_3=...=x_n=0
    \end{gather*}

    Переход
    \begin{gather*}
        a_{11} x_1 + a_{12}x_2 + ... + a_{1n} x_n = 0\\
        \exists i: a_{1i} \neq 0 \text{, иначе выкинем предыдущее уравнение}\\
        x_i = -\frac{a_{11}}{a_{1i}}x_1 - ...\text{ (без $i$-ого)} - -\frac{a_{1n}}{a_{1i}}x_n
    \end{gather*}
    Подставим выраженное $x_i$ во все остальные уравнения.
    Уравнений на 1 меньше, переменных на 1 меньше.
\end{proof}
\begin{example}
    \begin{equation*}
        \begin{cases}
            x+y+z=0 \\
            z+y-z=0
        \end{cases}
        \implies z=0 \qquad x+y=0
    \end{equation*}
\end{example}

\begin{theorem}
    Если $v_1, ..., v_k$ и $w_1, ...,w_n$ базисы $\in V$, то $k=n$.
\end{theorem}
\begin{proof}
    $v_1, ..., v_n$ -- порождающая система.
    \begin{gather*}
        w_1 = a_{11} v_1 + a_{21} v_2  + a_{31} v_3 + ... + a_{k1} v_k \\
        w_2 = a_{12} v_1 + a_{22} v_2  + a_{32} v_3 + ... + a_{k2} v_k \\
        ...                                                            \\
        w_n = a_{1n} v_1 + a_{2n} v_2  + a_{3n} v_3 + ... + a_{kn} v_k \\
    \end{gather*}
    \begin{equation}\tag{\star}\label{eq:star}
        x_1 w_1 + x_2 w_2 + ... + x_n w_n = \zv, x_i \in \R
    \end{equation}
    т.к. $w_1, ..., w_n$ -- ЛНЗ $\implies$ все $x_i=0$
    \begin{multline*}
        x_1 (a_{11} v_1 + a_{21} v_2 + ... + a_{k1} v_k) +
        x_2 (a_{12} v_1 + a_{22} v_2 + ... + a_{k2} v_k) \\
        + ... +
        x_n (a_{1n} v_1 + a_{2n} v_2 + ... + a_{kn} v_k) = \zv
    \end{multline*}
    \begin{multline*}
        v_1 (a_{11} x_1 + a_{12} x_2 + ... + a_{1n} x_n ) +
        v_2 (a_{21} x_1 + a_{22} x_2 + ... + a_{2n} x_n ) \\
        + ... +
        v_k (a_{k1} x_1 + a_{k2} x_2 + ... + a_{kn} x_n ) = \zv
    \end{multline*}
    $v_1, v_2, ..., v_k$ -- ЛНЗ $\implies$ все коэффициенты равны 0.
    \begin{equation*}
        \begin{cases}
            a_{11} x_1 + a_{12} x_2 + ... + a_{1n}x_n = 0 \\
            a_{21} x_1 + a_{22} x_2 + ... + a_{2n}x_n = 0 \\
            ...                                           \\
            a_{k1} x_1 + a_{k2} x_2 + ... + a_{kn}x_n = 0
        \end{cases}
    \end{equation*}
    Если $n>k \implies \exists$ ненулевые решения $\implies$ противоречие
    с \eqref{eq:star} и ЛНЗ $w_i \implies n \le k$. Аналогично $k \le n \implies
        n=k$.
\end{proof}

Если $\exists$ хотя бы один конечный базис, то все базисы будут равномощными.

\section{Координаты вектора в базисе}
Пусть $v_1, v_2, ..., v_n$ -- базис.
\[\forall w \in V \implies \exists ! \alpha_1, \alpha_2, ..., \alpha_n:
    w = \alpha_1 v_1 + ... + \alpha_n v_n\]
$w = (\alpha_1, \alpha_2, ..., \alpha_n)$ -- координаты $w$ в базисе $\{v_i\}_{i=1}^n$
\begin{gather*}
    v_1 = (1, 0, ..., 0)\\
    v_2 = (0, 1, ..., 0)\\
    ...\\
    v_n = (0, 0, ..., 1)
\end{gather*}
\end{document}