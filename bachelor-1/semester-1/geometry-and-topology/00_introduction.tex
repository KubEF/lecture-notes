% !TeX root = ./main.tex
\documentclass[main]{subfiles}
\begin{document}
\part{Векторные пространства}
\chapter{Введение}
\section{Множества}
\begin{definition}
    Множество --- неопределяемое понятие.
\end{definition}

A, B --- множества

$A \cup B = \{x:x \in A \text{ или } x \in B\}$ --- объединение

$A \cap B = \{x:x \in A \text{ и } x \in B\}$ --- пересечение

$A\setminus B = \{x:x \in A \text{ or } x \not\in B\}$ --- разность

$A \bigtriangleup B = (A \setminus B) \cup (B \setminus A)$ --- симметрическая разность

$A \times B = \{(x,y):x \in A; y \in B\}$ --- декартово произведение множеств

\subsection{Примеры декартового произведения множеств}

\begin{enumerate}
    \item Координатная плоскость $\R \times \R$
    \item Множество полей шахматной доски $\{A,B,C,D,E,F,G,H\}\times \{1,2,3,4,5,6,7,8\}$
    \item Колода карт $\{\text{масти}\} \times \{\text{достоинства}\}$
    \item Нумерация мест в театре
    \item Нумерация аудиторий на ММ
\end{enumerate}

\section{Отображения}
\begin{definition}
    Пусть A, B --- множества.
    Говорим, что задано отображение $f: A \to B$,
    если задано правило, сопоставляющее каждому $x \in A$ ровно один $y \in B$.\\
    Пишем: $y = f(x)$.
\end{definition}

\begin{example}
    $f(x) = \frac{1}{x}$ --- не отображение, т.к. $f(0) \not\exists$.\\
    Однако при $\R \setminus \{0\} \to \R$ такое отображение существует.
\end{example}
\begin{example}
    \begin{gather*}
        +:\R \times \R  \to \R \\
        (x,y)           \mapsto x+ y
    \end{gather*}
    Любая операция является отображением
\end{example}
\begin{example}
    $A \subset B \quad i:A\to B \quad i(a)=a$\\
    $ A \hookrightarrow B$ --- отображение включения\\
    $\id:A \to A \quad  \id(x)=A$ --- тождественное отображение
\end{example}


\section{Отношения эквивалентности}
\begin{definition}
    $M$ -- множество, $\mu \subset M \times M \implies \mu$ называется
    отношением над $M$.

    $\forall a,b \in M$ два случая
    \begin{enumerate}
        \item $(a,b) \in \mu$ пишем $a \mu b$
        \item $(a,b) \not\in \mu$ пишем $a \not\mu b$
    \end{enumerate}
\end{definition}
\begin{example}
    $=, <,\le , >, \ge, \vdots$

    $\subset$ тоже отношение, но только на множестве некоторых множеств.

    Если $M$ -- множество людей, то слова <<отец>>, <<мать>>, <<муж>>, <<жена>> и т.д.
\end{example}

\begin{definition}
    Отношение $\mu$ называется рефлексивным, если \[\forall a: a\mu a\]
\end{definition}
\begin{definition}
    Отношение $\mu$ называется симметричным, если \[a \mu b \implies b \mu a\]
\end{definition}
\begin{definition}
    Отношение $\mu$ называется транзитивным, если \[\begin{rcases}
            a \mu b \\
            b \mu c
        \end{rcases}\implies a\mu c\]
\end{definition}

\begin{definition}
    Отношение называется отношением эквивалентности, если оно рефлексивно,
    симметрично и транзитивно. Обозначение:$\sim$.
\end{definition}

\begin{definition}
    Если $a \in M, K_a = \{b: a \sim b\}$ -- класс эквивалентности.
\end{definition}

\begin{theorem}
    $K_a = K_b$ либо $K_a \cap K_b = \emptyset$
\end{theorem}
\begin{proof}
    Допустим противоречие, тогда $\exists c \in K_a \cap K_b,
        \exists d \in K_a \setminus K_b$ (или $\in K_b \setminus K_a$)
    \begin{gather*}
        a\sim c; b\sim c \qquad a \sim d; b \not\sim d\\
        \begin{rcases}
            a\sim c \\
            c\sim b
        \end{rcases} \implies a\sim b \qquad
        \begin{rcases}
            d \sim a \\
            a \sim b
        \end{rcases} \implies d \sim b \qquad\contradiction
    \end{gather*}
\end{proof}

\begin{definition}
    Множество классов эквивалентности называется фактор-множество. Обозначается $M/\sim$
\end{definition}

\section[Определители]{Определители $2\times2$ и $3 \times 3$}\label{introduction:deteminants}
\begin{definition}[Неформальное]
    На плоскости: $\va = (a_1, a_2); \vb = (b_1, b_2)$
    \[\begin{vmatrix}
            a_1 & a_2 \\
            b_1 & b_2
        \end{vmatrix}
        =S_{\va, \vb}\text{ (ориентированная площадь)}\]

    В пространстве: $\va = (a_1, a_2, a_3); \vb = (b_1, b_2, b_3); \vc = (c_1, c_2, c_3)$
    \[\begin{vmatrix}
            a_1 & a_2 & a_3 \\
            b_1 & b_2 & b_3 \\
            c_1 & c_2 & c_3
        \end{vmatrix}
        = V_{\va, \vb, \vc}\text{ (ориентированный объем)}\]
\end{definition}
\begin{definition}[Формальное]
    \[\begin{vmatrix}
            a_1 & a_2 \\
            b_1 & b_2
        \end{vmatrix}
        =a_1 b_2-a_2 b_1\]
    \[\begin{vmatrix}
            a_1 & a_2 & a_3 \\
            b_1 & b_2 & b_3 \\
            c_1 & c_2 & c_3
        \end{vmatrix}
        =a_1 b_2 c_3 + a_2 b_3 c_1 + a_3 b_1 c_2 - a_1 b_3 c_2 - a_2 b_1 c_3 - a_3 b_2 c_1\]
\end{definition}

Мнемоническое правило:

\begin{tikzpicture}
    \matrix (m) [matrix of math nodes,left delimiter={|},right delimiter={|}]{
        a_1 & a_2 & a_3 & a_1 & a_2 \\
        b_1 & b_2 & b_3 & b_1 & b_2 \\
        c_1 & c_2 & c_3 & c_1 & c_2 \\
    };
    \draw [-,red] (m-1-3.north east) -- (m-3-3.south east);
    \draw [-Latex,green] (m-1-5.north east) -- (m-3-3.south west);
    \draw [-Latex,green] (m-1-4.north east) -- (m-3-2.south west);
    \draw [-Latex,green] (m-1-3.north east) -- (m-3-1.south west);
    \draw [-Latex,cyan] (m-1-1.north west) -- (m-3-3.south east);
    \draw [-Latex,cyan] (m-1-2.north west) -- (m-3-4.south east);
    \draw [-Latex,cyan] (m-1-3.north west) -- (m-3-5.south east);
\end{tikzpicture}

По бирюзовой стрелке сложение, по зеленой -- вычитание.

\subsection{Свойства}
\begin{remark}
    Данные свойства справедливы для матриц любого порядка.
\end{remark}
\begin{enumerate}
    \item Если  строку или столбец умножить на $\alpha$, то определитель тоже
          умножится на $\alpha$.
    \item Если меняем 2 строки или столбца, то знак определителя меняется.
    \item Если есть 2 одинаковых строки, то определитель равен 0.
    \item Если к одному из векторов прибавить вектор кратный другому,
          то определитель не поменяется
\end{enumerate}
\begin{multline*}
    \begin{vmatrix}
        a_1 + \alpha b_1 & a_2+ \alpha b_2 & a_3 + \alpha b_3 \\
        b_1              & b_2             & b_3              \\
        c_1              & c_2             & c_3
    \end{vmatrix}=\\
    =\begin{vmatrix}
        a_1 & a_2 & a_3 \\
        b_1 & b_2 & b_3 \\
        c_1 & c_2 & c_3
    \end{vmatrix}+ \alpha
    \begin{vmatrix}
        b_1 & b_2 & b_3 \\
        b_1 & b_2 & b_3 \\
        c_1 & c_2 & c_3
    \end{vmatrix}=
    \begin{vmatrix}
        a_1 & a_2 & a_3 \\
        b_1 & b_2 & b_3 \\
        c_1 & c_2 & c_3
    \end{vmatrix}
\end{multline*}
\begin{assertion}
    Эти свойства и $\left| \begin{smallmatrix}
            1 & 0 &0\\
            0& 1& 0\\
            0&0&1
        \end{smallmatrix} \right| = 1$
    полностью определяют функцию объема
\end{assertion}

\end{document}