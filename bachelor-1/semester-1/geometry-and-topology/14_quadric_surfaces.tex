% !TeX root = ./main.tex
\documentclass[main]{subfiles}
\begin{document}
\chapter{Поверхности II порядка}
\begin{multline*}
    a_{11} x^2 + 2 a_{12} xy + 2 a_{13} xz + 2 a_{23} yz + a_{22} y^2 + a_{33}z^2 +\\
    2b_1x +2 b_2y + 2b_3z + b_4 = 0
\end{multline*}
\begin{theorem}
    С помощью вращений можно привести уравнение к виду.
    \[a_{11} x'^2 + a_{22} y'^2 + a_{33}z'^2 + 2b_1'x +2 b_2'y + 2b_3'z + b_4' = 0\]
    Штрихи снимаются для удобства
\end{theorem}

\section{Виды поверхностей}
\subsection{Эллиптический тип}
\[a_{11} > 0 \qquad a_{22} > 0 \qquad a_{33} > 0\]
Если все $<0$, то умножим на -1

\begin{lemma}
    Если $a_{11} \neq 0$, то считаем $b_1 = 0$
    \[a_{11} x^2 + 2b_1 x = a_{11} \left(x + \frac{b_1}{a_{11}}\right) - \frac{b_1^2}{a_{11}}\]
\end{lemma}

Получили уравнение:
\[a_{11}x^2 + a_{22}y^2 + a_{33}z^2 + b_4 = 0\]
\begin{enumerate}
    \item Эллипсоид
          \[\frac{x^2}{a^2} + \frac{y^2}{b^2} + \frac{z^2}{c^2} = 1\]
    \item Точка
          \[\frac{x^2}{a^2} + \frac{y^2}{b^2} + \frac{z^2}{c^2} = 0\]
    \item Мнимая эллипсоид
          \[\frac{x^2}{a^2} + \frac{y^2}{b^2} + \frac{z^2}{c^2} = -1\]
\end{enumerate}

\subsection{Гиперболический тип}
\[a_{11} \neq 0 \qquad a_{22} \neq 0 \qquad a_{33} \neq 0\]
Все НЕ одного знака, считаем, что $a_{11} >0$
\begin{enumerate}
    \setcounter{enumi}{3}
    \item Однополостный гиперболоид
          \begin{gather*}
              \frac{x^2}{a^2} + \frac{y^2}{b^2} - \frac{z^2}{c^2} = 1\\
              z = const \implies \frac{x^2}{a^2} + \frac{y^2}{b^2} = 1 + \frac{z^2}{c^2}\\
              y = const \implies \frac{x^2}{a^2} + \frac{z^2}{c^2} = 1 - \frac{y_2}{b_2}\\
              |y|=b \implies \frac{x^2}{a^2} - \frac{z^2}{b^2} = 0 \text{ -- пара пересекающихся прямых}
          \end{gather*}
    \item Двуполостный гиперболоид
          \begin{gather*}
              \frac{x^2}{a^2} + \frac{y^2}{b^2} - \frac{z^2}{c^2} = -1\\
              z = const \implies \frac{x^2}{a^2} + \frac{y^2}{b^2} = \frac{z^2}{c^2} -1\\
              |z| < c \implies \emptyset\\
              |z| = c \implies \text{ точка}\\
              |z| > c \implies \text{ эллипс}\\
              y = const \implies \frac{x^2}{a^2} - \frac{z^2}{c^2} = -1 - \frac{y^2}{b^2}\text{ -- гипербола}
          \end{gather*}
    \item Конус
          \begin{gather*}
              \frac{x^2}{a^2} + \frac{y^2}{b^2} - \frac{z^2}{c^2} = 0\\
              (x,y,z) \in \text{ конусу, то}\\
              (\alpha x, \alpha y, \alpha z) \in \text{ конусу}
          \end{gather*}
\end{enumerate}

\subsection{Параболический случай}
\[a_{33} = 0\]
\begin{lemma}
    Если $a_{33} = 0, b_3 \neq 0$,то считаем $b_4=0$
\end{lemma}
\[b_3 \neq 0 \quad a_{11} \neq 0 \quad a_{22} \neq 0\]
\begin{enumerate}
    \setcounter{enumi}{6}
    \item Эллиптический параболоид
          \b
    \item Гиперболический параболоид (седло)
          \[\frac{x^2}{a_2} - \frac{y^2}{b^2} = 2z\]
    \item
\end{enumerate}

\section{}
\begin{multline*}
    a_{11} x^2 + a_{22} y^2 + a_{33} z^2 + 2 a_{12} xy + 2 a_{13} xz + 2 a_{23} yz = f(x,y,z)
\end{multline*}

Рассмотри значения $f(x)$ на $S^2 = \{x^2 + y^2 + z^2 =1\}$:
\[f(\alpha x; \alpha y; \alpha z) = \alpha^2 f(x,y,z)\]
Пусть $M \in S^2$ -- точка, в которой $f(x,y,z)$ принимает $\max$ значение (почему $\exists M$?)
Через $M$ проведем OX $(x,y,z)$ новые координаты

\begin{gather*}
    f(x,y,z) = a_{11} x^2 + a_{22} y^2 + a_{33} z^2 + 2 a_{12} xy + 2 a_{13} xz + 2 a_{23} yz \\
    M(1,0,0) \max
    \intertext{В окрестности $M$ уравнение сферы}
    x = \sqrt{1-y^2-z^2}\\
    f(x,y,z) = a_{11} (1 - y^2 -z^2) + a_{22} y^2 + a_{33} z^2 \\+ 2 a_{12} y\sqrt{1-y^2-z^2} + 2 a_{13} z\sqrt{1-y^2-z^2} + 2 a_{23} yz\\
\end{gather*}

\end{document}
