% !TeX root = ./main.tex
\documentclass[main]{subfiles}
\begin{document}
\chapter{Вещественные числа}
\section{Обозначения и нотация}

В дальнейшем множество будем понимать как совокупность объектов, называемых
его элементами. Приведенное высказывание не является определением, однако в
дальнейшем при операциях с конкретными множествами, математический контекст
рассматриваемые множества определяет.

Если $a,b$ -- некие элементы, $A$ -- множество, то запись $a\in A$ означает,
что $a$ принадлежит множеству $A$; запись $b \not\in A$ означает,
что элемент $b$ не принадлежит множеству $A$.

Символ $\forall$ означает высказывание <<для всякого>>, далее всегда будет следовать
текст конкретизирующий это высказывание.

Символ $\exists$ означает высказывание <<существует>> и также будет задан
математическим контекстом.

Запись $A \Rightarrow B$ или $B \Leftarrow A$ означает <<из $A$ следует $B$>>;
запись $A \Leftrightarrow B$ означает <<$A$ эквивалентно $B$>>.

Множества $A$ и $B$ называются совпадающими, что записывают формулой $A=B$,
если $(\forall a \in A) \Rightarrow (a \in B)$ и
$(\forall b \in B) \Rightarrow (b \in A)$;
приведенная формальная запись означает, что $A=B$ в том и только в том случае,
когда они состоят из одних и тех же элементов.

Если множества $A$ и $B$ не совпадают, то пишут $A \neq B$.

Определяют также пустое множество, в котором нет элементов,
которое будем обозначать символом $\varnothing$.

Запись $A \subset B$ читается <<$A$ содержится в $B$>> и означает, что
$(\forall a \in A) \Rightarrow (a \in B)$. Полагаем, что $\varnothing \subset A$
для любого множества $A$. Понятно, что
\[A = B \Leftrightarrow (A \subset B) \text{ и } (B \subset A).\]

В дальнейшем при рассмотрении сразу нескольких множеств в качестве синонима
слова <<множество>> будем использовать слова <<семейство>>, <<класс>>,
<<совокупность>>.

\section{Операции над множествами}

Объединением $ A \cup B$ множеств $A$ и $B$ будем называть множество:
\[(a \in A \cup B) \Leftrightarrow (a \in A) \text{ или } (a \in B).\]

Если множество $A$ задается каким-то условием, обозначим его <<условие>>,
то для задания множества $A$ будем использовать обозначение
\[A=\{a:\text{<<условие>> на } a\}\]
\begin{case}
    \[A_1 \cup A_2 = \{a: a \in A_1 \text{ или } a \in A_2\} \]
\end{case}

Если имеется произвольное непустое множество $I$ и $\forall \alpha \in I$
имеется множество $A_\alpha$, то
\[\bigcup_{a \in I} A_\alpha = \{a: \exists \alpha \in I \text{ такое, что } a \in A_\alpha\}\]

Пересечением $A \cap B$ назовем множество
\[A \cap B = \{a: (a \in A) \text{ и } (a \in B)\}.\]

Если элементов $a$, принадлежащих $A$ и $B$, не существует, пишем
\[A \cap B = \varnothing\]
и называем $A$ и $B$ дизъюнктивными. Если есть непустое множество $I$,
то, предполагая, что $\forall\alpha\in I \exists A_\alpha$, Полагаем
\[\bigcap_{\alpha\in I} A_\alpha = \{a: \forall \alpha \in I \quad a \in A_\alpha\}\]

Теоретико-множественной разностью множеств $A$ и $B$, обозначаемой $A \\ B$,
называется множество
\[A \\ B = \{a: a\in A, a \not\in B\}\]

\begin{theorem}
    Предположим, что имеется непустое множество $I$ и для любого $\alpha \in I$
    имеется множество $A_\alpha$. Справедливы следующие формулы:
    \begin{equation}
        \label{thm:1.1}
        B \cap \left( \bigcup_{\alpha \in I} A_\alpha \right) =
        \bigcup_{\alpha \in I} (B \cap A_\alpha)
    \end{equation}
    \begin{equation}
        B \cup \left( \bigcap_{\alpha \in I} A_\alpha \right) =
        \bigcap_{\alpha \in I} (B \cup A_\alpha)
    \end{equation}
    \begin{equation}
        B \setminus \left( \bigcup_{\alpha \in I} A_\alpha \right) =
        \bigcap_{\alpha \in I} (B \setminus A_\alpha)
    \end{equation}
    \begin{equation}
        B \setminus \left( \bigcap_{\alpha \in I} A_\alpha \right) =
        \bigcup_{\alpha \in I} (B \cup A_\alpha)
    \end{equation}
\end{theorem}

\begin{proof}
    Докажем \eqref{thm:1.1}, остальные соотношения доказываются аналогично.
    Обозначим левую часть \eqref{thm:1.1} через $C$, а правую через $D$.
    Если $a \in C$, то $a\in B$ и $a \in \bigcup_{\alpha \in I} A_\alpha$,
    т.е. $\exists \alpha_0 \in I$, такое что $a \in A_\alpha$, тогда
    $a \in B \cap A_{\alpha_0}$, $a \in \bigcup_{\alpha \in I}(B \cap A_\alpha)$,
    $a \in D$, то есть $C \subset D$. Если $b \in D$, то $\exists \alpha_1 \in I$
    такое что $b \in B \cap A_{\alpha_1}$, то есть $b \in B$ и $b \in A_{\alpha_1}$,
    тогда $b \in \bigcup_{\alpha \in I} A_\alpha$, 
    $b \in B \cap \bigcup_{\alpha \in I} A_\alpha$, т.е. $b \in C$ и $D \subset C$,
    т.е. $C = D$, что и требовалось доказать.
\end{proof}

\section{Определение вещественных чисел по Р.Дедекинду}
Далее будем считать известными натуральные числа, множество которых всегда 
обозначается через \N, множество целых чисел \Z, множество 
рациональных чисел \Q. Считаем, что свойства арифметических действий
с числами из \Q и свойства, связанные с упорядочиванием рациональных 
чисел по возрастанию, известны.

\begin{definition}
    Пусть $\alpha$ - непустое множество, состоящее из рациональных чисел. Будем 
    называть множество $\alpha$ сечением, если выполняются следующие условия:
    \begin{enumerate}
        \item $\alpha \neq \Q$
        \item Если $p \in \alpha, q \in \alpha, q < p$, то $q \in \alpha$
        \item В $\alpha$ нет наибольшего числа, т.е. не существует $p_0 \in \alpha$,
    такого что $\forall p \in \alpha$ выполнено $p \leq p_0$
    \end{enumerate}
\end{definition}

\begin{assertion}
    Пусть $\alpha$ -- сечение. Если $q \in \Q, p \in \alpha, q \notin 
    \alpha$, то $p < q$.
\end{assertion}
\begin{proof}
    Из условия следует, что $p \neq q$. Если бы выполнялось $q < p$, то по п.2
    определения сечения $q \in \alpha$, чего нет. Следовательно $p > q$, чтд.
\end{proof}

\begin{term}
    Пусть $\alpha$ -- сечение. Числа из \Q, принадлежащие $\alpha$, называются
    нижними числами сечения $\alpha$, а числа из \Q, не принадлежащие $\alpha$,
    называются верхними числами сечения $\alpha$.
\end{term}

Сопоставим теперь $\forall z \in \Q$ сечение, которое будем обозначать $z^*$.
Далее запись $A \overset{def}{=} B$ означает, что объект $A$ определяется через 
объект $B$. Полагаем:
\begin{equation}
    \label{thm:1.5}
    z^* = \{ p \in \Q : p < z \} 
\end{equation}

Запись \eqref{thm:1.5} является сокращением формальной записи \eqref{thm:1.6}
\begin{equation}
    \label{thm:1.6}
    z^* = \{ p: p \in \Q \wedge p < z \} 
\end{equation}

Проверим, что $z^*$ -- сечение. $z - 1 < z$, т.е. $z - 1 \in z^*$,
множество $z^*$ непустое. $z + 1 > z, z + 1 \notin z^*, z^* \neq \Q$.
Если $p \in z^* \wedge q \in \Q, q < p$, то $q < p < z \Rightarrow 
q < z, q \in z^*$. Если $p_1 \in z^*$, то ${p_1 < z}$; 
пусть $p_2 = \frac{p_1 + z}{2}$, тогда $p_1 < p_2 < z$, $p_2 \in z^*$, т.е.
в $z^*$ нет наибольшего числа.

\begin{definition}
    Множество всех сечений будет называться множеством вещественных чисел, а любое
    конкретное сечение будем называть вещественным числом. Обозначаем множество 
    вещественных чисел \R.
\end{definition}

Приведенный подход к определению вещественных чисел принадлежит немецкому 
математику Р. Дедекинду, поэтому сечения называются сечениями множества
рациональных чисел по Дедекинду.

\section{Упорядочивание по возрастанию и арифметические действия над \R числами}
\begin{definition}
    Пишем $\alpha < \beta$, говорим, что $\alpha$ меньше $\beta$, если 
    $\exists p \in \Q$, т.ч. $p \in \beta \wedge p \notin \alpha$. 
    Пишем $\alpha \leq \beta$, говорим, что $\alpha$ не превосходит $\beta$,
    если $\alpha < \beta \vee \alpha = \beta$.
\end{definition}

\begin{theorem}
    Пусть $\alpha$, $\beta$ -- сечения. Тогда либо $\alpha < \beta$, либо
    $\alpha = \beta$, либо $\alpha > \beta$.
\end{theorem}
\begin{proof}
    Если $\alpha = \beta$, то определение влечёт, что не может быть при этом
    $\alpha < \beta$ или $\alpha > \beta$. Пусть $\alpha \neq \beta$. Докажем,
    что выполнено только одно соотношение $\alpha < \beta$ или $\alpha > \beta$.
    Предположим, что выполнены оба, т.е. $\alpha < \beta$ и $\beta < \alpha$.
    Тогда $(\alpha < \beta) \Rightarrow (\exists p \in \Q | 
    p \in \beta, p \notin \alpha)$; $(\beta < \alpha) \Rightarrow 
    (\exists q \in \Q | q \in \alpha, q \notin \beta$). 
    По утверждению из предыдущей лекции $(p \in \beta, q \notin \beta) \Rightarrow
    p < q; (q \in \alpha, p \notin \alpha) \Rightarrow q < p$ -- получили
    противоречие.

    Таким образом, $\alpha < \beta$ и $\beta < \alpha$ вместе не могут выполняться.
    Но, если $\alpha \neq \beta$, то в каком-то из этих множеств, например в 
    $\beta$ имеется элемент $r \in \Q$, не принадлежащий $\alpha$,
    тогда по определению имеем $\alpha < \beta$. Аналогично для $\beta < \alpha$.
    Следовательно, в случае $\alpha \neq \beta$ обязательно выполнится только
    одно условие $\alpha < \beta$ или $\beta < \alpha$. Теорема доказана.
\end{proof}

\begin{theorem}
    Теорема о трех сечениях. Пусть $\alpha, \beta, \gamma$ -- сечения. 
    Если $\alpha < \beta \wedge \beta < \gamma$, то $\alpha < \gamma$.
\end{theorem}
\begin{proof}
$(\alpha < \beta) \Rightarrow (\exists p \in \Q | p \in \beta, p \notin \alpha)$;
$(\beta < \gamma) \Rightarrow (\exists q \in \Q | q \in \gamma, q \notin \beta)$.
Далее, $(p \in \beta, q \notin \beta)$ $\Rightarrow$ по утверждению из прошлой лекции
$p < q$. Поскольку $p \notin \alpha$, то тогда и $q \notin \alpha$, 
в противоположном случае по свойству 2 в определении сечения было бы и $p \in \alpha$.
Таким образом, $q \in \gamma, q \notin \alpha$, т.е. $\alpha < \gamma$.
Теорема доказана.    
\end{proof}

\begin{definition}
    Сумма вещественных чисел = сумма сечений.
\end{definition}
\begin{theorem}
    Пусть $\alpha$ и $\beta$ -- сечения, $\gamma$ -- множество рациональных чисел
    $r$, т.ч. $r = p + q$, где $p \in \alpha$ - произвольное число, 
    $q \in \beta$ - произвольное число. Тогда $\gamma$ -- сечение.
\end{theorem}
\begin{proof}
    Поскольку $\alpha \neq \varnothing$, $\beta \neq \varnothing$, то 
    $\gamma \neq \varnothing$. Поскольку $\alpha \neq \Q, \beta \neq \Q$, то
    $\exists s \in \Q, s \notin \alpha$ и $\exists t \in \Q, t \notin \beta$.
    Пусть $p \in \alpha$, $q \in \beta$. 
    По удтверждению из прошлой лекции $(p \in \alpha, s \notin \alpha) \Rightarrow
    (p < s); (q \in \beta, t \notin \beta) \Rightarrow (q < t)$. 
    Отсюда следует, что $p + q < s + t$ $\forall p \in \alpha \wedge 
    \forall q \in \beta$, т.ч. $\forall r \in \gamma$ выполнено $r < s + t$, т.е.
    $s + t \notin \gamma$, т.е. $\gamma \neq \Q$ -- проверен п.1 в определении
    сечения.

    Пусть $r \in \gamma$, $s < r$. Тогда $r = p + q, p \in \alpha, q \in \beta$.
    Пусть $t = s - q$, тогда $t < r - q = (p + q) - q = p$, из $p \in \alpha$ и 
    $t < \alpha$ следует $t \in \alpha$, т.е. $s = t + q, t \in \alpha, 
    q \in \beta$, т.е. $s \in \gamma$ -- проверен п.2 в определении сечения.
    
    Пусть $r \in \gamma, r = p + q, p \in \alpha, q \in \beta$. По п.3 
    определения сечения $\exists p_1 \in \alpha, p_1 > p$, тогда 
    $r_1 = p_1 + q > p + q = r$, в $\gamma$ нет наибольшего элемента, проверен
    п.3 определения сечения.

    Теорема доказана.
\end{proof}

\begin{definition}
    Сечение $\gamma$, построенное в предыдущей теореме, называется суммой 
    сечений $\alpha$ и $\beta$.

    Поскольку вещественные числа определены как сечения, то вещественное число 
    $\gamma$ называют суммой вещественных чисед $\alpha$ и $\beta$, пишут
    $\gamma = \alpha + \beta$.
\end{definition}

\paragraph{Свойства сложения}
\begin{theorem}
    Пусть $\alpha, \beta, \gamma$ -- вещественные числа. Тогда:
    \begin{enumerate}
        \item $\alpha + \beta = \beta + \alpha$
        \item $(\alpha + \beta) + \gamma = \alpha + (\beta + \gamma)$
        \item $\alpha + 0^* = \alpha$
    \end{enumerate}
\end{theorem}
\begin{proof}
    Пункты 1 и 2 следуют из определения сложения и свойств сложения
    рациональных чисел. Докажем п.3.

    Пусть $r \in \alpha + 0^*$, тогда $r = p + q$, $p \in \alpha, q \in 0^*$,
    т.е. $q < 0$, поэтому $r = p + q < p$, тогда $r \in \alpha$ по условию 2 
    определения сечений, т.ч. $\alpha + 0^* \subset \alpha$, если мы делаем 
    акцент на том, что $\alpha + 0^*$ и $\alpha$ -- множества. Пусть теперь
    $t \in \alpha$. Выберем $s > t$, но $s \in \alpha$, что возможно по п.3
    определения сечений. Полагаем $q_0 = t - s$, тогда $t - s < 0 \Rightarrow
    t - s \in 0^*, t = s + (t - s) \in \alpha + 0^*$, т.е. $\alpha \subset 
    \alpha + 0^*$, тогда $\alpha = \alpha + 0^*$. Теорема доказана.
\end{proof}

\begin{theorem}
    Теорема о разности верхних и нижних чисел сечения. Пусть $\alpha$ -- сечение,
    и пусть $r \in \Q, r > 0$. Тогда $\exists p \in \Q, \exists q \in \Q$,
    такие что $p \in \alpha, q \notin \alpha$, $q$ не является наименьшим из
    верхних чисел $\alpha$ и $q - p = r$.
\end{theorem}
\begin{proof}
    Возьмем $s \in \alpha$, и пусть $s_n = s + nr, s_0 = s, n = 0, 1, \ldots$.
    Найдется $m_0$, т.ч. $s_{m_0} \notin \alpha$: если бы $s_n \in \alpha 
    \forall n \in \N$, то возьмем $\forall t \in \Q, t > s$. По свойствам
    рациональных чисел $\exists n_0$ т.ч. $s = n_0r > t$, и тогда 
    $s_{n_0} \in \alpha \Rightarrow t \in \alpha$, т.е. $\alpha = \Q$ в силу
    произвольности $t$, что противоречит условию 1.

    Таким образом, $\exists m_0 \in \N$, т.ч. $s_{m_0} \notin \alpha$.
    Поскольку $s_0 \in \alpha$, то имеется максимальное $m \in \N$, т.ч. 
    $s_m \in \alpha, m < m_0$, тогда $s_{m + 1} \notin \alpha$.
    Если $s_{m + 1}$ не является минимальным из верхних чисел сечения, 
    то полагаем $p = s_m, q = s_{m+1}$, тогда $q - p = s_{m + 1} - s_m = 
    (s + (m + 1)r) - (s + mr) = r$. Если же $s_{m + 1}$ является наименьшим
    из верхних чисел сечения, то пусть $p = s_m + \frac{r}{2}$,
    $q = s_{m + 1} + \frac{r}{2}$, $q - p = r$, $q > s_{m + 1} \Rightarrow
    q \notin \alpha$, $s_{m + 1}$ -- наименьшее из верхних чисел $\alpha$ и
    $p = s_m + \frac{r}{2} = s + mr + \frac{r}{2} < s + (m + 1)r$, поэтому
    $p \in \alpha$. Теорема доказана.  
\end{proof}

\paragraph{Существование противоположного числа}
\begin{theorem}
    Пусть $\alpha$ - вещественное число. Тогда существует единственное число 
    $\beta$ такое, что $\alpha + \beta = 0^*$
\end{theorem}
\begin{proof}
    Вначале докажем единственность $\beta$. Предположим, что $\exists \beta_0$ т.ч.
    $\alpha + \beta_0 = 0^*$. Тогда, по теореме о свойствах сложения имеем
    \[
        \beta_0 = 0^* + \beta_0 = (\alpha + \beta) + \beta_0 = (\beta + \alpha) +
        \beta_0 = \beta + (\alpha + \beta_0) = \beta + 0^* = \beta  
    \]
    т.е. $\beta$ - единственный, если существует.

    Найдем теперь какое-то $\beta$, т.ч. $\alpha + \beta = 0^*$. Пусть $\beta$ --
    множество всех рациональных чисел таких, что $-p$ является верхним числом 
    $\alpha$, но не наименьшим из верхним чисел.
    
    Проверим, что $\beta$ -- сечение (= вещественное число). Взяв любое верхнее не
    наименьшее число $t$ сечения $\alpha$, полагая $p = -t$, имеем $p \in \beta$,
    т.е. $\beta \neq \emptyset$. Взяв любое $s \in \alpha$, получаем, что 
    $-s \notin \beta$, т.к. $-(-s) = s \in \alpha$, $s$ - нижнее число $\alpha$,
    т.е. $\beta \neq \Q$ -- проверено условие 1.

    Если $p \in \beta$, $q \in \Q$ и $q < p$, то $-q > -p$, $-p$ -- верхнее число
    $\alpha \Rightarrow -q$ -- верхнее число $\alpha$ и $-q$ -- не наименьшее
    верхнее в $\alpha$, т.е. $q \in \beta$ -- проверено условие 2.
    
    Если $p \in \beta$, то $-p$ -- врехнее число $\alpha$ и $\exists$ верхнее число
    $\alpha$, обозначим его $-q$, т.ч. $-q < -p$; пусть $-z=^{def}-\frac{q + p}{2}$,
    тогда $-z > -q$, т.е. $-z$ -- верхнее число в $\alpha$ и не наименьшее, поэтому
    $z \in \beta$. Поскольку $-z < -p$, то $z > p$, в $\beta$ нет наибольшего --
    проверено условие 3. Таким образом $\beta$ -- сечение.

    \underline{Проверка свойства $\alpha + \beta = 0^*$}
    Пусть $p \in \alpha + \beta$, тогда $p = q + z, q \in \alpha, z \in \beta$;
    $z \in \beta \Rightarrow -z \notin \alpha$, тогда $q \in \alpha \Rightarrow
    q < -z, q + z < 0, p < 0, p \in 0^*$, т.е. $\alpha + \beta \subset 0^*$, 
    если трактовать $\alpha, \beta, 0^*$ как множества.

    Пусть $p \in 0^*$, тогда $p < 0$. По теореме о разности верхних и нижних 
    чисел сечения $\exists q \in \alpha, s \notin \alpha, s$ не является наименьшим
    верхним числом $\alpha$, т.ч. $s - q = -p$. Поскольку $-s \in \beta$, то тогда
    $p = q - s = q + (-s) \in \alpha + \beta$, т.е. $0^* \subset \alpha + \beta$;
    в итоге $0^* = \alpha + \beta$, теорема доказана.
\end{proof}

\begin{definition}
    Вещественное число $\beta$, построенное в предыдущей теореме обозначается
    $-\alpha$, и называется числом, противоположным $\alpha$. 
\end{definition}

\begin{assertion}
    О сохранении неравенства. Пусть $\beta < \gamma$, тогда $\alpha + \beta <
    \alpha + \gamma$. В частности, если $0^* < \gamma, 0^* < \alpha$, то 
    $(\alpha = 0^* + \alpha < \alpha + \gamma, 0^* < \alpha) \Rightarrow 
    0^* < \alpha + \gamma$.
\end{assertion}
\begin{proof}
    Из определения сложения вещественных чисел следует, что $\alpha + \beta \leq 
    \alpha + \gamma$. Если было бы $\alpha + \beta = \alpha + \gamma$, то тогда
    \[\beta = 0^* + \beta = ((-\alpha) + \alpha) + \gamma = 0^* + \gamma = \gamma\],
    что противоречит условию. Утверждение доказано.
\end{proof}

\paragraph{Определение разности вещественных чисел}
\begin{theorem}
    Пусть $\alpha, \beta$ -- вещественные числа. тогда существует единственное
    вещественное число $\gamma | \alpha + \beta = \gamma$.
\end{theorem}
\begin{proof}
    Полагаем $\gamma = \beta + (-\alpha)$. Тогда $\alpha + \gamma = \alpha +
    (\beta + (-\alpha)) = \alpha + ((-\alpha) + \beta) = (\alpha + (-\alpha)) +
    \beta = 0^* + \beta = \beta$.

    Если бы существовало $\gamma_1 | \alpha + \gamma_1 = \beta$, то если бы
    $\gamma \neq \gamma_1$, то тогда либо $\gamma < \gamma_1$, либо $\gamma_1 <
    \gamma$. Не уменьшая общности, считаем $\gamma < \gamma_1$. Тогда по удтверждению
    о сохранении неравенства мы получаем $\alpha + \gamma < \alpha + \gamma_1$,
    но $\alpha + \gamma = \beta, \alpha + \gamma_1 = \beta$, противоречие.

    Итак, вещественное число $\gamma$ одно. Оно называется разностью $\beta$ и 
    $\alpha$, $\gamma = \beta - \alpha$.
\end{proof}

\begin{definition}
    $|\alpha|$. Полагаем
    \begin{equation*}
        |\alpha| =
        \begin{cases}
            \alpha, & \alpha \geq 0^* \\
            -\alpha, & \alpha < 0^*
        \end{cases}
    \end{equation*} 
\end{definition}

\begin{assertion}
    $|\alpha| \geq 0^* \forall \alpha \in \R$
\end{assertion}
\begin{proof}
    Если $\alpha \geq 0^*$, это следует из определения $|\alpha|$. Пусть
    $\alpha < 0^*$, тогда $\alpha \neq 0^*$ и, если неверно, что $\alpha > 0^*$.
    то $-\alpha < 0^*$. По удтверждению о сохранении неравенства тогда бы 
    выполнялось $\alpha + (-\alpha) < \alpha + 0^* = \alpha$, но $\alpha < 0^*$,
    тогда $\alpha + (-\alpha) < 0^*, 0^* < 0^*$, что невозможно. Итак 
    $|\alpha| \geq 0^*$. Из определения видно, что $|\alpha| = 0^* \Leftrightarrow
    \alpha = 0^*$. Удтверждение доказано.
\end{proof}

\begin{theorem}
    $p^* < \alpha, p \in \Q. p^* < \alpha \Leftrightarrow p \in \alpha, p \in \Q$
\end{theorem}
\begin{proof}
    Пусть $p \in \alpha; p \notin p^* \Rightarrow p^* < \alpha$. Пусть теперь
    $p^* < \alpha$, тогада $\exists q \in \Q | q \notin p^*$, т.е. $q \geq p$, 
    и $q \in \alpha$. Тогда $p \in \alpha$. Теорема доказана.
\end{proof}

\end{document}
