% !TeX root = ./main.tex
\documentclass[main]{subfiles}
\begin{document}

\section{Свойства сочетаний}
\begin{enumerate}
    \item $C_n^k = \frac{n!}{k!(n-k)!}$
    \item $C_n^k = C_n^{n-k}$
    \item $C_{n+1}^{k+1} = C_n^k + C_n^{k+1}$ (геометрический интерпретация
$\rightarrow$ треугольник паскаля $\rightarrow$ биномиальные коэффициенты 
$\rightarrow$ бином Ньютона) 
\end{enumerate}


\section{Бином Ньютона}
\[
    (a + b)^{n}=\sum _{k=0}^{n}{\binom {n}{k} a^k b^{n-k}}
\]

\section{Перебор сочетаний}
Сочетание - вектор $x = (1, 2, \ldots, k)$
\begin{enumerate}
    \item Просматривать вектор $x$ справа и искать компонент, который можно
увеличить
    \item Если такой элемент не нашелся -- закончить процесс
    \item Увеличить найденный компонент $x_i$ на 1
    \item Все последющие компоненты заполнить последовательными натуральными
числами, начиная с $x_i + 1$
\end{enumerate}

\section{Нумерация перестановок}
Представим сочетание в виде характеристического вектора: 
$b=(1,0,0,1,1, \ldots)$,
где $k$ - число единиц в нем.
\begin{equation*}
    num(b[1:n], k) = \begin{cases}
        num[b[1:n-1], k], & b[n] = 0 \\
        C_{n-1}^k + num([1:n-1], k - 1), & b[n] = 1
    \end{cases}
\end{equation*}
\begin{case}
    Пусть $b=(1,0,0,1,1)$, тогда:
    \begin{multline*}
        num((1,0,0,1,1), 3) = \binom{4}{3} + num((1,0,0,1), 2) = \\ 
        \binom{4}{3} + \binom{3}{2} + num((1), 1) = 4 + 3 + 0 = 7     
    \end{multline*}
\end{case}

\end{document}
