% !TeX root = ./main.tex
\documentclass[main]{subfiles}
\begin{document}
\chapter{Комбинаторика}

\section{Основные правила}
\subsection{Правило суммы}
Если комбинации можно разбить на классы $A$ и $B$, то общее число кобинаций 
$|A| + |B|$.

\subsection{Правило произведения}
При составлении пары из двух элементов $(A, B)$ известно, что первый элемент пары
можно выбрать $|A|$ способами, а второй $|B|$.

\section{Основные объекты комбинаторики}
\subsection{Множество перестановок}
\[
    |P_k| = k!
\]

\subsection{Множество размещений}
\[
    |A_n^k| = \frac{n!}{(n - k)!}
\]

\subsection{Множество сочетаний}
\[
    |C_n^k| = \frac{A_n^k}{k!} = \frac{n!}{k!(n - k)!}
\]

\subsection{Перестановки с повторениями}
Пусть имеется $n_k$ элементов типа $k$. Всего элементов $n=\sum_{i = 1}^{k}n_i$.
\[
    P(n_1, \ldots, n_k) = \frac{n!}{n_1!n_2! \ldots n_k!}
\]

\subsection{Сочетания с повторениями}
Имеются предметы $n$ различных типов. Сколько $k$ комбинаций можно из них сделать, 
если не учитывать порядок.

Если $n$ различных типов. Пусть сначала идут все элементы первого типа, потом
второго и тд. Всего $k$ элементов. Добавим к ним $n-1$ перегородок. Всего будет
$n - 1 + k$ мест. Выбор расположения перегородок = сочетания.
\[
    \overline{C_n^k} = C_{n + k - 1}^{n - 1} = \frac{(n + k - 1)!}{k!(n - 1)!}
\]

\section{Формула включений-исключений}
\subsection{Для двух множеств ($A$ и $B$)}
\[
    |A \cup B| = |A| + |B| - |A \cap B|    
\]

\subsection{Общая}
\begin{multline*}
    \left|\bigcup_{i = 1}^n A_i\right| = 
    \sum_{i = 1}^{n} |A_i| - 
    \sum_{1 \leq i < j \leq n} |A_i \cap A_j| + \\
    \sum_{1 \leq i < j < k \leq n} |A_i \cap A_j \cap A_k| -
    \ldots + 
    (-1)^{n-1} |A_1 \cap A_2 \cap \ldots \cap A_n|
\end{multline*}

\subsection{Доказателсьво по индукции}
База: $n = 2$ -- верно.

Переход: предположим, что верно для $n = 1$, докажем что верно для $n$.
\begin{multline*}
    \left| \bigcup_{i = 1}^n A_i \right| = 
    \left| \bigcup_{i = 1}^{n-1} A_i \cup A_n \right| =
    \textcolor{red}{ \left| \bigcup_{i = 1}^{n-1} A_i \right| } + |A_n| - 
        \textcolor{blue}{ \left| \left( \bigcup_{i = 1}^n A_i \right) \cap A_n \right| } = \\
    \textcolor{red}{ \sum_{i = 1}^{n - 1} |A_i| - 
        \sum_{1 \leq i < j \leq n-1} |A_i \cap A_j| + 
        \sum_{1 \leq i < j < k \leq n-1} |A_i \cap A_j \cap A_k| -
        \ldots + } \\
        \textcolor{red}{ (-1)^{n-2} |A_1 \cap A_2 \cap \ldots \cap A_{n-1}| } + 
        |A_n| - 
        \textcolor{blue}{ \left| \bigcup_{i = 1}^{n - 1} (A_i \cap A_n) \right| } = \\
    \textcolor{red}{\sum_{i = 1}^{n - 1} |A_i|} - 
        \textcolor{blue}{ \sum_{1 \leq i < j \leq n-1} |A_i \cap A_j| } + 
        \textcolor{magenta}{ \sum_{1 \leq i < j < k \leq n-1} |A_i \cap A_j \cap A_k| } -
        \ldots + \\
        (-1)^{n-2} |A_1 \cap A_2 \cap \ldots \cap A_{n-1}| + 
        \textcolor{red}{|A_n|} - \\
        \Bigl(\textcolor{blue}{ \sum_{i = 1}^{n - 1} |A_i \cap A_n| } - 
            \textcolor{magenta}{ \sum_{1 \leq i < j \leq n-1} |A_i \cap A_j \cap A_n| } +
            \ldots + \\ 
            (-1)^{n-2} |A_1 \cap A_2 \cap \ldots \cap A_{n}| \Bigl) = \\
    \sum_{i = 1}^{n} |A_i| - 
        \sum_{1 \leq i < j \leq n} |A_i \cap A_j| + 
        \sum_{1 \leq i < j < k \leq n} |A_i \cap A_j \cap A_k| -
        \ldots + \\
        (-1)^{n-1} |A_1 \cap A_2 \cap \ldots \cap A_n|
\end{multline*}

\subsection{Примеры}
\paragraph{Пример 1.}
В отделе 67 человек. 47 знают английский, 35 - немецкий, 23 - оба языка. Тогда знают 
языки по формуле.

\paragraph{Пример 2.}
5 писем адресатам раскладывают по 5 конвертам. Сколько вариантов, в которых ни одно письмо
не попадет к адресату?

Пусть $A_i$ - количество вариантов, в которых адресат $i$ получил свое письмо. Тогда 
количество вариантов, когда хотя бы один получил свое письмо:
\[
    |A_1 \cup \ldots \cup A_5| = 5 * 4! - C_5^2 * 3! + C_5^3 * 2! - C_5^4 * 1! + 1 = 76
\]
Всего вариантов: $120$. Ответ: $120 - 76 = 44$.

\end{document}
