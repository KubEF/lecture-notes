% !TeX root = ./main.tex
\documentclass[main]{subfiles}
\begin{document}
\chapter{Китайская теорема об остатках}
\begin{theorem}
    Пусть $m,n \in \N,(m,n)=1, a,b \in \Z$, тогда
    \[\exists x \in \Z: \begin{cases}
            x\equiv a \pmod{m} \\
            x \equiv b \pmod{n}
        \end{cases}\]
    Далее, если
    \[x' \in \Z, то \begin{cases}
            x' \equiv a \pmod{m} \\
            x' \equiv b \pmod{n}
        \end{cases}
        \Leftrightarrow x' \equiv x \pmod{mn}\]
\end{theorem}
\begin{proof}
    \begin{gather*}
        x' \equiv x \pmod{mn} \Leftrightarrow
        \begin{cases}
            x' \equiv x \pmod{m} \\
            x' \equiv x \pmod{n}
        \end{cases}\Leftrightarrow
        \begin{cases}
            x' \equiv a \pmod{m} \\
            x' \equiv b \pmod{n}
        \end{cases}\\
        \begin{multlined}
            (m,n) = 1 \implies \overline{m} \in (\Z/n\Z)^* \\
            \implies \exists x_1 \in \Z : \overline{m} \overline{x_1}
            = \overline{1} \in (\Z/m\Z)^* \implies mx_1 \equiv 1 \pmod{n}
        \end{multlined}\\
        \intertext{Аналогично}
        \exists x_2 \in \Z: nx_2 \equiv 1 \pmod{m}\\
        \begin{cases}
            mx_1 \equiv 0 \pmod{m} \\
            mx_1 \equiv 1 \pmod{n}
        \end{cases} \qquad
        \begin{cases}
            nx_2 \equiv 1 \pmod{m} \\
            nx_2 \equiv 0 \pmod{n}
        \end{cases}\\
        \begin{array}{l}
            b(mx_1) + a(nx_2) \equiv b\cdot 0  + a \cdot 1\pmod{m} \\
            b(mx_1) + a(nx_2) \equiv b\cdot 1  + a \cdot 0\pmod{n}
        \end{array}
        \implies \begin{cases}
            x \equiv a \pmod{m} \\
            x \equiv b \pmod{n}
        \end{cases}
    \end{gather*}
\end{proof}
\begin{proof}
    \begin{gather*}
        \Z/mn\Z \xrightarrow{\text{инъ.}} (\Z/m\Z) \times (\Z/n\Z)\\
        [a]_{mn} \mapsto ([a]_m, [a]_n)\\
        |\Z/mn\Z| = mn = |(\Z/m\Z) \times (\Z/n\Z)|\implies \text{отображение биекция}
    \end{gather*}
\end{proof}
\begin{example}
    Сколько решением имеет уравнение $x^2 \equiv 1 \pmod{77}$
    \begin{gather*}
        x^2 \equiv 1 \pmod{77} \Leftrightarrow
        \begin{cases}
            x^2 \equiv 1 \pmod{7} \\
            x^2 \equiv 1 \pmod{11}
        \end{cases}\\
        \Leftrightarrow
        \begin{cases}
            \left[
            \begin{array}{l}
                x \equiv 1 \pmod{7} \\
                x \equiv -1 \pmod{7}
            \end{array}
            \right. \\
            \left[
            \begin{array}{l}
                x \equiv 1 \pmod{11} \\
                x \equiv -1 \pmod{11}
            \end{array}
            \right.
        \end{cases}
        \Leftrightarrow
        \left[
        \begin{array}{l}
            \begin{cases}
                x \equiv 1 \pmod{7} \\
                x \equiv 1 \pmod{11}
            \end{cases} \\
            \begin{cases}
                x \equiv 1 \pmod{7} \\
                x \equiv -1 \pmod{11}
            \end{cases} \\
            \begin{cases}
                x \equiv -1 \pmod{7} \\
                x \equiv 1 \pmod{11}
            \end{cases} \\
            \begin{cases}
                x \equiv -1 \pmod{7} \\
                x \equiv -1 \pmod{11}
            \end{cases}
        \end{array}
        \right.\\
        \Leftrightarrow
        \left[
        \begin{array}{l}
            x \equiv 1 \pmod{77}   \\
            x \equiv 43 \pmod{77}  \\
            x \equiv -43 \pmod{77} \\
            x \equiv -1 \pmod{77}  \\
        \end{array}
        \right.
    \end{gather*}
\end{example}

\end{document}