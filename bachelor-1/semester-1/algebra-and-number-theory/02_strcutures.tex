% !TeX root = ./main.tex
\documentclass[main]{subfiles}
\begin{document}
\chapter{Группы}
\section{Введение}
\begin{definition}
    Бинарная операция на множестве $M$ -- отображение из $M \times M \to M$
\end{definition}

\subsection{Примеры}
\begin{enumerate}
    \item $+, -, \cdot$ на \Z
    \item $+$ на векторном пространстве
    \item $M= X^X = \{f: X \to X\}$\\
          $(f,g) \mapsto f\circ g$\\
          $M \times M \mapsto M$
\end{enumerate}

\subsection{Свойства}
Есть операция $M \times M \to M$, обозначим ее $(a,b) \mapsto a*b$
\begin{enumerate}
    \item Если $\forall a,b \in M: a*b = b*a$, то $*$ коммутативна
    \item $*$ ассоциативна, если  $\forall a,b,c \in M: (a*b)*c = a*(b*c)$
    \item $e\in M$ называется левым нейтральным, если $\forall a \in M: e*a=a$
          \marginpar{В вычитании целых чисел ноль нейтрален справа}\\
          $e\in M$ называется правым нейтральным, если $\forall a \in M: a*e=a$\\
          $e\in M$ называется нейтральный, если он и левый, и правый нейтральный
          \begin{lemma}
              Пусть $*$ -- операция, $e_L, e_R$ -- нейтральные слева
              и справа относительно $*$, тогда $e_L = e_R$.
          \end{lemma}
          \begin{proof}
              \[e_R = e_L \cdot e_R = e_L\]
          \end{proof}
    \item \marginpar{Обратное к $a$ обозначается $a^{-1}$}
          Пусть $e$ нейтральный относительно $*$, $a\in M$. Элемент $b \in M$
          называется обратным к $a$, если $b*a = a*b = e$\\
          Если $b*a = e \Rightarrow b$ обратный слева\\
          Если $a*b = e \Rightarrow b$ обратный справа
          \begin{lemma}
              Если $*$ ассоциативна и у $a$ если левый и правый обратный,
              тогда они равны. $b*a=e, a*c=e$
          \end{lemma}
          \begin{proof}
              \begin{gather*}
                  (b*a)*c=b*(c*a)\\
                  e*c =b*e\\
                  c=b
              \end{gather*}
          \end{proof}
\end{enumerate}

Если $*$ -- ассоциативная операция, $m \in \Z$:
\begin{equation*}
    a^m =
    \begin{cases}
        a_1 * a_2 * \ldots * a_m                  & m > 0 \\
        e                                         & m = 0 \\
        a_1^{-1}* a_2^{-1} * \ldots * a_{-m}^{-1} & m < 0
    \end{cases}
\end{equation*}

\[a^m * a^n = a^{m+n} \qquad (a^m)^n = a^ {mn}\]

\section{Определение группы}

\begin{definition}
    Группой называется множества $G$ с операцией $*$, такие что:
    \begin{enumerate}
        \item $*$ ассоциативна
        \item У $*$ есть нейтральный элемент
        \item У любого $g\in G$ есть обратный
    \end{enumerate}
    Группа $G$ называется абелевой (коммутативной), если $*$ коммутативна
\end{definition}

\subsection{Примеры}
\begin{enumerate}
    \marginpar{1--4 абелевы группы}
    \item $(\Z, +)$
    \item $(\Q, +), (\R, +)$
    \item $(\Q \setminus \{0\}, \cdot), (\R \setminus \{0\}, \cdot)$
    \item $(\{1, -1\}, \cdot)$
    \item $(X^X, \cdot)$ -- не группа, при $|X| > 1$
    \item $(S(X), \cdot)$, что
          $S(x)=\{f:x\to x : x \text{ - биекция}\}$ -- группа,
          не абелева при $|X| = 2$
\end{enumerate}

\section{Подгруппы}

\begin{example}
    $(\Z, +)$ -- группа, $2\Z = \{2n: n \in \Z\}$ -- подгруппа
\end{example}

\begin{definition}
    $G$ -- группа, $H \subset G$ называется подгруппой, если:
    \begin{enumerate}
        \item $H$ замкнуто относительно умножения, т.е. $\forall h_1, h_2 \in H: h_1 h_2 \in H$
        \item $e \in H$
        \item $H$ замкнуто относительно обратного, т.е. $\forall h \in H: h^{-1} \in H$
    \end{enumerate}
\end{definition}

\subsection{Примеры} \marginpar{$\subset \Leftrightarrow <$}
\begin{itemize}
    \item  $2\Z < \Z$
    \item $\{0\}< \Z$
    \item $\Z \in \Q$
    \item $(\{-1, 1\}, \cdot) < \Q^*$
    \item  $\{2^n: n \in \Z\} < \Q^*$
    \item Группы самосовмещений (симметрий) фигур, $\Pi$ - плоскость,
          $S(\Pi)$, $T(\Pi) < S(\Pi)$ -- перемещения плоскости (движения)
\end{itemize}

\subsection{Законы сокращения}
\begin{lemma}
    Пусть $G$ - группа, $g, h_1, h_2 \in G$

    \begin{enumerate}
        \item $gh_1 = gh_2 \Rightarrow h_1 = h_2$
        \item $h_1 g = h_2 g \Rightarrow h_1 = h_2$
    \end{enumerate}
\end{lemma}
\begin{proof}
    \begin{gather*}
        g^{-1}gh_1 = g^{-1}gh_2 \Rightarrow h_1 = h_2
    \end{gather*}
\end{proof}

\section{Таблицы Кэли}
Дана группа $G = \{g_1, g_2, \ldots g_n\}$:

\begin{center}
    $\begin{array}{c|cccc}
                   & g_1     & g_2     & \cdots & g_n     \\
            \hline
            g_1    & g_1 g_1 & g_1 g_2 & \cdots & g_1 g_n \\
            g_2    & g_2 g_1 & g_2 g_2 & \cdots & g_2 g_n \\
            \vdots & \cdots  & \cdots  & \cdots & \cdots  \\
            g_n    & \cdots  & \cdots  & \cdots & \cdots
        \end{array}$
\end{center}

Дана группа $\Z^* = (\{ \pm 1\}, \cdot)$:

\begin{center}
    $\begin{array}{c|cc}
               & 1  & -1 \\
            \hline
            1  & 1  & -1 \\
            -1 & -1 & 1
        \end{array}$
\end{center}

Дана группа самосовмещений правильного прямоугольника:
\marginpar{Таблица Кэли является латинским квадратом}
\begin{center}
    $\begin{array}{c|cccc}
            \square & e     & S_1   & S_2   & R_\pi \\
            \hline
            e       & e     & S_1   & S_2   & R_\pi \\
            S_1     & S_1   & e     & R_\pi & S_2   \\
            S_2     & S_2   & R_\pi & e     & S_1   \\
            R_\pi   & R_\pi & S_2   & S_1   & e     \\
        \end{array}$
\end{center}
Группа абелева, т.к. симметрична относительно диагонали

Рассмотрим $\Z^* \times \Z^* = \{(1,1), (1, -1), (-1,1), (-1,-1)\}$.
Операции будем производить покомпонентно: $(a,b)(a', b')= (aa', bb')$.
\begin{center}
    $\begin{array}{c|cccc}
                    & e       & (1, -1) & (-1,1)  & (-1,-1) \\
            \hline
            e       & e       & (1, -1) & (-1,1)  & (-1,-1) \\
            (1,-1)  & (1, -1) & e       & (-1,-1) & (-1,1)  \\
            (-1,1)  & (-1,1)  & (-1,-1) & e       & (1, -1) \\
            (-1,-1) & (-1,-1) & (-1,1)  & (1, -1) & e       \\
        \end{array}$
\end{center}

Последние 2 группы изоморфны (если заменить все элементы, например, буквами,
то они и их таблицы Кэли будут идентичны)

Теория групп изучает группы с точностью до изомфорфизма

\begin{axiom}
    Любые группы третьего порядка изоморфны.
\end{axiom}

С группами порядка 4 это уже не выполняется
\end{document}