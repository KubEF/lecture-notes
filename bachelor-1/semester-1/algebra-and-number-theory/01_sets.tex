% !TeX root = ./main.tex
\documentclass[main]{subfiles}
\begin{document}
\part{Алгебраические структуры}
\chapter{Множества}
\section{Нотация}
Стандартная запись:
\begin{gather*}
    A' = \{1,3,5,7\}\\
    A=\{1,3,5, ..., 99\}
\end{gather*}

Общий вид:
\[B = \{2,4,6,...\} = \{2n: n\in \N\}\]

Стандартные числовые множества:
\begin{gather*}
    \N =\{1,2,3,...\} \qquad \Z =\{...,-1,0,1,2,...\}\\
    \Q = \left\{ \frac{p}{q}: p,q \in \Z, q \neq 0 \right\} \qquad \R, \C
\end{gather*}

Подмножества:
\begin{gather*}
    A' \subset A \subset \N, A' \not\subset B\\
    \begin{aligned}
        C & = \{1,2,3\} & \emptyset, & \{1\},  \{2\}, \{3\}     \\
          &             &            & \{1,2\},\{1,3\}, \{2,3\} \\
          &             &            & \{1,2,3\} = C
    \end{aligned}
\end{gather*}

Предикат для подмножеств: $\{n \in \N : n < 5\} = \{1,2,3,4\}$

\section{Операции на множествах}
Пусть A, B --- множества
\begin{gather*}
    A \cap B = \{a \in A \land  a \in B\}\\
    A \cup B = \{a: a \in A \lor a \in B\}\\
    A \setminus B = \{a\in A \land a\not\in B\}\\
    A \bigtriangleup B = (A \setminus B) \cup (B \setminus A)\\
    A \times B = \{(a,b):a\in A, b \in B\}\\
    \bigcap_{i=1}^n A_i \quad \bigcup_{i=1}^n A_i\\
    A \cup (B \cap C) = (A \cup B) \cap (A \cup C)\\
    A \cap (B \cup C) = (A \cap B) \cup (A \cap C)
\end{gather*}
\begin{example}
    \begin{gather*}
        A = \{1,2,3\} \quad B = \{-1, 1\} \\
        A \times B = \{ (1,-1), (1,1), (2, -1), (2,1), (3, -1), (3,1)\}
    \end{gather*}
\end{example}
\section{Отображение}
$A, B$ --- множества

\begin{definition}
    Задать отображение $A$ в $B$, значит для каждого $a \in A$ задать
    некоторый элемент $B$ (т.н. образ элемента $A$)
\end{definition}

\noindent\begin{minipage}{0.45\textwidth}
    \begin{gather*}
        A = \{1,2,3,4\}\\
        B = \R
    \end{gather*}
\end{minipage}
\begin{minipage}{0.45\textwidth}
    \begin{center}
        \begin{tabular}{c|c}
            $a$ & $f(a)$     \\
            \hline
            1   & $\sqrt{2}$ \\
            2   & 0          \\
            3   & $7^5$      \\
            4   & 0          \\
        \end{tabular}
    \end{center}
\end{minipage}

\noindent\begin{minipage}{0.45\textwidth}
    \begin{gather*}
        f: \R \to \R    \\
        f(a)=a-3        \\
        \Leftrightarrow \\
        f: \R \to \R    \\
        a \mapsto a-3
    \end{gather*}
\end{minipage}
\begin{minipage}{0.45\textwidth}
    \begin{center}
        \begin{tabular}{c|c}
            $a$ & $f(a)$ \\
            \hline
            1   & -2     \\
            2   & -1     \\
            3   & 0      \\
            4   & 1
        \end{tabular}
    \end{center}
\end{minipage}

\begin{gather*}
    f: \R \to \Z \\
    a \mapsto \begin{cases}
        1,  & a > 0 \\
        0,  & a =0  \\
        -1, & a < 0
    \end{cases}\\
    \varphi: \N \to \N \\
    n \mapsto |\{m \in \N: m \le n \& (m,n) = 1\}|
\end{gather*}

\begin{definition}
    $|M| = \#M = \text{Card } M$ --- мощность множества
\end{definition}
\begin{definition}
    $2^M$ --- множество всех подмножеств M, его мощность $|2^M| = 2^{|M|}$
\end{definition}

\subsection{Свойства}
\begin{prop}
    $f: A \to B$ называется инъекцией, если
    \[\forall a_1, a_2 \in A: a_1 \neq a_2 \implies f(a_1)\neq f(a_2)\]
\end{prop}
\begin{prop}
    $f: A \to B$ называется сюръекцией, если \[\forall b \in B, \exists a \in A: f(a) = b\]
\end{prop}
\begin{prop}
    $f: A \to B$ называется биекцией, если оно одновременно инъекция и сюръекция
\end{prop}
\begin{definition}
    Пусть $f: A \to B$, тогда $b\in B$ -- полный прообраз $b$ относительно $f$, если
    \[f^{-1}(b) = \{a \in A : f(a) = b\}\]
\end{definition}

\begin{corollary}
    \begin{itemize}
        \item $f$ -- инъекция $\Leftrightarrow \forall b \in B: |f^{-1}(b)| \le 1$
        \item $f$ -- сюръекция $\Leftrightarrow \forall b \in B:|f^{-1}(b)| \ge 1$
        \item $f$ -- биекция $\Leftrightarrow \forall b \in B:|f^{-1}(b)| = 1$
    \end{itemize}
\end{corollary}

\begin{definition}[Сужение отображения]
    Пусть $f: A \to B$ и $A' \subset A$, тогда
    \begin{gather*}
        f|_{A'} :  A' \to B    \\
        a \mapsto f(a)
    \end{gather*}
\end{definition}

\begin{definition}[Образ подмножества]
    Пусть $f: A \to B$ и $M \subset A$, тогда
    \begin{gather*}
        f (M) = \{ f(m): m \in M\}\\
        f (A) = \Im A
    \end{gather*}
\end{definition}

\section{Композиция}
\begin{definition}
    Пусть $f: A \to B$ и  $g: B \to C$, тогда
    \begin{gather*}
        g \circ  f:A \to C\\
        a \mapsto g(f(a))
    \end{gather*} --- композиция $f$ и $g$
\end{definition}
\begin{example}
    \begin{gather*}
        f, g: \R \to \R\\
        f(x) = x+1\\
        g(x) = 2x\\
        g \circ f :\R  \to \R \qquad f \circ g: \R \to \R        \\
        x \mapsto 2x+2 \qquad x \mapsto 2x +1
    \end{gather*}
\end{example}

\section{Тождественное отображение}
\begin{definition}[Тождественное отображение]
    Пусть $M$ -- множество
    \begin{gather*}
        \id_M : M \to M \\
        m \mapsto m
    \end{gather*}
\end{definition}

\begin{definition}[Обратное отображение]
    Пусть $f: X \to Y$, тогда отображение $g: Y \to X$ называется обратным, если
    $g\circ f = \id_X, f \circ g = \id_Y$
\end{definition}

\begin{theorem}
    У $f: X \to Y$ есть обратное $\Leftrightarrow f $ -- биекция
\end{theorem}
\begin{proof}
    Прямое доказательство:
    Зададим обратное отображение $g: Y \to X$, так что $g\circ f  = \id_X$ и $f\circ g  = \id_Y$.
    Тогда $\forall y \in Y$ верно следующее:
    \begin{gather*}
        g(y) = x \qquad f^{-1}(y) = \{x\}\\
        (g \circ f) (x) = g(f(x)) = x\\
        (f \circ g) (y) = f(g(y)) = y
    \end{gather*}
    Обратное доказательство:
    Если $g\circ f = \id_X$ верно, то и $f$ -- инъекция
    \[f(x_1) = f(x_2) \implies g(f(x_1)) = g(f(x_2)) \implies x_1 = x_2\]
    Если $f \circ g  = \id_Y$ верно, то и $f$ -- сюръекция
    \[y \in Y \implies \exists x \in X: f(x)=y \implies f(g(y)) =y \]
\end{proof}
\end{document}