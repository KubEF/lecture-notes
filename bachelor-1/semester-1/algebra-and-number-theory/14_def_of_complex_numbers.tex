% !TeX root = ./main.tex
\documentclass[main]{subfiles}
\begin{document}
\part{Комплексные числа}
\chapter{Определение}
Комплексные числа --- это числа вида $a+bi$, где $i^2 = -1$ и $a,b \in \R$.
Тогда определим комплексные числа таким образом:
\begin{gather*}
    \C = \R \times\R \qquad (a,b)\\
    (a,b) + (a',b') = (a+a', b+b')\\
    (a,b) \cdot (a',b') = (aa'-bb',ab'+a'b)
\end{gather*}

\begin{theorem}
    $(\C, +, \cdot)$ -- коммутативное ассоциативное кольцо с 1.
\end{theorem}
\begin{proof}
    \begin{enumerate}
        \item Коммутативность сложения очевидна
        \item Ассоциативность сложения очевидна
        \item $(0,0)$ -- нейтральный по сложению
        \item $(-a,-b) = -(a,b)$
        \item Коммутативность умножения очевидна
        \item Ассоциативность -- непосредственная проверка
        \item Дистрибутивность -- непосредственная проверка
        \item $(1,0)$ -- нейтральный элемент
              \[(a,b) \cdot (1,0)=(a \cdot 1 - b \cdot 0, a \cdot 0 + 1 \cdot b) = (a,b)\]
    \end{enumerate}
\end{proof}
Элементы $\C$ -- комплексные числа

Если бы берем пару $(a,0)$:
\begin{gather*}
    (a,0) + (b,0) = (a+b,0)\\
    (a,0) (b,0) = (ab-0c\cdot 0, a \cdot 0 + 0 \cdot b) = (ab,0)
\end{gather*}
Тогда $\{(a,0) : a \in \R\}$ -- подкольцо. Будем отождествлять $(a,0)$ с $a$.
\begin{gather*}
    (0,b) = (0,1)(b,0)\\
    (a,b) = (a,0) + (0,b) = a (0,1)b
    \intertext{Обозначим пару $(0,1)$ за $i$ и получим запись}
    (a,b) = a + ib
    \intertext{Теперь справедливо следующее}
    i^2 = (0,1)(0,1) = (-1,0) = -1\\
    (a+ib)+(a'+ib') = a + a' + i(b+b')
\end{gather*}

\section*{Обозначения}
Задано \[z = a + ib, a,b \in \R \] тогда:
\begin{itemize}
    \item $a$ -- вещественная часть $z \Leftrightarrow \Re z =a$
    \item $b$ -- мнимая часть $z \Leftrightarrow \Im z = b$
    \item $i$ -- мнимая единица
\end{itemize}
Из такого отождествления следует, что $\R \subset \C$

$\C \setminus \R$ -- мнимые числа, т.е. числа вида $ib (b\in \R)$
\end{document}