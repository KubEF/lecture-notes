% !TeX root = ./main.tex
\documentclass[main]{subfiles}
\begin{document}
\chapter{Отношения на множестве}
\begin{definition}
    Отношения на множестве $M$ -- это подмножество в $M\times M$
\end{definition}
\begin{example}
    $\le$ на $\{1,2,3\}$  -- $\{(1,1), (1,2), (1,3), (2,2),(2,3),(3,3)\}$
\end{example}

\begin{definition}
    $R$ на $M$ называется рефлексивным, если \[\forall m \in M: (m,m) \in R\]
\end{definition}
\begin{definition}
    $R$ на $M$ называется симметричным, если
    \[\forall m,n \in M: (m,n) \in R \implies (n,m) \in R\]
\end{definition}
\begin{definition}
    $R$ на $M$ называется антисимметричным, если
    \[\forall m,n \in M:  (m,n)\in R, (n,m) \in R \implies m = n\]
\end{definition}
\begin{definition}
    $R$ на $M$ называется транзитивным, если
    \[\forall a,b \in M: (a,b) \in R, (b,c) \in R \implies (a,c) \in R\]
\end{definition}

\begin{definition}
    $R$ называется отношением эквивалентности, если оно рефлексивно,
    симметрично и транзитивно.
\end{definition}

\begin{definition}[Класс эквивалентности]
    Пусть $R$ -- отношения эквивалентности на $M$, $a \in M$.
    Класс $[a] = \{b\in M: bRa\}$%
    \footnote{Будем использовать запись $(a,b)\in R = aRb$}
\end{definition}

\begin{lemma}
    \[\forall m,n \in M: [m] = [n] \text{ или } [m]\cap[n] = \emptyset\]
\end{lemma}
\begin{proof}
    \begin{gather*}
        [m] \cap [n] \neq \emptyset\\
        \exists l \in [m] \cap [n] \implies lRm, lRn \implies mRl \implies mRn\\
        a\in [m] \implies aRm \implies aRn \implies a \in [n]\\
        \text{Таким образом } [m]\subset [n]. \text{ Аналогично } [n]\subset[m]
        \implies [m]=[n]
    \end{gather*}
\end{proof}

\begin{theorem}
    Пусть $R$ отношение эквивалентности на множестве $M$, тогда
    $M = \bigcup_{i \in I} C_i$, т.ч.  $C_i \cap C_j = \emptyset (i\neq j)$
    и $mRn \Leftrightarrow m,n \in C_i$ для некоторого $i$.
\end{theorem}
\begin{proof}
    \begin{gather*}
        C_i \text{-- всевозможные } [m] \in R\\
        M = \bigcup_{m\in M}[m] \text{ т.к. } m \in [m]\\
        a,b \in [m] \implies \begin{cases}
            aRm \\
            bRm
        \end{cases}
        \implies aRb\\
        \begin{rcases}
            a \in [m] \\
            b \in [n] \\
            aRb
        \end{rcases}
        \implies [m] = [n]\\
        \begin{rcases}
            bRn \\
            aRb \\
        \end{rcases}
        \implies aRn \implies a \in [m] \cap [n] \implies [m] \cap [n] \neq \emptyset
        \implies [m] = [n]
    \end{gather*}
\end{proof}

\begin{definition}
    Если $\sim$  -- отношение эквивалентности на $M$, то множество
    классов эквивалентности: $M/\sim$ -- фактормножество $M$ относительно $\sim$
\end{definition}
\end{document}