% !TeX root = ./main.tex
\documentclass[main]{subfiles}
\begin{document}
\chapter{Комплексное сопряжение и модуль}
Рассмотрим отображение:
\begin{gather*}
    \C \to \C\\
    a+bi \mapsto a-bi\\
    z \mapsto \overline{z}
\end{gather*}
Оно называется комплексным сопряжением

\begin{proposition}
    \begin{enumerate}
        \item $\overline{z+w} = \overline{z} + \overline{w}$
        \item $\overline{zw} = \overline{z} \cdot \overline{w}$
        \item $\overline{\overline{z}} = z$
        \item $z = \overline{z} \Leftrightarrow z \in \R$
        \item $z \cdot \overline{z} \in \R_{\ge 0}; z \cdot \overline{z} = 0\Leftrightarrow z = 0$
    \end{enumerate}
\end{proposition}
\begin{proof}
    \begin{gather*}
        z = a+bi \qquad w = c+di
        \intertext{Докажем 1:}
        \begin{multlined}
            \overline{(a+bi)+(c+di)} = \overline{(a+c) +(b+d)i} = (a+c) -(b+d)i =\\
            (a-bi) + (c-di) = \overline{a+bi} + \overline{c+di}
        \end{multlined}
        \intertext{Докажем 2:}
        \begin{multlined}
            \overline{(a+bi)(c+di)} = \overline{(ac-bd) + (ad+bc)i} = (ac-bd) - (ad+bc)i =\\
            (a-bi)(c-di) = \overline{a+bi} \cdot \overline{c+di}
        \end{multlined}
        \intertext{Доказательство 3 очевидно, докажем 4:}
        z = \overline{z} \Leftrightarrow \Im z = 0 \Leftrightarrow z \in \R
        \intertext{Докажем 5:}
        z\cdot \overline{z} = (a+bi)(a-bi) = a^2 - (bi)^2 = a^2 +b^2 \in \R_{\ge0}\\
        a^2 + b^2 =0 \Leftrightarrow a = b= 0
    \end{gather*}
\end{proof}

\begin{definition}
    Пусть $z \in \C$. Его модулем называется:
    \[|z| = \sqrt{z\cdot \overline{z}} \qquad |a+bi| = \sqrt{a^2 +b^2}\]
\end{definition}
\begin{remark}
    Для числа $a\in\R$ новый модуль совпадает со старым.
\end{remark}

\begin{proposition}
    $\C$ -- поле
\end{proposition}
\begin{proof}
    \begin{gather*}
        z \in \C, z \neq 0\\
        z\cdot \overline{z} = |z|^2 \neq 0 \implies z\cdot \frac{1}{|z|^2} \overline{z} =1
    \end{gather*}
\end{proof}

Теперь мы можем использовать деление:
\[\frac{z}{w}= z \cdot w^{-1} = w^{-1}z\]
а также возведение в степень и соответствующие свойства:
\begin{gather*}
    z^m = \begin{cases}
        \overbrace{z\cdot ... \cdot z}^m                & m > 0 \\
        1                                               & m = 0 \\
        \underbrace{z^{-1} \cdot ... \cdot z^{-1}}_{-m} & m < 0
    \end{cases} \qquad m \in \Z\\
    z^{m+n} = z^m + z^n\\
    z^{mn} = (z^m)^n\\
    (zw)^n = z^n w^n
\end{gather*}

\begin{proposition}[Свойства модуля]
    \begin{enumerate}
        \item $|zw|=|z||w|$
        \item Если $w\neq 0$, то $\left|\frac{z}{w}\right| = \frac{|z|}{|w|}$
    \end{enumerate}
\end{proposition}
\begin{proof}
    Докажем 1:
    \begin{gather*}
        |zw|^2 = (zw)\overline{(zw)} = zw \overline{z} \cdot \overline{w} =
        z\overline{z} w \overline{w} = |z|^2 |w|^2
        \intertext{Докажем 2:}
        z = \frac{z}{w}w \implies |z| = \left| \frac{z}{w}\right| \cdot |w|
    \end{gather*}
\end{proof}

\section{Геометрическое представление комплексного числа}
\noindent\begin{minipage}{0.45\textwidth}
    \import{figures/}{complex_addition_on_plane.pdf_tex}
\end{minipage}
\begin{minipage}{0.45\textwidth}
    \begin{gather*}
        z = a + bi\\
        z'= a'+b'i\\
        (a+bi) + (a'+b'i) = \\
        (a+a')+(b+b')i
    \end{gather*}
\end{minipage}

\noindent\begin{minipage}{0.45\textwidth}
    \import{figures/}{complex_modulo.pdf_tex}
\end{minipage}
\begin{minipage}{0.45\textwidth}
    \[\sqrt{a^2+b^2} = |z|\]
    Таким образом $|z|$ -- расстояние от точки изображающей число $z$ до начала координат.
\end{minipage}
\begin{remark}
    \[|\overline{z}| = |z|\]
\end{remark}

\begin{proposition}[Неравенство треугольника]
    Для $z,w \in \C$:
    \[||z| - |w|| \le |z+w| \le |z| + |w|\]
\end{proposition}
\begin{proof}
    Докажем правое неравенство:
    \begin{gather*}
        z+w = 0 \implies |z+w| \le |z| + |w| \text{ очевидно верно}\\
        z+w \neq 0 \qquad 1 = \frac{z}{z+w} + \frac{w}{z+w}\\
        1 = \Re 1 = \Re\left(\frac{z}{z+w}\right) + \Re \left(\frac{w}{r+w}\right)\\
        \sqrt{a^2 + b^2} \ge \sqrt{a^2} = |a| \ge a\\
        \Re\left(\frac{z}{z+w}\right) + \Re \left(\frac{w}{r+w}\right) \le \left|\frac{z}{z+w}\right| + \left| \frac{w}{z+w}\right|\\
        |z+w| \le |z| + |w|\\
        \intertext{Докажем левое неравненство:}
        z = (z+w) + (-w) \implies\\
        |z| \le |z+w| + |-w| \implies |z+w| \ge |z| - |w|\\
        \text{Аналогично } |z+w| \ge |w| - |z|\\
        \implies |z+w| \ge ||z|-|w||
    \end{gather*}
\end{proof}
\end{document}