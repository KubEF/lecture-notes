% !TeX root = ./main.tex
\documentclass[main]{subfiles}
\begin{document}
\chapter{Аксиомы счетности}
\section{Сепарабельность}
\begin{definition}
    $(X, \Omega)$ -- топологическое пространство.
    Говорят, что $X$ обладает второй аксиомой счетности, если у $X$ есть счетная база.
\end{definition}

\begin{definition}
    $(X, \Omega)$ -- топологическое пространство.
    $A \subset X$ называется всюду плотным в $X$, если $\Cl A =  X$
\end{definition}

\begin{definition}
    $X$ называется сепарабельным, если существует счетное всюду плотное множество в $X$.
\end{definition}

\begin{theorem}
    Из второй аксиомы счетности следует сепарабельность
\end{theorem}
\begin{proof}
    $\{U_i\}_{i \in I}^\infty$ -- счетная база.
    $x_i \in U_i \implies \{x_i\}_{i=1}^\infty$ -- счетное всюду плотное.
    Тогда $\Cl \{x_i\}_{i=1}^\infty = X$?

    Допустим противное:
    $y \in \Ex \{x_i\}_{i=1}^\infty$, значит $\Ex \{x_i\}_{i=1}^\infty = \bigcup_j U_{i_j} \ni x_{i_j}$.
    Внешность множества $\{x_i\}_{i=1}^\infty$ содержит $x_{i_j}$ -- противоречие.
\end{proof}
\begin{remark}
    Вторая аксиома счетности и сепарабельность -- топологические свойства.
\end{remark}

Здесь и далее в этой главе под словом <<счетное>> подразумевается <<не более чем счетное>>.

\begin{example}
    $X$ НБЧС, тогда $X$ сепарабельно.
\end{example}
\begin{example}
    $X$ -- антидискретное, тогда вторая аксиома счетности и сепарабельность есть.
\end{example}
\begin{example}
    $X$ -- дискретное:
    \begin{enumerate}
        \item $X$ счетное, тогда есть вторая аксиома счетности: база -- одноточечные подмножества
        \item $X$ более чем счетное, нет ни сепарабельности, ни второй аксиомы счетности.
              $\Cl A=A$ в дискретной топологии.
    \end{enumerate}
\end{example}
\begin{example}
    На $\R^n$ со стандартной топологией, есть вторая аксиома счетности и сепарабельность

    Рассмотрим $\mathfrak{B} = \{B(x, \epsilon): x, \epsilon > 0 \in \Q\}$.
    Это счетная база.
    Возьмем  $y_0 \in B(x_0, \epsilon)$, где $x_0, \epsilon$ не обязательно рациональные.
    $\rho\coloneqq \rho(x_0, y_0)$, тогда существует $z_0$ с рациональными координатами:
    $\rho(z_0, y_0) < \frac{\epsilon - \rho}{2}$, выберем $r\in \Q_+Ж \rho(z_0, y_0)< r < \frac{\epsilon- \rho}{2}$.
    Рассмотрим $B(z_0, r)$ такой что $y_0$ принадлежит ему.
    $B(z_0, r) \subset B(x_0, \epsilon)$.

    Сепарабельность: множество точек с рациональными координатами -- счетное всюду плотное.
\end{example}

\begin{remark}
    НЕ любое метрическое пространство обладает второй аксиомой счетности или сепарабельностью.

    Пусть $X$ континуальное, $\rho(x,y) = 1$ если $x \neq y$, тогда порождается дискретная топология.
\end{remark}

\begin{example}
    $X = \R$ с топологией Зариского (замкнутые, значит конечные).
    $X$ сепарабельно (любое бесконечное множество всюду плотно).
    Второй аксиомы счетности нет.

    Предположим, что она есть: $\{U_i\}_{i \in I}^\infty$ -- счетная база.
    $U_i = X \setminus \{x_{i_1}, ..., x_{i_{n_i}}\}$, тогда $\bigcup_{i=1}^\infty \{x_{i_1}, ..., x_{i_{n_i}}\}$ счетно.
    А $\R$ несчетно, тогда $\exists y \in U_i\ \forall i$.
    $U = X \setminus \{y\}$, $y \not \in U \neq \bigcup_j U_j \ni y$, значит счетной базы нет.
\end{example}

\begin{theorem}[Линделёфа]
    Если на $X$ есть вторая аксиома счетности, тогда из любого открытого покрытия $X$ можно выбрать НБЧС подпокрытие.
\end{theorem}
\begin{proof}
    $X = \bigcup_{i \in I} U_i, \mathfrak{B} = \{B_i\}_{i=1}^\infty$  -- счетная база.
    $\forall i\ U_i = \bigcup_{j} B_{j}$

    %% Неверное рассуждение?
    % Составим бесконечный двудольный граф: I доля -- множества $U_i$, II доля -- множества $B_j$,
    % проводим ребро если $U_i \supset B_j$.

    % $\forall B_j$ выберем один $U_i$ (если есть): $U_i \supset B_j$.
    % Выбрали НБЧС множество $U_i$.
    % Они покрывают всё $X$. (не подходит)

    $\{U_i\}$ вполне упорядочены (по теореме Цермело так можно).
    Рассмотрим $U_{i_1}$, отметим все $B_j \subset U_{i_1}$.
    Пусть $x_2 \not\in U_{i_1} \implies \exists U_{i_2} \ni x_2$
    тогда $U_{i_2} = \bigcup_j B_j$, отметим все такие $B_j$.
    На этом шаге мы отметили как минимум одно новое $B_j$.

    Продолжаем: $x_3 \not\in U_{i_1} \cup U_{i_2} \implies x_3 \in U_3 = \bigcup_j B_j$,
    отметили новое $B_j$.

    Таких шагов нельзя сделать более чем счетное количество.
    Таких  $U_{i_k}$  НБЧС количество, после которых новую точку,
    не входящую в их объединение, нельзя выбрать.
\end{proof}

\section{Секвенциальная компактность}
\begin{definition}
    $\{x_n\}_{n=1}^\infty$ -- последовательность в $X$.
    Говорим, что $x_0 \in \lim_{n \to \infty} x_n$ если
    $(\forall \epsilon >0\ \exists N \in \N : \forall n > N$ выполнено $\rho(x_n, x_0) < \epsilon$ или $x_n \in B(x_0, \epsilon))$

    $\forall U_{x_0}$ окрестность $\exists N \in \N: \forall n > \N \ x_n \in U_{x_0}$
\end{definition}

\begin{example}
    $\R$ с топологией Зариского.
    Пусть $x_i \neq x_j \implies \forall x_0\ x_n \to x_0$
    \[\forall U_{x_0} = X \setminus \{a_1, a_2, ..., a_n\}\ \exists N: \forall n > N\ x_n \neq a_n \implies x_n \in U_{x_0}\]
\end{example}
\begin{example}
    $\R$ с топологией типа Зариского: замкнутые = НБЧС.
    Если $x_i \neq x_j$, то $\not\exists x_0: x_n \to x_0$.
    Возьмем любой $x_0$ (считаем, что $x_0 \neq x_n$, иначе начнем последовательность с $x_{n+1}$)
    \[U_{x_0} = \R \setminus \{x_1, x_2, ..., \} \text{ -- открыто}\]
    $U_{x_0}$ не содержит ни одного члена последовательности.
\end{example}
\begin{remark}
    Если $X$ хаусдорфово, тогда предел не более чем единственный.
\end{remark}
\begin{proof}
    Допустим $x_0$ и $\tilde{x}_0$ -- пределы $x_n$.
    \[U_{x_0} \cap U_{\tilde{x}_0} = \varnothing\]
    Тогда с некоторого места:
    все $x_n \in U_{x_0}$ и  все $x_n \in U_{\tilde{x}_0}$.
    Но это противоречит с хаусдорфовостью.
\end{proof}

\begin{definition}
    $X$ называется секвенциально компактным пространством, если
    $\{x_i\}_{i = 1}^\infty \subset X\ \exists x_{n_k} \xrightarrow[k \to \infty]{} x_0$
    (из любой подпоследовательности можно выбрать сходящуюся).
\end{definition}
\begin{definition}
    $X$ обладает первой аксиомой счетности если $\forall x_0 \in X$, существует счетная база окрестностей $x_0$,
    т.е. $\exists \{B_{x_0, i}\}_{i = 1}^\infty: x_0 \in B_{x_0, i} \ x_0 \in \forall U$ открытому

    $\exists B_i: x_0 \in B_{x_0, i} \subset U \implies \{B_{x_0, i}\}_{i, x_0}$ -- база топологии.

    Это обобщение $B(x_0, \epsilon)$.
\end{definition}

\begin{remark}
    $X$ -- метрическое пространство, тогда $X$ обладает первой аксиомой счетности.
    $B(x_0, \epsilon)$, где $\epsilon \in \Q_+$.
\end{remark}

\begin{example}
    $\R$ с топологией Зариского не обладает первой аксиомой счетности.

    Допустим: есть счетное $\{U_{x_0, i}\}\ \forall x_0$.
    Рассмотрим $U_{x_0, 1}, U_{x_0, 2}$ и так далее, каждое из них НЕ содержит счетное число точек, в итоге счетный набор точек НЕ содержится в каком-то из этих множеств.
    Значит $\exists y \in U_{x_0, i}\ \forall i$.
    Возьмем $U = \R \setminus \{y\}$ -- окрестность $x_0$.
    $\not\exists U_{x_0, i} \subset U$ т.к. $U \not\ni y$.
\end{example}
\begin{remark}
    Из второй аксиомы счетности следует первая аксиома счетности.
\end{remark}

Дальнейший план:
\begin{enumerate}
    \item Компактность + первая аксиома счетности = cеквенциальная компактность

          Секвенциальная компактность + вторая аксиома счетности = компактность
    \item В метрических пространствах компактность равносильна cеквенциальной компактности
\end{enumerate}
\begin{example}
    Есть $\R$ с дискретной топологией. $\rho(x,y) = 1$, если $x \neq y$.

    $\forall U$ -- замкнуто и ограничено. $B(x_0, 2) = X \supset U$, значит $U$ -- ограничено,
    $U$ замкнуто, т.к. любое множество замкнуто в дискретной топологии.
    Но если $U$ бесконечное множество, тогда $U$ не компактно.
\end{example}
\end{document}