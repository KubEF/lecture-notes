% !TeX root = ./main.tex
\documentclass[main]{subfiles}
\begin{document}
\chapter{Числовые и функциональные ряды с комплексными слагаемыми}
\section{Числовые ряды}
\begin{gather*}
    i^2 = -1 \quad c = a + bi \quad a,b \in \vR \\
    \mathbb{C}\\
    \overline{c} = a -bi \\
    |c| = \sqrt{a^2 + b^2}
\end{gather*}
\begin{definition}[Ряд комплексных слагаемых]
     Cимвол \begin{gather*}
        \sum_{n=1}^\infty c_n , c_n \in \mathbb{C} \quad c_n = a_n + ib_n \tag{1}
    \end{gather*}
\end{definition}

\begin{definition}[Сходимость]
    \begin{gather*}
    \sum^\infty_{n=1} a_n \tag{2}\\
    \sum^\infty_{n=1} b_n \tag{3}\\
    \text{Будем говорить, что ряд (1) сходится, если сходятся (3) и (2)} \\
    \sum^\infty_{n=1} c_n \stackrel{def}{=} \sum^\infty_{n=1} a_n + i \sum^\infty_{n=1} b_n \tag{4}\\
    \end{gather*}
\end{definition}
Если сходится ряд из $c_n$, то сходится и ряд из комплексно сопряжённых. 

\begin{gather*}
    p \in \mathbb{C} \quad p \ne 0 \quad p = s + it \\
    \sum^\infty_{n=1} pc_n = p \sum_{n=1}^\infty c_n \tag{5}\\
    pc_n = (s+it)(a_n + ib_n) = (sa_n-tb_n) + i(sb_n+ ta_n) 
\end{gather*}
\begin{definition}[Абсолютная сходимость]
    Ряд (1) сходится абсолютно, если сходится 
    \[\sum_{n=1}^\infty |c_n| < \infty \tag{6} \]
\end{definition}
\begin{theorem}
    Ряд (1) сходится тогда и только когда, когда сходятся ряды (2) и (3)
\end{theorem}
% \begin{proof}
%     $\implies$ \\
%     По признаку сравнений рядов
%     \begin{gather*}
%         |a_n| \leq |c_n| \quad |b_n| \leq |c_n| \\
%         |c_n| \leq |a_n| + |b_n| \\
%         \Leftarrow \text{ очевидно }
%     \end{gather*}
% \end{proof}
\begin{corollary}
    Ряд (1) сходится, если он сходится абсолютно.
\end{corollary}

\section{Функциональные ряды}
\begin{definition}[Комплексно-значная функция]
    Будем говорить, что на множестве $E \ne \varnothing$ задана комплексно-значная функция, если
    \begin{gather*}
        u(x) \quad v(x) :  E \rightarrow \vR \\
        w(x) = u(x) + i v(x)
    \end{gather*}
\end{definition}

\begin{gather*}
    x \in E \quad \{ w_n(x) \}^\infty_{n=1}  \quad w_n(x) \in \mathbb{C}\\
    w_n(x) = u_n(x) + i v_n(x) \\
\end{gather*}
\begin{definition}[Равномерное стремление]
    Будем говорить, что $w_n$ равномерно стремится к $\phi(x)$
    \[ w_n(x) \darrow{\underset{x \in E}{n \to \infty}} \phi(x) \] если
    \begin{gather*}
        u_n(x) \darrow{\underset{x \in E}{n \to \infty}} \alpha(x) \text{ и }
        v_n(x) \darrow{\underset{x \in E}{n \to \infty}} \beta(x) \tag{7}\\
    \end{gather*}
\end{definition}
\begin{theorem}
    \begin{gather*}
        (7) \Leftrightarrow \quad \forall \varepsilon : \forall n > N \text{ и  } \forall x \in E \\
        |w_n(x) - \phi(x) | < \varepsilon \tag{8}
    \end{gather*}
\end{theorem}
\begin{longProof}
    $\Leftarrow$ проводится точно так же, как в предыдущей теореме\\
    $ \Rightarrow$ Если выполнено (7), то для  определения равномерной сходимости:
    \begin{gather*}
        \exists N_1 \text{ для } \{u_n(x) \}^\infty_{n=1} \\
        \exists N_2 \text{ для } \{v_n(x) \}^\infty_{n=1} \\
        N = \max{N_1, N_2} \\
        |w_n(x) - \phi(x)| \leq \underbrace{| u_n(x) - \alpha(x) |}_{< \frac{\varepsilon}{2}} + \underbrace{|v_n(x) - \beta(x)|}_{< \frac{\varepsilon}{2}}
    \end{gather*}
\end{longProof}

\begin{gather*}
    x \in E \quad \{c_n(x) \}^\infty_{n=1} \quad c_n(x) \in \mathbb{C} \\
    \sum^\infty_{n=1} c_n(x) \tag{9}\\
    S_n(x) = \sum^n_{k=1} c_k(x) 
\end{gather*}
\begin{definition}[Равномерная сходимость]
    Будем говорить, что ряд (9) сходится равномерно на множестве $E$, если
    \begin{gather*}
        \exists \phi(x) \quad S_n(x) \darrow{\underset{x \to E}{n \to \infty}} \phi(x) \tag{10}\\
    \end{gather*}
\end{definition}
\begin{theorem}[Признак Вейерштрасса равномерной сходимости функционального ряда с комплексными слагаемыми]
    Имеется функциональный ряд (9) с комплексными слагаемыми. 
    \begin{gather*}
        \text{ Имеется } A_N \geq 0 \\
        \sum^\infty_{n=1} A_n < \infty \tag{11}\\
        \forall n \text{ и } \forall x \in E |c_n(x)| \leq A_n \tag{12} \\
        \text{тогда ряд (9) сходится равномерно}
    \end{gather*}
\end{theorem}

\begin{longProof}
    \begin{gather*}
        \forall n \text{ и } \forall x \in E \quad |c_n(x)| \leq A_N \tag{12} \\
        c_n(x) = a_n(x) = ib_n(x) \\
        (12) \implies |a_n(x) | \leq A_N \quad \forall n \text{ и } \forall x \in E \tag{13}\\
        (12) \implies |b_n(x) | \leq A_N \quad \forall n \text{ и } \forall x \in E \tag{14}\\
    \end{gather*}
    (13), (14) с вещественными слагаемыми, (11) и (13) $\implies$
    \begin{gather*}
        \sum^\infty_{n=1} a_n(x) \text{ сходится равномерно на } E \tag{15}\\
        \sum^\infty_{n=1} b_n(x) \text{ сходится равномерно на } E \tag{16}\\
        s=S_n(x) = \sum^n_{k=1} a_n(x) + i \sum^n_{k=1} b_k(x)
    \end{gather*}
    Поскольку частичные суммы выглядят таким образом, то по определению равномерной сходимости ряда, $a_n$
    равномерно стремятся к некоторой функции. Суммы мнимых частей $b_n$ тоже равномерно стремятся к какой-то функции.
    По определению равномерной сходимости получаем, что требовалось доказать.
\end{longProof} 
\begin{gather*}
    \alpha \in \mathbb{C} \quad \{ c_n \} ^\infty_{n=0} \\
    c_n \in \mathbb{C} \\
    z = x + iy \\
    z - \alpha
\end{gather*}
\begin{definition}[Степенной ряд]
    \[ c_0 + \sum^\infty_{n=1} c_n(z-\alpha)^n \tag{17} \]
\end{definition}
\begin{lemma}[Абеля]
    Предположим, что ряд (17) сходится при фиксированном $z_0 \ne \alpha$ (это ряд с комплексными 
    слагаемыми)
    \begin{gather*}
        R = |z_0 - \alpha| > 0 \implies \\
        (17) \text { сходится }\forall z : |z-\alpha| < R \tag{18}\\
        \text{ фиксируем } 0 < r < R  \\
        \text{ при } |z-\alpha| \leq r \text{ ряд сходится равномерно } \tag{19}
    \end{gather*} 
\end{lemma}
Пока отвлечемся от доказательства, рассмотрим ряд.
    \begin{gather*}
        w_n = p_n + iq_n \\
        \sum^\infty_{n=1} w_n \text{ сходится } \implies
        \sum^\infty_{n=1} q_n \text{ сходится и } \sum^\infty_{n=1}p_n \text{ сходится } 
        \intertext{по необходимому признаку сходимости}
        p_n \underset{n \to \infty}{\longrightarrow} 0 \implies \exists M_1 : |p_n| \leq M_1 \forall n \\
        q_n \underset{n \to \infty}{\longrightarrow} 0 \implies \exists M_2 : |q_n| \leq M_2 \forall n \\
        \implies |w_n| \leq |p_n| + |q_n| \leq M_1 + M_2
    \end{gather*}
    Провели такое рассуждение, показав, что сумма ограниченна
\begin{longProof}
    Напомню свойство комплексных чисел  
    \[|z_1| \cdot |z_2| = |z_1 \cdot z_2| \]
    В условии сказано, что 
    \begin{gather*}
        c_0 + \sum^\infty_{n=1} c_n(z_0-\alpha)^n \text{ сходится } \implies\\
         \text{по предыдущему рассуждению} \exists M : \forall n \\
        |c_n(z_0 - \alpha)^n | \leq M \tag{20} \\
        (20) : |c_n| \cdot |z_0 - \alpha|^n \leq M \Leftrightarrow |c_n| \leq \frac{M}{R^n} \tag{21}\\
        0 < \frac{|z-\alpha|}{R} = q < 1 \\
        (21) \implies |c_n(z-\alpha)^n | = |c_n| \cdot |z-\alpha|^n \leq \frac{M}{R^n} \cdot |z-\alpha|^n = Mq^n \tag{22}\\
        \sum^\infty_{n=1} Mq^n = \frac{Mq}{1-q} \\
    \end{gather*}
    Получаем, что ряд (17) сходится абсолютно, значит, он просто сходится
    \begin{gather*}
        \frac{r}{R} = q_0 < 1
        \text{Предположим, что выполнено(19)} \\
        (19),(21) \implies |c_n(z-\alpha)^n| < |c_n| \cdot |z-\alpha|^n \leq \frac{M}{R^n} \cdot r^n = Mq_0^n \tag{23}\\
    \end{gather*}
    Ряд из $Mq_0^n$ сходится так как $q_0< 1$.  $q_0$ не зависит от $z$, 
    если выполнено соотношение (19). Значит мы получаем, что сходимость ряда равномерная.
\end{longProof}

\section{Радиус сходимости и круг сходимости степенного ряда}
   \[ c_0 + \sum^\infty_{n=1} c_n(z-\alpha)^n  \quad R \quad B\tag{1} \]
\begin{enumerate}
    \item \begin{gather*}
        (1) \text{ сходится лишь при } z = \alpha \\
        R \stackrel{def}{=} 0 \quad B \stackrel{def}{=} \varnothing
    \end{gather*}
    \item \begin{gather*}
        (1) \text{ сходится при } \forall z \in \mathbb{C} \\
        R \stackrel{def}{=} +\infty \quad B \stackrel{def}{=} \mathbb{C}
    \end{gather*}
    \item \begin{gather*}
        \exists z_1 \ne \alpha : (1) \text{ сходится  в } z_1 \text{ и } \exists z_2 : (1) \text{ не сходится в } z_2 \\
        R_2 = |z_2 - \alpha| \implies \forall z : |z-\alpha| > R_2 \quad (1) \text{ расходится } \\
        \text{ т. ч. } E = \{ |z-\alpha|: (1) \text{ сходится в } z \} \\
        |z_1 - \alpha| \in E \quad \text{ и } E \text{ ограниченно сверху} \quad \forall \rho \in E \quad \rho \leq |z_2 - \alpha|
    \end {gather*}
\end{enumerate}
Понятно, что (3) дополняет (1) и (2)
\begin{definition}
    \[ R \stackrel{def}{=} \sup E \quad B \stackrel{def}{=} \{ z: |z-\alpha| < R \} \]
\end{definition}
\subsection*{Свойства сходимости} 
\begin{theorem}
    \begin{gather*}
        \forall z \in B \quad (1) \text{ сходится в } z \tag{2} \\
        \forall z \notin \overline{B} \text{ замыкание } \quad (1) \text{ расходится в } z \tag{3} 
    \end{gather*}
\end{theorem}

\begin{longProof}
    \begin{gather*}
        \rho = |z-\alpha| \quad \rho < R \implies \exists \rho_1, \rho < \rho_1 < R \tag{4\prime}\\
        \rho_1 \in E \tag{4} \\
        \text{ т.е. } \exists z_1 : |z_1-\alpha| = \rho_1 \tag{5} \\
        \text{ и } (1) \text{ сходится в } z_1 \tag{6} 
    \end{gather*} 
    В точке $z_1$ применима лемма Абеля.
    \begin{gather*}
        (4\prime), (4), (5),(6) \implies (2) \\
        z_0 \notin \overline{B} \Leftrightarrow |z_0 - \alpha| > R \\
        \text{ если бы (1) сходится в } z_0
        \forall z : |z-\alpha| < |z_0 - \alpha| \quad (1) \text{ сходится в } z \\
        \tilde{z} : |\tilde{z} - \alpha| = \frac{1}{2} (R + |z_0 - \alpha|) \quad |\tilde{z} - \alpha| < |z_0 - \alpha| \quad (1) \text{ сходится в } \tilde{z} \\
        |\tilde{z} - \alpha| > R \quad |\tilde{z} - \alpha | \in E
    \end{gather*}
    Получили противоречие с определением супремума
\end{longProof}

\subsection*{Непрерывность}
\begin{theorem}
    \begin{gather*}
        0 < r < R \quad \text{ если применить лемму Абеля к }  z_0\\
        \text{ ряд равномерно сходится в круге }\{ z: |z-\alpha| \leq r \} \\
         r < \rho < R \quad |z_0 - \alpha| = \rho \\
        z = x + iy \Leftrightarrow (x,y)  \quad \quad z - \alpha = x - \beta + i(y - \gamma)\\
        |z| = ||(x,y)|| \quad \quad \alpha = \beta + i\gamma 
        \intertext{тогда степенной ряд (1) имеет вид}
        c_0 + \sum^\infty_{n=1} u_n(z) + i \sum^\infty_{n=1} v_n(z)
    \end{gather*}
    $c_n(z-\alpha) = u_n(z) + iv_n(z)$, где $u_n$ и $v_n$ -- непрерывные функции, заданные на $\mathbb{C}$.
    Это ряды из непрерывных функций, поэтому их сумма тоже непрерывная функция.
\end{theorem}

\begin{definition}[Непрерывность]
    \[w(z) = A(Z) + iB(z) \] будем называть непрерывной, если непрерывны мнимая и вещественная часть
\end{definition}

(1) непрерывна на $B$ \\
Получено следующее утверждение: сумма степенного ряда по степеням $z-\alpha$ непрерывна в круге сходимости.

\subsection{Вычисление радиуса сходимости степенного ряда}
Имеется ряд (1).
    \[t = \overline{\underset{n \to \infty}{\lim}} \sqrt[n]{|c_n|} \tag{7} \]
Он может быть равен $+\infty$ \\
Теперь хотим определить \[R_0 = \begin{cases}
    0 \quad \text{ если } t = +\infty \\
    +\infty \quad \text{ если } t = 0  \\
    \frac{1}{t} \quad \text{ если } 0 < t < \infty
\end{cases} \tag{8} \]
\[ R_0 = R \tag{9} \]
\begin{theorem}
    $R_0$, опредленное в соотношении (8), является радиусом сходимости.
\end{theorem}
\begin{longProof}
    Надо рассмотреть 3 случая. Случаи $0$ и $+\infty$ мы рассматривать не будем, потому что это они простые. Рассмотрим
    более интересный случай
    \begin{gather*}
        |z_0 - \alpha| > R_0 \tag{10} \\
        \varepsilon = |z_0 - \alpha| - R_0 > 0 \\
        \delta_0 = \frac{t^2 \varepsilon}{1+t\varepsilon} \tag{11}
    \end{gather*}
    В силу свойств верхнего предела (это было еще в далёком первом семестре)
    \begin{gather*}
        \exists \{ n_k \}^\infty_{k=1} : \sqrt[n_k]{|c_{n_k}|} > t - \delta_0 \tag{12} \\
        (12) \Leftrightarrow |c_{n_k}| > (t-\delta_0)^{n_k} \tag{12\prime} \\
        (12^\prime) \implies |c_{n_k} (z_0-\alpha)^{n_k} | > (t-\delta_0)^{n_k} \cdot (R_0 + \varepsilon)^{n_k} = \\
       = ((t-\delta_0)(R_0 + \varepsilon))^{n_k}
    \end{gather*}
    Посчитаем теперь отдельно выражение в скобке
    \begin{multline*}
        (t-\delta_0)(R_0+ \varepsilon) =  \left (t- \frac{t^2\varepsilon}{1+t\varepsilon}\right ) \left (\frac{1}{t} + \varepsilon  \right ) = \\
        = \frac{t+t^2\varepsilon-t^2\varepsilon}{1+t\varepsilon} \cdot \frac{1 + t \varepsilon}{t} = 1\\
        (13) \implies c_{n_k} (z_0 - \alpha)^{n_k} \text{ не стремится к 0 при } k \to \infty
    \end{multline*}
    Ряд в точке $z_0$ разошёлся

\begin{gather*}
    |z_r - \alpha| < R_0 \quad \varepsilon = R_0 - |z_1 - \alpha| > 0 \\
    \delta_1 = \frac{1}{2} \cdot \frac{t^2 \varepsilon}{1 - t \varepsilon}
\end{gather*}
Опять же по свойствам верхнего предела из первого семестра. 
\begin{gather*}
    \exists N_1 : \forall n > N_1 \quad \sqrt[n]{|c_n|}  < t + \delta_1 \tag{14} \\
    (14) \Leftrightarrow |c_n| < (t+ \delta_1)^n \tag{14\prime} \\
    (14\prime) \implies |c_n| |z_1 - \alpha|^n < (t+\delta_1)^n (R_0 - \varepsilon) =
    ((t+\delta_1)(R_0 - \varepsilon))^n \tag{15} \\
    (t+2\delta_1)(R_0 - \varepsilon) = \left ( t + \frac{t^2\varepsilon}{1-t\varepsilon} \right ) \left ( \frac{1}{t} - \varepsilon \right ) =
    \frac{t - t^2\varepsilon + t^2\varepsilon}{1-t\varepsilon} \cdot \frac{1 -t \varepsilon}{t} = 1 \implies \\
    \implies (t+\delta_1)(R_0 - \varepsilon) = 1 - \delta_1(R_0 - \varepsilon) \stackrel{def}{=} q < 1 \tag{16}
\end{gather*} 
$q$ не зависит от $n$, поэтому
\begin{gather*}
    (15), (16) \implies |c_n(z-\alpha)^n| < q^n \\
    \sum_1^\infty q^n = \frac{q}{1-q} \\
    \intertext{получим}
    \forall z : |z-\alpha| > R_0 \quad  (1) \text{ расходится } \\
    \forall z : |z-\alpha| < R_0 (1) \quad \text{ сходится } \\
    (18) \implies R = R_0
\end{gather*}
Если бы $R \ne R_0$, то один из них больше другого. Возьмём $z$ между ними. 
Тогда посмотрим на радиус сходимости $R$ в точке $z$, там
будет расходимость, хотя должна быть сходимость.
\end{longProof}

\section{Вещественные степенные ряды} 

\[ \vR \leftrightarrow x + i0 \]
\begin{definition}[Вещественный степенной ряд]
    \begin{gather*}
        c_n \in \vR \quad n \geq 0 \quad \alpha \in \vR  \\
        c_0 + \sum^\infty_{n=1} c_n(x-\alpha)^n \quad x \in \vR \tag{1} \\
    \end{gather*}
\end{definition}

Рассмотрим
\begin{gather*}
    (2) c_0 + \sum^\infty_{n=1}  c_n (z-\alpha)^n \quad z \in \mathbb{C} \\
    R, B \text{ -- круг сходимости } (2) \\
    B = \{ z: |z-\alpha| < R  \} \quad R > 0 \\
    I = B \cap \vR = (\alpha - R, \alpha + R) \tag{3}
\end{gather*}

\begin{definition}[Интервал сходимости]
    $I$, которое может быть пустым, называется интервалом сходимости вещественного ряда (1).
\end{definition}

\begin{theorem*}
    Он обладает следующими свойствами: \\
  если $x_0 \in I \implies (1) $ сходится в $x_0$
    если $x_1 \notin \overline{I}  \implies (1) $ расходится в $x_1$
\end{theorem*}


\begin{proof}
    \[ x_0 \in I \implies x_0 \in B \implies \text{ ряд (2) сходится в }  x_0 \Leftrightarrow (1) \text{ сходится в } x_0 \] 
    \[ x_1 \notin \overline{I} \Leftrightarrow |x_1 - \alpha| > R \Leftrightarrow x_1 \notin \overline{B} \implies (3) \text{ расходится в } x_1  \] 
\end{proof}

\begin{gather*}
    R = \frac{1}{t} \\
    t = \overline{\underset{n \to \infty}{\lim}} \sqrt[n]{|c_n|}
\end{gather*}
Есть еще одна формула для радиуса сходимости.
\begin{theorem*}
    \begin{gather*}
        c_1 \in \mathbb{C} \quad c_n \ne \forall n \\
        \exists \underset{n \to \infty}{\lim} \frac{|c_n|}{|c_{n+1}|} = R
    \end{gather*}
\end{theorem*}
\begin{proof}
        Доказательство аналогично. Тут случай проще потому что предполагается,
         что предел существует.
\end{proof}

\begin{theorem}
    \begin{gather*}
        \forall r \quad 0 < r < R \quad (1) \text{ сходится равномерно при } |x-\alpha| \leq r \implies \\
        \implies c_0 + \sum^\infty_{n=1} c_n(x-\alpha)^n \in C(I) \quad I = [\alpha - r, \alpha + r] 
    \end{gather*}
\end{theorem}

\begin{theorem}[Абеля]
    \begin{gather*}
        (1) \text{ сходится в } \alpha - R \implies (1) \text{ равномерно сходится на } [\alpha - R,\alpha]  \quad \\
        \text{ сумма (1) } \in C([\alpha - R,\alpha]) \\
        (1) \text{ сходится в } \alpha + R \implies (1) \text{ равномерно сходится на } [\alpha,\alpha+R]  \quad \\
        \text{ сумма (1) } \in C([\alpha , \alpha +R])
    \end{gather*}
\end{theorem}

\begin{longProof}
    Оба случая доказываются аналогично, рассмотрим первый. $c_0$ и так добавляется, его писать не будем
    \begin{gather*}
        \sum^\infty_{n=1} c_n((\alpha-R)-\alpha)^n \text{ сходится } \\
        \sum^\infty_{n=1} (-1)^n c_n R^n \text{ сходится } \tag{5} \\
        \alpha - x > 0 \quad \quad \alpha - R < x < \alpha \\
        c_n(x-\alpha)^n = (-1)^n c_n (\alpha - x)^n = (-1)^n c_n R^n \cdot \left ( \frac{\alpha-x}{R} \right )^n \tag{6}
    \end{gather*}
    Хотим применить признак Абеля равномерной сходимости функциональных рядов \\
    \begin{gather*}
        u_n(x) = (-1)^n c_n R^n \\
        \sum^\infty_{n=1} u_n(x) v_n(x) \\
        \sum^\infty_1 u_n(x) \text{ сходится равномерно } \\
         v_n(x) \text{ монотонна } \forall x \\
         |v_n(x)| \leq M \\
         0 \leq v_n(x) = \left ( \frac{\alpha-x}{R} \right ) \leq 1 \\
         (6) \implies (1) \text{ равномерно сходится на } [\alpha - R, \alpha] 
    \end{gather*}
\end{longProof}

\section{Производная вещественного степенного ряда}
\begin{theorem}
     Пусть $S(x)=\su c_n(x-x_0)^n,\ x_0\in\R, c_n\in\R, n\geq 1$ - вещественный степенной ряд с радиусом сходимости $R,
      0<R\leq+\infty,$ и интервалом сходимости $I$. Тогда $\forall x\in I\ \exists S'(x)$ и справедлива формула
       \[ S'(x)=c_1+\su[\nu=1] c_{\nu+1}(x-x_0)^\nu (\nu+1)\tag{1} \]
\end{theorem}
\begin{longProof}
     Выберем $\forall r,\ 0<r<R$\\
Пусть \[ A(x)=\su[n=2] nc_n(x-x_0)^n\tag{2} \]
Вычислим радиус сходимости степенного ряда $A(x).$ Пусть $R_1$ - его радиус сходимости.\\
Пусть $t_1=\overline{\Lim}\sqrt[n]{n|c_n|},\ t=\overline{\Lim}\sqrt[n]{|c_n|}$\\
Пусть $\alpha_n=\sqrt[n]{n},$ тогда $\alpha_n\underset{n\to\infty}{\rightarrow} 1.$\\ Справедливо следующее утверждение.\\
\textbf{Лемма} Пусть $b_n\geq 0, \alpha_n\underset{n\to\infty}{\rightarrow}\alpha,\ \alpha_n,\ \alpha>0.$\\
Тогда \[ \overline{\Lim} \alpha_n b_n=\alpha\overline{\Lim} b_n\tag{3}\]
Доказательство (3) следует из свойства пределов и верхних пределов, примем его как факт.\\
Применим лемму с $\alpha_n=\sqrt[n]{n}, b_n=\sqrt[n]{|c_n|}$, тогда получим 
\[ t_1=\overline{\Lim} \sqrt[n]{n}\cdot\sqrt[n]{|c_n|}=1\cdot\overline{\Lim}\sqrt[n]{|c_n|}=t\tag{4} \]
Соотношение (4) и связь чисел $t$ и $t_1$ с радиусами сходимости $R$ и $R_1$ показывает, что $R=R_1.$
 Если $I_1$ - интервал сходимости ряда (2), то из предыдущего равенства следует $I_1=I,$ поэтому из свойства интервала сходимости следует,
  что $A(x)$ сходится при $\forall x\in I.$\\ Учтём, что при $x\ne x_0, x\in I$ имеем тождество 
  \[ \su[\nu=1] c_{\nu+1}(x-x_0)^{\nu}\cdot(\nu+1)=\frac{1}{x-x_0}A(x),\tag{5} \]
что проверяется почленным умножением слагаемых в $A(x).$\\ По свойствам радиуса сходимости степенного ряда ряд $A(x)$ равномерно
 сходится при $\{ x:|x-x_0|< r\}$, следовательно, формула (5) показывает, что ряд $\su[\nu=1]
  c_{\nu+1}(x-x_0)^{\nu}\cdot(\nu+1)$ равномерно сходится при\\ $\{ x: \frac{r}{2}\leq|x-x_0|< r\}$,
   поскольку при $|x-x_0|\geq\frac{r}{2}$ имеем $|\frac{1}{x-x_0}|\leq\frac{2}{r}.$\\ 
% \textbf{*}(от vs9lh: Как доказать равномерную сходимость (5):\\
%  $A(x)$ - равномерно сходится при $\{ x:|x-x_0|\leq r\}\implies\\ \forall \varepsilon\ \exists N: \forall m>n>N\ \forall\ x\in \{ x:|x-x_0|\leq r\}\
%   |\sum\limits_{k=n+1}^m kc_k(x-x_0)^k|\leq \varepsilon$\\
% $|\sum\limits_{k=n+1}^m \frac{1}{x-x_0}\cdot kc_k(x-x_0)^k|\leq \frac{2}{r}\cdot|\sum\limits_{k=n+1}^m kc_k(x-x_0)^k|<
%  \frac{2}{r}\cdot\varepsilon \implies$ (5) равномерно сходится при $\{ x: \frac{r}{2}\leq|x-x_0|\leq r\}$ )\textbf{*}
% \\
 Но $\su[\nu=1] c_{\nu+1}(x-x_0)^\nu\cdot(\nu+1)$ - степенной ряд, поэтому, если он сходится равномерно при $\{ x: \frac{r}{2}\leq|x-x_0|< r\}$,
 то он сходится равномерно при $\{ x: |x-x_0|< r\}$.\\
Пусть $v_n(x)=c_n(x-x_0)^n,$ тогда $v'_1(x)=c_1, v'_n(x)=nc_n(x-x_0)^{n-1},\ n\geq 2.$ Из вышесказанного следует, что ряд $\su v_n'(x)$ равномерно сходится при $\{ x: |x-x_0|< r\}$,
 а ряд $\su v_n(x)$ при $\{ x: |x-x_0|< r\}$ сходится.\\
Применяя теорему о производной функционального ряда, получаем, что при этих $x\ \exists S'(x)$ и справедливо равенство 
\[ S'(x)=\su v_n'(x)=c_1+\su[\nu=1]c_{\nu+1}(x-x_0)^{\nu}(\nu+1), x\in (x_0-r, x_0+r)\tag{6} \]
Поскольку $r<R$ - любое, то (6) доказывает теорему.
\end{longProof}
Учитывая утверждение предыдущей теоремы, в дальнейшем будем рассматривать степенные ряды вида $c_0+\su c_n(x-x_0)^n,$ производная 
такого выражения является в силу этого утверждения аналогичным выражением.
\subsection{Последующие производные степенного ряда}
Пусть $S(x)=c_0+\su c_n(x-x_0)^n$ - степенной ряд с добавленным слагаемым $c_0,$ пусть $R>0$ - его радиус сходимости,
 $I=(x_0-R,x_0+R)$. Было доказано, что \[ S'(x)=c_1+\su[\nu=1](\nu+1)c_{\nu+1}(x-x_0)^\nu;\ x\in I \]
ряд для $S'(x)$ - степенной ряд с добавленным слагаемым $c_1,$ поэтому при $x\in I$ у него существует производная,
 т.е. при $x\in I$ существует вторая производная \[ S''(x)=2c_2+\su[\mu=1](\mu+2)(\mu+1)c_{\mu+2}(x-x_0)^\mu\tag{7} \]
Ряд (7) опять имеет $I$ в качестве интервала сходимости. Этот процесс продолжается далее, и, тем самым, доказана следующая теорема.\\
\begin{theorem} Пусть $S(x)=c_0+\su c_n(x-x_0)^n, I\ne\varnothing$ - его интервал сходимости, $m\geq 1.$\\
Тогда $\forall x\in I\ \exists\ S^{(m)}(x)$ и справедлива формула
\[ S^{(m)}(x)=\su c_n((x-x_0)^n)^{(m)}\tag{8} \]
\end{theorem}
\subsection{Степенной ряд как ряд Тейлора}
Заметим, что справедливы следующие равенства:
\[ ((x-x_0)^n)'=n(x-x_0)^{n-1}, ((x-x_0)^n)''=n(n-1)(x-x_0)^{n-2},\ \dots \]
\[ ((x-x_0)^n)^{(m)}=n(n-1)\cdots(n-m+1)(x-x_0)^{n-m},\ m<n \]
\[ ((x-x_0)^n)^{(n)}=n!,\ ((x-x_0)^n)^{n+k}\equiv 0,\ k\geq 1 \]
Отсюда следует, что \[ ((x-x_0)^n)^{(m)}|_{x=x_0}=\begin{cases}
   0,& n\ne m
   \\
   n!,& n=m
   \end{cases}\tag{9} \]
Пусть \[ S(x)=c_0+\su c_n(x-x_0)^n,\ x\in I\tag{10} \]
Тогда (10) $\implies S(x_0)=c_0.$ Если $m\geq 1,$ то $(8),\ (10)\implies$ \[ S^{(m)}(x_0)=0+c_m\cdot m!,\  c_m=\frac{S^{(m)}(x_0)}{m!}\tag{11} \]
Из (11) находим, что \[ S(x)=S(x_0)+\su \frac{S^{(n)}(x_0)}{n!}(x-x_0)^n\tag{12} \]
Ряд (12) называется \textbf{рядом Тейлора} для функции $S(x).$

\subsection{Интегрирование вещественного степенного ряда}
\begin{theorem}
     Пусть $S(x)=c_0+\su c_n(x-x_0)^n$ - вещественный степенной ряд с интервалом сходимости $I\ne\varnothing.$ Пусть $a,b\in I.$ Тогда 
\[ \int_a^b S(x)dx=c_0(b-a)+\su c_n\frac{(b-x_0)^{n+1}-(a-x_0)^{n+1}}{n+1}\tag{13} \]
\end{theorem}
\begin{proof} Пусть $R$ - радиус сходимости данного ряда.\\
По условию, $R>0,\ |a-x_0|,\ |b-x_0|<R$. Выберем $r:\\ max(|a-x_0|,|b-x_0|)<r<R.$\\
Тогда данный ряд равномерно сходится при $\{ x:|x-x_0|\leq r\},$ в частности он равномерно сходится при $x\in [a,b].$\\
Тогда равенство (13) следует из теоремы об интегрировании функционального ряда.
\end{proof}

\subsection{Приложения предшествующей теоремы}
1. При $|x|<1$ имеем равенство \[\frac{1}{1+x}=1+\su(-1)^nx^n\tag{14}\]
Радиус сходимости ряда (14) равен 1, поэтому, полагая $a=0, b=x,\\ x_0=0,$ из (13) и (14) получаем
\[ ln(1+x)=\int_0^x\frac{dx}{1+x}=x+\su(-1)^n\frac{x^{n+1}}{n+1}=\su \frac{(-1)^{n-1}}{n}x^n\tag{15} \]
При $x=1$ ряд (15) имеет вид $\su (-1)^{n-1}\frac{1}{n},$ который сходится по признаку Лейбница.\\
По теореме Абеля ряд (15) равномерно сходится при $x\in [0,1]$, поэтому по теореме о непрерывности суммы равномерно сходящегося функционального ряда и из (15) получаем
\[ \su (-1)^{n-1}\cdot\frac{1}{n}=\lims{x\to 1-0}\su (-1)^{n-1}\frac{x^n}{n}=\lims{x\to 1-0} ln(1+x)=ln2\tag{16} \]
2. При $|y|<1$ положим в (14) $x=y^2,$ тогда (14) $\implies$ \[ \frac{1}{1+y^2}=1+\su (-1)^ny^{2n}\tag{17} \]
Полагая в (13) $a=0,\ b=y,\ x_0=0,$ при $|y|<1$ получаем \[ arctg\ y=\int_0^y \frac{dt}{1+t^2}=y+\su\frac{(-1)^n}{2n+1}y^{2n+1}\tag{18} \]
Радиус сходимости рядов (17) и (18) равен 1, ряд $1+\su\frac{(-1)^n}{2n+1}$ сходится по признаку Лейбница.\\
По теореме Абеля ряд $y+\su \frac{(-1)^n}{2n+1}y^{2n+1}$ равномерно сходится при $y\in[0,1]$ и является непрерывной на $[0,1]$ функцией. Поэтому \[ 1+\su\frac{(-1)^n}{2n+1}=\lims{y\to 1-0}(y+\su\frac{(-1)^n}{2n+1}y^{2n+1})=\lims{y\to 1-0}arctg\ y=\frac{\pi}{4}\tag{19} \]
 \subsection{Формула Тейлора с интегральным остатком}
 Далее, нам понадобится формула Тейлора с остатком в интегральной форме.\\
 \begin{theorem} Пусть $f\in C^n((a,b)),\ n\geq 1, x, x_0\in (a,b), x\ne x_0.$ Тогда справедливо равенство
 \[ f(x)=f(x_0)+f'(x_0)(x-x_0)+\dots+\frac{f^{(n-1)}(x_0)}{(n-1)!}(x-x_0)^{n-1}+\]\[+\frac{1}{(n-1)!}\int_{x_0}^{x}(x-t)^{n-1}f^{(n)}(t)dt\tag{20} \] \end{theorem}
 \begin{longProof}[по индукции]
 Если $n=1,$ то формула Ньютона-Лейбница даёт \[ f(x)=f(x_0)+\int_{x_0}^x f'(t)dt, \tag{21}\]
 и (21) совпадает с (20) при $n=1.$\\
 Предположим, что формула (20) справедлива при $n$, и пусть $f\in C^{n+1}((a,b)).$
  Применяя интегрирование по частям, получаем равенство 
  \[ \frac{1}{(n-1)!}\int_{x_0}^x(x-t)^{n-1}f^{(n)}(t)dt=\frac{1}{(n-1)!}\int_{x_0}^{x}(-\frac{(x-t)^n}{n})'f^{(n)}(t)dt= \]
 \[=\frac{1}{(n-1)!}\cdot(-\frac{f^{(n)}(t)}{n}(x-t)^n)|_{x_0}^x-\frac{1}{(n-1)!}\int_{x_0}^x(-\frac{(x-t)^n}{n})f^{(n+1)}(t)dt=\]
 \[=\frac{f^{(n)}(x_0)}{n!}(x-x_0)^n+\frac{1}{n!}\int_{x_0}^x(x-t)^nf^{(n+1)}(t)dt\tag{22} \]
 Применяя для функции $f\in C^{n+1}((a,b))$ формулу (20) со слагаемыми до $\frac{f^{(n-1)}(x_0)}{(n-1)!}(x-x_0)^{n-1},$
  а затем формулу (22), получим утверждение теоремы для $f\in C^{n+1}((a,b)).$\\
 \end{longProof}
 
 \subsection{Разложения элементарных функций в ряд Тейлора}
 \subsubsection{Разложение \boldmath$e^x$}
 Применим формулу Тейлора с остатком в форме Лагранжа: \[ e^x=1+\sum\limits_{m=1}^n\frac{x^m}{m!}+\frac{e^c}{(n+1)!}x^{n+1},\tag{23} \]
 где $0<|c|<|x|,\ cx>0$ Пусть $n_0>|x|, q=\frac{|x|}{n_0+1}<1$\\
 Тогда при $n>n_0$ имеем \[ |\frac{e^c}{(n+1)!}x^{n+1}|\leq \frac{e^{|x|}}{n_0!}|x|^{n_0}\frac{|x|}{n_0+1}\cdot\frac{|x|}{n_0+2}\dots\frac{|x|}{n+1}<\frac{e^{|x|}}{n_0!}|x|^{n_0}\cdot q^{n+1-n_0}\underset{n\to\infty}{\rightarrow}0,\textbf{(24)} \]
 поэтому (23) и (24) $(n\to \infty) \implies$
 \[ e^x=1+\su\frac{x^n}{n!},\ x\in\R\tag{25} \]
 \subsubsection{Разложение $\cos x$}
 Применим формулу Тейлора с остатком в форме Лагранжа:
 \[ \cos x=1+\sum\limits_{m=1}^n\frac{(-1)^m}{(2m)!}x^{2m}\pm \sin c\cdot\frac{x^{2n+1}}{(2n+1)!}\tag{25'} \]
 Поскольку, аналогично (24), имеем $\frac{|x|^{2n+1}}{(2n+1)!}\underset{n\to\infty}{\rightarrow} 0,$ и $|\sin c\cdot\frac{x^{2n+1}}{2n+1}|\leq\frac{|x|^{2n+1}}{(2n+1)!},$
 то (25') $\implies$ \[ \cos x=1+\su \frac{(-1)^n}{(2n)!}x^{2n},\ x\in\R\tag{26} \]
 \subsubsection{Разложение $\sin x$}
 Аналогично разложению $\cos x,$ получим \[ \sin x=\su (-1)^{n-1}\frac{x^{2n-1}}{(2n-1)!}\tag{27} \]
 \subsubsection{Разложение \boldmath$(1+x)^r,\ r\notin\N, r \ne 0$}
 Имеем соотношения: 
 \[ ((1+x)^r)'=r(1+x)^{r-1},\ ((1+x)^r)''=r(r-1)(1+x)^{r-2},\]
 \[ \dots ((1+x)^r)^{(n)}=r(r-1)\dots(r-n+1)(1+x)^{r-n}, \]
 тогда \[ ((1+x)^r)^{(n)}|_{x=0}=r(r-1)\dots(r-n+1),\ n\geq 1\tag{28} \]
Применим формулу Тейлора с остатком в интегральной форме при $n\geq 2, 0<|x|<1, (28)\implies$
 \[ (1+x)^r=1+rx+\frac{r(r-1)}{2!}x^2+\dots+\frac{r(r-1)\dots(r-n+2)}{(n-1)!}x^{n-1}+ \]
 \[ +\frac{1}{(n-1)!}\int_0^x (x-t)^{n-1}\cdot(r(r-1)\dots(r-n+1)(1+t)^{r-n})dt=\]
 \[ =1+rx+\frac{r(r-1)}{2!}x^2+\dots+\frac{r(r-1)\dots(r-n+2)}{(n-1)!}x^{n-1}+\]\[+\frac{r(r-1)\dots(r-n+1)}{(n-1)!}\cdot \int_0^x(x-t)^{n-1}\cdot(1+t)^{1-n}(1+t)^{r-1}dt\tag{29} \]
Если $1>x>0,$ то в интеграле (29) $0\leq t\leq x,$ тогда \[ 0<\frac{x-t}{1+t}\leq x\tag{30} \] 
 Если $-1<x<0,$ то в интеграле (29) $-1<x\leq t\leq 0,$ тогда
 \[ |\frac{x-t}{1+t}|=\frac{|x|-|t|}{1-|t|}\leq |x|\tag{31} \]
 Из (30) и (31) следует, что при $0<|x|<1, 0\leq |t|\leq |x|, tx>0$ имеем $|\frac{x-t}{1+t}|\leq|x|,$ поэтому
\[ |\int_0^x(\frac{x-t}{1+t})^{n-1}(1+t)^{r-1}dt|\leq \int_0^{|x|}|x|^{n-1}\cdot(1-|x|)^{-|r|-1}dt= \]
 \[ = |x|^n (1-|x|)^{-|r|-1}\tag{32} \]\newpage
 Положим \[ \alpha_n=\frac{|r(r-1)\cdot\dots\cdot(r-n+1)|}{(n-1)!}|x|^n(1-|x|)^{-|r|-1},\tag{33}\]
 тогда \[ \frac{\alpha_{n+1}}{\alpha_n}=\frac{|r(r-1)\cdot\dots\cdot(r-n+1)(r-n)|}{n!}\cdot|x|^{n+1}(1-|x|)^{-|r|-1}\times \]\[ \times \frac{(n-1)!}{|r(r-1)\cdot\dots\cdot(r-n+1)|}\cdot|x|^{-n}\cdot(1-|x|)^{|r|+1}=\frac{|r-n|}{n}|x|,\tag{34}\]
 Имеем $\frac{|r-n|}{n}\leq \frac{n+|r|}{n}=1+\frac{|r|}{n},$ поэтому при $n\geq n_1,$ где $n_1$ выбрано из условия $(1+\frac{|r|}{n_1})|x|<1,$ имеем
\[ \frac{\alpha_{n+1}}{\alpha_n}<1\tag{35} \]
\[(35) \implies \exists\Lim \alpha_n=\alpha\geq 0\tag{36}\]
 Из (34) и (36) имеем \[ \alpha=\Lim\alpha_{n+1}=\Lim (\frac{|r-n|}{n}|x|\alpha_n)=\]
 \[ =(\Lim \frac{|r-n|}{n}|x|)\Lim\alpha_n=|x|\cdot\alpha\tag{37} \]
 \[ (37) \implies \alpha=0\tag{38}\]
 Из (32), (36), (37) $\implies$ остаток в (29) стремится к 0 при $n\to\infty,$ поэтому при $0<|x|<1$ имеем
 \[ (1+x)^r=1+rx+\frac{r(r-1)}{2!}x^2+\dots+\frac{r(r-1)\dots(r-n+1)}{n!}x^n+\dots\tag{39}\]\newpage
 \subsubsection{Разложение $\arcsin x, 0<|x|<1$}
 Положим в (39) $x=-y^2,\ 0<|y|<1.$ Тогда \[ (1-y^2)^{-\frac{1}{2}}=1+(-\frac{1}{2})(-y^2)+\frac{(-\frac{1}{2})(-\frac{3}{2})}{2!}y^4+\dots+(-1)^n\frac{(-\frac{1}{2})(-\frac{3}{2})\dots(-\frac{2n-1}{2})}{n!}y^{2n}+\dots=\]
 \[=1+\frac{y^2}{2}+\frac{1\cdot 3}{2^2\cdot 2!}y^4+\dots+\frac{1\cdot 3\dots(2n-1)}{2^n n!}y^{2n}+\dots\tag{40} \]
 Теперь (40) $\implies$
 \[ \arcsin x=\int_0^x\frac{dy}{\sqrt{1-y^2}}=\int_0^x(1+\frac{y^2}{2}+\frac{1\cdot 3}{2^2\cdot 2!}y^4+\dots+\frac{1\cdot 3\dots(2n-1)}{2^n n!}y^{2n}+\dots)dy= \]
 \[ =x+\frac{1}{2\cdot 3}x^3+\frac{1\cdot 3}{2^2\cdot 2!\cdot 5}x^5+\dots+\frac{1\cdot 3\dots(2n-1)}{2^n\cdot n!(2n+1)}x^{2n+1}+\cdots\]


\end{document}