% !TeX root = ./main.tex
\documentclass[main]{subfiles}
\begin{document}
\chapter{Функциональные ряды}
\subsection{Функциональные последовательности и ряды}
\begin{definition} 
    Пусть $E\neq\varnothing$ - произвольное множество, $f_n:E\to{\vR}$ - функции,
     определённые на $E, n=1,2,\ldots, X_0\in E.$\\ Назовём $X_0$
      \textbf{точкой сходимости функциональной последовательности} $\{f_n(X)\},$ если $
       \exists \underset{n\to\infty}{lim}f_n(X_0)\in{\vR};$\\
        точку $X_1\in E$ 
       назовём \textbf{точкой рассходимости функциональной последовательности} $\{f_n(X)\}_{n=1}^\infty,$ 
       если предел $\Lim f_n(X_1)$ не существует или существут и равен $+\infty$ или $-\infty$.
\end{definition}
\begin{definition}
        Множество всех точек сходимости $\fnn$ назовём \textbf{множеством сходимости}, обозначим его $E_0,$ 
        а множество точек расходимости назовём \textbf{множеством расходимости} и обозначим его $E_1$. \\
        Какое-то из множеств $E_0, E_1$ может быть пустым, они обладают свойством
         $E_0\cap E_1=\varnothing, E_0\cup E_1=E,$
         поэтому оба они пустыми быть не могут. \end{definition}
         \begin{definition}
Если $v_n:E\to {\vR}$ - функции, заданные на $E$, то \textbf{функциональным рядом}, заданным на $E$,
 называемся символ $\sumn v_n(X), X\in E,$ его частичные суммы $S_n(X)$ определяются формулой
  $S_n(X)=v_1(X)+\dots+v_n(X).$ \end{definition}
  \begin{definition}
Точка $X_0$ - \textbf{точка сходимости ряда} $\sumn v_n(X),$ если $X_0$ - 
точка сходимости функциональной последовательности $\fnn[S_n(X)];
 X_1$ -
 \textbf{точка расходимости этого ряда}, если $X_1$ - точка расходимости 
 функциональной последовательности $\fnn[S_n(X)].$\\ 
 \textbf{Множества сходимости и расходимости функционального ряда}
  $\sumn v_n(X)$ - соответственно, множества сходимости и расходимости
   функциональной последовательности $\fnn[S_n(X)]$. 
\end{definition}
   \begin{definition}
Предположим, что для функциональной последовательности $\fnn[f_n(X)]$ имеем $E_0=E, E_1=\varnothing$, 
пусть $\forall X\in E\ f(X)\stackrel{def}{=}\Lim f_n(X).$\\ 
Тогда говорим, что функциональная последовательность 
$\fnn[f_n(X)]$ \textbf{сходится к функции $f$ на множестве $E$ поточечно}.
 Если для функционального ряда $\sumn v_n(X)$ имеем $E_0=E, E_1=\varnothing,$ 
 то для $X\in E$ полагаем $S(X)=\sumn v_n(X)$ и говорим, что функциональный ряд
  $\sumn v_n(X)$ \textbf{сходится к сумме $S(X)$ на множестве $E$ поточечно}. \end{definition}
\begin{definition}[Равномерная сходимость]
Пусть $\fnn[f_n(X)]$ - функциональная последовательность, заданная на $E, f:E\to{\vR}$ - функция. 
Будем называть $\fnn[f_n(X)]$ \textbf{равномерно сходящейся к $f$ на множестве $E$}, 
если $\forall \varepsilon>0 \ \exists N\in\N $ т.ч. $\forall n>N$ и $\forall X\in E$ 
выполнено соотношение \[ |f_n(X)-f(X)|<\varepsilon\tag{1} \]
То, что $\fnn[f_n(X)]$ равномерно сходится к $f$ на множестве $E$, будем записывать в виде 
\[ f_n(X) \darrow{\underset{X\in E}{n\to\infty}}  f(X)  \]
Пусть $\sumn v_n(X)$ - функциональный ряд, $S:E\to{\vR}$ - функция. Будем говорить, 
что этот функциональный ряд \textbf{равномерно сходится к $S$} на множестве $E$,
 если $S_n(X){\darrow{\underset{X\in E}{n\to\infty}}} S(X).$\\
Также говорят, что функциональный ряд $\sumn v_n(X)$ равномерно сходится на $E$,
 если $\exists S(X)$, к которой он равномерно сходится на $E$, при этом функция $S(X)$ может быть не указана.
\end{definition}
\subsection{Критерий Коши равномерной сходимости}
\begin{theorem}
     Пусть $\fnn[f_n(X)]$ - функциональная последовательность, заданная на $E\neq\varnothing.$ 
     Для того, чтобы нашлась функция $f: E\to {\vR}$ такая, что 
     $f_n(X){\darrow{\underset{X\in E}{n\to\infty}}}  f(X),$
      необходимо и достаточно, чтобы $\forall\ \varepsilon>0\ \exists N:\ \forall m, n>N$ и 
      $\forall X\in E$ выполнялось соотношение 
\[ |f_m(X)-f_n(X)|<\varepsilon\tag{2} \] \end{theorem}
\begin{longProof} \textbf{Необходимость.} \\
     Пусть $\ \exists f:E\to{\vR}$ т.ч. $f_n(X){\darrow{\underset{X\in E}{n\to\infty}}}  f(X).$ 
     По определению равномерной сходимости $\forall \varepsilon>0 \ \exists N: \forall n>N$ и $\ \forall X\in E$
      выполнено $|f_n(X)-f(X)|<\frac{\varepsilon}{2}$. Для $m>N$ это же неравенство выполняется.
       Тогда $\ \forall X\in E$ имеем при $m,n>N:$ \[ |f_m(X)-f_n(X)|=|(f_m(X)-f(X))-(f_n(X)-f(X))|\leq\] 
       \[\leq|f_m(X)-f(X)|+|f_n(X)-f(X)|<\frac{\varepsilon}{2}+\frac{\varepsilon}{2}=\varepsilon. \] 
       \textbf{Необходимость доказана.}
\textbf{Достаточность.} \\
Предположим, что соотношение (2) выполнено с сф ормулированными для него условиями. 
Фиксируем $X\in E,$ тогда (2) $\implies$ что для числовой последовательности $\fnn[f_n(X)]$ выполнено условие критерия Коши, поэтому $\ \exists \Lim f_n(X)\overset{def}{=}f(X).$ Таким образом, получена функция $f:E\to {\vR}$ т.ч. $f_n(X)\underset{n\to\infty}{\to} f(X)\ \forall X\in E.$\\
Возьмём $\forall \varepsilon>0$ и выберем $N$ так, чтобы выполнялось соотношение (2). 
Фиксируем $X\in E$ и возьмём $\ \forall m, n>N,$ тогда \[ |f_m(X)-f_n(X)|<\varepsilon \tag{2'} \]
Фиксируем $m$ в (2'), тогда $(2')\implies$ 
\[\Lim |f_m(X)-f_n(X)|\leq\varepsilon\tag{3} \]
Но $f_n(X)\underset{n\to \infty}{\to}f(X),$ поэтому $(3)\implies$
 \[|f_m(X)-f(X)|\leq\varepsilon \tag{4}\]
В соотношении (4) $X\in E$ было фиксированным, но его можно было выбирать произвольно. Таким образом, $(4) \implies \forall\varepsilon>0\ \exists N: \forall m>N$ и $\forall X$ выполнено \[ |f_m(X)-f(X)|<2\varepsilon \]
В силу произвольности $\varepsilon>0,$ достаточность в критерии Коши доказана.
\end{longProof}
\begin{theorem} 
    Пусть $\sumn v_n(X)$ - функциональный ряд, заданный на $E$. Для того, чтобы он равномерно сходился на $E$,
     необходимо и достаточно, чтобы $\forall \varepsilon>0\ \exists N: \forall m>n>N$ 
     и $\forall X\in E$ выполнялось соотношение: \[ |\sum_{k=n+1}^m v_k(X)|<\varepsilon\tag{5} \]
\end{theorem}
\begin{proof}
     Если $m>n$, то $S_m(X)-S_n(X)=\sum_{k=1}^m v_k(X)-\sum_{k=1}^n
 v_k(X)=\\=\sum_{k=n+1}^m v_k(X),$ поэтому утверждение теоремы следует из 
 определения равномерной сходимости функционального ряда и предыдущей теоремы.
\end{proof}
\subsection{Признак Вейерштрасса равномерной сходимости рядов.}
\begin{theorem}
     Пусть для функционального ряда $\sumn v_n(X) $ справедливо соотношение
      $|v_n(X)|\leq c_n, c_n\geq 0,$ для $\forall X\in E$ и $\forall n\geq 1.$ 
      Предположим, что ряд $\sumn c_n$ сходится. Тогда рассматриваемый функциональный ряд сходится равномерно.\end{theorem}
\begin{proof} 
    Возьмём $\forall \varepsilon >0.$ Тогда $\exists N: \forall m>n>N$ выполнено $\sum_{k=n+1}^m c_k<\varepsilon$. 
    Теперь при $\forall X\in E$ и $m>n>N$ имеем соотношение 
    \[ |\sum_{k=n+1}^m v_k(X)|\leq \sum_{k=n+1}^m |v_k(X)|\leq \sum_{k=n+1}^m c_k<\varepsilon \]
Теперь утверждение теоремы следует из критерия Коши равномерной сходимости функциональных рядов. \end{proof}
\subsection{Признак Дирихле равномерной сходимости функциональных рядов}
\begin{theorem} 
    Пусть $v_n:E\to{\vR}, b_n:E\to{\vR}$ - функции, заданные на множестве $E.$
     Предположим, что последовательность $\fnn[b_n(X)]$ монотонна при $\forall X\in E.
     $ Пусть $O_E:E\to{\vR}, O_E(X)=0\ \forall X\in E.$\\
Предположим, что $b_n(X)\darrow{\underset{X\in E}{n\to\infty}}O_E(X)$ 
и что $\exists C,$ не зависящая от $n$ и $X\in E$ такая, что $|\sum_{k=1}^n v_n(X)|\leq C.$ 
Тогда ряд $\sumn b_n(X)v_n(X)$ равномерно сходится при $X\in E.$ \end{theorem}
\begin{longProof} 
    Отметим, что $\forall X\in E$ и $\forall k>n\geq 1$ выполнено
     \[ |\sum_{l=n+1}^k v_l(X)|=|\sum_{l=1}^k v_l(X)-\sum_{l=1}^n v_l(X)|\leq
      |\sum_{l=1}^k v_l(X)|+|\sum_{l=1}^n v_l(X)|\leq 2C\tag{6} \]
Для $m>n\geq 1$ положим $V_n(X)\equiv0, V_{n+1}(X)=v_{n+1}(X),\\
 V_k(X)=v_{n+1}(X)
+\dots+v_k(X),$ если $k>n+1.$ Тогда
\begin{multline*}
 \sum_{k=n+1}^m b_k(X)v_k(X)=\sum_{k=n+1}^m b_k(X)(V_k(X)-V_{k-1}(X))= \\
=\sum_{k=n+1}^m b_k(X)V_k(X)-\sum_{k=n+1}^m b_k(X)V_{k-1}(X)= \\
=\sum_{k=n+1}^m b_k(X)V_k(X)-\sum_{l=n}^{m-1} b_{l+1}(X)V_l(X)= \\
=\sum_{k=n+1}^{m-1}(b_k(X)-b_{k+1}(X))V_k(X)+b_m(X)V_m(X)\tag{7} \end{multline*}
Равенства (7) - это преобразование Абеля, применённое для функциональных рядов. 
Для доказательства теоремы применим критерий Коши и соотношение (7). \\
Возьмём $\forall\varepsilon>0.$ Тогда $\exists N:\forall n>N$ и $\forall X\in E$ 
выполнено\\ $|b_n(X)|=|b_n(X)-O_E(X)|<\varepsilon.$ Тогда $\forall m>n>N\ (7)$ и
 $(6) \implies$
 \begin{multline*}
  |\sum_{k=n+1}^m b_k(X)v_k(X)|\leq |\sum_{k=n+1}^{m-1}(b_k(X)-b_{k+1}(X))V_k(X)|+|b_m(X)V_m(X)|\leq \\
\leq 2C\sum_{k=n+1}^{m-1}|b_k(X)-b_{k+1}(X)|+2C|b_m(X)|=2C|\sum_{k=n+1}^{m-1}(b_k(X)-b_{k+1}(X))|+ \\
+ 2C|b_m(X)|=2C|b_{n+1}(X)-b_m(X)|+2C|b_m(X)|\leq \\
 \leq 2C|b_{n+1}(X)|+4C|b_m(X)|<6C\varepsilon\tag{8} 
\end{multline*}
В равенстве (8) мы воспользовались монотонностью последовательности
 $\fn[b_n(X)], (8) \implies $ ряд $\sumn b_n(X)v_n(X)$ равномерно сходится при $X\in E.$ 
\end{longProof}
\subsection{Признак Абеля равномерной сходимости функциональных рядов.}
\begin{theorem}
     Предположим, что последовательность $\fnn[b_n(X)]$ монотонна $\forall X\in E$ 
     и что $\exists C_1$ т.ч. $|b_n(X)|\leq C_1 \forall n$ и $\forall X\in E.$ 
     Предположим, что ряд $\sumn v_n(X)$ равномерно сходится при $X\in E.$ 
     Тогда функциональный ряд $\sumn b_n(X)v_n(X)$ равномерно сходится 
     при $X\in E.$ \end{theorem}
\begin{proof}
     Применим критерий Коши. Возьмём $\forall \varepsilon>0.$ \\
      Тогда $\exists N: \forall m>n>N$ и $\forall X\in E$ выполнено
       $|S_m(X)-S_n(X)|=|v_{n+1}(X)+\dots+v_m(X)|<\varepsilon.$ В обозначениях
        соотношений (7) и (8) имеем $|V_k(X)|<\varepsilon, k\geq n+1.$ Тогда $(7) \implies$
       \begin{multline*}
 \left|\sum_{k=n+1}^m b_k(X)v_k(X)\right|\leq  \\ \leq 
 \left|\sum_{k=n+1}^{m-1}(b_k(X)-b_{k+1}(X))V_k(X)\right|+ |b_m(X)||V_m(X)|\leq \\
 \leq \sum_{k=n+1}^{m-1}|b_k(X)-b_{k+1}(X)|\varepsilon + |b_m(X)|\varepsilon =\\ =\varepsilon
  |\sum_{k=n+1}^{m-1}(b_k(X)-b_{k+1}(X))|+|b_m(X)|\varepsilon= \\
 =\varepsilon|b_{k+1}(X)-b_m(X)|+|b_m(X)|\varepsilon<3C_1\varepsilon\tag{9} \\
\end{multline*}
$(9)\implies$ ряд $\sumn b_n(X)v_n(X)$ равномерно сходится при $X\in E.$ \end{proof}
 
\subsection{Переход к пределу в равномерно сходящейся функциональной последовательности.}
\begin{theorem}
     Пусть $E$ - метрическое пространство, $X_0\in E, X_0 - $ точка сгущения,
      $f_n:E\to{\vR}, f:E\to{\vR}, f_n(X){\darrow{\underset{X\in E}{n\to\infty}}}  
      f(X).$\\ Предположим, что $\forall n\ \exists\underset{X\to X_0}{lim}f_n(X)=A_n, A_n\in{\vR}.$\\
       Тогда $\exists\Lim A_n=A\in{\vR}$ и $\ \exists\underset{X\to X_0}{lim}f(X)$ 
       и $\underset{X\to X_0}{lim}f(X)=A$ \end{theorem}
\begin{longProof}
     Возьмём $\forall\varepsilon>0.$ По критерию Коши $\exists N$ т.ч. $\forall m, n>N$ и $\forall X\in E$
      выполнено
       \[|f_m(X)-f_n(X)|<\varepsilon\tag{9'} \]
        Фиксируем $m,n>N$ и перейдём 
      в (9') к пределу при $X\to X_0,$\\ тогда $(9') \implies$
\[ |A_m-A_n|=|\underset{X\to X_0}{lim} f_m(X)-\underset{X\to X_0}{lim} f_n(X)|=\underset{X\to X_0}{lim} |f_m(X)-f_n(X)|\leq \varepsilon\tag{10} \] (У Николая Алексеевича здесь, вероятно, опечатка)\\
Соотношение (10), применённое к последовательности $\fnn[A_n]$, по критерию Коши влечёт существование $\Lim A_n\in{\vR}.$ Пусть $A=\Lim A_n$.\\
Опять выберем произвольное $\varepsilon>0.$ Тогда
\begin{gather*}
\exists N_1: \forall n>N_1 \text{ выполнено } |f_n(X)-f(X)|<\varepsilon\ \forall X\in E\tag{11} \\
\exists N_2: \forall n>N_2 \text{ выполнено } |A_n-A|<\varepsilon \tag{12} \\
\text{ Пусть } N_0=max(N_1,N_2)+1.\\
 \text{ Тогда } \exists\delta: \forall X\in E, d(X,X_0)<\delta, X\neq X_0 \text{ выполняется} \\
|f_{N_0}(X)-A_{N_0}|<\varepsilon\tag{13} \\
\text{ Теперь при } X\in E, X\neq X_0, d(X,X_0)<\delta \text{ выполнено}: \\
|f(X)-A|=|(f(X)-f_{N_0}(X))+(f_{N_0}(X)-A_{N_0})+(A_{N_0}-A)|\leq \\
 \leq |f(X)-f_{N_0}(X)|+|f_{N_0}(X)-A_{N_0}|+|A_{N_0}-A|<\varepsilon+\varepsilon+\varepsilon=3\varepsilon\tag{14} \\
 (14) \implies \underset{X\to X_0}{lim}{f(X)}=A. \end{gather*}
\end{longProof}


\begin{theorem}[О непрерывности]
     \begin{gather*}
          E \quad X_0 \in E \text{ - точка сгущения } \\
          \fnn \quad f_n(X) \darrow{n \to \infty, x \in E} f(X) \tag{1} \\
          f_n \text{ непрерывна в } X_0 \forall n \\
          \implies f \text{ непрерывна в } X_0 
     \end{gather*}
\end{theorem}
\begin{proof}
     \begin{gather*}
          f_n(X_0) \underset{n \to \infty}{\rightarrow} f(X_0) \tag {2\prime}\\
          \forall n \quad  \exists \underset{X \to X_0}{lim} f_n(X) = f(X_0) \tag{2} \\
          (1), (2) \implies \exists \underset{x \to x_0}{lim} f(X) = f_n(X_0)
     \end{gather*}
\end{proof}
\begin{corollary}[Следствие 1]
     \begin{gather*}
          \sumn v_n(X) \tag{3} \\
          v_n \text{ непрерывна в } X_0 \forall n \\
          S(X) = \sumn v_n(X) \quad S \text{ непрерывна в } X_0 \\
          S(X) = \sum_{k=1}^\infty v_k (X) \\
          S_n(X) \darrow{\stackrel{n \to \infty}{x \in E}} S(X)
      \end{gather*}
\end{corollary}
\begin{corollary}[Cледствие 2]
     \begin{gather*}
          E \quad \forall x \in E \text{ - точка сгущения} \\
          \fnn \quad f_n(X) \darrow{\stackrel{n \to \infty}{x \in E}} f(X) \\
          f_n \in C(E) \implies f \in C(E) \\
     \end{gather*}
\end{corollary}
\begin{proof}
     Доказательство следует прямо из теоремы.
      В каждой точке множества $E$ по этой теореме функция непрерывна.
\end{proof}
\begin{corollary}[Следствие 3]
     \begin{gather*}
          v_n \in C(E) \quad \forall n \implies S \in C(E)
     \end{gather*}
\end{corollary}
\begin{proof}
     Следует из Следствия 1, примененного к каждой точке
      множества $E$. Раз точка любая, то сумма $S$ будет 
      непрерывна на всем множестве $E$.
\end{proof}

\begin{theorem}[О переходе к пределу под знаком интеграла для функционального ряда]
     В следующих 2 теоремах у нас будут фигурировать обычные промежутки в качестве метрического пространства.
     \begin{gather*}
          \fnn \quad f_n : [a,b] \rightarrow \vR \quad f_n \in C([a,b]) \quad \forall n \\
          \text{Предположим, что } f_n(X) \darrow{\stackrel{n \to \infty}{x \in [a,b]}} f(X) \tag{5} \\
          \text{По следствию 2 функция f будет непрерывна на промежутке} \\
          \implies \int_a^b f(x) dx = \underset{n \to \infty}{lim} \int^b_a f_n(X)dx \tag{6}
     \end{gather*}
\end{theorem}
\begin{longProof}
     \begin{gather*}
          I = \int^b_a f(X) dx \quad I_n = \int^b_a f_n(x) dx \\
          \forall \varepsilon > 0 \quad (5) \implies \exists N : \forall n > N \text{ и }
          \forall x \in [a,b] \\ 
          |f_n(X) - f(X)| < \varepsilon \tag{7} 
     \end{gather*}
     \begin{multline*}
          (7) \implies |I_n - I| = \left | \int^b_a f_n(X)dx - f(X)dx \right |  = \\
         = \left | \int^b_a (f_n(X) - f(x))dx \right | \leq \int^b_a \left | f_n(X) - f(X) \right | dx \leq\\
          \leq \int^b_a \varepsilon dx = \varepsilon(b-a) \tag{8} \end{multline*}
          \begin{gather*}
          (8) \implies (6)\\
          \sumn v_n(X) \tag{9} \\
          v_n : [a,b] \rightarrow \vR \\
          v_n \in C([a,b]) \forall n \\
          \end{gather*}
          Предполагаем, что функциональный ряд (9) сходится равномерно. 
          Тогда по следствию 1 из предыдущей теоремы 
          сумма будет непрерывной функцией. Тогда справедливо соотношение
          \begin{gather*}
          \sumn \int^b_a v_n(x)dx = \int^b_a \sumn v_n(x)dx \tag{10} \\
          S_n(X) = \sumn v_k(X) \\
          S_n \in C([a,b]) \\
          S_n(X) \darrow{n \to \infty, x \in [a,b]} S(x)\\
          \text{Тогда по предыдущей теореме } \int^b_a S_n(X) dx \underset{n \to \infty}{\rightarrow}
          \int^b_a S(X) dx \tag{11} \\
          \text{В правой части (11) стоит правая часть (10)} \\
          \int^b_a S_n(X)dx = \int^b_a \sum^\infty_{k=1} v_k(X) dx = \sumn \int^b_a v_k(x) dx \tag{12} \\
          \text{ Правая часть (12) это частичная сумма числового ряда (10)} \\
          (12) \implies (10)
     \end{gather*}
\end{longProof}
\begin{theorem}[О последовательном дифференцировании функциональной последовательности]
     \begin{gather*}
          \fnn \quad f_n: [a,b] \rightarrow \vR \\
          \forall n { и } \forall x \in [a,b] \exists f^\prime_n(X) \tag{13} \\
          \text{ Предположим, что}
          f^\prime_n(X) \darrow{\stackrel{n \to \infty}{x \in [a,b]}} \phi(x) \tag{14} \\
          \text{ Также предположим, что} X_0 \in [a,b] \\
          \exists \underset{n \to \infty}{lim} f_n(X_0) = A \in \vR \tag{15} \\
          \implies \exists f: [a,b] \rightarrow \vR \\
          f_n(X) \darrow{\stackrel{n \to \infty}{x \in [a,b]}} f(X) \tag{16} \\
          f(X_0) = A \tag{17} \\
          \forall x \in [a,b] \exists f^\prime(X) \text{ и } f^\prime(X) = \phi(X) \tag{18}
     \end{gather*}
\end{theorem}
\begin{longProof}
     \begin{gather*}
          P_{mn}(x) = f_m(x) - f_n(x)   \\
          (13) \implies \forall x \in [a,b] \exists P^\prime_{mn}(x)  = f^\prime_m(x) - f^\prime_n(X) \tag{19} \\
          \text{ Собираемся доказать соотношение (16) с помощью критерия Коши}  \\
          \forall \epsilon > 0 (14) \implies \exists N : \forall m, n > N \text{ и } \forall x \in [a,b] \\
          |f^\prime_m(x) - f^\prime_n(x) | < \epsilon \tag{20} \\
          (19),(20) \implies |P^\prime_{mn}(X) | < \varepsilon \tag{20\prime} \\
          (15) \implies \exists N_1: \forall m, n > N_1 \\
          |f_m(x_0) - f_n(x_0)| < \varepsilon \tag{21} \\
          N_2 = max(N,N_1) 
     \end{gather*}
          \begin{multline*}
          m, n > N_2, \forall x \in [a,b] \quad (20), (21) \implies |P_{mn}(x) | = \\
          =|(P_{mn}(x) - P_{mn}(x_0)) + P_{mn}(x_0) | \leq \\
           \leq |P_{mn}(X) - P_{mn}(X_0) | + |P_{mn}(x_0)| = \\
          =|P^\prime_{mn} (C)(x-x_0) | + |P_{mn}(x_0)| < \\ <
          \varepsilon(b-a) + \varepsilon = \varepsilon(b-a+1) \tag{22} \\\end{multline*}
          \[ (22) \implies (16) \]
     Теперь будем доказывать что у функции $\fnn$ существует производная и она равна $\phi(x)$
          $x \in [a,b]$ Фиксируем точку. Будем доказывать что в ней существует производная.
          \begin{gather*}
          E = [a,b] \setminus \{ x \} \\
          \text{Будем рассматривать } y \in E \\
          g_n(Y) = \frac{f_n(y) - f_n(x)}{y-x} \tag{23} \\
          g(y) = \frac{f(y)-f(x)}{y-x} \tag{24} \\
          \text{ Давайте рассмотрим выражение }  \end{gather*}
          \begin{multline*}
          g_m(Y) - g_n(Y) = \frac{f_m(Y) -f_m(X)-(f_n(Y)-f_n(x))}{y-x} =\\ = 
          \frac{P_{mn}(Y)-P_{mn}(X)}{y-x} =
          \frac{P^\prime_{mn}(c_1 \cdot (y-x))}{y-x} = P^\prime_{mn}(c_1) \tag{25}
          \end{multline*}
          $P_{mn}$ имеет производные в каждой точке замкнутого пространства и поэтому по теореме Лагранжа существует $c_1$ \\
          \begin{gather*}
          \forall \varepsilon > 0  \quad (20\prime) \implies \exists N : \forall m,n > N \text{ и } \forall y \in [a,b] |P^\prime_{mn}(Y)| < \varepsilon \\
          (20\prime), (25) \implies \forall m, n > N \text{ и } \forall y \in E \\
          | g_m(Y) - g_n(Y)| = |P^\prime_{mn}(c_1)| < \varepsilon \tag{26} \\
          \end{gather*}
          Получено следующее: мы взяли $\epsilon$, нашли номер $N$, что для
           любого $Y$ выполнено соотношение (26). Тогда по критерию Коши сходимости 
           последовательности существует последовательность, к которой ... сходится равномерно
           \begin{gather*}
           (26) \implies \exists \tilde{g}: E \rightarrow \vR : g_n(Y) \darrow{\stackrel{n \to \infty}{y \in E}} (\tilde{g})(Y) \tag{27} \\
           \text{фиксируем } y \in E \\
           (16) \implies f_n(Y) \underset{n \to \infty}{\rightarrow} f(Y), f_n(X) \\
           \underset{n \to \infty}{\rightarrow} f(x) 
           \implies g_n(Y) \underset{n \to \infty}{\rightarrow} \frac{f(y) - f(x)}{y-x} = g(y) \tag{28} \\
     \end{gather*}
          Давйте посмотрим на (27) и (28). в (28) сказано, что $g_n$ равномерно стремится к $\tilde{g}$.
            Если возьмем конкретную точку, которая стремится к $g$, то это будет одна и та же функция.
            \[(23), (24),(27), (29) \implies \frac{f_n(Y) - f_n(x)}{y-x} \darrow{\stackrel{n \to \infty}{y \in E}} 
            \frac{f(y)-f(x)}{y-x} \tag{30} \]
            Теперь будем применять теорему о переходе к пределу \\
            \[\forall n \quad \exists \underset{y \to x}{lim} \frac{f_n(Y) - f_n(X)}{y-x} = f^\prime_n(x) \tag{31} \]
            \[(30), (31) \implies \exists \underset{y \to x}{lim} \frac{f(y)-f(x)}{y-x} = \underset{n \to \infty}{lim} f^\prime_n(x) = \phi(x) \implies (18) \]
\end{longProof}

\begin{corollary}
     \begin{gather*}
          \sumn v_n(x) \\
          v_n :[a,b] \rightarrow \vR \\
          \forall n, \forall x \in [a,b] \quad \exists v^\prime_n(X) \\
          \sumn v^\prime_n(X) \text{ равномерно сходится на } [a,b] \\
          x_0 \in [a,b] \\ 
          \sumn v_n(x_0) \text{ сходится} \implies \\
          \exists \left( v_n(x) \right)^\prime = \sumn v^\prime_n(x) \\
          \sumn v_k(X) = S_n(X)  \\
          \forall x \in [a,b] \exists S^\prime_n(x) = \sum_{k=1}^n v^\prime_k(x) \\
          \sum^n_{k=1} = S_n(X) \\
          \exists \phi : [a,b] \rightarrow \vR \\
          S^\prime_n(X) \darrow{\stackrel{n \to \infty}{x \in [a,b]}} \phi(x) \\
          S^\prime(X) = \phi(X) \forall x \in [a,b] \\
          \phi(x) = \underset{n \to \infty}{lim} \sum_{k=1}^n v^\prime_k(X)  = \sum^\infty_{k=1} v^\prime_k(x) 
     \end{gather*}
\end{corollary}
\begin{example}[Ван дер Варден]
     30-е годы XX века.
     \begin{gather*}
          \exists f \in C(\vR) \quad \forall x \in \vR \nexists f^\prime(x) \\
          \phi: [0,2] \rightarrow \vR \\
          \phi(x) = 1 - |x-1| \\
          \phi(0) = 0, \phi(1) = 1, \phi(2) = 0 \\
          x \in [2k, 2k+2] \quad k \in \mathbb{Z} \\
          \phi(x) \stackrel{def}{=} \phi(x-2k) \\
          f(x) \sumn \left( \frac{3}{4} \right)^n \phi(4^n(X)) \tag{1} \\
          0 \leq \phi(x) \leq 1 \\
     \end{gather*}
          Применим признак Вейерштрасса сходимости функциональных рядов.
          \begin{gather*}
          \left | \left(\frac{3}{4} \right)^n \phi(4^nx) \right | \leq \left(\frac{3}{4} \right) ^n \forall n \forall x \in \vR \\
          \begin{cases}
          \sumn \left( \frac{3}{4} \right)^n < \infty \\
          \sumn \left( \frac{3}{4} \right)^n \phi(4^nx) \in C(\vR)
          \end{cases}  \implies f \in C(\vR) \\
     \end{gather*}
  Наличие производной эквивалентно диффиеренцируемости функции в точке 
  (из 1 семестра) \\
     Пусть $\exists f^\prime(x) \text{ в } x \in vR$
     \begin{gather*}
          \implies f(x + h) = f(X) = f^\prime(x)h + r(h) \tag{2} \\
          \frac{|r(h)|}{|h|} \underset{h \to 0}{\rightarrow} 0 \tag{3} \\
          (3) \prime \exists \delta > 0  : \text{ при } 0 < |h| < \delta \\
          \frac{|r(h)|}{|h|}{|h|} < 1 \implies |f(x+h) -f(x)| \leq |f^\prime(x)| \cdot |h| +
          |r(h)| < (|f^\prime(x)| + 1)|h| \tag{4} \\
          A = |f^\prime(x)| + 1 \quad |f(x+h) - f(x)| \leq A|h| \tag{4\prime} \\
          h_1 > 0 \quad h_2 > 0, 0 < h_1 \delta, h_2 < \delta \end{gather*}
          \begin{multline*}
          (4\prime) \implies |f(x+h_2) - f(x)-h_1|  -f(x) + f(x)| \leq \\ 
          \leq  |f(x+h_2)-f(x)| + |f(x) - f(x-h_1)| \leq \\
          \leq Ah_2 + Ah_1 = A(h_2 + h_1) \tag{5} \end{multline*}
          \begin{gather*}
          (5): \quad x_1 < x < x_2, x_2 - x  < \delta, x - x_1 < \delta \\
          |f(x_2) - f(x_1)| \leq A(x_2-x_1) \tag{5\prime} \\
          4^m > \frac{1}{\varepsilon} \Leftrightarrow 4^{-m} < \delta \\
          k \leq 4^m(x) < k+1 \quad k \in \mathbb{Z} \\     
          a_m = \frac{k}{4m} \quad b_m = \frac{k+1}{4^m} \quad a_m \leq x < b_m \\
          b_m - a_m = \frac{1}{4^m} 
     \end{gather*}
          \begin{enumerate}
          \item n > m          
          $4^nb_m-4^na_m \quad 1. n > m$ \\
          \[\frac{4^n}{4^m}k + \frac{4^n}{4^m} - \frac{4^n}{4^m}k = 4^{n-m} \in \mathbb{Z}, \text{ чётное } \tag{6} \]
          \[(6) \implies \phi(4^nb_m) = \phi(4^na_m + \text { чётное }) =  \phi(4^na_n) \tag{7} \]
          \item $n = m$ \\
           \[4^mb_m - 4^ma_m = \frac{4^m}{4^m} = 1 \tag{8} \]
          \[(8): 4^mb_m = 4^m a_m + 1 \] 
          \item $ n < m $
          \[ 4^nb_m - 4^na_m = 4^{n-m} \tag{9} \]
          \end{enumerate}
          \begin{multline*}
          (1) \implies f(b_m) - f(a_m) = \sumn \left( \frac{3}{4} \right) ^n (\phi(4^nb_m) - \phi(4^na_m)) \\
          \stackrel{(7)}{=} \sum^m_{n=1} \left( \frac{3}{4} \right)^n(\phi(4^nb_m) - \phi(4^na_m)) \tag{10} \end{multline*}
          \[\forall l \in \mathbb{Z} |\phi(l+1) - \phi(l)| = 1 \tag{11} \]
          \begin{multline*}
          (10), (11) \implies |f(b_m) - f(a_m)|  \geq \\ \geq \left( \frac{3}{4} \right)^m |\phi(4^,b_m) - \phi(4^ma_m)|
          -\sum_{n=1}^{m-1} \left( \frac{3}{4} \right)^n | \phi(4^nb_m) - \phi(4^na_m)| \geq \\
           \geq   \left( \frac{3}{4} \right)^m \cdot 1 
          - \sum^{m-1}_{n=1} \left( \frac{3}{4} \right)^n |\phi(4^nb_m)-\phi(4^na_m)|\end{multline*}
          \begin{gather*}
          \forall x_1, x_2 |\phi(x_2) - \phi(x_1)| \leq |x_2 -x_1| \tag{12} \\
          |(1-|x_2-2|) - (1-|x_1-2|)| = (|x_2-2| - |x_1-2|) \leq \\
          \text{ дальше не будем расписывать неравенство треугольника} 
          \end{gather*}
          \begin{multline*}
 |x_2-x_1|\geq \left( \frac{3}{4} \right)^m - \sum_{n=1}^{m-1} \left( \frac{3}{4} \right)^n
          4^n(b_m - a_m) =\\
          = \left( \frac{3}{4} \right)^m - \frac{1}{4^m} \sum_{n=1}^{m-1}3^n =
          \left( \frac{3}{4} \right)^m - \frac{1}{4^m} \cdot \frac{3^m-1}{3-1} > \\ 
          > \left( \frac{3}{4} \right)^m - \frac{1}{2} \left( \frac{3}{4} \right)^m
          = \frac{1}{2} \left( \frac{3}{4} \right)^m \tag{13} \end{multline*}
          \begin{gather*}
          \frac{1}{4^m} = b_m - a_m \\
          (13\prime) : |f(b_m) - f(a_m)| > \frac{1}{2} 3^m \cdot (b_m-a_m) \\
          b_m - x < \delta, x - a_m < \delta \quad |f(b_m) - f(a_m)| < A (b_m-a_m) \\
          \frac{1}{2} 3^m < A
     \end{gather*} 
\end{example}
\subsection*{Равномерная сходимость семейства функций}
\begin{gather*}
     E \subset \vR^m \quad g \subset \vR^n \\
     f: E \times G \rightarrow \vR \\
     x \in E \quad y \in G \\
     f(X,Y) \\
     Y_0 \in G \\
     f(X, Y_0) \rightarrow \vR \\
     \intertext{Частный случай} G = \mathbb{N} \\
     \fnn \\
     Y_* \in \vR^n
     Y_* - \text{ точка из } G \\
     \phi : E \rightarrow \vR 
\end{gather*}
\begin{definition}
     Пусть  $f(X,Y) \darrow{\underset{Y \to Y_*}{x \in E}} \phi(X)$
      Если $\forall \varepsilon > 0 \exists U(Y_*) \subset  \forall Y \in (U \cap G) \setminus \{Y_*\} $
     \[\forall X \in E \text{ имеем } |f(X, Y) - \phi(X) | < \varepsilon \tag{1} \]
     
\end{definition}
\begin{theorem}[Критерий Коши сходимости функцоинального семейства]
     Пусть $E,G$ такие, как было написано раньше. Для того, чтобы существовала некоторая функция $\phi$, к которой стремилось функциональное семейство,
     необходимо и достаточно, чтобы
     \begin{gather*}
          \forall \varepsilon > 0 \quad \exists Y_* \in U : \forall Y_1, Y_2 \in U \cap G \setminus \{Y_*\} \\
          \forall x \in E \text{ выполнялось соотношение } \\
          |f(X, Y_1) - F(X,Y_2)| < \varepsilon \tag{2} \\
     \end{gather*}
\end{theorem}
\begin{longProof}
     Необходимость \\ 
     \begin{gather*}
          \phi: E \rightarrow \vR : f(X,Y) \darrow{\underset{Y \to Y_*}{X \in E}} \phi(X) \tag{3}\\
          (1) \implies Y_* \in U  : \\
           \forall Y \in U \cap G \setminus \{ Y_* \} \text{ и } \forall x \in E F\|f(X,Y) - \phi(X)| < \frac{\varepsilon}{2} \tag{4} 
     \end{gather*}
     \begin{multline*}
          (4) \implies \forall Y_1, Y_2 \in U \cap G \setminus \{Y_*\} \quad |f(X,Y_2) - f(X,Y_1)| = \\
          =|(f(X,Y_1) - \phi(x)) - (f(X,Y_1) - \phi(x))| \leq \\ 
          \leq |f(X,Y_2) - \phi(x)| + |f(X,Y_1) - \phi(X)| 
          < \frac{\varepsilon}{2} + \frac{\varepsilon}{2} = \varepsilon \implies (2)
     \end{multline*}
     Достаточность \\
     Предположим, что выполнено соотношение (2). Поскольку $Y_*$ -- точка сгущения, давайте выберем подпоследовательность, сходящуюся к ней
     \begin{gather*}
          \{Y_k\}^\infty_{k=1} \quad Y_k \in G \quad Y_K \neq Y_* \forall k \\
          Y_k \underset{k \to \infty}{Y_*} \tag{5} \\
          \{F_k(X) \}^\infty_{k=1} \quad F_k(X) = F(X, Y_k) \tag{6}\\
          \varepsilon > 0 \quad (5) \implies \exists K_0 : \forall k > K_0 \quad Y_k \in U \cap G \tag{7} \\
          (2), (7) \implies \forall k > K_0 \text{ и } \forall l > K_0 \text{ и } \forall X \in E  \\
          \text{применяем Критерий Коши сходимости функционального ряда ??} \\
          |F_k(X) - F_l(X)| = (f(X,Y_k) - f(X,Y_l)) < \varepsilon \tag{8} \\
          (2), (8) \implies \exists \phi: E \rightarrow \vR : F_k(X) \darrow{\underset{k \to \infty}{x \in E}} \phi(X) \tag{9}\\
          \{Y_k^\prime\}^\infty_{k=1} \quad Y^\prime_k \in G \setminus Y_* (Y_k^\prime - \text{ произвольная последовательность }) \\
          Y^\prime_k \rightarrow Y_* \tag{10} \\
          Y^0_k = \begin{cases}
               Y_q \text{ если } k = 2q-1 \\
               Y^\prime_q \text{ если } k = 2q
          \end{cases} \tag{11} \\
          (11),(5),(10) \implies Y^0_k \to Y_* \tag{12}\\
          (12) \implies \exists K_1 : \forall k > K_1 \quad Y^0_k \in U \cap G \setminus Y_* \tag{13} \\
          \text{Возьмём } \forall k > K_1 \text{ и } l = 2l_1 - 1, l_1 > K_1 \text{ и } \forall x \in E \\
          (2),(3) \implies |f(X,Y^0_k) - f(X,Y^0_{2l-1})| = |f(X,Y^0_k) - f(X,Y_l)| < \varepsilon \tag{14}\\
     \end{gather*}
     Давайте посмотрим на соотношение (14), оно действует при любом $X \in E$, при любом $l_1$ 
     \begin{gather*}
          k = 2k, k_1 > K_1 \\
          |f(X,Y^\prime_{k_1}) - f(X,Y_{l_1})| < \varepsilon {\tag{14\prime}} \\
          \text{ по соотношению (9) если мы устремим } l_1 \text{ к } \infty \\
          (14 \prime) \implies |f(X,Y^\prime_{k_1}) - \phi(X)| = \underset{l_1 \to \infty}{lim} |f(X,Y^\prime{k_1}) - f(X,Y_{l_1}) \leq \varepsilon \tag{15} \\
     \end{gather*}
     Мы взяли любую последовательнось $Y_k^\prime$, любое $\varepsilon < 0$, поэтому мы по этому $\varepsilon$ нашли
     окрестность $U$ и такую функцию $\phi$
\end{longProof}
\begin{theorem}[О пределе равномерно сходящегося семейства функций]
     \begin{gather*}
          E \subset \vR^m \quad G \subset \vR^n \quad X_0 - \text{ точка сгущения } E \quad Y_0 - { точка сгущения } G \\
          f(X,Y) \darrow{\underset{Y \to Y_0}{X \in E}} \phi(X) \tag{16} \\
          \text{Далее предполагаем } \forall Y \in G \\
          \exists \underset{X \to X_0}{f(X,Y)} = h(Y) \tag{17} \\
          \implies \exists \underset{X \to X_0}{\phi(X)} = \underset{Y \to Y_0}{h(Y)} \tag{18} \\
          \text{ Рассмотрим любую последовательность } \{Y_k\}^\infty_{k=1} \quad Y_k \in G \setminus Y_0 \quad Y_k \to Y_0 \\
          \text{Рассмотрим функциональную последовательность} F_k(X) f(X,Y_k) \tag{19} \\
     \end{gather*}
          Поскольку есть соотношение (16), то понятно, что 
          \[(16) \implies F_k(X) \darrow{\underset{k \to \infty}{x \in E}} \tag{20} \]
          Если соотношение (16) годится  для всего множества $G$, то годится оно и для любого его подмножества
          \begin{gather*}
               (17) \implies \forall k \exists \underset{x \to x_0}{lim} F_k(X) = h(Y_k) \tag{21} \\
               (20), (21) \exists \underset{X \to X_0}{lim} \phi(X) \tag{22} \\
               \exists \underset{k \to \infty}{lim} h^0_k \stackrel{def}{=} A \\
               A = \underset{X \to X_0}{lim} \phi(X) \tag{23} \\
          \end{gather*}
          Поскольку $A$ не зависит от последовательности $Y_k$
          \[ (23) \implies (18) \]
\end{theorem}
     \begin{corollary}
          \begin{gather*}
               f(X,Y) \quad X \in E \quad Y \in G \subset \vR^n \\
               X_0 \in E \quad X_0 - \text{ точка сгущения } E \quad Y_0 \text{ точка сгущения } G \\
               f(X,Y) \darrow{\underset{Y \to Y_0}{X \in}} \phi(X) \tag{ 24} \\
               f(X,Y) \text{ непрерывна в } X_0 \forall Y \in G \tag{25} \\
               \implies \phi \text{ непрерывна в } X_0 \\
          \end{gather*}
     \end{corollary}
     \begin{proof}
          \begin{gather*}
               (25) \implies \underset{X \to X_0}{lim} f(X,Y) = f(X_0,Y) \forall Y \in G \tag{27} \\
               (24),(27) \implies \exists \underset{X \to X_0}{lim} \phi(X) = \underset{Y \to Y_0}{lim} f(X_0,Y) \tag{28} \\
               (24) \implies f(X_0,Y) \underset{Y \to Y_0}{\rightarrow} \phi(X_0) \tag{29} \\
               (28),(29) \implies \exists \underset{X \to X_0}{lim} \phi(X) = \phi(X_0) \implies (26)
          \end{gather*}
     \end{proof}
     \begin{corollary}[Следствие из следствия???]
          \begin{gather*}
               f(X,Y) \quad X \in E \quad Y \in G \subset \vR^n \quad Y_0 - \text{ точка сгущения } G \\
               E \subset \vR^m \quad \forall x \in E - \text{ точка сгущения } E \\
               f(X,Y) \darrow{\underset{Y \to Y_0}{x \in E}} \phi(X)
               \text{Предположим, что} \forall Y \in G \quad f(X,Y) \in C(E) \tag{30} \\
               \implies \phi \in C(E) \tag{31}
          \end{gather*}
     \end{corollary}
     \begin{proof}
          Доказательство следует из предыдущего следствия. Мы можем применить его к каждой точке множества $E$
     \end{proof}
\end{document}
     