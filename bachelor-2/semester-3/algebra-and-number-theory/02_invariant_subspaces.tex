% !TeX root = ./main.tex
\documentclass[main]{subfiles}
\begin{document}


\chapter{Инвариантные подпространства}

\begin{definition} [Инвариантность пространств относительно оператора]
    $V$ — линейное конечномерное пространство, $\mca\in \End \ V$. Пусть $W  < V$ — линейное подпространство.
    $W$ — называется инвариантным относительно $\mca$, если $\forall w\in W : \mca(w) \in W$.
\end{definition}

\begin{propertylist}
    \
    \begin{enumerate}
        \item $0, W$ — $\mca$-инвариантны
        \item $\Ker \mca$ — $\mca$-инвариантно
        \item $\Im \mca$ — $\mca$-инвариантен
    \end{enumerate}
\end{propertylist}



Пусть $W$ — $\mca$-инвариант. Следовательно, $\mca|_W$ можно рассматривать как элемент $\End \ W$.
Более формально, $\exists\  \mca_1 \in \End \ W \ \forall w \in W : \mca_1 w=\mca w$.
\[\underset{w\mapsto \mca w}{W\xrightarrow{\mca_1} W}\]

$\mca_1$  —  оператор, индуцированный оператором $\mca$ на инвариантном подпространстве $W$.

$Пусть W < V,\ V/W = \{ v+w \ | \ v \in V\}$ — фактор-пространство.  $W$ — $\mca$-инвариант.
Определим $\mca_2$.

\[\mca_2: \underset{v+W \mapsto \mca v+W}{V/W \rightarrow V/W }\]

Проверка корректности: пусть $v_1+W = v_2+W$, нужно проверить, что $\mca v_1+W=\mca v_2 + W$. Так, $\mca v_2= \mca(v_1+(v_2-v_1))=\mca v_1+ \mca\underbrace{(v_2-v_1)}_{\in W} \Rightarrow
    \mca v_2 + W= \mca v_1+W$.

\begin{proposition} {}
    $\mca_2 \in \End \ V/W$
\end{proposition}

\begin{proof}
    Проверка линейности:
    \begin{multline*}
        \mca_2((v_1+W)+(v_2+W))= \mca_2((v_1+v_2)+W)=\\
        \mca(v_1+v_2)+W=  \mca v_1+\mca v_2+W=\\
        (\mca v_1+W)+(\mca v_2+W)
    \end{multline*}
    \begin{multline*}
        \mca_2(\alpha(v+W))= \mca_2(\alpha v+W)=\\
        \mca(\alpha v)+W= \alpha \mca v +W=\\
        \alpha(\mca v +W) = \alpha \mca_2(v+W)
    \end{multline*}
\end{proof}

$\mca_2$ — индуцированный оператор на фактор-пространстве.

\begin{proposition} {}
    Пусть $\mca \in \End \ V, \  W < V, \ e_1, \ldots ,e_m$ — базис $W$, $e_{m+1}, \ldots ,e_n$ — дополнение до базиса $V$. Тогда эквивалентны 2 утверждения:

    \begin{enumerate}
        \item $W$ — $\mca$-инвариант
        \item $[\mca]_{e_1, \ldots ,e_n}$ = $\left(
                  \begin{tabular}{c|c}
                          $A_1$ & $B$   \\
                          \hline
                          0     & $A_2$ \\
                      \end{tabular}
                  \right), \ A_1 \in M_m(K)$
    \end{enumerate}

    При этом $A_1=[\mca_1]_{e_1, \ldots ,e_m}, \ A_2=[\mca_2]_{e_{m+1}+W,\ldots ,e_n+W }$, где $\mca_1$ и $\mca_2$ — соответствующие индуцированные операторы.
\end{proposition}

\begin{proof}
    $1 \Rightarrow 2$:
    векторы $e_1, \ldots ,e_m \in W \Rightarrow \ \mca e_1, \ldots ,\mca e_m \in W = \Lin(e_1, \ldots ,e_m) \Rightarrow [\mca]_{e_1, \ldots ,e_n} = \left(
        \begin{tabular}{c|c}
                $A_1$ & $B$   \\
                \hline
                0     & $A_2$ \\
            \end{tabular}
        \right)$

    Очевидно, $[\mca_1]_{e_1, \ldots e_m} =A_1$.

    Пусть $[\mca]_{e_1, \ldots ,e_n} = (a_{ij})$.

    $\mca e_j = \underbrace{a_{1j}e_1+\ldots + a_{mj}e_m}_{\in W}+a_{m+1j}e_{m+1}+\ldots + a_{nj}e_n$, $j \geqslant m+1$

    $\underbrace{\mca e_j + W}_{=\mca_2(e_j+W)} = \underbrace{a_{m+1j}e_{m+1}+\ldots+a_{nj}e_n+W}_{=a_{m+1j}(e_{m+1}+W)+\ldots+a_{nj}(e_n+W)}$ (первые $m$ элементов станут нулевым классом).
    Таким образом, $[\mca_2]_{e_{m+1}+W, \ldots, e_n+W} =
        \begin{pmatrix}
            a_{m+1m+1} & \ldots & a_{m+1n} \\
            \vdots     & \ddots & \vdots   \\
            a_{nm+1}   & \ldots & a_{nn}   \\
        \end{pmatrix}
        = A_2$


    $2 \Rightarrow 1$:
    $[\mca]_{e_1, \ldots ,e_n} = \left(\begin{array}{c|c}
                A_1 & B   \\ \hline
                0   & A_2 \\
            \end{array}\right)
        \Rightarrow \mca e_1,\ldots,\mca e_m \in \Lin(e_1,\ldots ,e_m) \in W$.
    Пусть $w\in W \Rightarrow w=\beta_1e_1+\ldots+\beta_me_m \Rightarrow \mca w= \beta_1\underbrace{\mca_1e_1}_{\in W}+\ldots+\beta_m\underbrace{\mca_me_m}_{\in W}\in W$.
\end{proof}

\marginpar{14.09.22}

Итак, мы выяснили, что если в нашем подпространсве $V$ есть инвариантное подпространство меньшей размерности (ненулевое)
$W < V$, то это позволяет нам составить блочно-треугольную матрицу
$\left(\begin{array}{c|c}
            A_1 & B   \\ \hline
            0   & A_2 \\
        \end{array}\right)$, где $A_1,\ A_2$ -- квадратные матрицы, $\A_1 \in M_m(K),\ m = \dim W$.

В ситуации $V = W_1 \oplus W_2$ можно получить $\left(\begin{array}{c|c}
            A_1 & 0   \\ \hline
            0   & A_2 \\
        \end{array}\right)$.

\begin{proposition}
    Пусть $\mca \in \End V,\ V=W_1 \oplus W_2,\
        \underbrace{e_1, \ldots , e_m}_{E_1}$
    — базис $W_1$, $\underbrace{e_{m+1}, \ldots , e_{n}}_{E_2}$
    — базис $W_2$, $E = E_1+E_2$. Тогда эквивалентны 2 утверждения:
    \begin{enumerate}
        \item $ W_1,ё W_2 - \mca $-инвариантны
        \item $[\mca]_E = \left(\begin{array}{c|c}
                          A_1 & 0   \\
                          \hline
                          0   & A_2 \\
                      \end{array}\right),\ A_1 \in M_m(K),\ A_2 \in M_{n-m}(K)$.
              При этом $A_1 = [\mca|_{W_1}]_{E_1},\ A_2 = [\mca|_{W_2}]_{E_2}$
    \end{enumerate}
\end{proposition}

\begin{proof}
    Аналогично предыдущему предложению.

    $1 \Rightarrow 2$:
    $\mca e_1, \ldots, \mca e_m \in W_1 \Rightarrow$ $\left(\begin{array}{c|c}
                A_1 & \\
                \hline
                0   & \\
            \end{array}\right)$,\
    $\mca e_{m+1}, \ldots, \mca e_n \in W_2 \Rightarrow$ $\left(\begin{array}{c|c}
                 & 0   \\
                \hline
                 & A_2 \\
            \end{array}\right) $.

    $2\Rightarrow 1$:
    $\left(\begin{array}{c|c}
                  & \\
                \hline
                0 & \\
            \end{array}\right) $ $\Rightarrow \mca e_1, \ldots, \mca e_m \in \Lin(e_1, \ldots, e_m) = W_1
        \Rightarrow \forall w \in W_1,\ \mca w \in W_1,\ W_1$ — $\mca$-инвариант.

    $\left(\begin{array}{c|c}
                 & 0 \\
                \hline
                 &   \\
            \end{array}\right) $ $\Rightarrow \mca e_{m+1}, \ldots, \mca e_n \in \Lin(e_{m+1}, \ldots, e_n) = W_2
        \Rightarrow \forall w \in W_2,\ \mca w \in W_2,\ W_2$ — $\mca$-инвариант.
\end{proof}

Что означает в терминах оператора, что матрица получилась диагональной? Например, образ первого базисного вектора будет
прямо пропорционален первому базисному вектору: $\mca e_1 = \lambda_1 e_1,\ \mca e_2 = \lambda_2 e_2$ и т.д.
\end{document}