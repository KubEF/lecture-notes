% !TeX root = ./main.tex
\documentclass[main]{subfiles}
\begin{document}


\chapter{Многочлены от операторов}

Пусть $\mca \in \End V$ ($V$ над $K$), $f \in K[X]$. $f = \alpha_n x^n + \ldots + \alpha_1 x + \alpha_0$.
$f(\mca) = \alpha_n {\mca}^n + \alpha_{n-1} {\mca}^{n-1} + \ldots + \alpha_1 \mca + \alpha_0 \epsilon_V \in \End V$.

\begin{proposition}
    $f, g \in K[X]$
    \begin{enumerate}
        \item $(f+g)(\mca) = f(\mca) + g(\mca)$.
        \item $(fg)(\mca) = f(\mca)g(\mca) =g(\mca)f(\mca)=gf(\mca)$ (т.к. многочлены коммутируют).
    \end{enumerate}
\end{proposition}

\begin{proof}
    Непосредственная проверка.
\end{proof}

\begin{corollary}
    Пусть $\mca \in \End V$, $f \in K[X]$, тогда $\Ker f(\mca), \Im f(\mca)$ -- $\mca$-инвариантные подпространства.
\end{corollary}

\begin{proof}
    $v \in \Ker f(\mca)$, т.e. $f(\mca) (v) = 0$. Действуем на $v$ оператором $\mca$: $f(\mca)(\mca v) =
        (f(\mca)\mca)(v) = (\mca f(\mca))(v) = \mca(0) = 0$.
    Таким образом, $\mca v \in \Ker f(\mca)$.

    $\Im f(\mca)$ -- $\mca$-инвариантное подпространство.
    Пусть $v \in \Im f(\mca)$. Это означает, что $\exists \ w$: $v = f(\mca)(w) \Rightarrow
        \mca v = (\mca f(\mca))(w) = (f(\mca)\mca)(w) = f(\mca)(\mca w) \in \Im f(\mca)$.
\end{proof}

$R$ -- коммутативноe ассоциативное кольцо с 1. Подмножество $I\subset R$ называется идеалом, если:
\begin{enumerate}
    \item $I$ -- подгруппа по сложению.
    \item $\forall a \in I \ \forall r \in \R : ra \in I$.
\end{enumerate}

Пусть $c \subset R$, $(c) = \{cx | x \in K\}$ -- идеал в $R$, главный идеал, порожденный $c$.

\begin{corollary}
    В $K[X]$, где $K$ -- поле, все идеалы главные.
\end{corollary}

\begin{proof}
    Пусть $I \subset K[X]$.

    $I = 0$. $I = (0)$.

    $I \neq 0 $, $h$ - многочлен наименьшей степени, входящий в $I\setminus \{0\}$.
    Докажем: $I = (h)$. $h \in I \Rightarrow (h) \subset I$.
    Осталось: $I \subset (h)$. $f \in I \Rightarrow f=hq+r$, где $\deg r < \deg h
        \Rightarrow r = \underbrace{f}_{\in I} - \underbrace{h}_{\in I} q \in I \Rightarrow r = 0 \Rightarrow f = hq \in (h)$.
\end{proof}

\begin{remark}
    Аналогично доказывается, что любая евклидова область -- ОГИ (область главных идеалов).
\end{remark}

\begin{remark}
    $\Z[x]$ - не ОГИ. Например, $I = \{f|f(0) \vdots 2\} \neq (2),\ \neq (x),\ \neq (\pm 1)$.
\end{remark}

\begin{definition} [Аннулятор]
    Пусть $\mca \in \End V : v\in V$. $f \in K[X]$ называется аннулятором $v$ по отношению к оператору $\mca$, если $f(\mca)(v) = 0$.
\end{definition}

\begin{lemma}
    $I = \{f \ | \ f$ -- аннулятор $v \}$ -- идеал в $K[X]$.
\end{lemma}

\begin{proof}
    $f, g \in I,\ (f-g)(\mca)(v) = (f(\mca)-g(\mca))(v) = f(\mca)(v) - g(\mca)(v) = 0$.

    $f \in I,\ h \in K[X]$. $(hf)(\mca)(v) = h(\mca)(\underbrace{f(\mca)(v)}_{0}) = 0 \Rightarrow hf \in I$.
\end{proof}

Пусть $\dim V = n$. $v, \mca v, {\mca}^2 v, \ldots, {\mca}^n v$ - ЛЗС.
$f(\mca)(v) = \alpha_0 v + \alpha_1 \mca v + \ldots + \alpha_n {\mca}^n v = 0$, не все $\alpha_i =0$.
$f = \alpha_0 + \alpha_1 x + \ldots + \alpha_n x^n \neq 0 \Rightarrow f$ -- аннулятор $v$.

$I$ -- главный идеал $\Rightarrow I=(f_0), f_0 \neq 0$. $f_0$ -- минимальный аннулятор $v$ (минимальный аннулирующий многочлен).

Есть вектор $v \in V$, можно ли найти минимальное подпространство, содержащее этот вектор? $v\in W$, $W$ -- инвариант $\Rightarrow \mca v \in W \Rightarrow {\mca}^2 v \in W \Rightarrow \ldots \Rightarrow  {\mca}^k v \in W, \ \forall k \in \N$.
Отсюда понятно, как построить искомое подпрострнство: нужно «натянуть» пространство на все векторы вида ${\mca}^k v \in W$. Тогда пространство представляет собой линейную оболочку и как раз является инвариантным.

\begin{definition} [Циклическое подпространствo]
    Циклическим подпространством, порожденным $v$, называется $C_v = Lin(v, \mca v, {\mca}^2 v, \ldots)$.
\end{definition}

\begin{proposition}
    Пусть $f_0$ -- минимальный аннулятор $v$, $d = deg f_0$. Тогда $C_v$ -- $\mca$-инвартное подпространство с базисом
    $v, \mca v, {\mca}^2 v, \ldots, {\mca}^{d-1} v$.
\end{proposition}

\begin{proof}
    Всякое подпространство -- линейная оболочка, это проверять не нужно.

    Проверим $\mca$-инвариантность: $w \in C_v \Rightarrow w = \alpha_0 v + \alpha_1 \mca v + \ldots + \alpha_m {\mca}^{m} v \Rightarrow
        \mca w = \alpha_0 \mca v + \alpha_1 {\mca}^2 v + \ldots + \alpha_m {\mca}^{m+1} v \in C_v$.

    Проверим базис. Предположим, $v, \mca v, {\mca}^2 v, \ldots, {\mca}^{d-1} v$ -- ЛЗС. $g(\mca)(v) = \beta_0 v + \beta_1 \mca v + \ldots + \beta_{d-1} {\mca}^{d-1} = 0$, не все $\beta = 0$.

    $g = \beta_0 + \beta_1 x + \ldots + \beta_{d-1} x^{d-1} \neq 0 \Rightarrow
        g$ -- аннулятор $v \Rightarrow g \in (f_0) \Rightarrow f_0|g \Rightarrow deg g \leq d-1$, пришли к противоречию.
    Таким образом, $v, \mca v, {\mca}^2 v, \ldots, {\mca}^{d-1} v$ -- ЛНC.

    Осталось проверить $C_v = \underbrace{Lin(v, \mca v, {\mca}^2 v, \ldots, {\mca}^{d-1} v)}_{W}$.

    Докажем индукцией по $k$: ${\mca}^{k} v \in W$.

    База: $k = 0, 1, \ldots, d-1 \Rightarrow {\mca}^k \in W$ по определению.

    Переход: $k\geq d$.

    По индукционному предположению: ${\mca}^{k-1} v \in W$,
    т.е. ${\mca}^{k-1}v = \gamma_0 v + \gamma_1 \mca v + \ldots + \gamma_{d-1} {\mca}^{d-1} v \Rightarrow
        {\mca}^k v = \underbrace{\gamma_0 \mca v + \gamma_1 {\mca}^2 v + \ldots + \gamma_{d-2} {\mca}^{d-1} v}_{W} + \gamma_{d-1} {\mca}^d v$.

    ${\mca}^d v \in W$?

    $f_0 = \beta_0 + \beta_1 x + \ldots + \beta_{d-1} x^{d-1} + \beta_d x^d$ -- минимальный аннулятор, $B_d \neq 0$.

    $0 = f_0(\mca) v = \underbrace{\beta_0 v +\beta_1 \mca v + \ldots + \beta_{d-1} {\mca}^{d-1} v}_{\in W} + \beta_d {\mca}^d v \Rightarrow {\mca}^d v \in W \Rightarrow
        {\mca}^k v \in W, \  \forall \ k \in \N$.

\end{proof}

\marginpar{28.09.22}

Пусть $v \in V$, $\mca \in \End V$.

$C_v = Lin(v, \mca v, {\mca}^2 v, \ldots) = Lin(v, \mca v, {\mca}^2 v, \ldots, {\mca}^{d-1} v)$, где $d = deg f_0$, $f_0$ -- минимальный аннулятор $v$.

Итак, что представляет собой $\chi_{\mca |_{C_v}} = ?$

$f_0 = \beta_0 + \beta_1 x + \ldots + \beta_{d-1} x^{d-1} + \beta_{d} x^{d}$, $\beta_d \neq 0$.

Домножив $f_0$ на ненулевую константу, мы можем считать $\beta_d = 1$.

$E = (v, \mca v, {\mca}^2 v, \ldots, {\mca}^{d-1} v)$ -- базис $C_v$.

Легко видеть:

\[[\mca |_{C_v}]_E = \begin{pmatrix}
        0      & 0      & 0      & \ldots & 0      & -\beta_0     \\
        1      & 0      & 0      & \ldots & 0      & -\beta_1     \\
        0      & 1      & 0      & \ldots & 0      & -\beta_2     \\
        0      & 0      & 1      & \ldots & 0      & -\beta_3     \\
        \vdots & \vdots & \vdots & \ddots & \vdots & \vdots       \\
        0      & 0      & 0      & \ldots & 0      & -\beta_{d-2} \\
        0      & 0      & 0      & \ldots & 1      & -\beta_{d-1} \\
    \end{pmatrix} = L_{f_0}\]

$L_{f_0}$ -- сопровождающая матрица многочлена $f_0$.

*получаем последний столбец*

$f_0(\mca) v = 0 = \beta_0 v = \beta_1 \mca v + \ldots + \beta_{d-1} \alpha_{d-1} v + \alpha_d v$.

${\mca_v}^d = -\beta_0 v - \beta_1 \mca v - \ldots - \beta_{d-1} {\mca}^{d-1} v$ -- разложение по базису $E$

*получаем последний столбец*



\begin{align*}
    \chi_{\mca |_{C_v}} & = \begin{vmatrix}
                                -x     & 0      & 0      & \ldots & 0      & -\beta_0         \\
                                1      & -x     & 0      & \ldots & 0      & -\beta_1         \\
                                0      & 1      & -x     & \ldots & 0      & -\beta_2         \\
                                \vdots & \vdots & \vdots & \ddots & \vdots & \vdots           \\
                                0      & 0      & 0      & \ldots & -x     & -\beta_{d-2}     \\
                                0      & 0      & 0      & \ldots & 1      & -\beta_{d-1} - x \\
                            \end{vmatrix} =                        \\
                        & = (-1)^{d+1} (-\beta_0) +                                                              \\
                        & + (-1)^{d+2} (-\beta_1) \underbrace{\begin{vmatrix}
                                                                      -x     & 0      & 0      & \ldots & 0      \\
                                                                      0      & 1      & -x     & \ldots & 0      \\
                                                                      0      & 0      & 1      & \ldots & 0      \\
                                                                      \vdots & \vdots & \vdots & \ddots & \vdots \\
                                                                      0      & 0      & 0      & \ldots & 1      \\
                                                                  \end{vmatrix}}_{-x} +         \\
                        & + (-1)^{d+3}(-\beta_2) \underbrace{\begin{vmatrix}
                                                                     -x     & 0      & 0      & 0      & \ldots & 0      \\
                                                                     1      & -x     & 0      & 0      & \ldots & 0      \\
                                                                     0      & 0      & 1      & -x     & \ldots & 0      \\
                                                                     0      & 0      & 0      & 1      & \ldots & 0      \\
                                                                     \vdots & \vdots & \vdots & \vdots & \ddots & \vdots \\
                                                                     0      & 0      & 0      & 0      & \ldots & 1      \\
                                                                 \end{vmatrix}}_{x^2} + \ldots + \\
                        & + (-1)^{2d-1}(-\beta_{d-1})(-x)^{d-2} +                                                \\
                        & + (-1)^{2d}(-\beta_{d-1}-x)(-x)^{d-1} =                                                \\
                        & = (-1)^{d+1} \sum_{i=0}^{d-1} (-1)^i (-\beta_i)(-x)^i + (-x)^d =                       \\
                        & = (-1)^{d+1}(\sum_{i=1}^{d-1} \beta_i)x^i + x^d) =                                     \\
                        & = (-1)^d f_0
\end{align*}

Короче, $\chi_{\mca |_{c_v}} = (-1)^d f_0$.

\end{document}