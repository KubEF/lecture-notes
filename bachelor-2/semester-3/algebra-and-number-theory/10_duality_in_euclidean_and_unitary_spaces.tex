% !TeX root = ./main.tex
\documentclass[main]{subfiles}
\begin{document}

\chapter{Двойственность в евклидовых и унитарных пространствах}

Будем считать. что $V$ -- евклидово или унитарное пространство над полем $K$ ($\R$ или $\C$), $\dim V = n \leq \infty$.

$w \in V, \ \rho_w: \underset{v \mapsto (v, w)}{V \rightarrow K}$, поскольку умножаем справа, отображение линейно: $(\alpha_1 v_1 + \alpha_2 v_2, w) = \alpha_1 (v_1, w) + \alpha_2 (v_2, w)$.
Тем самым, $\rho \in V^*$.

\begin{proposition}
    Пусть $V$ евклидово. Тогда $\rho_w: \underset{w \mapsto \rho w}{V \rightarrow V^*}$ -- изоморфизм линейных пространств.
\end{proposition}

\begin{proof}
    Знаем, что размерности одинаковы, остается проверить инъективность.

    Пусть $w \in \Ker \rho$, тогда $0 = \rho_w(w) = (w, w) \Rightarrow w = 0 \Rightarrow \rho$ -- инъективно
    $\xRightarrow{\text{по пр. Дирихле}} \rho$ -- сюръективно.
\end{proof}

$V_1, V_2$ -- линейное пространства над полем комплексных чисел. $\mca: V_1 \rightarrow V_2$ --  называется полулинейным, если:
\begin{enumerate}
    \item $\mca(v+v') = \mca v + \mca v'$
    \item $\mca(\alpha v) = \overline{\alpha} v$
\end{enumerate}
   
\begin{proposition}
    Пусть $V$ унитарное. Тогда $\rho_w: \underset{w \mapsto \rho w}{V \rightarrow V^*}$ -- полулинейная биекция.
\end{proposition}

\begin{proof}
    Полулинейность очевидна, $\Ker \rho = 0$ доказывается таким же образом.

    $\widetilde{V} = V$, сложение как в $V$, $\alpha \ast v = \overline{\alpha} v$.
    $(\widetilde{V}, \ +, \ \ast)$ -- линейное пространство над $\C$. Базисы $\widetilde{V}, \ V$ совпадают $\Rightarrow \dim \widetilde{V} = n$.
    Рассмотрим $\widetilde{\rho}: \underset{w \mapsto \rho w}{\widetilde{V} \rightarrow V^*}$ -- линеное отображение, ядро по-прежнему равняется 0, тогда
    $\Im \widetilde{\rho} = V^*$.
\end{proof}

\[ \mca: V \rightarrow W\]

% https://q.uiver.app/?q=WzAsNCxbMCwwLCJXXioiXSxbMSwwLCJWXioiXSxbMCwxLCJXIl0sWzEsMSwiViJdLFswLDEsIlxcbWF0aGNhbHtBfV5UIl0sWzMsMSwiXFxyaG9fdiIsMl0sWzIsMCwiXFxyaG9fdyJdLFsyLDMsIlxcbWF0aGNhbHtBfV4qIiwyLHsic3R5bGUiOnsiYm9keSI6eyJuYW1lIjoiZGFzaGVkIn19fV1d
\[\begin{tikzcd}
        {W^*} & {V^*} \\
        W & V
        \arrow["{\mathcal{A}^T}", from=1-1, to=1-2]
        \arrow["{\rho_V}"', from=2-2, to=1-2]
        \arrow["{\rho_W}", from=2-1, to=1-1]
        \arrow["{\mathcal{A}^*}"', dashed, from=2-1, to=2-2]
    \end{tikzcd}\]

Пусть $\mca \in \Hom(V, W)$. Рассмотрим отображение $\mca^* = \rho_V^{-1} \circ \mca^T \circ \rho_W$. Тогда $\mca^* \in \Hom(W, V)$ называется отображением, сопряженным к $\mca$.
Очевидно, $\mca^*$ -- гомоморфизм групп.
\begin{gather*}
    \mca^*(\alpha w) = (\rho_V^{-1} \circ \mca^T \circ \rho_W)(\alpha w) = (\rho_V^{-1} \circ \mca^T)(\overline{\alpha} \rho_W(w)) = \\
    = \rho_V^{-1} (\overline{\alpha} ( \mca^T \circ \rho_W)(w) = \\
    = \alpha (\rho_V^{-1} \circ \mca^T \circ \rho_W) (w) = \alpha \mca^* (w)
\end{gather*}

\begin{proposition} [Характеристическое свойство сопряженного отображения]
    Пусть $\mca \in \Hom(V, W)$.

    \begin{enumerate}
        \item $\forall v \in V, \ w \in W: \ (\mca v, w) = (v, \mca^* w)$
        \item Если $\mcb \in \Hom (W, V)$ такое, что $\forall v \in V, \ w \in W: \ (\mca v, w) = (v, \mcb w)$,  то $\mcb = \mca^*$.
    \end{enumerate}
\end{proposition}

\begin{proof}
    \begin{enumerate}
        \item $(v, \mca^* w) = \rho_{\mca^*, w} (v) = \rho_V (\mca^*w)(v) = \mca^T(\rho_w(w))(v) = (\rho_w(w) \circ \mca)(v) = \rho_w (\mca v) = (\mca v, w)$
        \item Имеем: $\forall v \in V, \ w \in W: \ (\mca v, w) = (v, \mca^* w) \Rightarrow \\ \Rightarrow (v_1 (\mcb - \mca)(w)) = 0 \Rightarrow \forall w \in W: \ ((\mcb - \mca^*)w, (\mcb - \mca^*)w) = \\ = 0 \Rightarrow \forall w \in W: (\mcb - \mca^*)(w) = 0$, т.е. $\mcb w = \mca^* w$.
    \end{enumerate}
\end{proof}



\end{document}