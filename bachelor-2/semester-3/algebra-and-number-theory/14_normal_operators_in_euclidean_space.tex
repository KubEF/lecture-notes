% !TeX root = ./main.tex
\documentclass[main]{subfiles}

\begin{document}

\chapter{Нормальные операторы в евклидовом пространстве}

$V$ -- евклидово пространство, $\mca \in \End V$ -- нормальный.

$\mca_{\C}$ -- нормальный. $\mca_{\C}\mca_{\C}^* = \mca_{\C}(\mca^*)_{\C} = (\mca\mca^*)_{\C} = (\mca^*\mca)_{\C} = \mca_{\C}^*\mca_{\C}$.

$\underbrace{\chi_{\mca}}_{\in \R[X]} = \chi_{\mca_{\C}} = \prod_{i=1}^s(x - \lambda_i)^{g_i} \prod_{i=1}^t(x - \mu_i)^{h_i} (x - \overline{\mu_i})^{h_i}, \ \lambda_i \in \R, \ \mu_i \not\in \R$.

\marginpar{09.11.22} 

$V_{\C} = V_{\lambda_1}\oplus\ldots\oplus V_{\lambda_1}\oplus \underbrace{V_{\mu_i}\oplus V_{\overline{\mu_i}}}_{W_1} \oplus \ldots \oplus\underbrace{V_{\mu_t} \oplus V_{\overline{\mu_t}}}_{W_t}$.

Что мы сейчас делаем? В каждом из подпространств существует свой ортонормированный базис, объедивнишись, они составят ортонормированный базис всего пространства, так как собственные векторы, принадлежащие 
разным собственным значениям, между собой ортогональны. Мы же будем строить базис только из вещественных векторов.

Пусть $v$ -- собственное значение $\mca_{\C}$, $V_v = \Ker (\mca_{\C} - v \mse_{V_{\C}})$. 

$W = W_l = V_{\mu_l} \oplus V_{\overline{\mu_l}}$. Выберем в $V_{\mu_l}$ какой-нибудь ортонормированный базис: $v_1 + iw_1, \ldots, v_q + iw_q \Rightarrow
v_1 - iw_1, \ldots, v_q - iw_q$ -- базис $V_{\overline{\mu_l}}$.

$\mu_1 \neq \overline{\mu_2} \Rightarrow (v_j - iw_j) \perp (v_k + iw_k), \ j, k = 1, \ldots, q$.

$W = V_{\mu_l} + V_{\overline{\mu_l}} = \Lin(v_1 + iw_1, \ldots, v_q + iw_q, v_1 - iw_1, \ldots, v_q - iw_q) = \Lin(v_1, \ldots, v_q, w_1, \ldots, w_q)$. 

$\subset$: очевидно, $\supset$: $\dim W = 2q \Rightarrow \dim \Lin (v_1, \ldots, v_q, w_1, \ldots, w_q) \geq 2q \ (\supset W, \ \dim W = 2q) \Rightarrow
\dim \underbrace{\Lin (v_1, \ldots, v_q, w_1, \ldots, w_q)}_{=W} = 2q$ (с другой стороны, $v_1 = \frac{1}{2}((v_1 + iw_1) + (v_1 - iw_1))$ и т.д.) $\Rightarrow
v_1, \ldots, v_q, w_1, \ldots, w_q$ -- базис $W$.

При $j \neq k$: 
\begin{gather*}
    (v_j + iw_j, v_k \pm iw_k) = (v_i, v_k) \pm (w_j, w_k) + i((v_k, w_i) \mp (v_j, w_k)) = 0 \Rightarrow \\ \Rightarrow
\left\{ \begin{gathered} 
    (v_j, v_k) + (w_j, w_k) = 0 \\
    (v_j, v_k) - (w_j, w_k) = 0 \\
    (v_k, w_j) - (v_j, w_k) = 0 \\
    (v_k, w_j) + (v_j, w_k) = 0 
\end{gathered} \right. \Rightarrow \left\{ \begin{gathered} 
    (v_j, v_k) = 0 \\
    (w_j, w_k) = 0 \\
    (v_k, w_j) = 0 \\
    (w_j, w_k) = 0 
\end{gathered} \right.
\end{gather*} 

\begin{gather*}
    \begin{gathered} 
        (v_j + iw_i, v_j + iw_j) = 1 \\
        (v_j + iw_i, v_j - iw_j) = 0 
    \end{gathered} \Rightarrow \left\{ \begin{gathered} 
        (v_j, v_j) + (w_j, w_j) + i((v_j, w_j) - (v_j, w_j)) = 1 \\
        (v_j, v_j) - (w_j, w_j) + i((v_j, w_j) + (v_j, w_j)) = 0 
    \end{gathered} \right.  \Rightarrow \\
    \Rightarrow 
    \left \{ 
        \begin{gathered} 
        (v_j, v_j) + (w_j, w_j) = 1 \\
        (v_j, v_j) - (w_j, w_j) = 0 \\
        2(v_j, w_j) = 0
    \end{gathered} \right. 
    \Rightarrow 
    \left \{ 
        \begin{gathered} 
        (v_j, v_j) = \frac{1}{2} \\
        (w_j, w_j) = \frac{1}{2} \\
        (v_j, w_j) = 0
    \end{gathered} \right. 
\end{gather*}

Таким образом, $v_1, \ldots, v_q, w_1, \ldots, w_q$ -- ортогональный базис $W$; $(v_j, v_j) = (w_j, w_j) = \frac{1}{2}$.

$\widetilde{v_j} = \sqrt{2}v_j, \ \widetilde{w_j} = \sqrt{2}w_j \Rightarrow \widetilde{v_1}, \widetilde{w_1}, \ldots, \widetilde{v_q}, \widetilde{w_q}$ -- ортонормированный базис $W$.

$\underbrace{V_{\lambda_l}}_{\lambda_l \in \R} = \Lin(v_1 + iw_1, \ldots, v_p + iw_p)$ для некоторых векторов $v_1, \ldots, v_p, w_1, \ldots, w_p$.

\begin{gather*}
    \underbrace{\mca_{\C}(v_j + iw_j)}_{=\lambda_l(v_j + iw_j) = \lambda_l v_j + i\lambda_l w_j} = \mca v_j + i\mca w_j \Rightarrow \\
    \Rightarrow \mca v_j = \lambda_l v_j, \ \mca w_j = \lambda_l w_j \Rightarrow v_j, w_j \in V_{\lambda_l} \Rightarrow \\
    \Rightarrow  V_{\lambda_l} = \Lin(v_1, \ldots, v_p, w_1, \ldots, w_p) \ (\subset: \text{очевидно}, \supset: \text{т.к. } v_j, w_j \in  V_{\lambda_l})
\end{gather*}

Выбрав из $v_1, \ldots, v_p, w_1, \ldots, w_p$ базис $V_{\lambda_l}$ и применив к нему процесс ортогонализации Грамма-Шмидта, 
получим ортонормированный базис $ V_{\lambda_l}$ из вещественных векторов. Объединив базисы все $ V_{\lambda_l}$ и $W_l$, получим ортонормированный вещественный базис $V_{\C}$.
$ V_{\lambda_l} \perp  V_{\lambda_k} \ (k \neq l), \ W_l \perp  W_k \ (k \neq l), \ W_l \perp V_{\lambda_k}$.

Его элементы -- ЛНС в $V_{\C} \Rightarrow$ ЛНС в $V \Rightarrow$ базис $V$, т.к. количество векторов $= \dim V_{\C} = \dim V$.

Теперь посмотрим на матрицу оператора в найденном базисе.

\begin{gather*}
    v \in V_{\lambda_l} \Rightarrow \underbrace{\mca_{\C}v}_{\mca v} = \lambda_l(v + i0) = \lambda_l v \\
    \begin{pmatrix}
    \lambda_l & \ldots       & 0    \\
    \vdots     & \ddots & \vdots     \\
    0       & \ldots       & \lambda_l      \\
  \end{pmatrix} = diag(\underbrace{\lambda_l, \ldots, \lambda_l}_{p}) \text{ -- соответствует } V_{\lambda_l}
\end{gather*}

\begin{gather*}
    W = V_{\mu_l} \oplus V_{\overline{\mu_l}}, \ \widetilde{v_1}, \widetilde{w_1}, \ldots, \widetilde{v_q}, \widetilde{w_q} \\
    \mca_{\C}(v_j+ iw_j) = \mu_l (v_j+ iw_j) \\
    \mu_l = \alpha + i \beta, \ \alpha, \beta \in \R, \ \beta \neq 0 \\
    \mca v_j + i \mca w_j =\alpha v_j - \beta w_j + i(\alpha w_j + \beta v_j) \Rightarrow \left\{ \begin{gathered}
        \mca v_j=\alpha v_j - \beta w_j \\
        \mca w_j=\alpha w_j + \beta v_j
     \end{gathered} \right.
    \Rightarrow  \\ \Rightarrow \left\{ \begin{gathered}
       \mca \widetilde{v_j}=\alpha \widetilde{v_j} - \beta \widetilde{w_j} \\
        \mca \widetilde{w_j}=\alpha w_j + \beta v_j
    \end{gathered} \right. \quad
    \begin{pmatrix}
        & & \vdots & \vdots &  \\
        \ldots & \ldots & \alpha & \beta & \ldots \\
        \ldots & \ldots & -\beta & \alpha & \ldots \\
        & & \vdots & \vdots &  \\
    \end{pmatrix}
\end{gather*}

\begin{theorem} [Спектральная теорема для нормальных операторов в евклидовом пространстве]
    $V$ -- евклидово, $\mca \in \End V$ -- нормальный. Тогда в $V$ есть ортонормированный базис $E$, т.ч. $[\mca]_E$ -- блочно-диагональная матрица с блоками $1\times 1, \ 2\times 2$ вида 
    $\begin{pmatrix}
        \alpha & \beta \\
        -\beta & \alpha
    \end{pmatrix} = \begin{pmatrix}
        \alpha & 0 \\
        0 & \alpha
    \end{pmatrix} + \begin{pmatrix}
        0 & \beta \\
        -\beta & 0
    \end{pmatrix}$ (оператор = самосопряж. + кососим.)
\end{theorem}

$\mca$ называется ортогональным, если $\mca \mca^* = \mse = E_n \Leftrightarrow A = [\mca]_E \in O_n$, где $E$ -- ортонормированный базис $V$.

\begin{gather*}
    A = diag(\ldots, \lambda, \ldots, \begin{pmatrix} \alpha & \beta \\ -\beta & \alpha \end{pmatrix}, \ldots) \\
    A^T = diag(\ldots, \lambda, \ldots, \begin{pmatrix} \alpha & -\beta \\ \beta & \alpha \end{pmatrix}, \ldots) \\
    AA^T = diag(\ldots, \lambda^2,\ldots, \begin{pmatrix} \alpha^2 + \beta^2 & 0 \\ 0 & \alpha^2 + \beta^2 \end{pmatrix}, \ldots) \\
    A \in O_n \Leftrightarrow \text{ все } \lambda^2 = 1 \text{ и все } \alpha^2 + \beta^2 = 1 \\
    R_{\phi} = \begin{pmatrix}
        \cos \phi & -\sin \phi \\
        \sin \phi & \cos \phi
    \end{pmatrix} \text{ (матрица поворота плоскости), } \phi, \text{т.ч. } \beta = -\sin \phi, \ -\beta = \sin \phi
\end{gather*}

\begin{corollary}
    Пусть $\mca \in \End V$ -- ортогональный. Тогда в $V$ существует ортонормированный базис $E$, т.ч.
    $[\mca]_E = diag(R_{\phi_1}, \ldots, R_{\phi_t}, \pm 1, \ldots, \pm 1)$.
\end{corollary}

\begin{corollary} [Теорема Эйлера]
    Пусть $V$ -- трехмерное евклидово пространство, $\mca \in \End V$ -- ортогональный. Тогда в некотором ортонормированном базисе его матрица имеет вид 
    $\left(\begin{array}{c c}
            R_{\phi}       & \begin{matrix}
                0      \\
                \vdots \\
                0      \\
            \end{matrix}   \\
            
            0 \ldots 0 & \pm 1 \\
        \end{array}\right)$.
\end{corollary}

$\dim V$ нечетный: $[\mca]_E = diag(R_{\phi_1}, \ldots, R_{\phi_s}, \pm 1)$

$\dim V$ четный, $|\mca| = 1$: $[\mca]_E = diag(R_{\phi_1}, \ldots, R_{\phi_s})$

$\dim V$ четный, $|\mca| = -1$: $[\mca]_E = diag(R_{\phi_1}, \ldots, R_{\phi_s}, 1, -1)$

\end{document}