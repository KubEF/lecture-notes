% !TeX root = ./main.tex
\documentclass[main]{subfiles}
\begin{document}


\chapter{Характеристический многочлен оператора}

$\mca \in \End V$, $[\mca]_E = A$.
Задача: найти собственное значение $\mca$.
$\lambda$ -- собственое значение $\mca \Leftrightarrow
    \Ker(\mca - \lambda\epsilon) \neq 0 \Leftrightarrow
    \underbrace{[\mca - \lambda\epsilon]_E}_{=A-\lambda E_n} \not\in \GL_n(K) \Leftrightarrow |\mca - \lambda\epsilon| = 0$. Задача сводится к нахождению таких
$\lambda$, при которых определитель матрицы равен нулю.


\[\A = \begin{pmatrix}
        a_{11} & \ldots & a_{1n} \\
        \vdots & \ddots & \vdots \\
        a_{n1} & \ldots & a_{nn} \\
    \end{pmatrix}\]

\[|\A - \lambda E_n| = \begin{pmatrix}
        a_{11} - \lambda & a_{12}           & \ldots & a_{1n}           \\
        a_{21}           & a_{22} - \lambda & \ldots & a_{2n}           \\
        \vdots           & \vdots           & \ddots & \vdots           \\
        a_{n1}           & a_{n2}           & \ldots & a_{nn} - \lambda \\
    \end{pmatrix}\]


Например, $\begin{pmatrix}
        a_{11} - \lambda & a_{12}           \\
        a_{21}           & a_{22} - \lambda \\
    \end{pmatrix} = (a_{11} - \lambda)(a_{22} - \lambda) - a_{12} a_{21} =
    \lambda ^2 - \lambda(a_{11}+a_{22}) + a_{11}a_{22} - a_{12} a_{21}$.

Определитель обращается в ноль, когда $\lambda$
является корнем этого многочлена.

\begin{definition} [Характеристический многочлен]
    Пусть $\A \in M_n(K)$. Его характеристический многочлен называется
    $\chi_\A = \underbrace{|A - X\cdot E_n|}_{\in M_n(K[x])\subset M_n(K(x))} \in K[x]$.

    $\begin{vmatrix}
            \ddots &        \\
                   & \ddots \\
        \end{vmatrix} = (a_{11} - x)(a_{22} - x)\ldots(a_{nn} - x) + G =
        (-1)^{n-1}x^n+(-1)^{n-1}\underbrace{(a_{11}+\ldots+a_{nn})}_{\Tr\A}x^{n-1}+\ldots+|\A|$,
    где $\A= (a_{ij}),\ \deg G \leq n-2,\ \Tr\A$ -- след матрицы.
\end{definition}

\begin{definition}
    Пусть $\mca \in \End V$. Его характеристическим многочленом $\chi_{\mca}$ называется
    $\chi_{[\mca]_E}$, где $E$  — любой базис $\V$.
\end{definition}

Проверка корректности независимости выбора базиса: пусть $\A = [\mca]_E,\ \A_1 = [\mca]_{E_1}$,
$C = M_{E \rightarrow E_1}$. Нужно: $\chi_{\mca} = \chi_{\mca_1}$.

\[\A_1 = C^{-1}\A C\]
\begin{multline*}
    \chi_{A_1} = |\A_1 - X E_n| =
    |C^{-1}\A C - XC^{-1}C| = \\ = |C^{-1}\A C - C^{-1}XE_nC| =
    |C^{-1}(\A - XE_n) C| = \underbrace{|C^{-1}}_{|C|^{-1}}|\A - XE_n||C| = \\ =
    |\A - XE_n| = \chi_{A}
\end{multline*}

У эквивалентных матриц след одинаков.

\marginpar{21.09.22}

\begin{definition} [Алгебраическая кратность]
    Кратность корня $\lambda$ многочлена $\chi_{\mca}$ называется алгебраической кратностью собственного значения $\lambda$ (обозначается $a_{\lambda}$).
\end{definition}

\begin{proposition}
    Пусть $\mca \in \End V$
    \begin{enumerate}
        \item Пусть $W$ -- ${\mca}$-инвариантное подпространство $V$; $\mca_1 = \mca|_w \in W$. Тогда $\chi_{\mca_1} | \chi_{\mca}$.
        \item Пусть $V = W_1 \oplus W_2;\ W_1,\ W_2$ -- $\mca$-инвариантны. $\mca_1 = \mca|_{W_1},\ \mca_2 = \mca|_{W_2} \Rightarrow
                  \chi_{\mca} = \chi_{\mca_1}\chi_{\mca_2}$.
    \end{enumerate}

\end{proposition}

\begin{proof}
    1: $E$ -- базис $V$, начальная часть которого -- базис $W$.

    $[\mca]_E = \left(\begin{array}{c|c}
                A_1 & B   \\ \hline
                0   & A_2 \\
            \end{array}\right),\ A_1 = [\mca_1]_{E_1},\ E_1$ -- начальная часть $E$.

    $\chi_{\mca } = |\left(\begin{array}{c|c}
                A_1 & B   \\ \hline
                0   & A_2 \\
            \end{array}\right) - XE_n| = |\left(\begin{array}{c|c}
                A_1 - XE_m & B              \\ \hline
                0          & A_2 - XE_{n-m} \\
            \end{array}\right)| = |A_1 - XE_m||A_2 - XE_{n-m}| = \underbrace{\chi_{A_1}}_{=\chi_{\mca_1}}\chi_{A_2}$.

    2: аналогично, в подходящем $E$ $[\mca]_E = \left(\begin{array}{c|c}
                A_1 & 0   \\ \hline
                0   & A_2 \\
            \end{array}\right): A_1 = [\mca_1]_{E_1},\  A_2 = [\mca_2]_{E_2} \Rightarrow
        \chi_{\mca} = \chi_{A_1}\chi_{A_2} = \chi_{\mca_1}\chi_{\mca_2}$.
\end{proof}

\begin{corollary}
    Допустим, $\lambda$ -- собственное значение $\mca$, тогда $g_{\lambda} \leq a_{\lambda}$.
\end{corollary}

\begin{proof}
    Применим предложение к $W=V_{\lambda}$. Очевидно, $W$ -- $\mca$-инвариантно
    $\Rightarrow \chi_{\mca|_{V_{\lambda}}}|\chi_{\mca}$.

    В любом базисе $[\mca|_{V_{\lambda}}] = diag(\underbrace{\lambda, \lambda, \ldots, \lambda}_{g_{\lambda}})
        \Rightarrow \chi_{\mca|_{V_{\lambda}}} = |diag(\underbrace{\lambda - x, \ldots, \lambda - x}_{g_{\lambda}})| =
        (\lambda - x)^{g_{\lambda}} \Rightarrow
        (\lambda - x)^{g_{\lambda}}|\chi_{\mca} \Rightarrow a_{\lambda} \geq g_{\lambda}$.
\end{proof}

\begin{theorem}
    Пусть $\mca \in \End V$. Тогда эквивалентны 2 условия:
    \begin{enumerate}
        \item $\mca$ -- диагонализируем.
        \item $\chi_{\mca}$ раскладывается на линейные множители, и для любого собственного значения $\lambda$ выполнено: $g_{\lambda} = a_{\lambda}$.
    \end{enumerate}
\end{theorem}



\begin{proof}
    1 $\Rightarrow$ 2: существует базис $E$, такой что $A = [\mca]_E = diag(\underbrace{\lambda_1, \ldots, \lambda_1}_{g_{\lambda_1}},
        \underbrace{\lambda_2, \ldots, \lambda_2}_{g_{\lambda_2}}, \ldots, \underbrace{\lambda_k, \ldots, \lambda_k}_{g_{\lambda_k}})$, где $\lambda_1, \ldots, \lambda_k$ -- различные собственные значения.

    $\chi_{\mca} = (\lambda_1 - x)^{g_{\lambda_1}}\ldots (\lambda_k - x)^{g_{\lambda_k}}$ -- раскладывается на линейные множители (кратность -- степень)
    $\Rightarrow g_{\lambda_i} = a_{\lambda_i}$.

    2 $\Rightarrow$ 1: $\chi_{\mca}$ раскладывается на линейные множители $\Rightarrow \chi_{\mca} = \pm(x-\lambda_1)^{a_{\lambda_1}}\ldots(x-\lambda_k)^{a_{\lambda_k}},\ g_{\lambda_i} = a_{\lambda_i}
        \Rightarrow g_{\lambda_1}+\ldots+g_{\lambda_k} = a_{\lambda_1}+\ldots+a_{\lambda_k} = n \Rightarrow \chi$ -- диагонализируем.
\end{proof}

Теорема указывает на два обстоятельства, которые мешают оператору быть диагонализируемым: многочлен может не раскладываться на линейные множители (т.е. не только не быть диагонализируемым, но и не иметь собственных значений), геометрическая и алгебраическиая кратности могут не совпадать.

\begin{example}
    \begin{enumerate}
        \item $A = \begin{pmatrix}
                      0 & -1 \\
                      1 & 0  \\
                  \end{pmatrix},\ \chi_{\mca} = \begin{vmatrix}
                      -x & -1 \\
                      1  & -x \\
                  \end{vmatrix} = x^2+1 \ (K = \R)$.
        \item $A = \begin{pmatrix}
                      0 & 0 \\
                      1 & 0 \\
                  \end{pmatrix},\ \chi_{\mca} = \begin{vmatrix}
                      -x & 0  \\
                      1  & -x \\
                  \end{vmatrix} = x^2$, единственное собственное значение -- $0$, $a_0 = 2$ и $g_0 = 1$.
    \end{enumerate}
\end{example}

\end{document}