% !TeX root = ./main.tex
\documentclass[main]{subfiles}
\begin{document}

\chapter{Алгебра линейных операторов} \marginpar{07.09.22}

\begin{definition} [Линейный оператор]
    $V$ — линейное пространство над полем $K$.
    Линейный оператор на $V$ — линейное отображение $V \to V$ (эндоморфизм линейного пространства $V$).
\end{definition}
\begin{definition} [Множество линейных операторов]
    $\End V = \Hom(V, V)$ —  множество линейных операторов.
\end{definition}

\[\mca \in \Hom (V, W) \quad [\mca]_{E,F}\]

Имея два пространства $V$ и $W$, базисы $E$ и $F$ можно выбрать так, что матрица получится окаймленной единичной.

Теперь же мы имеем одно пространство, соответственно, и один базис и все еще хотим, чтобы матрица была наиболее простой.

\begin{definition} [Алгебра]
    Говорят, что задана алгебра над полем $K$, если задано множество $A$, бинарные операции +, $\times$ на нем и отбражение $K \cdot A \to A$, т.ч.:
    \begin{enumerate}
        \item $(A,+, \times)$ - кольцо
        \item $(A,+, \cdot)$ - линейное пространство над полем $K$
        \item $\forall \alpha \in K \  \forall a, b \in A : \alpha \cdot (a \times b) = (\alpha \cdot a) \times b = a \times (\alpha \cdot b)$
    \end{enumerate}
\end{definition}

\begin{example}
    $A = M_n(K)$,
    $A_0 = \{ \alpha E_n | \alpha \in K\}$ —  подкольцо скалярных матриц, изоморфное полю $K$.
\end{example}

\begin{example}
    $A = K[x]$
\end{example}
\begin{example}
    Любая ситуация, где поле $K \subset R$ ($R$ —  кольцо) $\implies R$ — $K$-алгебра.

    В обратную сторону тоже верно, если алгебра содержит единицу. Тогда там найдется подкольцо, которое можно отожествить с полем К.

    Почему алгебра с единицей:

    Пусть $A$ —  алгебра c $1(\neq0)$ над полем $K$.
    Рассмотрим множество $A_0 = \{\alpha \cdot 1| \alpha \in K\}$.
    \[K \xrightarrow{\varphi} A_0 \quad \alpha \mapsto \alpha_1\]
    Идеал в поле либо нулевой, либо все поле.
    $\varphi(1) \neq 0 \Rightarrow Ker(\varphi)\neq K$.
    Значит,  $\varphi$ — изоморфизм. $A_0$ — подкольцо, изоморфное полю $K$.
\end{example}


Линейные операторы тоже образуют алгебру. Заметим, что в $\End \ V$ есть сложение и композиция операторов, а также умножение на скаляр.
$(\End \ V, +)$ — абелева группа. Проверка дистрибутивности операторов:
\begin{gather*}
    \mca\circ(\mcb_1+\mcb_2) = \mca\circ\mcb_1 + \mca\circ\mcb_2 \\
    (\mca_1 + \mca_2)\circ\mcb = \mca_1\circ\mcb+\mca_2\circ\mcb
\end{gather*}
$(\End \ V, +, \circ)$ — линейное пространство над полем $K$. Наконец,
$(\alpha\cdot\mca)\circ\mcb=\mca\circ(\alpha\cdot\mcb)=\alpha\cdot(\mca\circ\mcb)$.

Таким образом, $(\End \ V, +, \circ, \cdot)$ — алгебра над полем $K$.




\begin{proposition}
    Пусть $\dim V = n$. $E$ — базис $V$.
    Тогда отображение $\lambda_E : \End V \rightarrow M_n(K), \ \mca \mapsto [\mca]_E$ — изоморфизм алгебр над полем $K$ (т.е. биекция, сохраняющая все операции).
\end{proposition}

\begin{proof}
    Знаем: $\lambda_E$ — изоморфизм линейных пространств. $\lambda_E(\mcb\circ\mca)=[\mcb\mca]_E=[\mcb]_E\cdot[\mca]_E = \lambda_E(\mcb)\lambda_E(\mca)$.
\end{proof}

\begin{corollary}
    $\dim \End\ V = (\dim \ V)^2 $
\end{corollary}

lil friendly reminder: $U_E \xrightarrow{\mca} V_F \xrightarrow{\mcb} W_G,\
    [\mcb\mca]_{EG} = [\mcb]_{FG}[\mca]_{EF}
$ - стандартный случай.

$\mca: U_{EE'} \rightarrow V_{FF'}$. Как связаны матрицы этого линейного отображения в двух базисах?
$[\mca]_{EF}=A$ -- знаем,  $[\mca]_{E'F'}=?$ Нужны матрицы перехода:
$M_{E\rightarrow E'} = C,\ M_{F\rightarrow F'} = D$. Можем записать:
$E' = EC,\ E=(e_1, \ldots ,e_n),\ C=c_{ij}$ ($E$ - вектор, $C$ - квадратная матрица). Тогда
$EC=(c_{11}e_1+\ldots+c_{m1}e_n, c_{12}e_1 +\ldots +c_{n2}e_n, \ldots)$. Что происходит с матрицей при такой замене базиса?

\begin{proposition}
    Пусть $\mca \in \End V, E$ и $E'$ — базисы, $[\mca]_{E}=A,\ M_{E\rightarrow E'}=C$, тогда $[\mca]_{E'}=C^{-1}AC$.
\end{proposition}

\begin{proof}
    % https://q.uiver.app/?q=WzAsNCxbMCwxLCJVX0YnIl0sWzEsMCwiVl9GIl0sWzEsMSwiVl9GJyJdLFswLDAsIlVfRSJdLFsxLDIsIlxccXVhZCBcXHF1YWQgXFxxdWFkXFxlcHNpbG9uX1Y9aWRfViJdLFswLDMsIlxcZXBzaWxvbl9VIl0sWzMsMSwiXFxtYXRoY2Fse0F9Il0sWzAsMiwiXFxtYXRoY2Fse0F9IiwyXV0=
    \[\begin{tikzcd}
            {U_E} & {V_F} \\
            {U_F'} & {V_F'}
            \arrow["{ \epsilon_V=id_V}", from=1-2, to=2-2]
            \arrow["{\epsilon_U}", from=2-1, to=1-1]
            \arrow["{\mathcal{A}}", from=1-1, to=1-2]
            \arrow["{\mathcal{A}}"', from=2-1, to=2-2]
        \end{tikzcd}\]
    \[[\mca]_{E'F'}=\underbrace{[\mse_V]_{FF'}}_{D^{-1}}\underbrace{[\mca]_{EF}}_{A}\underbrace{[\mse_U]_{E'E}}_{C}\]

    В нашем случае $(U=V, E=F, E'=F')$.
\end{proof}

\begin{definition} [Эквивалентность матриц оператора в разных базисах]
    Пусть $A'$ эквивалентно $A$, если $\exists C \in \GL_n(K)$: $A'=C^{-1}AC$.
    Проверка симметричности и транзитивности:
    \begin{gather*}
        A= (C^{-1})^{-1}A'C^{-1} симметричность \\
        A'' = D^{-1}A'D=D^{-1}C^{-1}ACD=(DC)^{-1}A(CD)  \\
    \end{gather*}
\end{definition}

\end{document}