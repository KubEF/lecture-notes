% !TeX root = ./main.tex
\documentclass[main]{subfiles}
\begin{document}


\chapter{Примарные подпространства}

\begin{definition} [Примарное подпространство]
    Допустим, $p$ -- неприводимый многочлен, $\mca \in \End V, \ W < V$ называется $p$-примарным, если $W$ -- $\mca$-инвариантно
    и $\forall v \in W: \ \mu_{\mca, v} = p^m, \ m \geq 0$, $p$ -- неприводимый.

    \begin{center}$W_p = \{w \in V \ | \ \mu_{\mca, w} = p^m, \ m \geq 0\}$ \end{center}
\end{definition}

\begin{proposition}
    $W_p$ -- максимальное по включению $p$-примарное подпространство.
\end{proposition}

\begin{proof}
    Нужно проверить:
    \begin{enumerate}
        \item $W_p < V$:
              \begin{center}
                  $\lambda \neq 0 \Rightarrow \mu_{\mca, \lambda w} = \mu_{\mca, w}$.

                  $w_1, \ w_2 \in W_p$

                  $\mu_{\mca, w_1} = p^{m_1}, \ \mu_{\mca, w_2} = p^{m_2}$

                  $ m = max(m_1, \ m_2)$.

                  $p^m(\mca)(w_1+w_2) = p^m(\mca)(w_1)+ p^m(\mca)(w_2) = 0 $

                  $\mu_{\mca, w_1 +w_2} | p^m \Rightarrow \mu_{\mca, w_1 +w_2} = p^l$, где $l \leq m \Rightarrow w_1+w_2 \in W_p$
              \end{center}

        \item $W_p$ -- $\mca$-инвариантно:
              \begin{center}
                  $w \in W_p \Rightarrow \mu_{\mca, w} = p^m$

                  $p^m(\mca)(w) = 0$

                  $p^m(\mca)(\mca w) = \mca\underbrace{(p^m(\mca)(w))}_{0} = 0 $

                  $ \mu_{\mca, \mca_w} | p^m \Rightarrow \mu_{\mca, \mca_w} = p^l, \ l \leq m \Rightarrow \mca w \in W_p$
              \end{center}

    \end{enumerate}
\end{proof}

\begin{proposition}
    Допустим $f(\mca) = 0, \ f = gh, \ (g, h) = 1$, тогда $V = W_1 \oplus W_2$, где
    $W_1, W_2$ -- $\mca$-инвариантны,  $g(\mca|_{W_1}) = 0, \  h(\mca |_{W_2}) = 0$.
\end{proposition}

\begin{proof}
    Предположим, $W_1 = \Ker g(\mca), \ W_2 = \Ker h(\mca), \
        W_1, W_2$ -- $\mca$-инвариантные пространства.
    $(g, h) = 1 \Rightarrow \exists \ a, \ b \in K[X]$ такие, что $ag+bh = 1$.

    \begin{enumerate}
        \item Проверим, что $W_1 + W_2 = V$:
              \begin{center}
                  $g(\mca)a(\mca) + h(\mca)b(\mca) = \mse_V$

                  \text{Пусть } $v \in V, \ v = \underbrace{g(\mca)a(\mca)(v)}_{\in W_2} + \underbrace{h(\mca)b(\mca)(v)}_{\in W_1}$.

                  $h(\mca)(g(\mca)a(\mca)(v)) = \underbrace{(hg)}_{f}(\mca)a(\mca)(v) = 0 $

                  $\Rightarrow g(\mca)a(\mca)(v) \in W_2$
              \end{center}

              Аналогично, $h(\mca)b(\mca)(v) \in W_1$.

              Таким образом, $\forall v \in W_1 + W_2$.
        \item Проверим, что $W_1 \bigcap W_2 = 0$:
              \begin{center}
                  $v \in W_1 \bigcap W_2$

                  $a(\mca)g(\mca) + b(\mca)h(\mca) = \mse_V$

                  $a(\mca)\underbrace{g(\mca)(v)}_{0} + b(\mca)\underbrace{h(\mca)(v)}_{0} = v$
              \end{center}
              Таким образом, $v = 0$.
    \end{enumerate}
 \end{proof}

\marginpar{05.10.22}

\begin{proposition}
    Пусть $f(\mca) = 0$, $f = \epsilon {p_1}^{r_1} \ldots {p_s}^{r_s}$, $\epsilon \in K^*$,
    $p_i$ попарно неассоциированные неприводимые многочлены. Тогда $V = W_{p_1} \oplus \ldots \oplus W_{p_s}$, т.е.
    \begin{enumerate}
        \item $V = W_{p_1} + \ldots + W_{p_s}$
        \item $ 0 = w_1 + \ldots + w_s, \ w_j \in W_{p_j} \Rightarrow w_1 = \ldots = w_s = 0$ (равносильно условию, что $V$ единственным образом представляется в виде суммы $W_{p_1}, \ldots, W_{p_s}$: $v = w_1 + \ldots + w_s = w'_1 + \ldots + w'_s \Rightarrow (w_1 - w'_1) + \ldots + (w_s - w'_s) = 0 \Rightarrow w_i - w'_i = 0$).
    \end{enumerate}
\end{proposition}

\begin{proof}
    Индукция по $s$.

    База: $s = 1$. ${p_1}^{r_1}(\mca) = 0 \Rightarrow \mu_{\mca, v} | {p_1}^{r_1} \Rightarrow W_{p_1} = V$.

    Переход: $f = gh, \ g = \mse {p_1}^{r_1} \ldots {p_{s-1}}^{r_{s-1}}$, $h = {p_s}^{r_s}$, $(g, h) = 1 \Rightarrow
        V = V' \oplus V'', \ g(\mca |_{V'}) = 0, \ h(\mca |_{V''}) = 0$ (по предложению 7.2).

    По индукционному предположению, $V' =  \widetilde{W}_{p_1} \oplus \ldots \oplus \widetilde{W}_{p_s-1}$.

    $\widetilde{W}_{p_j}$ -- максимальной $p_j$-примарное подпространство для $\mca |_{V'}$.
    $\forall v \in \widetilde{W}_{p_j}$: $\mu_{\mca |_{V'}, v} = \mu_{\mca, v} \Rightarrow \widetilde{W}_{p_j} \subset W_{p_j}$.

    ${p_s}^{r_s}(\mca |_{V''}) = 0 \Rightarrow V'' - p_s$-примарное $ \Rightarrow
        V'' \subset W_{p_s}$.

    $V = \widetilde{W}_{p_1} \oplus \ldots \oplus \widetilde{W}_{p_{s-1}} \oplus V'' \subset W_{p_1} + \ldots + \widetilde{W}_{p_{s-1}} \subset W_{p_1} + \ldots + W_{p_s}$.

    Таким образом, $V = W_{p_1} + \ldots + W_{p_s}$.

    Предположим, $w_1 + \ldots + w_s = 0$, $w_j \in W_{p_j}, \ j = 1, \ldots, s$.

    Проверим: $w_s = 0$ (аналогично $w_j = 0 \ \forall j)$.

    $w_s = - w_1 - w_2 - \ldots - w_{s-1}$.

    $w_j \in W_{p_j} \Rightarrow \mu_{\mca, w_j} = p_j^{l_j}, \ j = 1, \ldots, s$.

    ${p_1}^{l_1} \ldots {p_{s-1}}^{l_{s-1}} (\mca) (w_j) = 0, \ j = 1, \ldots, s-1 \Rightarrow
        {p_1}^{l_1} \ldots {p_{s-1}}^{l_{s-1}} (\mca) \underbrace{(w_1 + \ldots + w_{s-1})}_{-w_s} = 0  \Rightarrow
        {p_s}^{l_s} \mid {p_1}^{l_1} \ldots {p_{s-1}}^{l_{s-1}} \Rightarrow l_s = 0 \Rightarrow w_s = 0$.
\end{proof}

\begin{corollary}
    Пусть $\mca \in \End V$. Тогда $V = \bigoplus_{p|\chi_{\mca}} W$, $p$ -- неприводимый приведеннный.
\end{corollary}

\begin{remark}
    $p \nmid \chi_{\mca} \Rightarrow W_p = 0$ (т.е. $\mu_{\mca, v} \mid \mu_{\mca} \mid \chi_{\mca}) \Rightarrow V = \bigoplus_p W_p$, $p$ -- неприводимый приведеннный.
\end{remark}

\begin{remark}
    $p \mid \chi_{\mca} \Rightarrow W_p \neq 0$.
\end{remark}

\begin{remark}
    Пусть $V$ -- $p$-примарное, тогда $V = \bigoplus_{i = 1}^{t} C_{v_i}, \ \mu_{\mca_i, v_i} = p^{m_i}$.
\end{remark}

В частном подходящем базисе:

$[\mca]_E = \left(\begin{array}{c|c|c}
            L_{p^{m_1}} &  &             \\
            \hline
                        &  &             \\
            \hline
                        &  & L_{p^{m_t}} \\
        \end{array}\right)$, где $L_f$ -- сопровождающая матрица многочлена $f$.

$f = x^d + \alpha_{d - 1} x^{d - 1} + \ldots + \alpha_1x + \alpha_0 \Rightarrow L_f = \begin{pmatrix}
        0      & 0      & \ldots & -\alpha_0     \\
        1      & 0      & \ldots & -\alpha_1     \\
        0      & 1      & \ldots & -\alpha_2     \\
        \vdots & \vdots & \ddots & \vdots        \\
        0      & 0      & \ldots & -\alpha_{d-1} \\
    \end{pmatrix}$


\end{document}