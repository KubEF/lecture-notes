% !TeX root = ./main.tex
\documentclass[main]{subfiles}
\begin{document}

\chapter{Жорданова нормальная форма}

Пусть $\mca \in \End V$, $\chi_{\mca}$ раскладывается на линейные множители.

$p| \chi_{\mca}$ неприводимый приведенный $\Rightarrow p = x - \lambda$, $\lambda$ -- собственное значение $\mca$.

$W_{x-\lambda} = R_{\lambda} = \{v \ | \ \exists j: \ (x-\lambda)^j (\mca) (v) = 0\} =
  \{v \ | \ (\mca - \lambda \mse )^j (v) = 0\}$ -- корневые векторы, принадлежащее собственному значению $\lambda$.

\begin{definition} [Корневой вектор и корневое подпространство]
  Корневым вектором, принадлежащим собственному значению $\lambda$ называется $v \in V$ такое, что $(\mca - \lambda \mse)^j (v) = 0$ для некоторого $j$.

  $R_{\lambda}$ -- корневое подпространство, принадлежащее собственному значению $\lambda$.
\end{definition}

\begin{propertylist}
  \begin{enumerate}
    \item $V_{\lambda} \subset R_{\lambda}$, т.к. $V_{\lambda} = \Ker(\mca - \lambda \mse)$
    \item $R_{\lambda} \neq 0 \Leftrightarrow \lambda$ -- собственное значение $\mca$
          \begin{proof}
            $\Rightarrow: \ R_{\lambda} = W_{x-\lambda}$ (по опр.). $R_{\lambda} \neq 0 \Rightarrow
              (x - \lambda) \mid \chi_{\mca} \Rightarrow \lambda$ -- собственное значение.

            $\Leftarrow: \lambda$ -- собственное значение $\Rightarrow V_{\lambda} \neq 0 \Rightarrow R_{\lambda} \neq 0$
          \end{proof}
    \item $V = \bigoplus_{\lambda} R_{\lambda}$ (интерпретация основного определения $V = \bigoplus_{p | \chi_{\mca}} W_p$)
  \end{enumerate}
\end{propertylist}

\begin{definition} [Высота корневого вектора]
  Высота корневого вектора $v$ -- это минимальное $h$ такое, что $(\mca - \lambda \mse)^h (v) = 0$.
\end{definition}

Корневой вектор высоты 0 -- это 0, 1 -- собственные векторы, 2 -- такие $v$, что $(\mca - \lambda \mse)(v)$ -- собственный вектор,
$h$ -- $\Ker(\mca - \lambda \mse)^h \setminus \Ker(\mca - \lambda \mse)^{h-1}$.

Очевидно, $R_{\lambda} = \bigcup_{h \in \N} \underbrace{\Ker(\mca - \lambda \mse)^h}_{:= R_{\lambda, h}}$.

$\underbrace{R_{\lambda, 0}}_{=0} \subset \underbrace{R_{\lambda, 1}}_{= V_{\lambda}} \subset R_{\lambda, 2} \subset \ldots \subsetneq  R_{\lambda, N_{\lambda}} = R_{\lambda, N_{\lambda} + 1} \Rightarrow
  R_{\lambda} = R_{\lambda, N_{\lambda}}$.

\begin{proposition}
  Пусть $N_{\lambda}$ -- минимальное натуральное число такое, что $R_{\lambda} = R_{\lambda, N_{\lambda}}, \ \lambda_1, \ldots, \lambda_s$ -- все собственные значения $\mca$.
  Тогда ${\mu}_{\mca} = \underbrace{\prod_{i = 1}^s(x-\lambda_i)^{N_{\lambda_i}}}_{=f}$.
\end{proposition}

\begin{proof}
  $R_{\lambda_i} = R_{\lambda_i, N_{\lambda_i}} \Rightarrow (x - \lambda_i)^{N_{\lambda_i}}(\mca|_{R_{\lambda_i}}) = 0 \Rightarrow
    f(\mca|_{R_{\lambda_i}}) = 0, \ i = 1, \ldots, s \Rightarrow f(\mca) = 0 \Rightarrow \mu_{\mca} | f$.

  Докажем: $(x - \lambda_i)^{N_{\lambda_i}} \mid \mu_{\mca}, \ i = 1, \ldots, s$.

  $\exists v \in R_{\lambda_i, N_{\lambda_i}}\setminus R_{\lambda_i, N_{\lambda_i} - 1}$

  $\mu_{\mca, v} = (x - \lambda_i)^{N_{\lambda_i}}, \ \mu_{\mca, v} \mid \mu_{\mca} \Rightarrow f \mid \mu_{\mca}$.
\end{proof}

\begin{corollary}
  Множества корней $\chi_{\mca}$ и $\mu_{\mca}$ совпадают.
\end{corollary}

\begin{proof}
  $\mu_{\mca}$ делит $\chi_{\mca}$, и у $\chi_{\mca}$ нет других корней, так как $N_{\lambda_i} \geq  1$.
\end{proof}

\begin{definition} [Жорданова клетка порядка $n$]
  Пусть  $\lambda \in K$, $n \in \N$. Жордановой клеткой порядка $n$ с собственным значением $\lambda$ называется
  \begin{gather*} \mathcal{J}_n(\lambda) =
    \begin{pmatrix}
      \lambda & 0       & \ldots & 0       & 0       \\
      1       & \lambda & \ldots & 0       & 0       \\
      0       & 1       & \ldots & 0       & 0       \\
      \vdots  & \vdots  & \ddots & \vdots  & \vdots  \\
      0       & 0       & \ldots & \lambda & 0       \\
      0       & 0       & \ldots & 1       & \lambda \\
    \end{pmatrix}
  \end{gather*}

  Жорданова матрица -- блочно-диагональная матрица, блоки которой -- некоторые жордановы клетки.

  $\chi_{\mathcal{J}_n(\lambda)} = (\lambda - x)^n = \pm (x - \lambda)^n, \ (a_{\lambda} = n,\ g_{\lambda} = 1)$ для оператора умножения на $\mathcal{J}_n(\lambda)$.

  Базис $E$ пространства $V$ называется жордановым базисом (для $\mca$), если $[\mca]_E$ жорданова.
\end{definition}

\begin{definition} [Нильпотентный оператор]
  $\mca$ называется нильпотентным, если $\exists \ N > 0: \ \mca^N = 0$ (т.е. $\mu_{\mca} = x^{...}$).
\end{definition}

$V = \bigoplus R_{\lambda}$

$(x - \lambda)^{N_{\lambda}}(\mca |_{R_{\lambda}}) = 0 \Rightarrow (\mca |_{R_{\lambda}} - \lambda\mse |_{R_{\lambda}})^{N_{\lambda}} = 0$.
Таким образом, $\mca |_{R_{\lambda}} - \lambda \mse_{R_{\lambda}}$ -- нильпотентный. Минимальное $N$ называется индексом нильпотентности $\mca$.

\begin{remark}
  $E$ -- жорданов базис для $\mca \Leftrightarrow E$ -- жорданов базис для $\mca + \alpha \mse$.
\end{remark}

\begin{proof}
  $[\mca + \alpha \mse]_E = [\mca]_E + [\alpha \mse]_E = [A]_E + \diag(\alpha, \ldots, \alpha)$.
  $\mathcal{J}_n(\lambda) + \alpha E_n = \mathcal{J}_n(\lambda + \alpha)$.
\end{proof}

Построение жорданового базиса для нильпотентного оператора.

$\mcb$ -- нильпотент, $N$ -- индекс нильпотентности.

$\mu_{\mcb} = x^N, \ V = R_0 = R_{0, N} \supsetneq R_{0, N-1} \supset \ldots \supset R_{0, 1} \supset R_{0, 0} = \{0\}$

Пусть $e_{N, 1}, e_{N,2}, \ldots, e_{N, q_N}$ -- базис $R_{0, N}$ относительно $R_{0, N-1}$.

\begin{lemma}
  Пусть $v_1, \ldots, v_s \in R_{0, m}$ ЛНЗ относительно $R_{0, m-1}$.
  Тогда: $\mcb v_1, \ldots, \mcb v_s \in R_{0, m-1}$ ЛНЗ относительно $R_{0, m-2}$. 
\end{lemma}

\begin{proof}
  $\mcb^m v_i = 0 = \mcb^{m-1}(\mcb v_i) = 0 \Rightarrow \mcb v_i \in R_{0, m-1}$.

  Пусть $\alpha_1 \mcb v_1 + \ldots + \alpha_s \mcb v_s \in R_{0, m-2}. \ \mcb^{m-2}(\alpha_1 \mcb v_1 + \ldots \alpha_s \mcb v_s) =
    \mcb^{m-1}(\alpha_1 v_1 + \ldots \alpha_s v_s) = 0 \Rightarrow \alpha_1 v_1 + \ldots \alpha_s v_s \in \Ker \mcb^{m-1} = R_{0, m-1}
    \Rightarrow \alpha_1 = \ldots = \alpha_s = 0$.
\end{proof}
\marginpar{12.10.22}
\begin{remark}
  $\mcb$  нильпотентный $\Rightarrow$ единственное собственное значение $\mcb$ -- это 0.
\end{remark}

\begin{proof}
  Допустим, $\mcb v = \lambda v, \ \lambda\neq 0$. Тогда $0 = \mcb^N v = \lambda^N v$  -- противоречие. Появляющиеся жордановы клетки имеют собственное значение ноль.
  \begin{gather*}
    \begin{array}{c|c|c}
       &                                     & \\
      \hline
       & \begin{matrix}
           0      & 0      & \ldots & 0 & 0 \\
           1      & 0      & \ldots & 0 & 0 \\
           0      & 1      & \ldots & 0 & 0 \\
           \vdots & \vdots & \ddots & 0 & 0 \\
           0      & 0      & \ldots & 1 & 0 \\
         \end{matrix} &      \\
      \hline
       &                                     & \\
    \end{array} \\
    e_l \stackrel{\mcb}{\mapsto} e_{l+1} \stackrel{\mcb}{\mapsto} \ldots \stackrel{\mcb}{\mapsto} 0
  \end{gather*}
  Базисные векторы разбиваются на такие цепочки, а последний переходит в ноль. Наша задача строить такие цепочки.
\end{proof}

$e_{N, 1}, \ e_{N, 2}, \ \ldots, e_{N, q_N}$ -- базис $R_{0, N}$ относительно $R_{0, N-1}$.
Подействуем на каждый вектор оператором $\mcb$, соответственно, их высота уменьшится на единицу.
$e_{N-1, 1}, \ e_{N-1, 2}, \ \ldots, e_{N-1, q_N} \in R_{0, N-1}$, линейно независимы относительно $ R_{0, N-2}$.
Дополним этот набор $e_{N-1, q_N+1}, \ \ldots, e_{N-1, q_{N-1}}$, получили базис  $R_{0, N-1}$ относительно $R_{0, N-2}$. Повторяем эти действия, пока не дойдем до высоты 1:
$e_{1, 1}, \ e_{1, 2}, \ \ldots, e_{1, q_2} \in R_{0, 1} = V_0$, линейно независимы относительно $ R_{0, 0} = \{0\}$.
Тогда $e_{1, 1}, \ e_{1, 2}, \ \ldots, e_{1, q_2}$, $e_{1, q_2+1}, \ e_{1, q_2+2}, \ \ldots, e_{1, q_1}$ -- базис $R_{0, 1}$,$e_{1, 1}, \ e_{1, 2}, \ \ldots, e_{1, q_1}$ и $e_{2, 1}, \ e_{2, 2}, \ \ldots, e_{2, q_2}$ в совокупности -- базис $R_{0, 2}$ \footnote{\url{https://youtu.be/qlrIV2_gFcU?t=760}}. 


Индукцией по $k$ мы получаем, что $(e_{i,j} \  | \ 1 \leq i \leq k_, \ 1 \leq j \leq q_i)$ -- базис $R_{0, k}$.
Найденный базис состоит из $\underbrace{(e_{i,j} \  | \ 1 \leq i \leq k-1, \ 1 \leq j \leq q_i)}_{\textrm{базис $R_{0, k-1}$ по ИП}}$
и $\underbrace{(e_{k,j} \  | \ 1 \leq j \leq q_k)}_{\textrm{базис $R_{0, k}$ отн. $R_{0, k-1}$}}$. Следовательно, все $e_{i,j}$ образуют базис
$R_{0, N} = R_0 = V$.

Легко видеть, что найденный базис -- жорданов.

\begin{gather*}
  [\mcb]_E = \diag(\underbrace{\mathcal{J}_N(0), \ldots, \mathcal{J}_N(0)}_{q_N}, \underbrace{\mathcal{J}_{N-1}(0), \ldots, \mathcal{J}_{N-1}(0)}_{q_{N-1}-q_N}, \ldots, \underbrace{\mathcal{J}_1(0), \ldots, \mathcal{J}_1(0)}_{q_1-q_2})
\end{gather*}

Пусть $\mca \in \End V$, такое что  $\chi_{\mca} = \pm \prod_{i=1}^m (x-\lambda^i)^{\alpha_{\lambda_i}} \Rightarrow
  V = \bigoplus_{i=1}^m R_{\lambda_i}$.

$\mcb_i = (\mca - \lambda_i\mse) |_{R_{\lambda_i}}$ -- нильпотентный $\Rightarrow$ в $R_{\lambda_i}$ существует жорданов базис для $\mcb_i \Rightarrow$ для $\mca|_{R_{\lambda_i}}$.
Объединяя эти базисы, получим жорданов базис для $\mca$.

В $R_{\lambda_i}$ есть базис $E_i$, такой что $[\mcb_i]_{E_i} = \diag(\mathcal{J}_m(0), \ldots) \Rightarrow
  [\mca|_{R_{\lambda_i}}]_{E_i} = [\mcb_i+\lambda_i \mse_{R_{\lambda_i}}]_{E_i} =
    [\mcb_i]_{E_i}+ \lambda_i E_{\dim R_{\lambda_i}} = \diag(\mathcal{J}_m(\lambda_i), \ldots)$.



\begin{proposition}
  Пусть $\chi_{\mca}$ раскладывается на линейные множители. Тогда жорданова форма $\mca$
  определена однозначно с точностью до порядка следования клеток.
\end{proposition}

\begin{proof}
  $\lambda$ -- собственное значение $\mca$. $m_{j, \lambda}$ -- количество жордановых клеток вида
  $\mathcal{J}_j(\lambda)$, докажем, что $m_{j, \lambda}$ -- инвариант $\mca$. $d_j = \dim R_{\lambda, j}$ -- инвариант $\mca$.
  $E$ -- жорданов базис $\mca$. Переупорядочивая $E$, считаем, что:
  $[\mca]_E = \diag(\underbrace{\mathcal{J}_1(\lambda), \ldots, \mathcal{J}_1(\lambda)}_{m_1, \lambda}, \underbrace{\mathcal{J}_2(\lambda), \ldots, \mathcal{J}_2(\lambda)}_{m_2, \lambda}, \ldots, \underbrace{\mathcal{J}_N(\lambda), \ldots, \mathcal{J}_N(\lambda)}_{m_N, \lambda}, \mathcal{J}_p(\mu), \ldots)$.



\newpage

Посчитаем $d_j$, для этого рассмотрим оператор $\mcb = \mca - \lambda \mse$.

\begin{gather*}
  \left.
  \begin{array}{ccc}
    \ \quad \quad \quad \quad \quad \quad \quad \ \bullet\longmapsto 0 \\
    \ \quad \quad \quad \quad \quad \quad \quad \ \vdots               \\
    \ \quad \quad \quad \quad \quad \quad \quad \ \bullet\longmapsto 0 \\
  \end{array}
  \right\} m_{1,\lambda} \\
  \left.
  \begin{array}{ccc}
    \ \quad \quad \quad \quad \quad \bullet \longmapsto \bullet \longmapsto 0  \\
    \ \quad \quad \quad \quad \ \ \vdots \quad \quad \ \vdots                  \\
    \ \quad \quad \quad \quad \quad  \bullet \longmapsto \bullet \longmapsto 0 \\
  \end{array}
  \right\} m_{2,\lambda} \\
  \left.
  \begin{array}{ccc}
    \bullet \longmapsto \ \ \  \dots \ \ \  \bullet\longmapsto\bullet\longmapsto 0                 \\
    \vdots \quad \quad \quad \quad \quad  \vdots \quad \quad \ \vdots                              \\
    \underbrace{\bullet\longmapsto \ \ \  \dots \ \ \  \bullet\longmapsto\bullet\longmapsto 0}_{N} \\
  \end{array}
  \right\} m_{N,\lambda} \\
  \text{ } \bullet\longmapsto \dots \\
  \text{ } \bullet\longmapsto \dots \\
  \text{ } \bullet\longmapsto \dots \\
\end{gather*}

$d_0 = 0$ 

$d_1 = \dim\Ker{\mcb} = m_{1,\lambda} + \ldots + m_{N,\lambda}$

$d_2 = \dim\Ker{\mcb}^2 = m_{1,\lambda} + 2m_{2,\lambda} + \ldots + 2m_{N,\lambda} = d_1 + m_{2,\lambda}+ \ldots + m_{N,\lambda}$

$d_3 = d_2 + m_{3,\lambda} + \ldots + m_{N,\lambda}$

$\dots \ \ \ \ \dots \ \ \ \ \dots$

$d_N = d_{N-1} + m_{N,\lambda}$

$d_{N+1} = d_N$

\[
  \rotatebox[origin=c]{270}{$\Rightarrow$}
\]

$d_0 = 0$

$d_1 = m_{1,\lambda} + \ldots + m_{N,\lambda}$

$d_2 - d_1 = m_{2,\lambda} \ldots + m_{N,\lambda} $

$d_3 - d_2 = m_{3,\lambda} + \ldots + m_{N,\lambda}$

$\dots \ \ \ \ \dots \ \ \ \ \dots$

$d_N - d_{N-1} = m_{N,\lambda}$

$d_{N+1} = d_N$

\[
  \rotatebox[origin=c]{270}{$\Rightarrow$}
\]

$d_1 - (d_2 -d_1) = m_{1,\lambda}$

$(d_2 - d_1) -(d_3- d_2) = m_{2,\lambda}$

$\dots \ \ \ \ \dots \ \ \ \ \dots$

$(d_{N-1} - d_{N-2}) -(d_N -d_{N-1}) = m_{N-1,\lambda}$

\[
  m_{k,\lambda} = -d_{k-1}+2d_k-d_{k+1},\ k = 1,\ldots, N
\]

\end{proof}

\begin{remark}
  Максимальный порядок жордановой клетки с собственным значением $\lambda$
  равен кратности $(x - \lambda)$ в $\mu_{\mca}$.
\end{remark}

\begin{proof}
  $e_1, \ldots, e_n$ -- жорданов базис, $\mu_{\mca}=$ НОК $(\mu_{\mca, e_i} | i = 1, \ldots, n)$. Смотрим, где столбец пересекается с жордановой клеткой.
  $(\mca - \lambda \mse)^s e_i = 0, \ (\mca - \lambda \mse)^{(s-1)} e_i \neq 0$, где $s$ -- количество столбцов, включая $i$, попавших в эту клетку.
  Следовательно, $\mu_{\mca, e_i} = (x - \lambda)^s$, максимальное значение $s$ -- максимальный порядок жордановой клетки с собственным значением $\lambda$.
\end{proof}




\end{document}