% !TeX root = ./main.tex
\documentclass[main]{subfiles}

\begin{document}

\chapter{Комплексификация}

\marginpar{02.11.22}

$V$ -- линейное пространство над полем $\R$. Построим $V_{\C}$ -- линейное пространство над
полем $\C$. Неформально: $V_{\C} = \{v +  iw \ | \ v, \ w \in V\}$ (сложение: вещественные части и мнимые отдельно, 
умножение на скаляр: $(\alpha + i\beta) (v + iw) = (\alpha v - \beta w) + i(\beta v + \alpha w)$).

$V_{\C} = V\times V$. На этом множестве введем сложение и умножение на комплексные скаляры.
\begin{gather*}
    (v_1,w_1) + (v_2, w_2) := (v_1 + v_2, w_1 + w_2)  \\
    (\alpha + i\beta) (v, w) := (\alpha v - \beta w, \beta v + \alpha w)
\end{gather*}

Нетрудно проверить, что $(V_{\C}, +, \cdot)$ -- линейное пространство над $\C$.

\begin{enumerate}
    \item Рассмотрим подмножество $V = \{(v, 0) \ | \ v \in V\}$ -- 
    замкнуто относительно сложения и умножения на вещественные скаляры 
    ($(\alpha + i0)(v, w) = (\alpha v, 0)$).
    
    $\underset{v \mapsto (v, 0)}{V \xrightarrow{\sim} V'}$ -- изоморфизм линейных пространств над $\R$. Следовательно, 
    $V$ можно рассматривать как линейное пространство над $\R$. Можно от отожествить $v$ с $(v, 0)$.
    \item Можно ли сделать что-то с $(0, w)$? Оказывается, что можно: $(0, w) = i(w, 0)$.
    \item Таким образом, $(v, w) = (v, 0) + i(w, 0) = v +iw$.
\end{enumerate}

$V_{\C}$ -- комплексификация $V$.

\begin{corollary}
    Пусть $e_1, \ldots, e_n$ -- базис $V$. Тогда $e_1, \ldots, e_n$ -- базис $V_{\C}$, где $e_k = e_k + i0, \ 1 \leqslant k\leqslant n $.
\end{corollary}

\begin{proof}
    Рассмотрим $u \in V_{\C}$.
    \begin{gather*}
        u = v + iw, \ v, \ w \in V \\
        v = \alpha_1 e_1 + \ldots + \alpha_n e_n \\
        w = \beta_1 e_1 + \ldots + \beta_n e_n \\
        \Rightarrow u = (\alpha_1 + i\beta_1) e_1 + \ldots + (\alpha_n + i\beta_n) e_n \in \Lin_{V_{\C}} (e_1, \ldots, e_n)
    \end{gather*} 
    Предположим, $u = (\alpha_1 + i\beta_1) e_1 + \ldots + (\alpha_n + i\beta_n) e_n  = 0$.
    $u = \alpha_1 e_1 + \ldots + \alpha_n e_n + i(\beta_1 e_1 + \ldots \beta_n e_n) \Rightarrow
    \left\{ \begin{gathered} 
          \alpha_1 e_1 + \ldots + \alpha_n e_n = 0 \hfill 
          \\ 
          \beta_1 e_1 + \ldots + \beta_n e_n = 0 \hfill 
          \\ 
        \end{gathered}   \right.\Rightarrow
        \left\{ \begin{gathered} 
            \alpha_1 = \ldots = \alpha_n = 0 \hfill 
            \\ 
            \beta_1 = \ldots  =\beta_n = 0 \hfill 
            \\ 
          \end{gathered} \right.$
\end{proof}

\begin{corollary}
    $\dim V_{\C} = \dim V$
\end{corollary}

По $\mca \in \End V$ построим $\mca_{\C} \in \End V_{\C}$.

\begin{gather*}
    \mca_{\C}: \ \underset{v+iw \mapsto \mca v + i \mca w}{V_{\C} \rightarrow V_{\C}}
\end{gather*}

\begin{proposition}
    $\mca_{\C} \in \End V_{\C}$
\end{proposition}

\begin{proof}
    \begin{multline*}
        \mca_{\C}((v_1+iw_1)+(v_2+ iw_2)) = \mca(v_1 + v_2) + i\mca(w_1 + w_2) = \\
   = \mca v_1 + \mca v_2 + i\mca w_1 + i\mca w_2 = \mca_{\C}(v_1 + iw_1) + \mca_{\C}(v_2 + iw_2)
    \end{multline*}
\begin{multline*}
    \alpha \in \R, \ \mca_{\C}(\alpha(v + iw)) = \mca(\alpha v) + i\mca(\alpha w) = \\
   = \alpha(\mca v + i\mca w) = \alpha \mca_{\C}(v+ iw)
\end{multline*}
\begin{multline*}
    \mca_{\C}(i(v+i w)) = \mca_{\C}(-w + iv) = \\
    = \mca(-w) + i\mca v = i(\mca v + i\mca w) = i \mca_{\C}(v + iw)
\end{multline*}
    

    $\Rightarrow \mca_{\C}((\alpha + i\beta)(v+iw)) = (\alpha + i\beta)\mca_{\C}(v+iw)$
\end{proof}

\begin{proposition}
    Пусть $E$ -- базис $V$. Тогда $[\mca_{\C}]_E = [\mca]_E$.
\end{proposition}

\begin{proof}
    $[\mca]_E = (a_{kj}), \ \mca e_j = \sum_{k=1}^n a_{kj} e_k. \ \mca_{\C}e_j = \sum_{k=1}^{n}a_{kj}e_k + i0 = \sum_{k=1}^n(a_{kj} + i0)e_k$.
\end{proof}

\begin{corollary}
    $\chi_{\mca_{\C}} = \chi_{\mca}$.
\end{corollary}

\begin{proposition}
    Пусть $\mca \in \End V$, $\lambda$ -- собственное значение $\mca_{\C}$, $\lambda \not\in \R$;
    $v_1+iw_1, \ldots, v_l+iw_l$ -- базис $V_{\lambda}$. Тогда $\overline{\lambda}$ -- собственное 
    значение $\mca_{\C}$ и $v_1-iw_i, \ldots, v_l-iw_l$ -- базис $V_{\overline{\lambda}}$.
\end{proposition}

\begin{proof}
    \begin{gather*}
        \chi_{\mca_{\C}} = \chi_{\mca} \in \R[X] \\
        \chi_{\mca}(\lambda) = 0 \Rightarrow \chi_{\mca}(\overline{\lambda}) = 0 \\
    \end{gather*}
    
    \begin{remark}
        $u \in V_{\C}, \ \gamma \in \C \Rightarrow \overline{\gamma u} = \overline{\gamma} \cdot \overline{u}. \ \overline{u} = \overline{v + iw} = v - iw$.
        Проверка: пусть $\gamma = \alpha + i\beta, \ u = v + iw$, тогда $\overline{(\alpha + i\beta)(v + iw)} =
         \overline{\alpha v - \beta w + i(\alpha w + \beta v)} = \alpha v - \beta w - i(\alpha w + \beta v) = (\alpha + i\beta)(v - iw) =
        \alpha v - \beta w -i(\beta v + \alpha w)$.
    \end{remark}

    \begin{gather*}
        \underbrace{\mca_{\C}(v_j + iw_j)}_{\mca v_j + i\mca w_j} = \lambda(v_j + iw_j) \\
        \mca_{\C}(v_j - iw_j) = \mca v_j - i\mca w_j = \overline{(\mca v_j + i\mca w_j)} = \\
        = \overline{\lambda(v_j + iw_j)} = \overline{\lambda}(v_j-iw_j)
    \end{gather*}

    Таким образом, $v_1 - iw_1, \ldots, v_l - iw_l \in V_{\overline{\lambda}}$.

    Проверим линейную независимость. Предположим, $\gamma_1(v_1 - iw_1) + \ldots +
    \gamma(v_l - iw_l) = 0 \Rightarrow \overline{\gamma}(v_1 + iw_1) + \ldots + \overline{\gamma}(v_l + iw_l) = 0 \Rightarrow
    \overline{\gamma_1} = \ldots = \overline{\gamma_l} = 0 \Rightarrow \gamma_1 = \ldots = \gamma_l = 0 \Rightarrow
    \dim V_{\overline{\lambda}} \geq \dim V_{\lambda}$.

    Аналогично, $\dim V_{\lambda} \geq \dim V_{\overline{\lambda}} \Rightarrow
    \dim V_{\lambda} = \dim V_{\overline{\lambda}} = l \Rightarrow 
    v_1 - iw_i, \ldots, v_l - iw_l$ -- базис $V_{\overline{\lambda}}$.
\end{proof}

    Далее $V$ -- евклидово пространство. Введем на $V_{\C}$ отображение $\underset{(u, u') \mapsto \mcb(u, u')}{V_{\C}\times V_{\C}\to\C}$, 
    где $\mcb(v+iw, v'+iw') = (v, v') + (w, w') + i((w, v') - (v, w'))$. 

    lil friendly reminder: сопряженный к $i = -i$, и при выносе скаляра из правой части, нужно брать его с сопряжением.

    \begin{proposition}
        $\mcb$ -- скалярное произведение на $V_{\C}$
    \end{proposition}
    \begin{proof}
        Полуторалинейность -- непосредственная проверка. 

        $\mcb(v'+iw', u+iw) = \overline{\mcb(v + i w, v' + iw')}.$ 

        $\mcb(v + iw, v + iw) = (v, v) + (w, w) \geq 0$, ($=0$ только при $v = w = 0$).

        Таким образом, $(V_{\C}, \mcb)$ -- унитарное пространство.
    \end{proof}

    \begin{remark}
         $e_1, \ldots, e_n$ -- базис V. Тогда: $e_1, \ldots, e_n$ -- ортонормированный базис $V \Leftrightarrow$  $e_1, \ldots, e_n$ -- ортонормированный базис $V_{\C}$.
    \end{remark}

    \begin{proposition}
        Пусть $\mca \in \End V$. Тогда $(\mca_{\C})^* = (\mca^*)_{\C}$.
    \end{proposition}

    \begin{proof}
        Нужно проверить: $\forall u, \ u' \in V_{\C} \ (\mca_{\C}u, u') = (u, (\mca^*)_{\C} u')$. 
        Пусть  $u = v + iw, \ u' = v' + iw'$.
        \begin{multline*}
            (\mca_{\C}u, u') = (\mca v + i \mca w, v' + iw') = \\
            = (\mca v, v') + (\mca w, w') + i((\mca w, v') - (\mca v, w')) = \\
            = (v, \mca^* v') + (w, \mca^* w') + i((w, \mca^* v') - (v, \mca^*w)) = \\
            = (v, \mca^*v' + i\mca^*w') + (w, \mca^*w' - i\mca^* v') = \\
            = (v, \underbrace{\mca^*v+ i\mca^* w'}_{(\mca^*)_{\C}u'}) + (iw, \underbrace{i(\mca^*w'-i\mca^*v')}_{((\mca^*)_{\C}u')}) = (u, (\mca^*)_{\C} u').
        \end{multline*}
    \end{proof}
\end{document}