% !TeX root = ./main.tex
\documentclass[main]{subfiles}
\begin{document}


\chapter{Теорема Гамильтона-Кэли}

\begin{theorem}
    Пусть $\mca \in \End V$, тогда $\chi_{\mca}(\mca) = 0$.
\end{theorem}

\begin{proof}
    Рассмотрим $v \in V$.  $\chi_{\mca |_{C_v}}$ -- минимальный аннулятор $v \Rightarrow \chi_{\mca |_{C_v}}(\mca)(v) = 0$.
    $\chi_{\mca |_{C_v}} | \chi_{\mca} \Rightarrow \chi_{\mca} (\mca)(v) = 0$,
    $v$ -- любой элемент $V$. Таким образом, $\chi_{\mca}(\mca) = 0$.
\end{proof}

\begin{corollary} [Матричная теорема Гамильтона -- Кэли]
    Допустим, $A \in M_n(K)$, тогда $\chi_A(A)=0$.
\end{corollary}

\begin{proof}
    Представим $A$ как $[\mca]_E$, тогда $\chi_A = \chi_{\mca}$.

    $\chi_A (A) = \chi_A ([\mca]_E) = [\chi_A(\mca)]_E =[\chi_{\mca}(\mca)]_E = 0$.
\end{proof}

Как подставляется матрица в многочлен?

Если $f = \alpha_0 + \alpha_1 x + \ldots + \alpha_n x^n$, $A \in M_n(K)$, то
$f(A) = \alpha_0 E_n + \alpha_1 A + \ldots + \alpha_n A^n$.

Почему мы можем представить матрицу как матрицу какого-то оператора?
\begin{enumerate}
    \item $V$ -- любое пространство с $\dim V = n$, $E$ -- фиксированный базис.
          $\End V \rightarrow M_n(K)$, $\beta \mapsto [\beta]_E$ -- изоморфизм $\Rightarrow$ существует $\mca \in \End V$: $[\mca]_E = A$.

    \item $V = K^n$, $\mca$: $\underset{b \mapsto Ab}{K^n \rightarrow K^n}$, $E$ -- стандартный базис $K^n$, $A = [\mca]_E$.
\end{enumerate}

\begin{proposition}
    Пусть $\mca \in \End V$, тогда $I_{\mca} = \{f\in K[X] | f(\mca) = 0\}$ -- идеал $K[X]$, где $f(\mca)$ -- нулевой оператор.
\end{proposition}

\begin{proof}
    $f, g \in I_{\mca} \Rightarrow f+g \in I_{\mca}$ - очевидно.

    $f \in I_{\mca}$, $g \in K[X]$: $(gf)(\mca) = g(\mca) \circ \underbrace{f(\mca)}_{0} = 0$.

    $\chi_{\mca} \in I_{\mca}$,  по теореме Гамильтона--Кэли $\Rightarrow I_{\mca} \neq 0$.
\end{proof}

Образующую $I_{\mca}$ называют минимальным многочленом  оператора $\mca$.

\begin{remark}
    $(f) = (g) \Leftrightarrow \begin{cases}
            f|g \\
            g|f
        \end{cases} \Rightarrow g = \epsilon f$, $\epsilon \in K^*$.
\end{remark}

\begin{definition} [Минимальный многочлен оператора]
    $\mu_{\mca}$ называется минимальным многочленом оператора, если $(\mu_{\mca}) = I_{\mca}$ (является порождающим идеала $I_{\mca}$).
\end{definition}

По теореме Гамильтона--Кэли $\mu_{\mca}|\chi_{\mca}$ ($\chi_{\mca} \in I_{\mca} = (\mu_{\mca})$).

$\mu_{\mca, v}$ - минимальный аннулятор вектора $v$.
\begin{proposition}
    \begin{enumerate}
        \item $\mu_{\mca, v} | \mu_{\mca}$
        \item $V = Lin(v_1, \ldots, v_n)$, тогда $\mu_{\mca} = $ НОК $(\mu_{\mca, v_1}, \ldots, \mu_{\mca, v_n})$
    \end{enumerate}
\end{proposition}

\begin{proof}
    \begin{enumerate}
        \item
              \
              \begin{center}
                  $\mu_{\mca}(\mca) = 0 \Rightarrow \mu_{\mca}(\mca)(v) = 0 \Rightarrow \mu_{\mca} \in (\mu_{\mca, v})$
              \end{center}
        \item
              \
              \begin{center}
                  $f = $ НОК ($\ldots$)

                  $\mu_{\mca, v_i} | f \Rightarrow f(\mca)(v_i) = 0$, $i = 1, \ldots, n$

                  $v \in V \Rightarrow v = \alpha_1 v_1 + \ldots + \alpha_n v_n$

                  $v_1, \ldots, v_n \in \Ker f(\mca) \Rightarrow v \in \Ker f(\mca)$

                  $f(\mca) = 0 \Rightarrow \mu_{\mca} \mid f$

                  $ 1 \Rightarrow \mu_{\mca, v_i} \mid \mu_{\mca}, \ i=1, \ldots, n \Rightarrow f \mid \mu_{\mca} \Rightarrow f = \epsilon \mu_{\mca}, \ \epsilon \in K^*$.
              \end{center}
    \end{enumerate}
\end{proof}

\end{document}