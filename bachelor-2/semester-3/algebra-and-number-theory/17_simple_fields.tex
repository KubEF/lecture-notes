% !TeX root = ./main.tex
\documentclass[main]{subfiles}
\begin{document}

\chapter{Простые поля}

\marginpar{23.11.22}

\begin{definition}[Подполе]
    Пусть $K$ -- поле. Подполем $K$ называется подкольцо $F \subset K$, такое что
     $F$ -- поле.
\end{definition}

\begin{definition}[Простое поле]
    Поле $K$ называется простым, если в нем нет собственных подполей
\end{definition}

\begin{example}
    Пусть $\Q$  -- простое поле.
       
    Пусть  $F \subset \Q$ -- подполе.
        $1 \in F \Rightarrow \N \subset F \Rightarrow \Z \subset F 
        \Rightarrow \forall n \in \N : n^{-1} \in F \Rightarrow F = \Q$.
\end{example}
\begin{example}
    Пусть $p$ -- простое. Тогда $ \F_p = \underbrace{\Z / (p)}_{\Z /p \Z}$ -- простое поле.
        $F \subset \F_p $ -- подполе.
        $1 \in F \Rightarrow \quad \forall k \in \N : \underbrace{1 + \ldots + 1 }_{k} \in F  \Rightarrow F = \F_p
         \\ (1 = \overline{1}, \ 
        2 := 1 + 1$  и т. д.)
    
\end{example} 

\begin{definition}[Характеристика кольца]
    $R$ -- кольцо с единицей.
        \[\text{char} R = \begin{cases}
            min \{ n : \underbrace{1 + \ldots + 1}_n = 0 \text{ в } R \} \\
            0, \quad \text{ если } \forall n \in \N: \ \underbrace{1 + \ldots + 1}_n \ne 0
        \end{cases}
    \]
\end{definition}

\begin{lemma}
    Пусть $F$ -- поле. Тогда $\text{char} F$ является простым числом или $= 0$.
\end{lemma}

\begin{proof}
    Будем доказывать от противного. Пусть $l = \text{char} F = mn, \ m,\ n > 1$.
    $ 0 = \underbrace{1 + \ldots + 1}_l = \underbrace{(1 + \ldots + 1)}_{\underset{\ne 0 \text{, т.к. } m < l}{m}}
    \underbrace{(1 + \ldots + 1)}_{\underset{\ne 0 \text{, т.к. } n < l}{n}}$ -- невозможно в поле.
\end{proof}

\begin{proposition}
    Пусть $F_1, F_2$ -- поля, $\alpha : F_1 \rightarrow F_2$ -- гомоморфизм.
    Тогда $\alpha$ отображает $F_1$ изоморфно на подполе $\alpha(F_1)$
     (индуцирует изоморфизм на свой образ).
\end{proposition}

\begin{proof}
        \[\Ker \alpha \text{ -- идеал в } F_1 \]
        \[\Rightarrow \left[
            \begin{gathered}
            \Ker \alpha = 0   \\
            \Ker \alpha = F_1 \Rightarrow \underbrace{\alpha(1)} _{=1}= 0 \ne 1 \text{ - не реализуется}
            \end{gathered}
        \right.\]
        \[\Rightarrow \alpha \text{ инъективно} \]
        \[\Rightarrow \alpha \text{ индуцирует изоморфизм } F_1 \xrightarrow{\sim}
         \alpha(F_1) \]
       \[ F_1 \text{ поле } \Rightarrow \alpha (F_1) \text{ поле } \]
\end{proof}
Термин гоморфизм полей не используется, так как он всегда инъективен, говорят "вложение".
\begin{theorem}
    Пусть $F$ -- простое поле. 
    \begin{enumerate}
        \item Если $\text{char} F = 0$, то $F \cong \Q$.
        \item Если $\text{char} F = p > 0 $, то $F \cong \F_p$.
    \end{enumerate}
\end{theorem}

\begin{longProof}
    \begin{enumerate}
        \item Будем строить $\alpha: \mathbb{Q} \rightarrow F$.
        \[\left. \begin{gathered}
         n \in \mathbb{N} \quad \alpha(n) := \underbrace{1 + \ldots 1}_n \text{ в } F \\
            \alpha(0) := 0 \\
            \alpha(-n) := -\underbrace{(1 + \ldots + 1)}_n 
        \end{gathered}
        \right. \tag*{(*)} \]
        \[\alpha \left(  \frac{a}{b} \right) := \frac{\alpha(a)}{\alpha(b)}, \ \alpha(b) \ne 0 
        \text{, т.е. } \alpha(b) = \underbrace{1 + \ldots + 1}_b \neq 0, \text{ т.к. } \text{char} F = 0\]
        Проверка корректности:
        \begin{gather*}  
            \frac{a}{b} = \frac{a\prime}{b\prime} \Rightarrow ab^\prime = a^\prime b \\
            \Rightarrow \underbrace{\alpha(a b^\prime)}_{=\alpha(a)\alpha(b^\prime) \ne 0}
             = \underbrace{\alpha(a^\prime b)}_{=\alpha(a^\prime)\alpha(b) \ne 0} \\
            \Rightarrow \frac{\alpha(a)}{\alpha(b)} = 
            \frac{\alpha(a^\prime)}{\alpha(b^\prime)} 
        \end{gather*}
        Непосредственно проверяется, что $\alpha$ -- гомоморфизм.
        \begin{gather*}
            \Rightarrow \alpha(\mathbb{Q}) \text{ -- подполе } F, \text{ изоморфное } \mathbb{Q}\\
            \Rightarrow \alpha(\mathbb{Q}) = F \\
            \Rightarrow F \cong \mathbb{Q}
        \end{gather*}
        \item
         $\mathbb{Z} \stackrel{\alpha}{\longrightarrow} F$. Зададим $\alpha$ формулами $(*)$. Легко видеть:
        $ \alpha $ -- гомоморфизм. $\Ker \alpha$ -- идеал в $\Z$.
        \begin{gather*}
            \text{char} F = p \Rightarrow \begin{cases}
                p \in \Ker \alpha \\
                1, \ldots, p-1 \notin \Ker \alpha
            \end{cases} \Rightarrow \Ker \alpha = (p)
        \end{gather*}
        По теореме о гомоморфизме $\alpha$ -- индуцированный изоморфизм. $\overline{\alpha}: \ \Z / (p) 
        \xrightarrow{\sim} \Im \alpha, \ p$ -- простое $\Rightarrow \Z / (p) $ -- поле 
            $\Rightarrow \Im \alpha$ -- поле
            $\Rightarrow \Im \alpha = F \Rightarrow
            \overline{\alpha}: \ \mathbb{F}_p \xrightarrow{\sim} F$.
    \end{enumerate}
\end{longProof}

\begin{proposition}
    Пусть $K$ -- поле. Тогда в $K$ содержится единственное простое подполе.
\end{proposition}
\begin{proof}
    $F_0 := \bigcap_{F \text{ -- подполе } K} F$ -- подполе $K$. $F_1 \subset F_0$-- подполе $\Rightarrow F_1$-- подполе 
    $K \Rightarrow F_0 \subset F_1 \Rightarrow F_1 = F_0$.
    
    $F_0^\prime$ -- еще одно простое подполе $ \Rightarrow  F_0 \subset F^\prime_0 \Rightarrow F^\prime_0 = F_0$.
\end{proof}

\end{document}