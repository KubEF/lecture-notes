% !TeX root = ./main.tex
\documentclass[main]{subfiles}

\begin{document}

\chapter{Самосопряженные операторы}

\begin{proposition}
    Пусть $V$ -- евклидово или унитарное пространство, $\mca \in \End V$. Тогда $\mca = \mca^* \Leftrightarrow$ в $V$ существует ортонормированный базис $E$, такой что
    $[\mca]_E = diag(\alpha_1, \ldots, \alpha_n), \  \alpha_1, \ldots, \alpha_n \in \R$.
\end{proposition}

\begin{proof}
    $\Leftarrow$:
     % https://q.uiver.app/?q=WzAsNCxbMCwwLCJbXFxtYXRoY2Fse0F9XV4qX0UiXSxbMSwwLCJbXFxtYXRoY2Fse0F9XV9FIl0sWzAsMSwiW1xcbWF0aGNhbHtBfV4qXV9FIl0sWzEsMSwiXFxtYXRoY2Fse0F9XiogPSBcXG1hdGhjYWx7QX0iXSxbMCwyLCIiLDAseyJsZXZlbCI6Miwic3R5bGUiOnsiaGVhZCI6eyJuYW1lIjoibm9uZSJ9fX1dLFswLDEsIiIsMix7ImxldmVsIjoyLCJzdHlsZSI6eyJoZWFkIjp7Im5hbWUiOiJub25lIn19fV0sWzIsMywiIiwwLHsibGV2ZWwiOjJ9XV0=
    \[\begin{tikzcd}
        {[\mathcal{A}]^*_E} & {[\mathcal{A}]_E} \\
        {[\mathcal{A}^*]_E} & {\mathcal{A}^* = \mathcal{A}}
        \arrow[Rightarrow, no head, from=1-1, to=2-1]
        \arrow[Rightarrow, no head, from=1-1, to=1-2]
        \arrow[Rightarrow, from=2-1, to=2-2]
    \end{tikzcd}\]
    $\Rightarrow$: для унитарных доказано ранее.
    \[\begin{pmatrix}
        \alpha & \beta \\
        -\beta & \alpha
    \end{pmatrix}^T = \begin{pmatrix}
        \alpha & -\beta \\
        \beta & \alpha
    \end{pmatrix} \Rightarrow \beta = 0 \]
\end{proof}

\marginpar{16.11.22}

Далее $V$ -- евклидово пространство.

Пусть $\mca \in \End V, \ \mca = \mca^*$. $\underset{(v, w) \mapsto (\mca v, w)}{V\times V \xrightarrow{\mcb} \R}$.

\begin{proposition}
    $\mcb$ -- симметрическая билинейная форма. 

    \begin{enumerate}
        \item биленейность очевидна.
        \item $\mcb(w, v) = (\mca w, v) = (w, \mca^*v) = (w, \mca v) = (\mca v, w) = \mcb(v, w)$
    \end{enumerate}
\end{proposition}

\begin{remark}
    Для любой симметрической  билинийной формы $\mcb$ на $V$ существует единственный $\mca \in \End V, \ \mca = \mca^*$, т.ч. 
    $\mcb(v, w) = (\mca v, w)$.
\end{remark}
\begin{gather*}
    \text{*упражнение*} \\
    \text{если } E \text{ -- ортонормированный базис } V, \text{ то } [\mca]_E = [\mcb]_E \\
    \text{*упражнение*}
\end{gather*}

Пусть $V$ -- евклидово пространство, $\mca \in \End V$ -- самосопряженный. $\mca$ называется положительно определенным, если $\forall
v \in V, \ v \neq 0: \ (\mca v, v)  > 0$.

\begin{proposition}
    Следующие 2 условия эквивалентны:
    \begin{enumerate}
        \item $\mca$  положительно определенный.
        \item В $V$ существует ортонормированный базис $E$, такой что $[\mca]_E = diag(\lambda_1, \ldots, \lambda_n), \ \lambda_i > 0, \ i = 1, \ldots, n$.
    \end{enumerate}
\end{proposition}

\begin{proof}
    1 $\Rightarrow$ 2: существует ортонормированный $E$, такой что $[\mca]_E = diag(\lambda_1, \ldots, \lambda_n)$. 
    $\mca e_i = \lambda_i e_i$. $\underbrace{(\mca e_i, e_i)}_{>0} = (\lambda_i e_i, e_i) = \lambda_i (e_i, e_i) = \lambda_1$.

    2 $\Rightarrow$ 1: $v= \alpha_1 e_1 + \ldots + \alpha_n e_n, \ \exists i: \ \alpha_i \neq 0$. $\mca v = \alpha_1 \lambda_1 e_1 + \ldots + \alpha_n \lambda_n e_n$. $(\mca v, v) = \alpha_1^2 \lambda_1 + 
    \alpha_n^2 \lambda_n >0$.
\end{proof}

\begin{proposition}
    Пусть $\mca$ -- положительно определенныйна $V$. Тогда существует единственный $\mcb \in \End V$, такой что $\mca = \mcb^2$ и 
    $\mcb$ -- положительно определенный. 
\end{proposition}

\begin{proof}
    Существование: есть ортонормированный базис $E$: $[\mca]_E = diag(\lambda_1, \ldots, \lambda_n), \ \lambda_i \geq  0$.
    $\mcb(e_i) = \sqrt{\lambda_i} e_i$. $[\mcb]_E = diag(\sqrt{\lambda_1}, \ldots, \sqrt{\lambda_n}) = B$.
    $B^* = B \Rightarrow \mcb = \mcb^*; \ \mcb^2 = \mca$.

    Единственность: пусть $\mcb_1 = \mcb_1^*$, $\mcb^*$ положительно определен, $\mcb_1^* = \mca$.
    Существует ортонормированный $F$, $[\mcb_1]_F$ -- диагональная. Пусть $\mu_1, \ldots, \mu_k$ -- все собственные значения $\mcb_1$, $g_{\mcb_1\mu_1}, \ldots, g_{\mcb_1\mu_k}$ --
    их геометричсекие кратности. 
    \begin{gather*} 
        $g_{\mcb_1\mu_1} + \ldots + g_{\mcb_1\mu_k} = n = \dim V \\
        V_{\mcb_1\mu_j} = \Ker(\mcb_1 - \mu_j \mse) \\
        \mcb_1|_{V_{\mcb_1\mu_j}} = \mu_j \mse|_{\ldots} \\
        \Rightarrow \mcb_1^2|_{V_{\mcb_1\mu_j}} = \mu_j^2 \mse|_{\ldots} \\
        \Rightarrow \mca | _{V_{\mcb_1\mu_j} } = \mu_j^2 \mse|_{\ldots} \\
        \Rightarrow V_{\mcb_1\mu_j} \subset V_{\mca{\mu_j}^2} \\
        g_{\mcb_1\mu_j} \leq g_{\mca{\mu_j}^2} \\
        n = g_{\mcb_1\mu_1} + \ldots + g_{\mcb_1\mu_k} \leq g_{\mca\mu_1} + \ldots + g_{\mca\mu_k} \leq n \\
        \Rightarrow g_{\mcb_1\mu_j} = g_{\mca{\mu_j}^2} \text{ и из } \mca \text{ нет собственных значений, кроме } \mu_1^2, \ldots, \mu_k^2 \\
        \text{Таким образом, } V_{\mcb_1\mu_j} = V_{\mca_1{\mu_j}^2} \Rightarrow \\
        \Rightarrow (\mca e = \lambda e \Rightarrow \mcb_1 e = \sqrt{\lambda}e, \text{в частности, }) \mcb_1 e_i = \sqrt{\lambda_i} = \mcb e _i
    \end{gather*}
\end{proof}

\begin{theorem} [Полярное разложение]
    Пусть $V$ -- евклидово, $\mca \in \GL (V)$. Тогда существует единственное $\mcc, \mco \in \End V$, такие что 
    $\mcc$ -- положительно определенный, $\mco$ ортогональный и $\mca = \mcc \mco$.
\end{theorem}

$\mca \in \GL(V) \Rightarrow \mca^* \in \GL(V) \ (\mca \mcb = \mse \Rightarrow \mcb^* \mca^* = \mse)$

\begin{proof}
    $(\mca \mca^*)^* = (\mca^*)^* \mca^* = \mca \mca^*$. $\mca \mca^*$ положительно определенный: 
    $v \neq 0, \ (\mca \mca^* v, v) = (\underbrace{\mca^* v}_{\neq 0}, \mca^* v) > 0 \Rightarrow \mca \mca^* = \mcc^2, \ \mcc \in \End V$ -- положетельно определенный. 

    $\mco = \mcc^{-1}\mca, \ \mca = \mcc \mco$.

    $\mcc^2 = \mca \mca^* = \mcc \mco \mco^* \mcc^* = \mcc \mco \mco^* \mcc \Rightarrow \mse = \mco \mco^*$.
\end{proof}

\begin{gather*}
    \mca = \mcc_1 \mco_1 = \mcc_2 \mco_2 \
    \mcc_1,\mcc_2 \text{ -- полож. опред.}, \ \mco_1, \mco_2 \text{ -- ортог.} \\
    \text{а потом нас обманули...} \\
    \mca \mca^* = \mcc_1 \mco_1 \mco_1^* \mcc_1 = \mcc_1^2 \\
    \mca \mca^* = \mcc_2 \underbrace{\mco_2 \mco_2^*}_{=\mse} \mcc_2 = \mcc_2^2 \\
    \text{из соображений единственности: } \mcc_1 = \mcc_2, \ \mco_1, \mco_2
\end{gather*}

\end{document}