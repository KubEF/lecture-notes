% !TeX root = ./main.tex
\documentclass[main]{subfiles}
\begin{document}
\part{Дифференциальная геометрия поверхностей}
\chapter{Касательная плоскость}\marginpar{03.10.22}
\begin{theorem*}[О неявной функции]
    Если $F(x,y,z) = 0$ и $F(x_0, y_0,z_0) = 0$,
    а так же $\frac{\partial F}{\partial z}\rvert_{z = z_0} \neq 0$,
    тогда в некоторой окрестности $(x_0, y_0)$,
    тогда существует $f: \fancyD \to \R$, что $F(x, y, f(x,y)) =0$.
\end{theorem*}

Как задать поверхность?
\begin{itemize}
    \item Явная формула: $z = f(x,y)$, если $(x,y) \in \fancyD \subset \R^2$
          ($\fancyD$ -- область, то есть открытое и связное множество)
          \begin{center}
              \import{figures/}{surface_z(x,y).pdf_tex}
          \end{center}
    \item Неявное задание: $F(x,y,z) = 0$. Классический пример -- сфера
          $F(x,y,z)=x^2 + y^2 + z^2 - R^2 = 0$
    \item Параметрическое задание:
          есть $(u,v) \in \fancyD \subset \R^2$ -- область,
          $(u,v)$ -- внутренние координаты.
          \[\begin{cases}
                  x = x(u,v) \\
                  y = y(u,v) \\
                  z = z(u,v)
              \end{cases}\]
          Иначе говоря, $\vr = \vf(u,v)$, где $f: \fancyD \to \R^3$.
\end{itemize}

\begin{example}
    Поверхность вращения.
    Есть кривая с параметризацией $x = f(t), z = g(t)$:

    \begin{minipage}{0.59\textwidth}
        \import{figures/}{surface_of_revolution.pdf_tex}
    \end{minipage}
    \begin{minipage}{0.35\textwidth}
        \[\begin{cases}
                x = f(t) \cos \phi \\
                y = f(t) \sin \phi \\
                z = g(t)
            \end{cases}\]
    \end{minipage}
\end{example}
\begin{example}
    Еще более частный пример: сфера.
    Зададим полуокружность в плоскости $Oxz$
    \[\begin{cases}
            x = R \cos \psi \\
            z = R \sin \psi
        \end{cases}\]
    % Нужен рисунок??
    Тогда параметризация поверхности вращения
    \[\begin{cases}
            x = R \cos \psi \cos \phi \\
            y = R \cos \psi \sin \phi \\
            z = R \sin \psi
        \end{cases}\]
    Получили стандартные географические координаты, где
    $\phi$ -- долгота ($\phi \in [0;2 \pi]$), $\psi$ -- широта ($\psi \in [-\pi/2 ; \pi/2]$)
\end{example}

\section{Перепараметризация поверхности}
\begin{definition}[Диффеоморфизм]
    Диффеоморфизм -- дифференцируемый гомеоморфизм, обратный к которому тоже дифференцируем.
\end{definition}
\begin{center}
    \import{figures/}{reparametrization_of_surfaces.pdf_tex}
\end{center}
Если $\theta$ -- диффеоморфизм, то $\vf = \vg \theta$ и $\vg = \vf \theta^{-1}$.

\section{Кривая лежит на поверхности}
\begin{center}
    \import{figures/}{curve_on_surface.pdf_tex}
\end{center}
Считаем, что поверхность несамопересекается, то есть $\vf$ -- инъекция.
Тогда $\vh = \vf^{-1}\vg$, где $\vh: [a,b] \to \fancyD$ и $u = h_1(t), v = h_2(t)$ -- внутренние уравнения кривой на поверхности.

\begin{definition}[Касательный вектор поверхности]
    Вектор $\vv$ называется касательным вектором поверхности в точке $M$,
    если существует кривая на поверхности, проходящая через $M$, так что
    $\vv$ -- касательный вектор к данной кривой в точке $M$.
\end{definition}

Верно ли что касательные векторы образуют плоскость? Ответ не тривиальный.

Есть 2 касательных вектора
$\frac{\partial \vf}{\partial u}$ и $\frac{\partial \vf}{\partial v}$ к кривым
\[\begin{cases}
        u = t \\
        v = v_0
    \end{cases}
    \text{ и }
    \begin{cases}
        u = u_0 \\
        v = t
    \end{cases}.\]
Векторы получены из
\[\frac{d\vf(u,v)}{dt} = \frac{\partial\vf}{\partial u} u_t' + \frac{\partial\vf}{\partial v} v_t'.\]

\begin{definition}[Регулярная параметризация]
    Параметризация называется регулярной, если $\forall (u_0, v_0) \in \fancyD$
    \[\frac{\partial\vf}{\partial u}|_{(u_0, v_0)}
        \not\parallel
        \frac{\partial\vf}{\partial v}|_{(u_0, v_0)}.\]
\end{definition}
Теперь будем последовательно фиксировать $v_0$ и строить кривые,
потом повторим то же самое с $u_0$, получим семейство координатных кривых:
\begin{center}
    \import{figures/}{coordinate_curves.pdf_tex}
\end{center}
Параметризация регулярная $\Leftrightarrow$ любые 2 линии пересекаются под углом, не равном 0.

Если $\vf$ регулярная, то
\[
    \frac{\partial \vf}{\partial u} = \vf'_u \quad
    \frac{\partial \vf}{\partial v} = \vf'_v \quad
    \vn = \frac{\vf_u' \times \vf_v'}{|\vf_u' \times \vf_v'|}
\]
образуют базис (свой для каждой точки).
При этом $\vf_u'$ и $\vf_v'$ не обязательно ортогональны.

\begin{example}[Поверхность с нерегулярной параметризацией]
    Есть кривая в пространстве, рассмотрим множество касательных прямых.
    Эти прямые будут заметать некоторую поверхность, в некоторой окрестности эта поверхность даже будет несамопересекающейся.
    \begin{center}
        \import{figures/}{not_regular_surface.pdf_tex}
    \end{center}
    Найдем параметризацию это поверхности:
    \[\vg(t, \tau) = \vf(t) + \vf'(t) \tau\]
    При $\tau = 0$ параметризация не регулярная.
    \[\vg_t' |_{\tau = 0} = \vg_\tau'|_{\tau =0} = \vf'(t)\]
\end{example}

\begin{theorem}
    $\vf$ -- регулярная параметризация, тогда касательные векторы в точке $(u_0, v_0)$ образуют плоскость с базисом $\vf_u'$ и $\vf_v'$.
\end{theorem}
\begin{proof}
    Рассмотрим кривую на поверхности с внутренними координатами $u(t), v(t)$,
    при этом $u(t_0) =u_0, v(t_0) = v_0$.
    \begin{gather*}
        \frac{d\vf}{dt} = \frac{\partial\vf}{\partial u} u_t' + \frac{\partial \vf}{\partial v} v_t' = \vf'_u u'_t + \vf_v' v_t' \in \langle \vf_u', \vf_v'\rangle
    \end{gather*}
    так как $u_t'$ и $v_t'$ константы в точке $t_0$.
\end{proof}
\begin{definition}[Касательная плоскость]
    Касательная плоскость это плоскость, состоящая из касательных векторов.
\end{definition}
\begin{definition}[Вектор нормали]
    \[\vn = \frac{\vf_u' \times \vf_v'}{|\vf_u' \times \vf_v'|}\]
\end{definition}

Формально докажем, что любой вектор плоскости $\langle \vf_u', \vf_v'\rangle$ -- касательный вектор.
Рассмотрим $\alpha \vf_u' + \beta \vf_v'$, пусть внутренние координаты $u = \alpha t, v = \beta t$,
которые задают кривую на поверхности, которая, очевидно, имеет касательный вектор $\alpha \vf_u' + \beta \vf_v'$.

Как написать уравнение касательной плоскости, если поверхность задана параметрически?

Пусть $\vf(u,v) = (x(u,v), y(u,v), z(u,v))$, тогда $\vf_u' = (x_u', y_u', z_u')$ и $\vf_v' = (x_v', y_v', z_v')$.
В таком случае нормаль к касательной плоскости -- $\vf_u' \times \vf_v'$.
А уравнение касательной плоскости:
\[\begin{vmatrix}
        y_u' & z_u' \\
        y_v' & z_v'
    \end{vmatrix} (X - x_0 ) +
    \begin{vmatrix}
        z_u' & x_u' \\
        z_v' & x_v'
    \end{vmatrix} (Y - y_0 ) +
    \begin{vmatrix}
        x_u' & y_u' \\
        x_v' & y_v'
    \end{vmatrix} (Z - z_0 ) = 0\]
(и все определители не равны 0)

Явный вид сводится к параметрическому:
\[\begin{cases}
        x = u \\
        y = v \\
        z = z(u,v)
    \end{cases}\]

В случае неявного вида: $F(x,y,z) = 0$,
у нас есть градиент $\nabla F = (F_x', F_y', F_z')$, и тогда
$\nabla F |_M$ -- вектор нормали к поверхности в точке $M$. Докажем это:

Пусть $x(t), y(t), z(t)$ -- любая кривая на поверхности.
Это означает, что $F(x(t), y(t), z(t)) \equiv 0\ \forall t$, соответственно
$\frac{d}{dt}F = F_x' x' + F_y' y' + F_z' z' =0$, а это
$\nabla F \cdot (x', y', z') = 0$, где
$(x', y', z')$ -- касательный вектор к кривой,
тогда градиент перпендикулярен касательному вектору.
По заданию у нас любая прямая, значит градиент -- вектор нормали.
\end{document}