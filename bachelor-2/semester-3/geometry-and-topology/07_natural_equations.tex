% !TeX root = ./main.tex
\documentclass[main]{subfiles}
\begin{document}
\chapter{Натуральные уравнения кривой}

Есть кривая в натуральной параметризации.
Знаем $k(s)$ и $\kappa(s)$.
Верно ли, что существует ровно одна кривая, что эти функции являются ее кривизной и кручением?
Единственность точно верно, с существованием~--- не всегда, $k$ должно быть положительно.

\begin{theorem}
    Если $\vg_1(s)$ и $\vg_2(s)$~--- кривые с одинаковыми $k$ и $\kappa$, тогда они отличаются друг от друга движением.
\end{theorem}
\begin{proof}
    В точке $s_0$ состыкуем реперы Френе.

    $(\vv_1(s), \vn_1(s), \vb_1(s))$~--- репер Френе для $\vg_1$.

    Аналогично, $(\vv_2(s), \vn_2(s), \vb_2(s))$~--- репер Френе для $\vg_2$.

    Что означает <<состыкуем>>?
    \begin{gather*}
        \vv_1 (s_0) = \vv_2(s_0)\\
        \vn_1 (s_0) = \vn_2(s_0)\\
        \vb_1 (s_0) = \vb_2(s_0)
    \end{gather*}

    Заведем скалярную функцию
    \[h(s) = \vv_1(s) \vv_2(s) + \vn_1(s) \vn_2(s) + \vb_1(s) \vb_2(s)\]
    заметим, что
    \begin{itemize}
        \item $h(s_0) = 3$
        \item $h(s) \le 3$ и $h(s) = 3 \Leftrightarrow \vv_1 = \vv_2, \vn_1 = \vn_2, \vb_1 = \vb_2$
    \end{itemize}
    Как доказать, что скалярная функция равна константе во всех точках, если мы знаем, что она равна константе в одной точке?
    Достаточно взять производную и показать, что она ноль.
    Возьмем производную $h$:
    \begin{multline*}
        h'(s) = \vv_1' \vv_2 + \vv_1 \vv_2' + \vn_1' \vn_2 + \vn_1 \vn_2' + \vb_1' \vb_2 + \vb_1 \vb_2' = \\
        = k \vn_1 \vv_2 + k \vv_1 \vn_2  + \kappa \vb_1 \vn_2 + \kappa \vn_1 \vb_2\\
        - k \vn_1 \vv_2 - k \vv_1 \vn_2  - \kappa \vb_1 \vn_2 - \kappa \vn_1 \vb_2 = 0
    \end{multline*}
    тогда $h(s) = 3 \ \forall s$.
\end{proof}
\begin{definition}[Натуральные уравнения кривой]
    $(k(s), \kappa(s))$~--- натуральные уравнения кривой.
\end{definition}

Практический вывод:
хотим спроектировать резьбу.
Что такое резьба~--- кривая.
Нам известна такая резьба~--- винтовая линия:
\[\begin{cases}
        x = R \cos t \\
        y = R \sin t \\
        z = at
    \end{cases}\]
Меняем $R$~--- меняем радиус резьбы, меняем $a$~--- меняем шаг резьба.
Вопрос: есть ли другая кривая, подходящая для нарезки резьбы?
Ответ: нет, нельзя.
Выясним почему:
когда мы завинчиваем
(то есть делаем движение одной кривой, относительно другой кривой)
болт в гайку, кривые болта и гайки должны самосовместиться, то есть кривизна и кручение у кривой в во всех точках одинаковые.
То есть наша кривая должна удовлетворять условию, что $k = const, \kappa = const$.
Подсчитаем их для винтовой кривой:
\begin{gather*}
    \vf = (R\cos t, R \sin t, at)\\
    \vf' = (-R \sin t, R \cos t, a) \\
    \vf'' = (-R \cos t, - R \sin t, 0)\\
    \vf''' = (R \sin t, -R \cos t, 0) \\
    \vf' \times \vf'' = (aR \sin t, - aR \cos t, R^2)\\
    |\vf' \times \vf''| = \sqrt{a^2 R^2 + R^4} = R \sqrt{R^2 + a^2}\\
    |\vf'| = \sqrt{R^2 + a^2}\\
    k = \frac{|\vf' \times \vf''|}{|\vf'|^3} = \frac{R \sqrt{R^2 + a^2}}{(R^2 + a^2) \sqrt{R^2 + a^2}} = \frac{R}{R^2 + a^2}\\
    (\vf', \vf'', \vf''') = \begin{vmatrix}
        -R \sin t & R \cos t   & a \\
        -R \cos t & - R \sin t & 0 \\
        R \sin t  & - R \cos t & 0
    \end{vmatrix} = aR^2\\
    \kappa =\frac{(\vf', \vf'', \vf''')}{|\vf' \times \vf''|^2} = \frac{aR^2}{R^2 (R^2+ a^2)} = \frac{a}{R^2 + a^2}.
\end{gather*}
\end{document}