% !TeX root = ./main.tex
\documentclass[main]{subfiles}
\begin{document}
\chapter{Кривизна кривой на поверхности} \marginpar{31.10.22}
Есть поверхность с параметризацией $\vf: \fancyD \to \R^3$.
$u = u(s), v = v(s)$~--- внутренние координаты кривой на поверхности.
Тогда $\vpsi(s) = \vf(u(s), v(s))$, где $s$~--- натуральный параметр.
И $|\vpsi'(s)| = 1$, $|\vpsi'' (s)| = k$.
\begin{gather*}
    \vpsi' = \vf'_u u' + \vf'_v v'\\
    \vpsi'' = \vf''_{uu}u'^2 + \vf''_{uv} u' v' + \vf'_u u'' + \vf''_{vu} u' v' + \vf''_{vv} v'^2 + \vf'_vv''
\end{gather*}
Хотим избавится от слагаемых $\vf'_u u''$ и $\vf'_vv''$, для этого умножим $\vpsi''$ на вектор нормали $\vn$, и т.к. $\vf'_u \perp \vn$, $\vf'_v \perp \vn$, получим
\begin{gather*}
    \vpsi'' \cdot \vn =\vf''_{uu} \vn u'^2 + 2 \vf''_{uv} \vn u'v' + \vf''_{vv} \vn v'^2\\
    \intertext{Обозначим}
    \begin{aligned}
        \vf''_{uu} \vn & = L(u,v) &
        \vf''_{uv} \vn & = M(u,v) &
        \vf''_{vv} \vn & = N(u,v)
    \end{aligned}
\end{gather*}
При этом $\vpsi'' \cdot \vn = k \cos\theta$, где $\theta$~--- угол и $|\vn| =1$.
\begin{center}
    \import{figures}{vpsi''_times_vn.pdf_tex}
\end{center}
Отсюда получаем:
\[k \cos \theta = L u'^2 + 2M u'v' + Nv'^2\]
\begin{definition}[Вторая квадратичная форма]
    $L, M, N$~--- коэффициенты $\sff$  квадратичной формы поверхности, где
    \begin{align*}
        L(u,v) & \coloneqq \vf''_{uu} \vn &
        M(u,v) & \coloneqq \vf''_{uv} \vn &
        N(u,v) & \coloneqq \vf''_{vv} \vn
    \end{align*}
    и
    \[\sff(u',v') \coloneqq L u'^2 + 2M u'v' + Nv'^2\]
\end{definition}
\begin{theorem}
    Если у кривых совпадает угол $\theta$ и $u', v'$ в данной точке во внутренних координатах, тогда у них совпадает кривизна.
\end{theorem}
\begin{theorem}
    Если $\cos \theta = \pm 1$, то кривизна такой кривой зависит только от касательного вектора $(u',v')$.
\end{theorem}

Существует ли ситуация, когда $\cos\theta = \pm 1$?

Пусть есть поверхность и точка на ней, для которой построены вектор нормали $\vn$ и касательная плоскость.
Проведем направление $l$  в касательной плоскости.
Хотим провести кривую на поверхности с касательным вектором в этом направлении.
Кривую рассматриваем в натуральной параметризации, т.ч. касательный вектор будет единичным.
И все еще хотим  $\cos\theta = \pm 1$, можем ли мы провести такую прямую?
Да, можем.
Для этого через $l$ и $\vn$ строим плоскость $\alpha$, она единственна.
$\alpha$ пересекает поверхность по какой-то кривой, эта кривая плоская, значит ее вектор нормали лежит в плоскости.
У этой кривой есть касательный вектор $\vpsi'$ и есть вектор $\vpsi''$, поскольку $\vpsi'$ единичный, то $\vpsi' \perp \vpsi''$ всегда.
И раз вся эта конструкция лежит в плоскости $\alpha$, то получается, что $\cos\theta = \pm 1$.
\begin{center}
    \import{figures}{cos_theta_pm_1.pdf_tex}
\end{center}

\begin{definition}[Нормальная кривизна]
    $l$~--- направление в касательной плоскости поверхности.
    $k_{n}(l)$~--- кривизна, построенной кривой с $\cos\theta = \pm 1$.
    Такая $k_{n}(l)$ называется нормальной кривизной поверхности в направлении $l$.
\end{definition}
\begin{theorem}[Мёнье]
    Нормальная кривизна $k_{n}(l)$ не зависит от кривой.
\end{theorem}

$\vpsi''$ можно разложить на 2 компоненты
\[\vpsi'' = \vpsi''_{n} + \vpsi''_{g}\]
где $\vpsi''_{n}$~--- проекция на вектор нормали, а $\vpsi''_{g}$~--- проекция на касательную плоскость.
В таком случае $\vpsi''_{n} = k_{n}(l)$.
Кроме того $k_{g}(l)$~--- геодезическая кривизна кривой,
\[\begin{cases}
        k_1 \coloneqq k_{n\ \min} \\
        k_2 \coloneqq k_{n\ \max}
    \end{cases}\]
главные кривизны поверхности в точке, а
\[\begin{cases}
        l_{\min} \\
        l_{\max}
    \end{cases}\]
главные направления соответственно.

\begin{definition}[Гауссова кривизна]
    \[K \coloneqq k_1k_2 \]
\end{definition}
\begin{definition}[Средняя кривизна]
    \[H \coloneqq \frac{k_1 + k_2}{2}\]
\end{definition}
\begin{theorem}[Egregium, Гаусс]
    $K$ зависит только от $E,F,G$ и их производных.
\end{theorem}

[Здесь воспроизведение магии вживую, посмотрите сами]

До этого все рассуждения были в натуральной параметризации, теперь перейдем к любой.

Пусть $\vphi(t)$~--- любая параметризация, а $\vpsi(s)$~--- натуральная, тогда
\[\vpsi'(s) = \frac{\vphi'(t)}{|\vphi'(t)|}\]
Тогда  $k \cos\theta = \vpsi'' \vn$.
$\vpsi(s)$ соответствует внутренним координатам $\tilde{u}(s), \tilde{v}(s)$.
А $\vphi(t)$ соответствует $u(t), v(t)$.
В таком случае
\begin{gather*}
    \tilde{u}'(s) = \frac{u'(t)}{|\vphi'(t)|} \quad \tilde{v}'(s) =\frac{v'(t)}{|\vphi'(t)|}\\
    \begin{multlined}
        k \cos \theta = L \tilde{u}'^2 + 2M \tilde{u}' \tilde{v}' + N \tilde{v}'^2 = \frac{L u'^2 + 2M u'v' + Nv'^2}{|\vphi'(t)|^2} =\\
        = \frac{L u'^2 + 2M u'v' + Nv'^2}{Eu'^2 + 2F u'v' + G v'^2} = \frac{\sff(u',v')}{\fff(u',v')}
    \end{multlined}
\end{gather*}
\begin{theorem}
    В произвольной параметризации
    \[k_{n(u', v')} = k \cos \theta = \frac{\sff(u',v')}{\fff(u',v')}\]
\end{theorem}
\end{document}