% !TeX root = ./main.tex
\documentclass[main]{subfiles}
\begin{document}
\chapter{Соприкасающаяся плоскость}\marginpar{19.09.22}
В натуральной параметризации $\vv = \vf'$ и $\vn = \vf''/k$.
Тогда плоскость $\langle \vf', \vf''\rangle$~--- соприкасающаяся плоскость для натуральной параметризации.

А что будет в случае не натуральной параметризации?
Посмотрим, что в таком случае происходит с вектором $\vf''$, будет ли он перпендикулярен $\vf'$?
Нет, не будет, потому что, если вектор $\vf'' \perp \vf'$, то $|\vf'| = const$
и параметризация почти натуральная, в том смысле, что наша скорость постоянная, но возможно не единичная.
Вывод: в обычной ситуации $\vf''$ не перпендикулярен $\vf'$, однако плоскость в которой он лежит не меняется.

\begin{theorem}
    Плоскость $\langle \vf', \vf''\rangle$ не зависит от параметризации.
\end{theorem}
\begin{proof}
    Пусть $\vf(t) = \vg(s)$, где $s$ не обязательно натуральный параметр и $s = \phi(t)$, тогда $\vg(\phi(t)) = \vf(t)$.
    Уже доказано, что $\vf' \parallel \vg'$.
    Теперь выясним, что
    \begin{multline*}
        \vf''(t) = (\vg(\phi(t)))'' = (\vg'(\phi(t)) \phi'(t))' = \\
        =\vg''(\phi(t)) \phi'^2(t) + \vg'(\phi(t)) \phi''(t) \in \langle \vg', \vg''\rangle
    \end{multline*}
\end{proof}

\begin{theorem}
    Есть регулярная параметризация $\vf(t)$,
    $\delta(t)$~--- расстояние от $\vf(t)$ до прямой, проходящей через точку $\vf(t_0)$.
    \begin{center}
        \import{figures/}{approximation1.pdf_tex}
    \end{center}
    \[\lim_{t \to t_0} \frac{\delta}{|\vf(t) - \vf(t_0)|} = 0.\]
    Такой $\lim = 0$ тогда и только тогда, когда касательная прямая.
\end{theorem}
\begin{proof}
    Выберем удобную для нас координатную систему:
    \begin{itemize}
        \item $\vf(t_0) = (0,0,0)$
        \item $t_0 = 0$
        \item Касательная прямая~--- прямая $OX$.
              Тогда $\vf'(0) = (a,0,0)$.
    \end{itemize}
    Пусть $\vf(t) = (f_1(t), f_2(t), f_3(t))$, выясним что такое $\delta$.
    \[\delta = \sqrt{f_2^2(t) + f_3^2(t)}\]
    Разложим $f_1$ по Тейлору:
    \begin{gather*}
        f_1(t) = f_1(0) + f_1'(0)t + o(|\vf(t) - \vf(t_0)|) = at + o(t)
        \intertext{На малом промежутке $|\vf(t) - \vf(t_0)| \approx t$. Аналогично с $f_2, f_3$:}
        f_2 (t) = f_2(0) + f_2'(0)t+o(t) = o(t)\\
        f_3 (t) = o(t)
    \end{gather*}
    Отсюда, $\delta(t) = o(t)$ и
    \[\lim_{t \to 0} \frac{\delta(t)}{t} = 0\]
    А так же
    \[\lim_{t \to 0} \frac{|\vf(t) - \vf(0)|}{t} = |\vf'(0)|\]

    Обратное доказательство~--- упражнение.
    (Hint: если $\delta$~--- расстояние от данной точки до любой прямой, кроме $OX$,
    то в формуле для $\delta$ появится слагаемое $f_1^2(t)$)
\end{proof}

% Здесь умирает звук в записи => снижается качество пояснений.
\begin{theorem}
    Пусть $\vf(t)$~--- бирегулярная параметризация.
    $\delta$~--- расстояние от $\vf(t)$ до плоскости $\alpha$.
    \begin{center}
        \import{figures/}{approximation2.pdf_tex}
    \end{center}
    \[\lim_{t \to t_0} \frac{\delta}{t^2} = 0 \left( \lim_{t \to t_0} \frac{\delta}{|\vf(t) - \vf(t_0)|^2} = 0 \right)\]
    Такой $\lim = 0 \Leftrightarrow \alpha$~--- соприкасающаяся плоскость.
\end{theorem}
\begin{proof}
    Введем удобную систему координат:
    \begin{enumerate}
        \item $t_0 = 0$
        \item $\vf(0) = (0,0,0)$
        \item $\vf'(0) = (a,0,0)$ и $\vf''(0) = (b,c,0)$
    \end{enumerate}
    Тогда соприкасающаяся плоскость описывается уравнением $z = 0$ (это следует из вида $\vf'(0)$ и $\vf''(0)$).

    Запишем вектор-функцию в координатах: \[\vf(t) = (f_1(t), f_2(t), f_3(t)).\]
    Пусть плоскость $\alpha$ задана уравнением $Ax+By+Cz+D = 0$ и $A^2 + B^2 + C^2 = 1$.
    Подсчитаем $\delta$ используя разложение по Тейлору:
    \begin{multline*}
        \delta = |Af_1(t) + Bf_2(t) + Cf_3(t) + D| =\\
        \left| A \left(f_1(0) + f_1'(0)t + \frac{f_1''(0)}{2}t^2 + o(t^2)\right) + \right.\\
        + B \left(f_2(0) + f_2'(0)t + \frac{f_2''(0)}{2}t^2 + o(t^2)\right) + \\
        + \left. C \left(f_3(0) + f_3'(0)t + \frac{f_3''(0)}{2}t^2 + o(t^2)\right) + D \right| = \\
        \left| Aat + \frac{Ab}{2} t^2 + \frac{Bc}{2}t^2 + D + o(t^2) \right|
    \end{multline*}
    Теперь хотим выяснить чему равносильно $\delta = o(t^2)$.
    \[\begin{cases}
            Aa = 0 \\
            Bc = 0 \\
            D = 0  \\
        \end{cases}\]
    При этом $a \neq 0$, т.к. это единственная ненулевая координата касательного вектора, она не может быть нулем.
    И $c \neq 0$, иначе $\vf''$ коллинеарно $\vf'$. Тогда
    \[\begin{cases}
            A = 0 \\
            B = 0 \\
            D = 0
        \end{cases}\]
    и $\alpha$ имеет единственно возможное уравнение $z = 0$.
\end{proof}

Посмотрим как задаются все эти плоскости в координатах.
Пусть
$\vf(t) = (f_1(t), f_2(t), f_3(t))$
и $\vf$ не натуральная параметризация (т.к. к натуральной параметризации тяжело перейти).
\[\vf'(t) = (f_1',f_2',f_3')\]

Построим нормальную плоскость:
\[f_1'(t_0)(x - f_1(t_0)) + f_2'(t_0)(y - f_2(t_0)) + f'_3(t_0)(z - f_3(t_0)) = 0\]

Построим соприкасающуюся плоскость:
для этого найдем вектор главной нормали
\[\vn = \vf' \times \vf'' = (f_2'f_3'' - f_3'f_2'',f_3'f_1'' - f_1'f_3'', f_1'f_2'' - f_2'f_1'')\]
\begin{center}
    \import{figures/}{main_normal_vector.pdf_tex}
\end{center}
и уравнение плоскости
\[(f_2'f_3'' - f_3'f_2'')(x - f_1) + (f_3'f_1'' - f_1'f_3'')(y - f_2) + (f_1'f_2'' - f_2'f_1'')(z - f_3) = 0\]
Построение спрямляющей плоскости опущено в виду громоздкости выкладок.
\end{document}