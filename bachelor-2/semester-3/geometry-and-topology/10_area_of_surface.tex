% !TeX root = ./main.tex
\documentclass[main]{subfiles}
\begin{document}
\chapter{Площадь поверхности}
\begin{theorem}
    \[S = \iint_{\fancyD} \sqrt{EG - F^2} dudv\]
\end{theorem}
\begin{proof}
    Не будет.
\end{proof}
[Здесь был ликбез по двойному интегралу]
\begin{theorem}
    \[EG > F^2\]
\end{theorem}
\begin{proof}
    \[\left( \frac{\partial f}{\partial u}\right)^2 \left( \frac{\partial f}{\partial v}\right)^2 > \left(\frac{\partial f}{\partial u}\frac{\partial f}{\partial v}\right)^2\]
    потому что по сути справа скалярное произведение векторов, то есть произведение длин векторов на косинус угла меду ними, а слева просто квадраты длин.
    Почему не равенство? Оно не достигается, т.к. иначе $\frac{\partial f}{\partial u} \parallel \frac{\partial f}{\partial v}$, что невозможно, т.к. мы рассматриваем только регулярные поверхности.
\end{proof}

\section{Что такое площадь поверхности?}
В данный момент, доказательство у теоремы о площади поверхности отсутствует не только,
из-за недостаточного математического инструментария, но и в силу того, что мы не определили, что такое площадь.
Однако площадь не просто определить, а некоторыми способами площадь определить вообще нельзя.

Как определялась длина кривой?
Разбивали кривую точками, проводили прямые, считали их длину, а потом переходили к пределу.
Можно ли сделать то же самое с поверхностями?
Заведем триангуляцию поверхности, дальше попытаемся перейти к пределу, если максимальная длина грани треугольника стремится к 0.
Но эта штука не работает, и таким образом нельзя даже подсчитать площадь цилиндра.
\begin{example}
    Контрпример (сапог) Шварца.

    Антипризма -- фигура, у которой в нижнем основании правильный $n$-угольник, в верхнем тоже, и они смещены друг от друга.

    Возьмем цилиндр и разобьем его на маленькие цилиндрики, а затем каждый кусочек заменим на антипризму.
    Пусть у нас $k$ кусочков по высоте.
    Найдем площадь получившегося барабана и поймем к чему она стремится если $k$ и $n$ стремятся к бесконечности.
    \begin{center}
        \import{figures/}{schwarz_lantern.pdf_tex}
    \end{center}

    Пусть $h = H/k$ -- высота каждого участка.
    Тогда $S = k \cdot 2n \cdot S_{\triangle}$ и нам нужно вычислить $S_{\triangle}$.
    \begin{center}
        \import{figures/}{schwarz_lantern_part.pdf_tex}
    \end{center}
    \begin{gather*}
        OD = r, \alpha = AOD\\
        \angle AOB = \frac{2\pi}{n} \implies \angle \alpha = \frac{\pi}{n}\\
        AD = AO\sin \alpha = R \sin \frac{\pi}{n}\text{ по определению антипризмы}\\
        AB = 2AD = 2 R \sin \frac{\pi}{n}\\
        \intertext{$CED$ -- прямоугольный треугольник в пространстве}
        h_{ABC} = \sqrt{CE^2 + DE^2} = \sqrt{h^2 + \left(R-R \cos \frac{\pi}{n}\right)^2}\\
        S_{ABC} = \frac{1}{2}AB h_{ABC} = R \sin \frac{\pi}{n} \sqrt{h^2 + R^2 \left(1 - \cos \frac{\pi}{n}\right)^2}\\
        S = 2kn\cdot R \sin \frac{\pi}{n} \sqrt{\frac{H^2}{k^2} + 4R^2\sin^4\frac{\pi}{2n}}\\
        n \sin \frac{\pi}{n} \to \pi\\
        \lim_{k,n \to \infty} S_{k,n} = 2 \pi R \lim_{\substack{n \to \infty\\ k\to \infty}} \sqrt{H^2 + 4R^2 k^2 \sin^4 \frac{\pi}{2n}}
    \end{gather*}
    если $k^2 \sin^4(\pi/2n) \to 0$ $\Leftrightarrow k^2/n^4 \to 0$ или $k/n^2 \to 0$, то
    \[\lim_{k,n \to \infty} S_{k,n} = 2\pi RH.\]
    Если нет, то есть проблемы.
    Например если $k = n^2$, то
    \[\lim_{k,n \to \infty} S_{k,n} = 2 \pi R \sqrt{H^2 + 4R^2 \left(\frac{\pi}{2}\right)^4}\]
    если $k = n^3$, то
    \[2 \pi R \lim_{n \to \infty} \sqrt{H^2 + 4R^2 n^6 \sin^4 \frac{\pi}{2n}} = \infty\]

    Почему вообще так происходит?
    Посмотрим на треугольник $ABC$, там как раз видно, что $k$ довольно велико относительно $n$,
    то есть полоса узкая, но нарезаем довольно грубо на $n$-угольник.
    В таком случае это треугольник НЕ параллелен плоскости боковой поверхности,
    и получается нарезка из треугольников стоящих довольно сильно поперек, за счет этого площадь поверхности увеличивается.
    Поэтому подбирая $k$ и $n$ можно прийти к любом пределу, большему чем настоящая площадь.
\end{example}

Тогда как правильно определить площадь поверхности?

\begin{definition}[Площадь поверхности]
    Есть поверхность, разобьем ее координатными линиями и рассмотрим конкретный параллелограмм.
    К нему есть 2 касательных вектора, они формируют касательную плоскость к данной поверхности.
    Спроецируем каждый параллелограмм на свою касательную плоскость.
    Посчитаем площадь в касательной плоскости, сложим их и перейдем к пределу.
    \begin{center}
        \import{figures}{surface_area.pdf_tex}
    \end{center}
\end{definition}
\begin{theorem}
    \[\left| \frac{\partial \vf}{\partial u} \times \frac{\partial \vf}{\partial v}\right| = \sqrt{EG - F^2}\]
\end{theorem}
\begin{proof}
    Пусть $\vf = (x,y,z)$, где $x=x(u,v), y = y(u,v), z = z(u,v)$, тогда $\vf'_u = (x'_u, y'_u, z'_u), \vf'_v = (x'_v, y'_v, z'_v)$.
    Посчитаем в координатах:
    \begin{gather*}
        \vf'_u \times \vf'_v = (y'_u z'_v - z'_u y'_v; z'_u x'_v - x'_u z'_v; x'_u y'_v - y'_u x'_v)\\
        \begin{multlined}
            |\vf'_u \times \vf'_v| = \\
            =\sqrt{
                \begin{multlined}
                    y'^2_u z'^2_v + z'^2_u y'^2_v + z'^2_u x'^2_v + x'^2_u z'^2_v + x'^2_u y'^2_v + y'^2_u x'^2_v\\
                    -2 (y'_u z'_v z'_u y'_v + z'_u x'_v x'_u z'_v + x'_u y'_v y'_u x'_v )
                \end{multlined}
            } = \\
            =\sqrt{
                \begin{multlined}
                    (x'^2_u + y'^2_u +z'^2_u)(x'^2_v + y'^2_v + z'^2_v) - \\
                    - (x'^2_u x'^2_v + y'^2_u y'^2_v + z'^2_u z'^2_v) - \\
                    -2 (y'_u z'_v z'_u y'_v + z'_u x'_v x'_u z'_v + x'_u y'_v y'_u x'_v )
                \end{multlined}
            }  \\
            =\sqrt{EG-F^2}
        \end{multlined}
    \end{gather*}
\end{proof}
\end{document}