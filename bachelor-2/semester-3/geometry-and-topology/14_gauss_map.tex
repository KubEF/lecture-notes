% !TeX root = ./main.tex
\documentclass[main]{subfiles}
\begin{document}
\chapter{Сферическое отображение}
\begin{definition}[Сферическое отображение]
    $\Gamma: S \to S^2$, $S$~--- поверхность, $S^2 = \{x^2 + y^2 + z^2 =1 \}$.
\end{definition}
\begin{center}
    \import{figures}{gauss_map.pdf_tex}
\end{center}
\begin{example}
    \begin{itemize}
        \item Для окружности оно тождественно
        \item Для плоскости это отображение в точку
        \item Для цилиндра~--- отображение в экватор сферы
    \end{itemize}
\end{example}
\begin{theorem}[Родрига]
    Формулы
    \begin{gather*}
        \vn'_u = -k_1 \vr'_u\\
        \vn'_v = -k_2 \vr'_v
    \end{gather*}
    верны, если $u, v$~--- главные направления и $k_1, k_2$~--- главные кривизны.
\end{theorem}
\begin{definition}[Основной оператор]
    Основной оператор в точке~--- это дифференциал сферического отображения.
    \[R: \R^2 \to \R^2\text{~--- линейный оператор}\]
    Если $l$~--- направление, считаем, что $|l| = 1$, тогда $R(l) \coloneqq \vn'_l \perp \vn$.
\end{definition}
Теорема Родрига $\Leftrightarrow$ $-k_1, -k_2$~--- собственные числа $R$, $\vr'_u, \vr'_v$~--- собственные векторы $R$, если $u,v$~--- главные направления.
\begin{proof}
    Вместо поверхности рассмотрим соприкасающийся параболоид:
    \[ \begin{cases}
            x = u \\
            y = v \\
            z = f(u,v) = k_1 u^2 + k_2 v^2 + o(u^2 + v^2)
        \end{cases}\]
    В таком случае $2k_1, 2k_2$~--- главные кривизны.
    Тогда
    \begin{gather*}
        \vn|_{(0,0,0)} = (0,0,1)\\
        \vr'_u = (1, 0, 2k_1 u + o(u))\\
        \vr'_v = (0,1,2k_2 v + o(v))\\
        \intertext{для удобства откинем $o$}
        \vn = \frac{\vr'_u \times \vr'_v}{|\vr'_u \times \vr'_v|} = \frac{(-2k_1 u; -2k_2v, 1)}{\sqrt{4k_1^2 u^2 + 4k_2^2 v^2 + 1}}\\
        \intertext{обозначим $t \coloneqq 4k_1^2 u^2 + 4k_2^2 v^2 + 1$}
        \begin{multlined}
            \vn'_u|_{\substack{u = 0\\ v = 0}}
            = \left(\frac{\frac{-2k_1}{\sqrt{t}} -2k_1 u(\sqrt{t})'_u}{t};-2k_2 v \left(\frac{1}{\sqrt{t}}\right)'_u; \left(\frac{1}{\sqrt{t}}\right)'_u\right) = \\
            = (-2k_1,0,0) = -2k_1 \vr'_u|_{\substack{u = 0\\ v = 0}}
        \end{multlined}
    \end{gather*}
    $\vn'_v$ аналогично.
\end{proof}
Что делает сферическое отображение с площадью?
Глобально не очень понятно, но есть геометрический смысл локально, т.е. в предельном случае, в окрестности одной точки.
\begin{theorem}
    Пусть $S(X)$~--- отображение площади. Тогда
    \[\lim_{\diam \fancyD \to 0} \frac{S(\Gamma(\fancyD))}{S(\fancyD)} = K\]
\end{theorem}
\begin{proof}
    Пусть $\fancyD'$~--- прообраз $\fancyD$.
    Вспомним, что
    \begin{multline*}
        S(\fancyD) = \iint_{\fancyD'} \sqrt{EG-F^2} dudv = \\
        = \iint_{\fancyD'}|\vr'_u \times \vr'_v| dudv = |\vr'_u \times \vr'_v|\Big|_{M} \cdot S(\fancyD')
    \end{multline*}
    где последнее равенство следует из теоремы о среднем.
    В $\Gamma(\fancyD)$ роль радиус-вектора $\vr$ выполняет вектор $\vn$, тогда
    \begin{gather*}
        S(\Gamma(\fancyD)) = \iint_{\fancyD'} |\vn'_u \times \vn'_v|dudv =  |\vn'_u \times \vn'_v|\Big|_{\tilde{M}} \cdot S(\fancyD')\\
        \lim_{\diam \fancyD \to 0} \frac{S(\Gamma(\fancyD))}{S(\fancyD)} =
        \lim_{\diam \fancyD \to 0}  \frac{S(\fancyD')|\vn'_u \times \vn'_v| \Big|_{\tilde{M}}}{S(\fancyD')|\vr'_u \times \vr'_v|\Big|_{M} } =
        \left.\frac{|\vn'_u \times \vn'_v|}{|\vr'_u \times \vr'_v|}\right|_M = K
    \end{gather*}
    предел можно снять в силу того, что при $\diam \fancyD \to 0$, точки $M$ и $\tilde{M}$ стремятся друг к другу и мы можем считать их одной точкой.
\end{proof}
\end{document}