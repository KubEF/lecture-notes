% !TeX root = ./main.tex
\documentclass[main]{subfiles}
\begin{document}
\chapter{Деривационные формулы}
Для кривых были формулы Френе в базисе $(\vv, \vn, \vb)$.
Для поверхностей тоже есть базис: $(\vf'_u, \vf'_v, \vn)$, однако векторы $\vf'_u, \vf'_v$ не единичные,
и более того могут быть не ортогональны.
Попробуем сделать аналог формул Френе продифференцировав базисные векторы.
\begin{definition}[Деривационные формулы]
    Основная тройка деривационных формул
    \begin{gather*}
        \vf''_{uu} = \Gamma^1_{11} \vf'_u + \Gamma^2_{11} \vf'_v + L \vn \\
        \vf''_{uv} = \Gamma^1_{12} \vf'_u + \Gamma^2_{12} \vf'_v + M \vn \\
        \vf''_{vv} = \Gamma^1_{22} \vf'_u + \Gamma^2_{22} \vf'_v + N \vn
    \end{gather*}
    есть еще отдельные формулы, которые тоже можно называть деривационными
    \begin{gather*}
        \vn'_u = \alpha \vf'_u + \beta \vf_v\\
        \vn'_v = \gamma \vf'_u + \delta \vf_v
    \end{gather*}
\end{definition}
\begin{definition}[Символы Кристоффеля]
    $\Gamma^k_{ij}$ -- символы Кристоффеля.
\end{definition}
Для начала выясним, что такое $\alpha, \beta, \gamma, \delta$.
Для начала скалярно умножим каждое равенство на $\vf'_u, \vf'_v$, получим
\begin{align*}
    \vn'_u \vf'_u & = \alpha E + \beta F = -L & \vn'_v \vf'_u & = \gamma E + \delta F = -M \\
    \vn'_u \vf'_v & = \alpha F + \beta G = -M & \vn'_v \vf'_v & = \gamma F + \delta G = -N
\end{align*}
где последние равенства следуют из
\begin{gather*}
    0 = (\vf'_u, \vn)'_u = \vf''_{uu} \vn + \vf'_u \vn'_u = L + \vf'_u \vn'_u\\
    0 = (\vf'_u, \vn)'_v = M + \vf'_u \vn_v
\end{gather*}
Отсюда $\alpha, \beta$ находятся
\begin{gather*}
    \begin{cases}
        \alpha E + \beta F = -L \\
        \alpha F + \beta G = -M
    \end{cases}\\
    \begin{aligned}
        \alpha & = \frac{-LG + MF}{EG - F^2} & \beta  & = \frac{-EM + LF}{EG - F^2} \\
        \gamma & = \frac{-MG + NF}{EG - F^2} & \delta & = \frac{-EN + MF}{EG - F^2}
    \end{aligned}
\end{gather*}
\begin{theorem}
    $\Gamma^k_{ij}$ относятся к внутренней геометрии поверхности.
\end{theorem}
\begin{proof}
    Возьмем деривационные формулы и помножим их на базисные векторы:
    \begin{gather*}
        \frac{1}{2} E'_u = \vf''_{uu}\vf'_u = \Gamma^1_{11} E + \Gamma^2_{11} F\\
        F'_u - \frac{1}{2} E'_v = \vf''_{uu}\vf'_v = \Gamma^1_{11} F + \Gamma^2_{11} G
    \end{gather*}
    получили систему
    \begin{gather*}
        \begin{cases}
            E \Gamma^1_{11} + F \Gamma^2_{11} = \frac{1}{2}E'_u \\
            F \Gamma^1_{11} + G \Gamma^2_{11} = F'_u - \frac{1}{2} E'_v
        \end{cases}\\
        \Gamma^1_{11} = \frac{\begin{vmatrix}
                \frac{1}{2} E'_u       & F \\
                F'_u - \frac{1}{2}E'_v & G
            \end{vmatrix}}{EG - F^2} \qquad
        \Gamma^2_{11} = \frac{\begin{vmatrix}
                E & \frac{1}{2} E'_u       \\
                F & F'_u - \frac{1}{2}E'_v
            \end{vmatrix}}{EG - F^2}
    \end{gather*}
    $\Gamma^1_{22}, \Gamma^2_{22}$ вычисляются аналогично.

    Верно ли, что $\Gamma^1_{12}$ и $\Gamma^2_{12}$ вычисляются так же?
    \begin{gather*}
        \vf''_{uv} \vf'_u =\Gamma^1_{12} E + \Gamma^2_{12} F = \frac{1}{2} E'_v\\
        \vf''_{uv} \vf'_v =\Gamma^1_{12} F + \Gamma^2_{12} G = \frac{1}{2} G'_v
    \end{gather*}
    Ответ: да.
\end{proof}

\begin{theorem}
    \begin{gather*}
        L = - \vr'_u \vn'_u\\
        M = - \vr'_u \vn'_v = - \vr'_v \vn'_u\\
        N = - \vr'_v \vn'_v
    \end{gather*}
\end{theorem}
\begin{proof}
    $L + \vr'_u \vn'_u = \vr''_{uu} \vn + \vr'_u \vn'_u = (\vr'_u \cdot \vn)'_u = 0$ , т.к. $\vr'_u \perp \vn$.
\end{proof}
\begin{theorem}
    \[|\vn'_u \times \vn'_v| = \frac{|LN-M^2|}{\sqrt{EG-F^2}}\]
\end{theorem}
\begin{proof}
    Вспомним:
    \begin{gather*}
        \vn'_u = \alpha \vr'_u + \beta \vr'_v \\
        \vn'_v = \gamma \vr'_u + \delta \vr'_v \\
        \intertext{тогда}
        \vn'_u \times \vn'_v = (\alpha \vr'_u + \beta \vr'_v) \times (\gamma \vr'_u + \delta \vr'_v) = (\alpha\delta - \beta\gamma) \vr'_u \times \vr'_v\\
        \intertext{минус появляется, т.к. $\vr'_v \times\ \vr'_u = - \vr'_u \times \vr'_v$}
        |\vn'_u \times \vn'_v| = |\alpha\delta - \beta\gamma| \sqrt{EG-F^2} = \\
        \intertext{коэффициенты $\alpha, \beta, \gamma, \delta$ посчитаны выше}
        \begin{multlined}
            =\frac{
                \left|\splitfrac
                {(FM - GL)(FM - EN) -}
                {(FL - EM)(FN - GM)}\right|
            }{(EG-F^2)^{3/2}} =\\
            =\frac{
                \left|\splitfrac
                {F^2 M^2 - EFMN - FGLM + EGLN}
                {- F^2 LN + FGLM + EFMN - EGM^2}\right|
            }{(EG-F^2)^{3/2}} = \\
            =\frac{\left|(EG-F^2)(LN-M^2)\right|}{(EG-F^2)^{3/2}}
        \end{multlined}
    \end{gather*}
\end{proof}
\begin{corollary}
    \[\frac{|\vn'_u \times \vn'_v|}{|\vr'_u \times \vr'_v|} = \frac{LN - M^2}{EG - F^2} = K\]
\end{corollary}
\section{Уравнения Петерсона-Майнарди-Кодацци}
Вспомним:
\begin{gather*}
    \vr''_{uu} = \Gamma^1_{11} \vr'_u + \Gamma^2_{11} \vr'_v + L \vn \\
    \vr''_{uv} = \Gamma^1_{12} \vr'_u + \Gamma^2_{12} \vr'_v + M \vn \\
    \vr''_{vv} = \Gamma^1_{22} \vr'_u + \Gamma^2_{22} \vr'_v + N \vn
    \intertext{тогда}
    \vr'''_{uuv} = \left(\Gamma^1_{11}\right)'_v \vr'_u + \Gamma^1_{11} \vr''_{uv} + \left(\Gamma^2_{11}\right)'_v \vr'_v + \Gamma^2_{11} \vr''_{vv} + L'_v \vn + L \vn'_v = \vr'''_{uvu}\\
    \intertext{умножим на $\vn$}
    \vr'''_{uuv} \vn = \Gamma^1_{11}M + \Gamma^2_{11}N + L'_v = \vr'''_{uvu}\vn= \Gamma^1_{12} L + \Gamma^2_{12} N + M'_u \\
    M'_u - L'_v = \Gamma^1_{11}M + \Gamma^2_{11} N - \Gamma^1_{12}L - \Gamma^2_{12} M\\
    \intertext{аналогично}
    \Gamma^1_{22} L + \Gamma^2_{22}M + N'_u = \vr'''_{uvv}\vn = \Gamma^1_{12}M +\Gamma^2_{12} N + M'_v
\end{gather*}
\begin{theorem}[Петерсона]
    Если даны функции $E,F,G, L,M,N$, такие, что $E >0, G>0, EG - F^2 >0$, $LN - M^2$ равно результату теоремы \ref{touching_paraboloid:gauss}
    и выполнены равенства выше, то существует единственная с точностью до положения в пространстве поверхность с такими коэффициентами.
\end{theorem}
\begin{proof}
    Не будет.
\end{proof}
\end{document}