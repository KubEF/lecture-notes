% !TeX root = ./main.tex
\documentclass[main]{subfiles}
\begin{document}
\chapter{Соприкасающийся параболоид}
Сейчас идея такая: хотим заменить поверхность на параболоид, так чтобы в данной конкретной точке $\fff, \sff$ квадратичные формы и все, что с ними связано не изменились.
Попробуем её реализовать.

Пусть у нас есть поверхность и точка $M$ на ней.
Введем координаты:
\begin{itemize}
    \item точка $M = (0,0,0)$
    \item $T_M = OXY$ -- касательная плоскость к точке $M$ поверхности это $OXY$
    \item $\vn_M = (0,0,1)$
\end{itemize}
\begin{center}
    \import{figures}{touching_paraboloid.pdf_tex}
\end{center}
Тогда можем параметризовать поверхность таким образом $z = f(x,y)$, по крайней мере в окрестности точки $M$.

Рассмотрим некоторое приближение этой поверхности, для этого применим формулу Тейлора:
\begin{align*}
    z & = \vf(x,y)                                                       \\
      & = \vf(x_0,y_0) + \vf'_x(x_0,y_0)(x-x_0) + \vf'_y(x_0,y_0)(y-y_0) \\
      & + \frac{\vf''_{xx}(x_0,y_0)}{2!}(x-x_0)^2                        \\
      & + \frac{2 \vf''_{xy}(x_0,y_0)}{2!}(x- x_0)(y-y_0)                \\
      & + \frac{\vf''_{yy}(x_0,y_0)}{2!}(y-y_0)^2                        \\
      & +o((x-x_0)^2 +(y-y_0)^2)
\end{align*}
Т.к. $M = (0,0,0)$, то $\vf(x_0,y_0) = 0$.
Из того, что $T_M = OXY$, следует, что $\vf'_x(0,0) = \vf'_y(0,0)=0$, тогда формула Тейлора приобретает вид:
\[z = \frac{\vf''_{xx}(0,0)}{2}x^2 + \vf''_{xy}xy + \frac{\vf''_{yy}(0,0)}{2} y^2 + o(x^2 + y^2)\]
Это в точности уравнение параболоида.
\begin{definition}[Соприкасающийся параболоид]
    Этот параболоид называется соприкасающимся параболоидом.
    Повернем $OXY$, так чтобы  слагаемое при $xy$ исчезло, получим:
    \[z = Ax^2 + By^2\]
    Считаем, что
    \begin{align*}
        \vf''_{xx}(0,0) & = 2A & \vf''_{yy}(0,0) & =2B & \vf''_{xy}(0,0) = 0
    \end{align*}
\end{definition}
\end{document}