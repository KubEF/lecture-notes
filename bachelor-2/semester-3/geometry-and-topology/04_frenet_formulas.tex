% !TeX root = ./main.tex
\documentclass[main]{subfiles}
\begin{document}
\chapter{Реп\'ер Френ\'е}

Есть кривая и $\vf(s)$ -- ее натуральная параметризация, тогда $\vv(s)$ -- ее касательный вектор.
$|\vv(s)| = 1$.
Тогда $\vv'(s) \perp \vv(s)$ по лемме \ref{dfoc:very_important_lemma}.
\begin{definition}[Кривизна кривой]
    Определим $\vn(s)$: $\vn(s) \upuparrows \vv'(s)$, $|\vn(s)| = 1$,
    такой $\vn$ -- вектор главной нормали.
    \[k = \frac{\vv ' (s)}{\vn} \Leftrightarrow \vv' = k \vn\]
    Такая $k$ -- кривизна кривой.
    А выражение $\vv' = k \vn$ называется первой формулой Френе.
\end{definition}
\begin{remark}
    $k \ge 0$.
\end{remark}
\begin{remark}
    $\vn$ -- не везде определен, необходима бирегулярность.
\end{remark}

\begin{definition}[Бирегулярная параметризация]
    Кривая называется бирегулярной, если $\vf''(t) \not\parallel \vf'(t)$ для любой параметризации.
    Или, если $\vv'(s) \neq 0$ для натуральной параметризации.
    Или $\vn$ корректно определен.
    (почему они эквивалентны -- вопрос будущего)
\end{definition}

По умолчанию считаем, что все кривые бирегулярны.

У нас есть вектор $\vv$ и перпендикулярный ему $\vn$.
Они единичные, хотим превратить их в базис пространства.
Для этого построим вектор $\vb$ перпендикулярный им обоим и тоже единичный.
\begin{center}
    \import{figures/}{binormal_vector.pdf_tex}
\end{center}

\begin{definition}[Вектор бинормали]
    \[\vb \coloneqq \vv \times \vn\]
    Правая тройка $(\vv,\vn, \vb$) -- репер Френе.
\end{definition}

Изучим $\vb'$:
$\vb'(s) \perp \vb(s)$ из леммы \ref{dfoc:very_important_lemma},
также $\vb' \perp \vv$. Почему?
\[\vb' = (\vv \times \vn)' = \vv' \times \vn + \vv\times \vn' = 0 + \vv\times \vn' \perp \vv\]
Таким образом, $\vb' \parallel \vn$ и $\vb' = -\kappa \vn$ -- вторая формула Френе.
\begin{definition}[Кручение кривой]
    $\kappa$, определенная выше -- кручение кривой.
\end{definition}

Изучим $\vn$:
\[\vn' = (\vb \times \vv)' = \vb' \times \vv + \vb \times \vv' = -\kappa \vn \times \vv + \vb \times k\vn = \kappa\vb - k\vv\]
получили третью формулу Френе.
\begin{definition}[Формулы Френе]
    \[
        \begin{array}{cccc}
            \toprule
                 & \vv & \vn     & \vb    \\ \midrule
            \vv' & 0   & k       & 0      \\ \midrule
            \vn' & -k  & 0       & \kappa \\ \midrule
            \vb' & 0   & -\kappa & 0      \\
            \bottomrule
        \end{array}
    \]
    Производная везде берется по натуральному параметру.
\end{definition}
\begin{definition}[Нормальная плоскость кривой]
    Плоскость $(\vn, \vb)$ -- нормальная плоскость кривой.
\end{definition}
\begin{definition}[Соприкасающаяся плоскость кривой]
    Плоскость $(\vv, \vn)$ -- соприкасающаяся плоскость кривой.
\end{definition}
\begin{definition}[Спрямляющая плоскость кривой]
    Плоскость $(\vv, \vb)$ -- спрямляющая плоскость кривой.
\end{definition}

Вопрос: а как это посчитать?
\begin{example}
    Есть окружность:
    \[\begin{cases}
            x = R \cos t \\
            y = R \sin t \\
            z = 0
        \end{cases}\]
    Хотим найти натуральную параметризацию:
    сейчас мы проходим окружность за время $2\pi$, наверное нужно проходить окружность за время $2\pi R$.
    Тогда получим:
    \[\begin{cases}
            x = R\cos (t/R)  \\
            y = R \sin (t/R) \\
            z = 0
        \end{cases}
        \implies
        \begin{cases}
            x' = - \sin (t/R) \\
            y' = \cos (t/R)
        \end{cases}\]
\end{example}
\end{document}