% !TeX root = ./main.tex
\documentclass[main]{subfiles}
\begin{document}
\chapter{Геодезическая кривизна}
\begin{definition}[Вектор кривизны]
    Есть кривая и её кривизна $\vk = k \vn_1$, где $\vn_1$ -- вектор главной нормали к кривой.
\end{definition}
\begin{definition}[Разложение вектора кривизны]
    \[k_g = \proj_{TM_{x_0}} \vk\]
    проекция $\vk$ на касательную плоскость к точке $x_0$.
    \[k_n = \proj_{\vn_2} \vk, \]
    где $\vn$ -- нормаль к поверхности.
    Тогда
    \[k^2 = k^2_n + k^2_g\]
    (из теоремы Пифагора)
\end{definition}
\begin{theorem}
    $k_g$ зависит только от $E,F,G$ и их производных.
\end{theorem}
\begin{proof}
    Кривая задана внутренней натуральной параметризацией: $u = u(t), v = v(t)$.
    Поверхность задана параметризацией $\vf: \fancyD \to \R^3$.

    Подсчитаем $\vk = \vf''_{tt}$ (это равенство следует из натуральности параметризации):
    \begin{gather*}
        \vf'_t = \vf'_u u' + \vf'_v v'\\
        \vf''_{tt} = \vf''_{uu}u'^2 + 2 \vf''_{uv} u' v' + \vf''_{vv} v'^2 + \vf'_u u'' + \vf'_v v''
        \intertext{вспомним деривационные формулы}
        \begin{multlined}
            \vf''_{tt} = \Gamma^1_{11} \vf'_u u'^2 + \Gamma^2_{11} \vf'_v u'^2
            +2 \Gamma^1_{12} \vf'_u u'v' + 2 \Gamma^2_{12} \vf'_v u'v' + \\
            + \Gamma^1_{22} \vf'_u  v'^2+ \Gamma^2_{22} \vf'_v v'^2 + \vf'_u u'' + \vf'_v v''
            + \vn (L u'^2 + 2M u'v' + N v'^2)
        \end{multlined}
    \end{gather*}
    При таком разложении
    \begin{multline*}
        \vk_g = \Gamma^1_{11} \vf'_u u'^2 + \Gamma^2_{11} \vf'_v u'^2
        +2 \Gamma^1_{12} \vf'_u u'v' + 2 \Gamma^2_{12} \vf'_v u'v' + \\
        + \Gamma^1_{22} \vf'_u  v'^2+ \Gamma^2_{22} \vf'_v v'^2 + \vf'_u u'' + \vf'_v v''
    \end{multline*}

    Умножим на $\vf'_u$:
    \begin{multline*}
        \vk_g \vf'_u = \Gamma^1_{11} E u'^2 + \Gamma^2_{11} F u'^2 + 2 \Gamma^1_{12} E u'v' + 2 \Gamma^2_{12} F u'v'+\\
        + \Gamma^1_{22} E v'^2 + \Gamma^2_{22} F v'^2 + E u'' + F v''
    \end{multline*}
    $\vk_g \vf'_u$ зависит только от $\fff$
    Аналогично, $\vk_g \vf'_v$ зависит только от $\fff$.

    Заметим, что если есть вектор на плоскости,
    его проекции на $\vf'_u$ и $\vf'_v$ зависят только от $\fff$,
    то и сам вектор зависит только от $\fff$.
    Почему?
    \begin{gather*}
        \vk_g = A \vf'_u + B \vf'_v
        \intertext{домножим на $\vf'_u$ и $\vf'_v$}
        \begin{cases}
            \vk_g \vf'_u = AE + BF \\
            \vk_g \vf'_v = AF + BG
        \end{cases}\\
        A = \frac{\vk_g \vf'_u G - F \vk_g \vf'_v}{EG-F^2} \qquad  B = \frac{\vk_g\vf'_v E - F \vk_g \vf'_u}{EG-F^2}
    \end{gather*}
    при этом $\vk_g \vf'_u$ и $\vk_g \vf'_v$ зависят только от $\fff$.
\end{proof}

\begin{theorem}
    \[k_g = \left|\frac{(\phi'', \phi', \vn)}{|\phi'|^3}\right|,\]
    где $\phi(t) = \vf(u(t), v(t))$
\end{theorem}
% \begin{proof}
%     \begin{gather*}
%         % \vk = \frac{\vphi' \times \vphi''}{|\vphi'|^3}
%         % \phi'(t) = \vf'_u u' + \vf'_v v'\\
%         % \phi''(t) = \vf''_{uu}u'^2 + 2 \vf''_{uv} u' v' + \vf''_{vv} v'^2 + \vf'_u u'' + \vf'_v v''\\
%         % (\phi'', \phi', \vn) = \\
%         % = |(\vn, \vf'_u u' + \vf'_v v', \vf''_{uu}u'^2 + 2 \vf''_{uv} u' v' + \vf''_{vv} v'^2 + \vf'_u u'' + \vf'_v v'' )| =\\
%         % = |(\vn, \vf'_u u')
%         % \vk = \frac{\phi' \times \phi''}{|\phi'|^3}\\
%         % \va \perp \phi' \vn, \phi'' \in \langle \phi', \va\rangle\\
%         % \phi'' = \alpha \phi' + \beta \va\\
%         % \phi'' \times \phi' = \beta \va\times \phi'\\
%         % (\phi'' \times \phi')\vn = \beta (\va\times \phi')\vn
%     \end{gather*}

% \end{proof}
\end{document}