% !TeX root = ./main.tex
\documentclass[main]{subfiles}
\begin{document}
\chapter{Касательный вектор}
\begin{lemma}\label{dfoc:very_important_lemma}
    Пусть $\vf(t)$~--- вектор-функция, тогда
    \[|\vf(t)| = const \Leftrightarrow \vf'(t) \perp \vf(t)\]
\end{lemma}
\begin{proof}
    $\Longleftarrow$: Из $\vf'(t) \perp \vf(t)$ получаем: $ (\vf'(t), \vf(t)) = 0\ \forall t$.
    Возьмем производную скалярного квадрата и получим:
    \[(\vf(t), \vf(t))' = 2(\vf'(t), \vf(t)) = 0 = |\vf(t)|^{2'}\]
    Тогда, $|\vf(t)| = const$.

    $\Longrightarrow$: Если $|\vf(t)| = const$, то и $\vf^2(t) = const$, а дальше обратить доказательство $\Longleftarrow$.
\end{proof}
\begin{definition}[Касательный вектор]
    $\vf'(t_0)$ называется касательным вектором к кривой в точке $t_0$.
    Прямая, на которой лежит $\vf'(t_0)$~--- касательная прямая.
\end{definition}
\begin{theorem}
    Касательная прямая не зависит от параметризации, если она регулярна.
\end{theorem}
\begin{proof}
    $\phi$~--- скалярная функция, $\vf(t)$~--- вектор-функ\-ция.
    Также $\vf(\phi(t)) = \vg(t)$.
    $\vf'(t)$, $\vg'(t) = \vf'(\phi(t)) \phi'(t)$~--- касательные векторы $\vf$ и $\vg$ соответственно.
    Обозначим $\tau = \phi(t)$.
    $\vf'(\tau)$ и $\vg'(t)$ отличаются друг от друга на скаляр,
    тогда $\vf'(\tau) \parallel \vg'(t)$.
    Следовательно, при перепараметризации касательный вектор будет параллелен предыдущему, значит касательная прямая инвариантно определена.
\end{proof}
\begin{remark}
    Регулярная параметризация~--- это параметризации для которой в любой точке существует касательная прямая.
\end{remark}
\begin{definition}[Натуральная параметризация]
    Параметризация $\vf(t)$ называется натуральной, если $|\vf'(t)| \equiv 1\ \forall t$.
\end{definition}
По сути, мы идем по кривой с единичной скоростью.
Но пока не ясно существует и единственна ли натуральная параметризация.
\begin{proof}
    Проверить единственность достаточно просто: $\vg(t) = \vf(\phi(t))$, $\phi(t) = \tau$
    \[|\vg'(t)| = |\vf'(\tau)| |\phi'(t)| \implies |\phi'(t)| = 1\]
    тогда $\phi = t + t_0$ (с точностью до выбора начального момента времени).
\end{proof}

\begin{theorem}
    Натуральная параметризация существует.
\end{theorem}
\begin{proof}
    Вспомним про длину кривой.
    Глобальная идея: параметризация говорит сколько мы проходим по кривой за данное время;
    чтобы перейти к натуральной параметризации мы откажемся от стандартного времени,
    и скажем, что новое время это тот участок кривой, за которое мы его проходим, или
    единичное расстояние мы проходим за единичное время, значит параметр времени~--- это участок дуги.

    Реализуем эту идею:
    \[s = \int_{t_0}^t |\vf'(\tau)| d\tau\]
    $s$~--- искомый натуральный параметр.
    Будем считать $t - t_0$ временем.
    Обозначим $s = \phi(t)$:
    \[\phi(t) = \int_{t_0}^t |\vf'(\tau)| d\tau\]
    Заметим, что $\phi(t)$ возрастает и непрерывна.
    Значит существует $t = \phi^{-1}(s)  = \psi(s)$.
    Тогда $\vf(t) = \vf(\psi(s))$ должна быть натуральной параметризацией.

    Теперь докажем, что $\vf(\psi(s))$ есть натуральная параметризация.
    Хотим убедиться, что
    \[\left| \frac{d \vf(\psi(s))}{ds}\right| = 1.\]
    Для этого
    \begin{gather*}
        \psi'(s) = \frac{1}{\phi'(t(s))} = \frac{1}{|\vf'(t)|}\\
        \frac{d}{ds}\vf(\psi(s)) = \vf'(\psi(s)) \psi'(s) = \frac{\vf'(t)}{|\vf'(t)|}\\
        \left| \frac{\vf'(t)}{|\vf'(t)|} \right| = 1
    \end{gather*}
\end{proof}
\end{document}