% !TeX root = ./main.tex
\documentclass[main]{subfiles}
\begin{document}
\chapter{Геодезические линии}
\begin{theorem}
    Следующие условия равносильны:
    \begin{enumerate}
        \item $k_g = 0$
        \item спрямляющая плоскость кривой в точности касательная плоскость поверхности
        \item соприкасающаяся плоскость содержит $\vn$
        \item $k = k_n$ в данном направлении
        \item $k$~--- минимальная среди всех кривизн в данном направлении
    \end{enumerate}
\end{theorem}
\begin{proof}
    Из равенства
    \[\vk = \vk_n + \vk_g\]
    следует равносильность (1) и (4).

    А из
    \[k^2 = k^2_n + k^2_g\]
    где $k^2_n$ зависит от поверхности, следует равносильность (1) и (5).

    Спрямляющая плоскость~--- плоскость на касательном векторе и векторе бинормали,
    т.е. плоскость перпендикулярная вектору главной нормали.
    (2) означает, что вектор главной нормали к кривой совпадает с вектором нормали к поверхности, а это означает (1).
    % TODO: Было ли доказательство равносильности с (3)?
\end{proof}
\begin{definition}[Геодезическая кривая]
    Если эти условия выполнены, то кривая называется геодезической.
    Уравнение геодезической линии: $(\phi'', \phi', \vn) = 0$.
\end{definition}

\begin{theorem}
    В любой точке в любом направлении проходит ровно одна геодезическая линия.
    (в локальном смысле, т.е. в окрестности точки, насколько она продолжима, мы не знаем)
\end{theorem}
\begin{proof}
    Зададим эту геодезическую координатами $u = t, v = v(t)$,
    тогда $u'=1, v' = v'(t), u''=0$.

    Обозначим
    \[
        A(v') = \Gamma^1_{11} \vf'_u + \Gamma^2_{11} \vf'_v
        +2 \Gamma^1_{12} \vf'_u v' + 2 \Gamma^2_{12} \vf'_v v'
        + \Gamma^1_{22} \vf'_u  v'^2+ \Gamma^2_{22} \vf'_v v'^2
    \]
    т.к. $u' = 1$.

    Теперь уравнение геодезической можно записать в виде:
    \begin{gather*}
        (A(v') + \vf'_v v''; \vf'_u + \vf'_v v'; \vn) = 0\\
        (A(v'); \vf'_u + \vf'_v v'; \vn) + v''(\vf'_v; \vf'_u+ \vf'_v v';\vn) = 0\\
        v'' = - \frac{(A(v'); \vf'_u + \vf'_v v'; \vn)}{(\vf'_v; \vf'_u; \vn)} \\
        (\vf'_v, \vf'_u, \vn) \neq 0
    \end{gather*}
    Дифференциальное уравнение $v'' = F(t,v,v')$ имеет решения, если $F$ непрерывна.
    Решение единственно, если $\frac{\partial F}{\partial v}, \frac{\partial F}{\partial v'}$ существуют, у нас именно так.
\end{proof}

\section{Полугеодезическая система координат}
\begin{definition}[Полугеодезическая система координат]
    Система координат называется полугеодезической, если одно семейство координатных линий~--- геодезические,
    а второе им перпендикулярно.
    \begin{center}
        \import{figures/}{semi_geodesic_coordinates.pdf_tex}
    \end{center}
    В первом семействе все линии в натуральной параметризации.
    Это означает, что $E = 1, F = 0$.
\end{definition}
\begin{theorem}
    Полугеодезическая параметризация существует (в окрестности любой точки).
\end{theorem}
\begin{proof}
    Рассмотрим некоторую кривую на поверхности,
    заданную координатами $\vphi(t)$,
    где $\vphi: [a,b] \to \R^3$.
    \begin{center}
        \import{figures}{semi_geodesic_parametrization.pdf_tex}
    \end{center}
    Возьмем каждую точку и выберем направление перпендикулярное $\vphi'(t)$.
    В этом перпендикулярном направлении проведем геодезическую
    $\psi_t(s)$  в точке $\vphi(t)$, где $\psi_t(s)$  в натуральной параметризации.
    Тогда участок поверхности в окрестности какой-то точки этой кривой характеризуется уравнением $\vr(s,t) \coloneqq \psi_t(s)$.
    Это в точности нужная нам параметризация. Почему?

    Поймем что такое $\vr'_s$.
    $|\vr'_s(s,t)| = 1$, т.к. $|(\psi_t)'_s(s)| = 1$ это означает, что $|\vr'_s(s,t)|^2 = E = 1$.

    А почему второе семейство ортогонально?
    Знаем: $F(0,t) = 0$
    (на линии $\vphi(t)$ считаем $s = 0$).
    Хотим доказать, что $F$ в любой точке окрестности равен 0.
    Для этого достаточно проверить, что $F'_s = 0$.
    \begin{gather*}
        F = \vr'_s \vr'_t \\
        F'_s = (\vr'_s \vr'_t)'_s = \vr''_{ss} \vr'_t + \vr'_s \vr''_{st}
    \end{gather*}
    При этом $\vr''_{ss} \perp \vr'_s$, т.к. $|\vr'_s| = 1$;
    $\vr''_{st} \perp \vr'_s$ по той же причине.
    Почему $\vr''_{ss} \vr'_t = 0$?
    Линии геодезические, значит $(\vr'_s; \vr''_{ss}, \vn) = 0$.
    \begin{gather*}
        \vr'_s \times \vr''_{ss} \perp \vn\\
        \vr'_s \perp \vr''_{ss} \implies \vn = \alpha \vr'_s + \beta \vr''_{ss}
    \end{gather*}
    вектор нормали лежит в плоскости $\langle\vr'_s, \vr''_{ss}\rangle$
    \begin{gather*}
        \vr''_{ss} = \frac{1}{\beta} (\vn - \alpha \vr'_s)\\
        (\beta \neq 0, \text{ т.к. }\vn \not\parallel \vr'_s)\\
        \vn \parallel \vr''_{ss}, \alpha = 0\\
        \vr''_{ss} = \frac{\vn}{\beta}\\
        \intertext{подставим это в $\vr''_{ss}\vr'_t=0$}
        \frac{1}{\beta} \vn \vr'_t = 0
    \end{gather*}
\end{proof}
\begin{theorem}
    Геодезические линии локально кратчайшие.
\end{theorem}
\begin{proof}
    Пусть у нас есть $g(s)$~--- геодезическая линия (в натуральной параметризации).
    Пусть эта линия $t=0$  в некоторой полугеодезической параметризации.
    Допустим, что есть кривая $s = 1, t = t(\tau)$,
    длина которой короче, чем длина геодезической.
    \begin{gather*}
        l = \int_a^b \sqrt{E s'^2 + 2 F s' t' + G t'^2} d \tau
        \intertext{но у нас $E=1, F=0$}
        l = \int_a^b \sqrt{s'^2 + G t'^2} d \tau = \int_0^{s_0}\sqrt{1 + G t'^2} ds
        \intertext{В то же время длина геодезической}
        \int_0^{s_0}1ds
    \end{gather*}
    Получили противоречие.
\end{proof}
\end{document}