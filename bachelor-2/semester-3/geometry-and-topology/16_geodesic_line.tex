% !TeX root = ./main.tex
\documentclass[main]{subfiles}
\begin{document}
\chapter{Геодезические линии}
\begin{theorem}
    Следующие условия равносильны:
    \begin{enumerate}
        \item $k_g = 0$
        \item спрямляющая плоскость кривой в точности касательная плоскость поверхности
        \item соприкасающаяся плоскость содержит $\vn$
        \item $k = k_n$ в данном направлении
        \item $k$ -- минимальная средняя всех кривизн в данном направлении
    \end{enumerate}
\end{theorem}
\begin{proof}
    Из равенства
    \[\vk = \vk_n + \vk_g\]
    следует равносильность (1) и (4).

    А из
    \[k^2 = k^2_n + k^2_g\]
    где $k^2_n$ зависит от поверхности, следует равносильность (1) и (5).

    Спрямляющая плоскость -- плоскость на касательном векторе и векторе бинормали,
    т.е. плоскость перпендикулярная вектору главной нормали.
    (2) означает, что вектор главной нормали к кривой совпадает с вектором нормали к поверхности, а это означает (1).
    % TODO: Было ли доказательство равносильности с (3)?
\end{proof}
\begin{definition}[Геодезическая кривая]
    Если эти условия выполнены, то кривая называется геодезической.
    Уравнение геодезической линии: $(\phi'', \phi', \vn) = 0$.
\end{definition}

\begin{theorem}
    В любой точке в любом направлении проходит ровно одна геодезическая линия.
    (в локальном смысле, т.е. в окрестности точки, насколько она продолжима, мы не знаем)
\end{theorem}
\begin{proof}
    Зададим эту геодезическую координатами $u = t, v = v(t)$,
    тогда $u'=1, v' = v'(t), u''=0$.

    Обозначим
    \[
        A(v') = \Gamma^1_{11} \vf'_u + \Gamma^2_{11} \vf'_v
        +2 \Gamma^1_{12} \vf'_u v' + 2 \Gamma^2_{12} \vf'_v v'
        + \Gamma^1_{22} \vf'_u  v'^2+ \Gamma^2_{22} \vf'_v v'^2
    \]
    т.к. $u' = 1$.

    Теперь уравнение геодезической можно записать в виде:
    \begin{gather*}
        (A(v') + \vf'_v v''; \vf'_u + \vf'_v v'; \vn) = 0\\
        (A(v'); \vf'_u + \vf'_v v'; \vn) + v''(\vf'_v; \vf'_u+ \vf'_v v';\vn) = 0\\
        v'' = - \frac{(A(v'); \vf'_u + \vf'_v v'; \vn)}{(\vf'_v; \vf'_u; \vn)}
        (\vf'_v, \vf'_u, \vn) \neq 0
    \end{gather*}
    Дифференциальное уравнение $v'' = F(t,v,v')$ имеет решения, если $F$ непрерывна.
    Решение единственно, если $\frac{\partial F}{\partial v}, \frac{\partial F}{\partial v'}$ существуют, у нас именно так.
\end{proof}

\end{document}