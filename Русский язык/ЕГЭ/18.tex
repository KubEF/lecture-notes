% !TeX root = ./main.tex
\documentclass[main]{subfile}
\begin{document}
\section{Знаки препинания при словах и конструкциях, не связанных с членами предложения.}
\textbf{Обращения:}

\begin{enumerate}
      \item Обращение вместе с относящимися к нему словами \textbf{выделяется запятыми}. \newline
            \case{Шуми, шуми с крутой вершины, не умолкай, \textbf{поток седой}!}
      \item Если обращение, стоящее в начале предложения, произносится с восклицательной интонацией, то после него \textbf{ставится восклицательный знак}, а следующее за обращением слово пишется с заглавной буквы. \newline
            \case{\textbf{Колокольчики мои, цветики степные!} Что глядите на меня, темно-голубые?}
      \item Если обращение является распространенным и его части отделены другими частями предложения, то \textbf{каждая часть обращения выделяется запятыми}. \newline
            \case{Крепче, \textbf{конское}, бей, \textbf{копыто}, отчеканивая шаг.}
      \item Частица \textib{о}, стоящая перед обращением, никаким знаком от него \textbf{не отделяется}. \newline
            \case{\textbf{Казбек, о страж востока}, принес я, странник, свой поклон.}
            \begin{itemize}
                  \item[!] Если \textib{о} выступает в роли междометия (в значении <<ах>>), то оно \textbf{в соответствии с правилами обособляется}. \newline
                        \case{\textbf{О, моя утраченная свежесть}, буйство глаз и половодье чувств.}
            \end{itemize}
      \item Личные местоимения \textib{ты, вы}, как правило, не входят в состав распространенного обращения и только в некоторых случаях могут выступать в роли обращения. \newline
            \case{Что ж ты, \textbf{милая}, вся, как лист дрожишь? // \newline
                  Эй, \textbf{вы}, сходитесь! // \newline
                  Уж \textbf{вы девицы, вы затейницы, вы головушки неразумные}, что вы шутите над невестою?}
\end{enumerate}

\textbf{Вводные слова:}

Группы вводных слов:

\noindent\begin{longtabu}{X[l] X[l]}
      \toprule
      \endfirsthead
      \midrule
      \endhead
      \endfoot
      \bottomrule
      \endlastfoot

      Выражающие чувства говорящего и его отношение к высказываемому                                                 &
      \case{К сожалению, к несчастью, к счастью и т. д.}                                                                         \\
      \midrule
      Выражающие оценку реальности сообщаемого                                                                       &
      \case{Конечно, возможно, вероятно, очевидно, несомненно, наверно и т. п.}                                                  \\
      \midrule
      Указывающие на источник высказывания                                                                           &
      \case{Говорят, по-моему, по словам и др.}                                                                                  \\
      \midrule
      Указывающие на последовательность высказывания, на связь мыслей                                                &
      \case{Во-первых, во-вторых, итак, наконец, следовательно, главное, кстати, между прочим, впрочем, стало быть, в частности} \\
      \midrule
      Употребляемые с целю привлечения внимания собеседника, чтобы внушить ему определенное отношение к сообщаемому. &
      \case{Видишь ли, простите, допустим, предположим и т. д.}                                                                  \\
      \midrule
      Указывающие на приемы и способы оформления мыслей                                                              &
      \case{Словом, одним словом, иначе говоря, вообще, вернее, точнее, как говорится, так сказать и т. п.}                      \\
      \midrule
      Указывающие меру того, о чем говорится                                                                         &
      \case{По крайней мере, самое большое и т. п.}                                                                              \\
      \midrule
      Выражающие экспрессивность                                                                                     &
      \case{Честно говоря, смешно сказать, не в обиду будь сказано, по правде и т. д.}                                           \\
      \midrule
      Показывающие степень обычности того, о~чем говорится                                                           &
      \case{Бывало, по обычаю, по обыкновению и т. п.}                                                                           \\
\end{longtabu}


\begin{enumerate}
      \item Вводные слова могут находится в начале, середине и конце предложения. На письме вводные слова \textbf{выделяются запятыми}. \newline
            \case{Мать, \textbf{наверное}, приехала усталой, раздраженной.}
      \item При наличии двух вводных слов \textbf{между ними ставится запятая}. \newline
            \case{\textbf{Итак, к нашему удивлению}, горы оказались всего в двух километрах от нас.}
      \item От предшествующих сочинительных союзов вводные слова обычно \textbf{отделяются запятой}. Вводное слово при этом можно переставить в другое место предложения. \newline
            \case{Но, \textbf{быть может}, читателю уже наскучило сидеть со мной у однодворца Овсянникова. \newline
                  (Но читателю, \textbf{быть может}, уже наскучило\dots // \newline
                  Но читателю уже наскучило, \textbf{быть может}\dots)}

            \begin{itemize}
                  \item[!] В вводных словах типа \textib{а значит, а впрочем} и т. п., где \textib{а} --- компонент вводного слова, перестановка вводного слова невозможна, и \textbf{запятая после \textit{а} не ставится}. \newline
                        \case{\textbf{А значит}, он ни в чем не виноват.}
                  \item[!] \textbf{Запятая не ставится} перед вводными словами после присоединительного союза. \newline
                        \case{\textbf{И кроме того}, предстояло закупить продукты и снаряжение для двухлетней экспедиции.}
            \end{itemize}

      \item Вводное слово, стоящее в начале или конце обособленного оборота, обычно \textbf{не отделяется} от него никакими знаками. \newline
            \case{Он достал лист бумаги и ручку, \textbf{должно быть} специально приготовленные заранее.}
      \item Не являются вводными и, следовательно, \textbf{не выделяются запятыми} слова и сочетания слов: \textib{авось, небось, вдобавок, вряд ли} и др. \newline
            \case{\textbf{Авось} чего-нибудь подыщу.}
\end{enumerate}

Слова \textib{наконец, значит, однако} могут функционировать по-разному:

\begin{enumerate}
      \item Слово \textib{наконец} является вводным, если обозначает порядок мыслей (=\textib{еще}) или дает оценку факта, показывает на отношение говорящего к событиям.
            В значении «после всего», «напоследок», «в результате всего» оно не является вводным и \textbf{не выделяется запятой}. \newline
            \case{Михей поднимает, \textbf{наконец}, от лаптей свою старую седую голову. // \newline
                  \textbf{Наконец} мы встали и опять пошли бродить до вечера.}
      \item Слово \textib{значит} вводное, если оно =\textib{следовательно, стало быть}; если оно =\textib{означает}, то оно является членом предложения и \textbf{запятой не выделяется}. \newline
            \case{Если мы сможем найти там машину, \textbf{значит}, прибудем к месту встречи вовремя. // \newline
                  Любовь для Катерины \textbf{значит} много больше, чем для Варвары.}
      \item Слово \textib{однако} является вводным, если стоит в начале предложения, в начале предложения слово \textib{однако} имеет значение противительного союза. \newline
            \case{Давыдов, \textbf{однако}, уверен в необходимости поделить все по справедливости. // \newline
                  \textbf{однако} Давыдов уверен в необходимости поделить все по справедливости.}
\end{enumerate}

\textbf{Вставные конструкции:}

\textbf{Вставные конструкции} --- это словосочетания и целые предложения, которые содержат различного рода добавочные, попутные замечания, уточнения, поправки и т. д.
Вставные конструкции грамматически не связаны с предложением и расположены обычно в его середине или конце.

Вставные конструкции выделяются \textbf{скобками} или \textbf{тире}.

По структуре вставная конструкция может быть (сложным) предложением, а также состоять из нескольких предложений.
Пунктуационно эти конструкции оформляются в соответствии с действующими правилами; в конце могут стоять точка, восклицательный или вопросительный знак, многоточие.
Если конструкция стоит в середине предложения, то после нее основной предложение пишется со строчной буквы.

\case{А Каракатица \textbf{(так по-настоящему называют госпожу К.)} не спеша --- \textbf{ведь она была уверена, что добыча от нее не уйдет!} --- бесшумно подплывала все ближе. // \newline
      В черных сучьях дерев обнажённых желтый зимний закат за окном. \textbf{(К эшафоту на казнь осужденных поведут на закате таком.)}}

\end{document}