% !TeX root = ./main.tex
\documentclass[main]{subfile}
\begin{document}
\section{Знаки препинания в сложноподчиненном предложении.}

\begin{enumerate}
    \item Придаточное предложение \textbf{отделяется} от главного \textbf{запятой}. \newline
          \case{Хороши летние туманные дни, \textbf{хотя охотники их и не любят}.}
    \item Придаточное предложение, стоящее внутри главного, \textbf{выделяется запятыми с обеих сторон}. \newline
          \case{В пору цветения, \textbf{если тронуть сосну}, она окутывается золотистым душистым облаком пыльцы.}
    \item Если придаточное присоединяется к главному при помощи сложного союза \textib{(в силу того что, в то время как, для того чтобы, оттого что, потому что и т. д.)}, то в зависимости от смысла и интонации запятая ставится или перед союзом, или перед второй его частью \textib{(для того, чтобы; потому, что и т. д.)}. \newline
          \case{Мы стали делать зарубки на деревьях,\textib{для того чтобы} не заблудиться в лесу. // \newline
              Мы стали делать зарубки на деревьях \textib{для того, чтобы} не заблудиться в лесу.}
    \item \textbf{Двоеточие ставится} перед придаточным при наличии в главном слов, предупреждающих о дальнейшем разъяснении (можно подставить \textib{а именно}). \newline
          \case{Мне хотелось одного: \textib{(а именно) чтобы он пришел вовремя}.}
    \item При интонационном подчеркивании придаточных (изъяснительных, обстоятельственных), стоящих перед главным, вместо запятой \textbf{может ставиться тире}. \newline
          \case{Пахарь ли песню вдали запоет – долгая песня за сердце берет\dots}
\end{enumerate}

\textbf{Исключения:}
\begin{enumerate}
    \item Не являются придаточными и поэтому \textbf{не выделяются запятыми} устойчивые сочетания: \textib{во что бы то ни стало; что есть мочи; кто во что горазд} и т. п. \newline
          \case{Теперь мне оставалось \textbf{во что бы то ни стало} доказать свою правоту.}
    \item \textbf{Не отделяются запятой} придаточные, если перед подчинительным союзом стоит отрицание \textib{не} или повторяющиеся сочинительные союзы \textib{и, или, либо} и т. п. \newline
          \case{Я спал и не слышал \textbf{ни} как пришел отец, \textbf{ни} как он ушел.}
\end{enumerate}

\end{document}