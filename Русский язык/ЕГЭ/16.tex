% !TeX root = ./main.tex
\documentclass[main]{subfiles}
\begin{document}
\begin{center}
    {\large \textit{\textbf{Задание №16}}}

    \textit{\textbf{Пунктуация в сложносочиненном предложении и в предложении с~однородными членами.}}
\end{center}
\begin{longtabu}[c]{X[l]X[l]}
    \toprule
    \centering {\large Запятая ставится} &
    \centering {\large Запятая НЕ ставится} \\
    \midrule
    \endfirsthead
    \midrule
    \centering {\large Запятая ставится} &
    \centering {\large Запятая НЕ ставится} \\
    \midrule
    \endhead
    \endfoot
    \bottomrule
    \endlastfoot

    \multicolumn{2}{c}{\large \textbf{Определения}} \\
    \midrule
    \multicolumn{1}{c}{\textbf{Однородные}} &
    \multicolumn{1}{c}{\textbf{Неоднородные}} \\
    \midrule
    Обозначают отличительные признаки разных предметов. \newline
    \case{Пионы белые, розоватые, розовые, темно-вишневые.} &
    Характеризуют предмет с~разных сторон. \newline
    \case{Большая железная печь.} \\
    \midrule
    Обозначат признаки одного предмета, с~одной стороны. \newline
    \case{Крупный, короткий, благодатный дождь.} &
    Выражены сочетанием качественного и относительного прилагательных. \newline
    \case{Кожаные большие сапоги.} \\
    \midrule
    Являются контекстными синонимами или эпитетами. \newline
    \case{Бледно-голубые, стеклянные глаза.} &
    \multirow[t]{7}{=}{Имеют предшествующее определение, которое относится не непосредственно к~определяемому существительному, а к~сочетанию последующего определения и определяемого существительного. \newline
        \case{Профессор открыл свой новенький кожаный саквояж.}}\\
    \cmidrule{1-1}
    Образуют смысловую градацию. \newline
    \case{Темная, мрачная, душная комната.} & \\
    \cmidrule{1-1}
    Выражены причастным оборотом, стоящим за одиночным определением. \newline
    \case{Пожилая, гладко причесанная на прямой пробор женщина.} & \\
    \cmidrule{1-1}
    Стоят после определяемого имени существительного. \newline
    \case{Лицо Егора, бледное, одутловатое, […]} & \\
    \cmidrule{1-1}
    Противопоставлены другим определениям при одном существительном. \newline
    \case{Цветы яркие, красочные, но неестественно крупные и благоухающие.} & \\
    \midrule


    \multicolumn{1}{c}{\textbf{Однородные}} &
    \multicolumn{1}{c}{\textbf{Неоднородные}} \\
    \midrule
    Характеризуют предмет с~одной стороны, обозначают его близкие признаки. \newline
    \case{Доктор педагогических наук, профессор Ступин.} &
    Характеризуют предмет с~разных сторон, обозначают его разные признаки. \newline
    \case{Командир дивизии генерал-майор Панфилов.} \\
    \midrule

    \multicolumn{2}{c}{\large \textbf{Однородные члены с~неповторяющимися союзами}} \\
    \midrule
    С~союзами \textib{а, но, да}~(в значении \textib{но})\textib{, однако, зато, тем не менее, хотя}~(с~уступительными значением). \newline
    \case{Резв, но мил.} &
    \multirow[t]{3}{=}{С~союзами \textib{и, да}~(в значении \textib{и})\textib{, или, либо}. \newline
        \case{Читать да писать.}} \\
    \cmidrule{1-1}
    Между двумя однородными сказуемыми, соединенными одиночным \textib{и}, СТАВИТЬСЯ ТИРЕ для указания следствия во втором сказуемом или резкой смены. \newline
    \case{Хотел объехать целый свет – и не объехал сотой доли.} & \\
    \midrule

    \multicolumn{2}{c}{\large \textbf{Однородные члены с~повторяющимися союзами}} \\
    \midrule
    С~союзами \textib{и…~и, да…~да, ни…~ни, или…~или, ли…~ли, то…~то, не то…~не то, либо…~либо} и др. \newline
    \case{Разлюбил и брань, и саблю, и свинец.} &
    Если образуется смысловое единство между двумя однородными членами. \newline
    \case{И день и ночь, и день и ночь!} \\
    \midrule
    >3 слов и союз не стоит перед первым. \newline
    \case{Заставляли бледнеть мужчин, и женщин, и детей.} &
    Члены соединяются попарно, запятая ставится между парными группами. \newline
    \case{Пьяные и трезвые, робкие и отчаянные.} \\
    \midrule
    \multirow[t]{5}{=}{С~союзами \textib{как…~так и, не только…~но и, не столько…~сколько}, запятая ставится только перед второй частью. \newline
        \case{Не только для писателей, но и для людей.}} &
    В фразеологических выражениях из двух слов с~повторением и или ни. \newline
    \case{Ни то ни се.} \\
    \cmidrule{2-2}
    & В союзах \textib{не то что…~а, не то чтобы…~а~(но)}, запятая перед \textib{что} и \textib{чтобы} не ставится.\\
    \midrule

    \multicolumn{2}{c}{\large \textbf{Сложносочиненные предложения}} \\
    \midrule
    Если части соединены соединительными, противительными или разделительными союзами. \newline
    \case{Зацвела черемуха, и весь город тащил себе ветки с~белыми цветами.} &
    Если есть общий для обеих частей второстепенный член. \newline
    \case{\textbf{Во время частых зимних штормов} в порту ревели басами океанские пароходы и скрипело от ветра окно.} \\
    \midrule
    Если безличные предложения в составе ССП неоднородны по своему составу. \newline
    \case{В комнате было душно, и мне захотелось выйти на свежий воздух.} &
    \multirow[t]{4}{=}{Если обе части восклицательные или вопросительные предложения, объединенные общей интонацией. \newline
        \case{Зачем ты послан был и кто тебя послал?}} \\
    \cmidrule{1-1}
    >2 номинативных предложений. \newline
    \case{Ночь, тишина, и звезды, звезды.} & \\
    \midrule
    \multicolumn{2}{p{\linewidth}}{\textib{Точка с~запятой} ставится между частями ССП, которые значительно распространены и имеют знаки препинания внутри.\newline
        \case{Обыкновенно Вернер исподтишка насмехался над своими больными; но раз я видел, как он плакал над умирающим солдатом.}} \\
    \midrule
    \multicolumn{2}{p{\linewidth}}{\textib{Тире} в ССП ставится между его частями, которые содержат неожиданное присоединение или резкое противопоставление.\newline
        \case{Я сажусь на трамвай – и через 20 минут я опять в поле.}}
\end{longtabu}
\end{document}