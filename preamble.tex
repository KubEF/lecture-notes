%%% PDF settings
\pdfvariable minorversion 7 % Set PDF version to 1.7.

%%% Fonts and language setup.
\usepackage{polyglossia}
% Setup fonts.
\usepackage{fontspec}
\setmainfont{CMU Serif}

\usepackage{microtype} % Add fancy-schmancy font tricks.
\usepackage{multicol}

\usepackage{xcolor} % Add colors support.

%% Math
\usepackage{amsmath, amsfonts, amssymb, amsthm, mathtools} % Advanced math tools.
\usepackage{unicode-math} % Allow TTF and OTF fonts in math and allow direct typing unicode math characters.
\unimathsetup{
    warnings-off={
            mathtools-colon,
            mathtools-overbracket
        }
}
\setmathfont{Latin Modern Math} % default
\setmathfont[range={\setminus,\varnothing,\smashtimes}]{Asana Math}

%%% Images
\usepackage{graphicx}
\graphicspath{{images/}}
\usepackage{wrapfig} % Floating images.
\usepackage{import}

%%% Bibtex
% \usepackage[style=gost-inline,
%     language=auto]{biblatex}

%%% Polyglossia setup after (nearly) everything as described in documentation.
\setdefaultlanguage{russian}
\setotherlanguage{english}


%%% Custom commands
\newcommand{\R}{\mathbb{R}}
\newcommand{\N}{\mathbb{N}}
\newcommand{\Z}{\mathbb{Z}}
\newcommand{\Q}{\mathbb{Q}}
\newcommand{\C}{\mathbb{C}}
\newcommand{\id}{\mathrm{id}}
\newcommand{\zv}{\mathbf{0}}
\renewcommand{\le}{\leqslant}
\renewcommand{\ge}{\geqslant}
\renewcommand*{\&}{%
    \relax
    \ifmmode
        \mathbin{\char`\&}%
    \else
        \char`\&\relax
    \fi
}
\newcommand{\va}{\mathbf{a}}
\newcommand{\vb}{\mathbf{b}}
\newcommand{\vc}{\mathbf{c}}
\newcommand{\ve}{\mathbf{e}}
\newcommand{\vf}{\mathbf{f}}
\newcommand{\vi}{\mathbf{i}}
\newcommand{\vj}{\mathbf{j}}
\newcommand{\vk}{\mathbf{k}}
\newcommand{\vn}{\mathbf{n}}
\newcommand{\vv}{\mathbf{v}}
\newcommand{\vw}{\mathbf{w}}
\newcommand{\vu}{\mathbf{u}}
\newcommand{\proj}{\operatorname{Пр}}
\newcommand{\dist}{\operatorname{dist}}
\newcommand{\contradiction}{\smashtimes}
\AtBeginDocument{\renewcommand{\Re}{\operatorname{Re}}}
\AtBeginDocument{\renewcommand{\Im}{\operatorname{Im}}}
\AtBeginDocument{\renewcommand{\phi}{\varphi}}
\AtBeginDocument{\renewcommand{\epsilon}{\varepsilon}}

%%% amsthm envs
\theoremstyle{plain}
\newtheorem{assertion}{Утверждение}[chapter]
\newtheorem{theorem}{Теорема}[chapter]
\newtheorem{lemma}{Лемма}[chapter]
\newtheorem{corollary}{Следствие}[chapter]
\newtheorem{proposition}{Предложение}[chapter]
\theoremstyle{definition}
\newtheorem{definition}{Определение}[chapter]
\newtheorem{term}{Термин}[chapter]
\newtheorem{designation}{Обозначение}[chapter]
\newtheorem{prop}{Свойство}[chapter]
\newtheorem{axiom}{Аксиома}[chapter]
\theoremstyle{remark}
\newtheorem*{example}{Пример}
\newtheorem*{remark}{Замечание}
\newtheorem*{addition}{Дополнение}
\renewcommand{\qedsymbol}{\blacksquare}
%%% Memoir settings
\chapterstyle{veelo}
\setlength{\headheight}{2\baselineskip}

%%% HyperRef
\usepackage{hyperref}