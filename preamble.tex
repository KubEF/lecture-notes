%%% PDF settings
\pdfvariable minorversion 7 % Set PDF version to 1.7.

%%% Fonts and language setup.
\usepackage{polyglossia}
% Setup fonts.
\usepackage{fontspec}
\setmainfont{CMU Serif}
\setsansfont{CMU Sans Serif}

\usepackage{microtype} % Add fancy-schmancy font tricks.

\usepackage{xcolor} % Add colors support.

%% Math
\usepackage{amsmath, amsfonts, amssymb, amsthm, mathtools} % Advanced math tools.
\usepackage{thmtools}
\usepackage{unicode-math} % Allow TTF and OTF fonts in math and allow direct typing unicode math characters.
\unimathsetup{
    warnings-off={
            mathtools-colon,
            mathtools-overbracket
        }
}
\setmathfont{Latin Modern Math} % default
\setmathfont[range={\setminus,\varnothing,\smashtimes}]{Asana Math}

%%% Images
\usepackage{graphicx}
\graphicspath{{figures/}}
\usepackage{import}

%%% Polyglossia setup after (nearly) everything as described in documentation.
\setdefaultlanguage{russian}
\setotherlanguage{english}


%%% Custom commands
\newcommand{\R}{\mathbb{R}}
\newcommand{\N}{\mathbb{N}}
\newcommand{\Z}{\mathbb{Z}}
\newcommand{\Q}{\mathbb{Q}}
\newcommand{\C}{\mathbb{C}}
\newcommand{\id}{\mathrm{id}}
\renewcommand{\le}{\leqslant}
\renewcommand{\leq}{\leqslant}
\renewcommand{\ge}{\geqslant}
\renewcommand{\geq}{\geqslant}
\AtBeginDocument{\renewcommand{\Re}{\operatorname{Re}}}
\AtBeginDocument{\renewcommand{\Im}{\operatorname{Im}}}
\AtBeginDocument{\renewcommand{\phi}{\varphi}}
\AtBeginDocument{\renewcommand{\epsilon}{\varepsilon}}

%%% theorem-like envs
\theoremstyle{definition}

\declaretheoremstyle[spaceabove=0.5\topsep,
    spacebelow=0.5\topsep,
    headfont=\bfseries\sffamily,
    bodyfont=\normalfont,
    headpunct=.,
    postheadspace=5pt plus 1pt minus 1pt]{myStyle}
\declaretheoremstyle[spacebelow=\topsep,
    headfont=\bfseries\sffamily,
    bodyfont=\normalfont,
    headpunct=.,
    postheadspace=5pt plus 1pt minus 1pt,]{myStyleWithFrame}
\declaretheoremstyle[spacebelow=\topsep,
    headfont=\bfseries\sffamily,
    bodyfont=\normalfont,
    headpunct=.,
    postheadspace=5pt plus 1pt minus 1pt,
    qed=\blacksquare]{myProofStyleWithFrame}

\usepackage[breakable]{tcolorbox}
\tcbset{sharp corners=all, colback=white}
\tcolorboxenvironment{theorem}{}
\tcolorboxenvironment{theorem*}{}
\tcolorboxenvironment{axiom}{}
\tcolorboxenvironment{assertion}{}
\tcolorboxenvironment{lemma}{}
\tcolorboxenvironment{proposition}{}
\tcolorboxenvironment{corollary}{}
\tcolorboxenvironment{definition}{}
\tcolorboxenvironment{proofReplace}{toprule=0mm,bottomrule=0mm,rightrule=0mm, colback=white, breakable }

\declaretheorem[name=Теорема, numberwithin=chapter, style=myStyleWithFrame]{theorem}
\declaretheorem[name=Теорема, numbered=no, style=myStyleWithFrame]{theorem*}
\declaretheorem[name=Аксиома, sibling=theorem, style=myStyleWithFrame]{axiom}
\declaretheorem[name=Преположение, sibling=theorem, style=myStyleWithFrame]{assertion}
\declaretheorem[name=Лемма, sibling=theorem, style=myStyleWithFrame]{lemma}
\declaretheorem[name=Предложение, sibling=theorem, style=myStyleWithFrame]{proposition}
\declaretheorem[name=Следствие, numberwithin=theorem, style=myStyleWithFrame]{corollary}

\declaretheorem[name=Определение, numberwithin=chapter, style=myStyleWithFrame]{definition}
\declaretheorem[name=Свойство, numberwithin=chapter, style=myStyle]{property}
\declaretheorem[name=Свойства, numbered=no, style=myStyle]{propertylist}

\declaretheorem[name=Пример, numberwithin=chapter, style=myStyle]{example}
\declaretheorem[name=Замечание, numbered=no, style=myStyle]{remark}

\declaretheorem[name=Доказательство, numbered=no, style=myProofStyleWithFrame]{proofReplace}
\renewenvironment{proof}[1][\proofname]{\begin{proofReplace}}{\end{proofReplace}}
\declaretheorem[name=Доказательство, numbered=no, style=myProofStyleWithFrame]{longProof}

%%% Memoir settings
\chapterstyle{ger}
\setlength{\headheight}{2\baselineskip}

%%% HyperRef
\usepackage{hyperref}
